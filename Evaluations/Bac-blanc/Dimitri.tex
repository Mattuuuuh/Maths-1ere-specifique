\documentclass[10pt,a4paper]{article}
\usepackage[utf8]{inputenc}          %type de codage des caractères
\usepackage{amsmath}                 %base pour les théorèmes: American Maths Society
\usepackage{amsfonts}
\usepackage{amssymb}
\usepackage[left=2cm,right=2cm,top=2cm,bottom=2cm]{geometry}
%%%%%%%%%%%%%%%%%%%%%%Des tas de packages pour faire de jolis titres et Ptés%%%%%%%%
\usepackage{framed}                   \usepackage{tikz,tkz-tab}                                                %
\usepackage{mdframed}                                                             %
\usepackage{color}                                                                 %
\usepackage[usenames,dvipsnames]{xcolor}                                           %

\usepackage{tikz}  
%


\usepackage{cancel}                  %pour barrer lors des simplifications dans un calcul

\usepackage{multicol}                %pour écrire du texte sur plusieurs colonnes
\usepackage{enumerate}               % pour numéroter les questions de mathtom
\usepackage{algorithm}  
\usepackage{algorithmic} 
\usepackage{algpseudocode}
\usepackage{enumitem}
\usepackage{color}   
\usepackage{hyperref}


%%%%%%%%%%%%%%%%%%%%%%%%%%%%%%%%%%%%%%%%%%%%%
% Commandes permettant d'écrire cm, dm, mm  %
% et m en mode maths (sans italique)        %
\newcommand{\m}{\ensuremath{\textrm{~m}}}   %
\newcommand{\dm}{\ensuremath{\textrm{~dm}}} %
\newcommand{\cm}{\ensuremath{\textrm{~cm}}} %
\newcommand{\mm}{\ensuremath{\textrm{~mm}}} %
%%%%%%%%%%%%%%%%%%%%%%%%%%%%%%%%%%%%%%%%%%%%%

%%%%%%%%%%%%%%%%%%%%%%%%%%%%%%%%%%%%%%%%%%%%%
% Commandes permettant d'écrire les noms    %
% des ensembles, et les équivalences        % 
\newcommand{\R}{\ensuremath{\mathbb{R}}}    %
\newcommand{\N}{\ensuremath{\mathbb{N}}}    %
\newcommand{\Q}{\ensuremath{\mathbb{Q}}}    %
\newcommand{\Z}{\ensuremath{\mathbb{Z}}}  
\newcommand{\C}{\ensuremath{\mathbb{C}}}

\newcommand{\U}{\ensuremath{\mathbb{U}}}
\newcommand{\K}{\ensuremath{\mathbb{K}}}
\newcommand{\SSI}{\Longleftrightarrow}      %
\newcommand{\ssi}{\Leftrightarrow}          %
\newcommand{\Zp}[1]{\mathbb{Z}/p^{#1}\mathbb{Z}}
\newcommand{\Zn}[1]{\mathbb{Z}/n^{#1}\mathbb{Z}}

\usepackage{tikz}

\newcommand{\titlebox}[2]{%
\tikzstyle{titlebox}=[rectangle,inner sep=10pt,inner ysep=10pt,draw]%
\tikzstyle{title}=[fill=white]%
%
\bigskip\noindent\begin{tikzpicture}
\node[titlebox] (box){%
    \begin{minipage}{0.94\textwidth}
#2
    \end{minipage}
};
%\draw (box.north west)--(box.north east);
\node[title] at (box.north) {#1};
\end{tikzpicture}\bigskip%
}

\date{}
\title{Bac Blanc (proposition)}
\begin{document}

\maketitle

\section*{Exercice 2 :}

On étudie le trajet d'une voiture (voiture A) en fonction du temps. A l'instant $t = 2$ secondes, la voiture a parcouru 10 mètres. A l'instant $t = 5$ secondes, la voiture a parcouru 25 mètres. On considère que la voiture avance à vitesse constante et en ligne droite. \\
On modélise la distance (en mètres) parcourue par la voiture à l'aide d'une fonction affine $f$. Ainsi, si $t$ représente le temps en secondes, on pose :
$$f(t) = mt+p$$
avec $m$ et $p$ des coefficients inconnus.

\begin{enumerate}
     \item A-t-on ici choisi un modèle discret ou continu ? Justifier.
    \item Ce modèle est-il à croissance linéaire ou exponentielle ? Justifier la réponse.
    \item Donner les valeurs de $f(2)$ et $f(5)$.
    \item A l'aide d'un calcul, justifier que $m=5$.
    \item A quelle vitesse va la voiture ? (donner le résultat en m/s puis en km/h)
    \item A l'aide d'un calcul, justifier que $p=0$.
    \item Une autre voiture (voiture B) démarre au même moment que la voiture A, mais avec 4 mètres d'avance. Par contre, elle ne roule qu'à 3 mètres/seconde. On utilise une fonction $g$ qui représente la distance parcourue par cette voiture B en fonction du temps $t$ (en secondes toujours).
    \begin{enumerate}
        \item On a tracé dans le repère suivant les fonctions $f$ et $g$. A laquelle de ces deux fonctions correspond la droite en pointillés ? \\[0.5cm]
                    \includegraphics[width=0.5\linewidth]{courbes.png}
          
               

        \item Résoudre graphiquement l'équation $f(t)=g(t)$.
        \item A l'aide des questions précédentes, répondre à la problématique suivante : la voiture A va-t-elle rattraper la voiture B ? Si oui, au bout de combien de secondes ? Justifier les réponses.        
    \end{enumerate}
\end{enumerate}


\section*{Correction}

\begin{enumerate}
    \item On a choisi un modèle continue car on utilise une fonction et non pas une suite pour modéliser la distance.
    \item Ce modèle est à croissance linéaire car la fonction utilisée est affine.
    \item D'après l'énoncé, $f(2) = 10$ et $f(4)=25$.
    \item $m$ est le coefficient directeur de $f$, donc d'après la formule du cours :
    $$m = \dfrac{f(5)-f(2)}{5-2}$$
    $$ = \dfrac{25-10}{3}$$
    $$ = 5$$
    \item En 3 secondes, la voiture a parcouru 15 mètres. Donc, sa vitesse est :
    $$v = \dfrac{distance}{temps}$$
    $$ = \dfrac{15}{3}$$
    $$ = 5m/s$$
    Pour convertir en km/h, on multiplie par 3600 et on divise par 1000. On trouve $18$ km/h.
    \item On sait que $f(2)=10$, donc $5 \times 2 +p = 10$, donc $p=0$.
    \item 
    \begin{enumerate}
        \item La fonction en pointillés est $g$ car son ordonnée à l'origine est 4, ce qui représente les 4m d'avance qu'elle au démarrage. On peut aussi dire que son coefficient directeur est inférieur à celui de la droite en trait plein, or c'est bien la voiture B qui va moins vite que la voiture A.
    \item Le point d'intersection des deux droites a pour abscisse 2, donc $S = \{2 \}$.
    \item D'après la question précédente, on en déduit qu'au bout de 2 secondes, les voitures auront parcouru la même distance (en prenant en compte les 4m d'avance de la voiture B), donc la voiture A aura rattrapé la B au bout de deux secondes.
    \end{enumerate}
    
\end{enumerate}

\end{document}
