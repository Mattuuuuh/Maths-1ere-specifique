\documentclass[a4paper, 12pt]{extarticle}

\usepackage[utf8x]{inputenc}
%fonts
\usepackage{libertinus,libertinust1math}
\usepackage{amsmath,amsthm,amssymb,mathtools}

% SOLUTION SWITCH

\ifsolutions
	\newcommand{\exe}[2]{
		\begin{ex} #1  \end{ex}
		\begin{sol} #2 \end{sol}
	}
\else
	\newcommand{\exe}[2]{
		\begin{ex} #1  \end{ex}
	}
	
\fi


\usepackage[french]{babel}
\usepackage[
a4paper,
margin=2cm,
nomarginpar,% We don't want any margin paragraphs
]{geometry}

% HEADER, ARRAY, ENUM, MULTIOCL
\usepackage{fancyhdr}
\usepackage{array}
\usepackage{multicol, enumitem}
\newcolumntype{P}[1]{>{\centering\arraybackslash}p{#1}}
\usepackage{stackengine}
\newcommand\xrowht[2][0]{\addstackgap[.5\dimexpr#2\relax]{\vphantom{#1}}}

% theorems

\theoremstyle{theorem}
\newtheorem{thm}{Théorème}
\theoremstyle{plain}
\newtheorem*{sol}{Solution}
\theoremstyle{definition}
\newtheorem{ex}{Exercice}
\newtheorem{dfn}{Définition}
\newtheorem*{dfn*}{Définition}


%couleurs
\usepackage{tcolorbox}
\definecolor{myg}{RGB}{56, 140, 70}
\definecolor{myb}{RGB}{45, 111, 177}
\definecolor{myr}{RGB}{199, 68, 64}
\definecolor{mygr}{HTML}{2C3338}


\tcbuselibrary{theorems,skins,hooks}
\newcounter{commonbox}
\makeatletter
\newtcbtheorem[use counter=commonbox]{theorem}{Théorème }%
{
	enhanced,
	colback=white,
	colframe=mygr,
	attach boxed title to top left={yshift*=-\tcboxedtitleheight},
	fonttitle=\bfseries,
	title={#2},
	boxed title size=title,
	boxed title style={%
			sharp corners,
			rounded corners=northwest,
			colback=tcbcolframe,
			boxrule=0pt,
		},
	underlay boxed title={%
			\path[fill=tcbcolframe] (title.south west)--(title.south east)
			to[out=0, in=180] ([xshift=5mm]title.east)--
			(title.center-|frame.east)
			[rounded corners=\kvtcb@arc] |-
			(frame.north) -| cycle;
		},
	#1
}{th}
\newtcbtheorem[use counter=commonbox]{rappel}{Rappel }%
{
	enhanced,
	colback=white,
	colframe=mygr,
	attach boxed title to top left={yshift*=-\tcboxedtitleheight},
	fonttitle=\bfseries,
	title={#2},
	boxed title size=title,
	boxed title style={%
			sharp corners,
			rounded corners=northwest,
			colback=tcbcolframe,
			boxrule=0pt,
		},
	underlay boxed title={%
			\path[fill=tcbcolframe] (title.south west)--(title.south east)
			to[out=0, in=180] ([xshift=5mm]title.east)--
			(title.center-|frame.east)
			[rounded corners=\kvtcb@arc] |-
			(frame.north) -| cycle;
		},
	#1
}{th}
\newtcbtheorem[use counter=commonbox]{strategie}{Stratégie }%
{
	enhanced,
	colback=white,
	colframe=mygr,
	attach boxed title to top left={yshift*=-\tcboxedtitleheight},
	fonttitle=\bfseries,
	title={#2},
	boxed title size=title,
	boxed title style={%
			sharp corners,
			rounded corners=northwest,
			colback=tcbcolframe,
			boxrule=0pt,
		},
	underlay boxed title={%
			\path[fill=tcbcolframe] (title.south west)--(title.south east)
			to[out=0, in=180] ([xshift=5mm]title.east)--
			(title.center-|frame.east)
			[rounded corners=\kvtcb@arc] |-
			(frame.north) -| cycle;
		},
	#1
}{th}
\newtcbtheorem[use counter=commonbox]{outil}{Outil }%
{
	enhanced,
	colback=white,
	colframe=mygr,
	attach boxed title to top left={yshift*=-\tcboxedtitleheight},
	fonttitle=\bfseries,
	title={#2},
	boxed title size=title,
	boxed title style={%
			sharp corners,
			rounded corners=northwest,
			colback=tcbcolframe,
			boxrule=0pt,
		},
	underlay boxed title={%
			\path[fill=tcbcolframe] (title.south west)--(title.south east)
			to[out=0, in=180] ([xshift=5mm]title.east)--
			(title.center-|frame.east)
			[rounded corners=\kvtcb@arc] |-
			(frame.north) -| cycle;
		},
	#1
}{th}
\newtcbtheorem[use counter=commonbox]{but}{Buts du chapitre }%
{
	enhanced,
	colback=white,
	colframe=mygr,
	attach boxed title to top left={yshift*=-\tcboxedtitleheight},
	fonttitle=\bfseries,
	title={#2},
	boxed title size=title,
	boxed title style={%
			sharp corners,
			rounded corners=northwest,
			colback=tcbcolframe,
			boxrule=0pt,
		},
	underlay boxed title={%
			\path[fill=tcbcolframe] (title.south west)--(title.south east)
			to[out=0, in=180] ([xshift=5mm]title.east)--
			(title.center-|frame.east)
			[rounded corners=\kvtcb@arc] |-
			(frame.north) -| cycle;
		},
	#1
}{th}
\newtcbtheorem[use counter=commonbox]{propriete}{Propriété }%
{
	enhanced,
	colback=white,
	colframe=mygr,
	attach boxed title to top left={yshift*=-\tcboxedtitleheight},
	fonttitle=\bfseries,
	title={#2},
	boxed title size=title,
	boxed title style={%
			sharp corners,
			rounded corners=northwest,
			colback=tcbcolframe,
			boxrule=0pt,
		},
	underlay boxed title={%
			\path[fill=tcbcolframe] (title.south west)--(title.south east)
			to[out=0, in=180] ([xshift=5mm]title.east)--
			(title.center-|frame.east)
			[rounded corners=\kvtcb@arc] |-
			(frame.north) -| cycle;
		},
	#1
}{th}
\newtcbtheorem[number within=commonbox]{definition}{Définition }%
{
	enhanced,
	colback=white,
	colframe=mygr,
	attach boxed title to top left={yshift*=-\tcboxedtitleheight},
	fonttitle=\bfseries,
	title={#2},
	boxed title size=title,
	boxed title style={%
			sharp corners,
			rounded corners=northwest,
			colback=tcbcolframe,
			boxrule=0pt,
		},
	underlay boxed title={%
			\path[fill=tcbcolframe] (title.south west)--(title.south east)
			to[out=0, in=180] ([xshift=5mm]title.east)--
			(title.center-|frame.east)
			[rounded corners=\kvtcb@arc] |-
			(frame.north) -| cycle;
		},
	#1
}{th}
\newtcbtheorem[number within=commonbox]{exemples}{Exemples }%
{
	enhanced,
	colback=white,
	colframe=mygr,
	attach boxed title to top left={yshift*=-\tcboxedtitleheight},
	fonttitle=\bfseries,
	title={#2},
	boxed title size=title,
	boxed title style={%
			sharp corners,
			rounded corners=northwest,
			colback=tcbcolframe,
			boxrule=0pt,
		},
	underlay boxed title={%
			\path[fill=tcbcolframe] (title.south west)--(title.south east)
			to[out=0, in=180] ([xshift=5mm]title.east)--
			(title.center-|frame.east)
			[rounded corners=\kvtcb@arc] |-
			(frame.north) -| cycle;
		},
	#1
}{th}
\newtcbtheorem[number within=commonbox]{exemple}{Exemple }%
{
	enhanced,
	colback=white,
	colframe=mygr,
	attach boxed title to top left={yshift*=-\tcboxedtitleheight},
	fonttitle=\bfseries,
	title={#2},
	boxed title size=title,
	boxed title style={%
			sharp corners,
			rounded corners=northwest,
			colback=tcbcolframe,
			boxrule=0pt,
		},
	underlay boxed title={%
			\path[fill=tcbcolframe] (title.south west)--(title.south east)
			to[out=0, in=180] ([xshift=5mm]title.east)--
			(title.center-|frame.east)
			[rounded corners=\kvtcb@arc] |-
			(frame.north) -| cycle;
		},
	#1
}{th}
\newtcbtheorem[number within=commonbox]{questions}{Questions guidantes }%
{
	enhanced,
	colback=white,
	colframe=mygr,
	attach boxed title to top left={yshift*=-\tcboxedtitleheight},
	fonttitle=\bfseries,
	title={#2},
	boxed title size=title,
	boxed title style={%
			sharp corners,
			rounded corners=northwest,
			colback=tcbcolframe,
			boxrule=0pt,
		},
	underlay boxed title={%
			\path[fill=tcbcolframe] (title.south west)--(title.south east)
			to[out=0, in=180] ([xshift=5mm]title.east)--
			(title.center-|frame.east)
			[rounded corners=\kvtcb@arc] |-
			(frame.north) -| cycle;
		},
	#1
}{th}
\makeatother

% corps
\newcommand{\R}{\mathbb{R}}
\newcommand{\Rnn}{\mathbb{R}^{2n}}
\newcommand{\Z}{\mathbb{Z}}
\newcommand{\N}{\mathbb{N}}
\newcommand{\Q}{\mathbb{Q}}

% domain
\newcommand{\D}{\mathcal{D}}
% for calligraphic C
\usepackage{calrsfs}
\newcommand{\C}{\mathcal{C}}

% date
\usepackage{advdate}

% ensembles tq. 
\newcommand{\xRtq}[1]{
	$\left\{ x \in \R \text{ tq. } #1 \right\}$
}

% vabs
\newcommand{\vabs}[1]{
	\left| #1 \right|
}

%pinfty minfty
\newcommand{\pinfty}{{+}\infty}
\newcommand{\minfty}{{-}\infty}

% plots
\usepackage{pgfplots}

%virgules
\usepackage{icomma}
\pgfplotsset{/pgf/number format/use comma}

%subfigures
\usepackage{subcaption}

%hyperlink footnote
\usepackage{hyperref}

%wider tabulars
\def\arraystretch{2}
\setlength\tabcolsep{15pt}

% tableaux var, signe
\usepackage{tkz-tab}

\SetDate[10/01/2026]
\reversemarginpar

\setlength{\marginparsep}{.5cm}

\renewcommand{\ExerciseHeader}{
	\textbf{Partie \Alph{Exercise}.}
	\ifnum\ExerciseDifficulty=0
	\else
		(\theExerciseDifficulty)
	\fi
}
\renewcommand{\AnswerHeader}{
	\textbf{Partie \Alph{Exercise}.}
}

\begin{document}
\pagestyle{fancy}
\fancyhead[L]{Première spécifique}
\fancyhead[C]{\textbf{Baccalauréat blanc}}
\fancyhead[R]{\today}

\null\vspace{-30pt}
Consignes particulières : 
\begin{itemize}[label=$\bullet$]
	\item 
	La calculatrice est {interdite}.
	\item
	L'évaluation fait \pageref{final-page} pages.
\end{itemize}

\hrule

\begin{center}
\section*{PREMIÈRE PARTIE : AUTOMATISMES — QCM (6pts)}
\end{center}

\textbf{
	Pour cette première partie, aucune justification n'est demandée et une seule réponse est possible par question.
	Pour chaque question, reportez son numéro sur votre copie et indiquez votre réponse.
}

\vspace{30pt}


\newcommand{\cm}{\text{ cm}}

\begin{enumerate}[label=\textbf{\arabic*.}]
	\item 
	Le nombre $x$ vérifiant $6x + 8 = 16x - 2$ est
	\begin{multicols}{4}
	\begin{enumerate}[label=\textbf{\alph*)}]
		\item 
		$x = 1$
		\item 
		$x = \frac{10}{22}$
		\item 
		$x = -1$
		\item 
		$x = 0$
	\end{enumerate}
	\end{multicols}
	
	\item 
	La suite de nombre $2 ; 4 ; 8 ; 16$ est en progression
	\begin{multicols}{2}
	\begin{enumerate}[label=\textbf{\alph*)}]
		\item
		 arithmétique uniquement.
		\item
		géométrique uniquement.
		\item
		arithmétique et géométrique.
		\item
		ni arithmétique ni géométrique.
	\end{enumerate}
	\end{multicols}
	
	\item 
	Une diminution de 30\% correspond à
	\begin{multicols}{2}
	\begin{enumerate}[label=\textbf{\alph*)}]
		\item
		une multiplication par 0,3.
		\item
		une multiplication par 1,3.
		\item
		une multiplication par 0,7.
		\item
		une division par 1,3.
	\end{enumerate}
	\end{multicols}
	
	\item 
	Le prix d'un objet passe de 50€ à 75€.
	Ceci correspond à
	\begin{multicols}{2}
	\begin{enumerate}[label=\textbf{\alph*)}]
		\item
		une augmentation de 25\%.
		\item
		une augmentation de 125\%.
		\item
		une augmentation de 150\%.
		\item
		une augmentation de 50\%.
	\end{enumerate}
	\end{multicols}
	
	\item 
	Considérons la fonction $f(x) = \dfrac{1}{2x+1}$.
	L'image de 2 par la fonction $f$ est égale à
	\begin{multicols}{4}
	\begin{enumerate}[label=\textbf{\alph*)}]
		\item
		0,2
		\item
		5
		\item
		0,5
		\item
		1,5
	\end{enumerate}
	\end{multicols}
	
	\item 
	Soit $A = \dfrac12 - \dfrac12 \times \dfrac43$.
	Alors
	\begin{multicols}{4}
	\begin{enumerate}[label=\textbf{\alph*)}]
		\item
		$A = 0$
		\item
		$A = -\frac16$
		\item
		$A = \frac23$
		\item
		$A = -1$
	\end{enumerate}
	\end{multicols}
	
	\item 
	Soit $E = \dfrac{2^3}{2^8}\times2^{-1}$.
	Alors
	\begin{multicols}{4}
	\begin{enumerate}[label=\textbf{\alph*)}]
		\item
		$E = 2^{12}$
		\item
		$E = 2^{-6}$
		\item
		$E = 2^{10}$
		\item
		$E = 2^{-4}$
	\end{enumerate}
	\end{multicols}
	
	\item 
	Le nombre $5^2 + 10^{-100}$ est environ égal à
	\begin{multicols}{4}
	\begin{enumerate}[label=\textbf{\alph*)}]
		\item
		$-1~000$
		\item
		$10$
		\item
		$25$
		\item
		$1~000$
	\end{enumerate}
	\end{multicols}
	
	\item 
	L'ordonnée à l'origine de la fonction affine $f(x) = 3x - 4$ est égale à
	\begin{multicols}{4}
	\begin{enumerate}[label=\textbf{\alph*)}]
		\item
		$3$
		\item
		$-3$
		\item
		$4$
		\item
		$-4$
	\end{enumerate}
	\end{multicols}
\end{enumerate}

\begin{multicols}{2}
	\begin{enumerate}[label=\textbf{\arabic*.}, resume]
		\item 
		Considérons la droite de la figure ci-contre.
		La seule équation pouvant lui correspondre est
		\begin{multicols}{2}
		\begin{enumerate}[label=\textbf{\alph*)}]
			\item $y=1-x$
			\item $y=-x-1$
			\item $y=1+x$
			\item $y=x-1$
		\end{enumerate}
		\end{multicols}
	\end{enumerate}
	\vfill\null
	\centering
	\begin{tikzpicture}[scale=.7]
	\begin{axis}[xmin = -2, xmax=2, ymin=-2, ymax=5, axis x line=middle, axis y line=middle, axis line style=<->, xlabel={}, ylabel={}, grid=none, grid style = {opacity=.5}, clip=true, ticks = none]
		\addplot[BLUE_E, very thick, domain =-2:4, samples=2] {1-x};% node[below right, pos=.6]{$\C_f$} ;
	\end{axis}
	\end{tikzpicture}
\end{multicols}

\begin{enumerate}[label=\textbf{\arabic*.}]\setcounter{enumi}{10}
	
	\item 
	En 2019, les recettes fiscales recouvrées par les Finances publiques s'élèvent à 464 milliards d'euros.
	En 2022, elles atteignent 544,4 milliards d'euros.
	
	Le calcul permettant d'obtenir le pourcentage d'augmentation annuel moyen est
	\begin{multicols}{2}
	\begin{enumerate}[label=\textbf{\alph*)}]
		\item
		$\left( \dfrac{544,4}{464} - 1\right)\times100$
		\item
		$\left(\left( \dfrac{464}{544,4}\right)^{1/3} - 1 \right)\times100$
		\item
		$\left(\left( \dfrac{544,4}{464}\right)^{3} - 1 \right)\times100$
		\item
		$\left(\left( \dfrac{544,4}{464} \right)^{1/3} - 1 \right)\times100$
	\end{enumerate}
	\end{multicols}
		
\end{enumerate}


\begin{multicols}{2}
	\begin{enumerate}[label=\textbf{\arabic*.}]\setcounter{enumi}{11}
		\item 
		Considérons la fonction exponentielle $f$ représentée ci-contre.
		La seule expression algébrique pouvant lui correspondre est
		\begin{multicols}{2}
		\begin{enumerate}[label=\textbf{\alph*)}]
			\item $f(x)=0,5 \times 2^x$
			\item $f(x)=0,5x + 3$
			\item $f(x)=2 \times 0,5^x$
			\item $f(x)=3x + 0,5$
		\end{enumerate}
		\end{multicols}
	\end{enumerate}
	\vfill\null
	\centering
	\begin{tikzpicture}[scale=1]
	\begin{axis}[xmin = -3, xmax=4, ymin=-3, ymax=7, axis x line=middle, axis y line=middle, axis line style=<->, xlabel={}, ylabel={}, grid=both, grid style = {opacity=.5}, clip=true, xtick distance=1, ytick distance=1]
		\addplot[RED_E, very thick, domain =-4:4, samples=50] {.5*2^x} node[right, pos=.7]{$\C_f$} ;
	\end{axis}
	\end{tikzpicture}
\end{multicols}

\newpage

\begin{center}
\section*{DEUXIÈME PARTIE (14pts)}
\end{center}
%\marginpar{[pts]}

\subsection*{Exercice 1 (8 points)}


Un élève du lycée Marguerite Yourcenar a répandu une rumeur.
Aujourd'hui, jour 0, on compte 100 élèves ayant appris la rumeur.
Chaque jour, la rumeur se propage jusqu'à inévitablement atteindre les 900 élèves du lycée.
Le but des questions suivantes est d'étudier deux propagations possibles de la rumeur.


\exe{}{
	Le jour d'après, 20 nouveaux élèves ont appris la rumeur.
	\begin{enumerate}
		\item Combien d'élèves connaissent la rumeur au jour 1 ?
		\item Quel pourcentage d'augmentation cela représente-t-il ?
	\end{enumerate}
}{exe:1}{
	\begin{enumerate}
		\item
		$100 + 20 = 120$ élèves connaissent la rumeur au jour 1.
		\item 
		Le coefficient multiplicateur est donné par $CM = \frac{120}{100} = 1,2$.
		Comme $CM - 1 = 0,2 = 20\%$, l'augmentation est de 20\%.
	\end{enumerate}
}

\exe{}{
	Dans cette partie, on suppose que 20 nouveaux élèves apprennent la rumeur chaque jour, tant que cela est possible.

	\begin{enumerate}
		%\item
		%De quelle nature est la propagation ? Donner son terme initial et sa raison.
		\item
		Combien d'élèves connaitront la rumeur au jour 10 ?
		%\item
		%Soit $n$ un entier naturel.
		%Déterminer, en fonction de $n$, le nombre d'élèves connaissant la rumeur au jour $n$.
		\item
		Est-il possible qu'un jour, 290 élèves connaissent la rumeur ?
		\item
		Au bout de combien de jours le lycée tout entier aura-t-il appris la rumeur ?
	\end{enumerate}
}{exe:2}{
	\begin{enumerate}
		\item
		En partant de 100 élèves, et en ajoutant 20 dix fois, on obtient
			\[ 100 + 10\times20 = 100 + 200 = 300. \]
		Donc 300 élèves connaitront la rumeur au jour 10.
		\item
		Au jour 9, $300 - 20 = 280$ élèves connaitront la rumeur.
		Dans notre modèle discret, il n'est pas possible que 290 élèves connaissent la rumeur, car le nombre d'élèves passe de 280 à 300 d'un jour à l'autre.
		\item
		Le nombre d'élèves connaissant la rumeur est une suite arithmétique de terme initial 100 et de raison 20.
		Au jour $n$, il est donc égal à $100 + 20n$.
		Posons $100 + 20n = 900$ et trouvons $n$.
			\begin{align*}
				100 + 20n &= 900 \\
				20n &= 800 \\
				n &= \dfrac{800}{20} = 40
			\end{align*}
		Par conséquent, le lycée tout entier aura appris la rumeur au jour 40.
		On peut vérifier que $100 + 40\times20 = 900$.
	\end{enumerate}
}

\exe{}{
	Dans cette partie, on suppose que le nombre d'élèves ayant appris la rumeur augmente de 20\% chaque jour, tant que cela est possible.
	
	On pourra s'aider du tableau ci-dessous pour répondre aux questions suivantes.
	\begin{enumerate}
		%\item
		%De quelle nature est la propagation ? Donner son terme initial et sa raison.
		\item
		Combien d'élèves connaitront la rumeur au jour 4 ?
		\item
		Soit $n$ un entier naturel.
		Déterminer, en fonction de $n$, le nombre d'élèves connaissant la rumeur au jour $n$.
		\item
		Au bout de combien de jours le lycée tout entier aura-t-il appris la rumeur ?
	\end{enumerate}
	
	
	\begin{center}
\setlength\tabcolsep{12pt}
	\begin{tabular}{|c|c|c|c|c|c|c|c|c|c|c|}\hline
		$n =$ & 0 & 1 & 2 & 4 & 5 & 10 & 12 & 13 & 14 & 15 \\ \hline
		$1,2^n \approx$ & 1 & 1,2 & 1,44 & 2,07 & 2,49 & 6,19 & 8,92 & 10,70 & 12,84 & 15,40 \\ \hline
	\end{tabular}
	\end{center}
	
}{exe:3}{
	\begin{enumerate}
		\item
		Après quatre évolutions successives de $+20\%$, le nombre d'élèves connaissant la rumeur s'élève à 
			\[ 100 \times 1,2^4 = 100 \times 2,07 = 207. \]
		\item
		Après $n$ évolutions successives de $+20\%$, le nombre d'élèves connaissant la rumeur s'élève à 
			\[ 100 \times 1,2^n. \]
		\item
		Il s'agit de trouver le plus petit entier naturel $n\in\N$ vérifiant 
			\begin{align*}
				100 \times 1,2^n \geq 900 && \iff && 1,2^n \geq 9.
			\end{align*}
		D'après le tableau, $n=13$ est le premier $n$ vérifiant l'inégalité : le lycée tout entier aura appris la rumeur après 13 jours. 
	\end{enumerate}
}


%\exe{4}{
%	Réaliser sur votre copie un \emph{schéma} sur lequel apparaissent l'allure des nuages de points traduisant la progression du nombre d'élèves connaissant la rumeur, aussi bien dans le cas de la question \ref{exe:2} que dans le cas de la question \ref{exe:3}.
%	Dans chacun des cas, y faire figurer le jour où le lycée tout entier apprend la rumeur. 
%}{exe:4}{
%	todo
%}

\newpage

\subsection*{Exercice 2 (6 points)}

On étudie le trajet d'une voiture (voiture A) en fonction du temps. À l'instant $t = 2$ secondes, la voiture a parcouru 10 mètres. À l'instant $t = 5$ secondes, la voiture a parcouru 25 mètres. On considère que la voiture avance à vitesse constante et en ligne droite. 

On modélise la distance (en mètres) parcourue par la voiture à l'aide d'une fonction affine $f$. Ainsi, si $t$ représente le temps en secondes, on pose :
$$f(t) = mt+p$$
avec $m$ et $p$ des coefficients inconnus.

\begin{enumerate}
    %\item A-t-on ici choisi un modèle discret ou continu ? Justifier.
    %\item Ce modèle est-il à croissance linéaire ou exponentielle ? Justifier la réponse.
    \item Donner les valeurs de $f(2)$ et $f(5)$.
    \item A l'aide d'un calcul, justifier que $m=5$.
    \item A quelle vitesse va la voiture ? (donner le résultat en m/s puis en km/h)
    \item A l'aide d'un calcul, justifier que $p=0$.
    \item Une voiture B démarre au même moment que la voiture A, mais avec 4 mètres d'avance. Par contre, elle ne roule qu'à 3 mètres/seconde. On utilise une fonction $g$ qui représente la distance parcourue par cette voiture B en fonction du temps $t$ (toujours en secondes).
    \begin{enumerate}
        \item On a tracé dans le repère suivant les fonctions $f$ et $g$. À laquelle de ces deux fonctions correspond la droite en pointillés ?
		\begin{center}
		\begin{tikzpicture}[>=stealth, scale=1.2]
		\begin{axis}[xmin = -.5, xmax=3.9, ymin=-.5, ymax=17, axis x line=middle, axis y line=middle, axis line style=->, xlabel={}, ylabel={}, grid=both, grid style = {opacity=.5}, ytick distance = 5, xtick distance = 1, extra x ticks = {0}, extra y ticks = {0}, clip=true, xlabel = temps $t$ (secondes), ylabel = distance (mètres), ylabel near ticks, xlabel near ticks]
		
			\addplot[BLUE_E, very thick, domain =-1:4, samples=2, dashed] {4+3*x}  node[pos = .8, below right, transparent] {$\C_g$};	
			\addplot[PURPLE_E, very thick, domain =-1:4, samples=2] {5*x}  node[pos = .8, above left, transparent] {$\C_f$};			
		\end{axis}
		\end{tikzpicture}
		\end{center}
			
          
               

        \item Résoudre graphiquement l'équation $f(t)=g(t)$.
        \item À l'aide des questions précédentes, répondre à la problématique suivante : la voiture A va-t-elle rattraper la voiture B ? Si oui, au bout de combien de secondes ? Justifier les réponses.        
    \end{enumerate}
\end{enumerate}

%%%%%%%%%%%%
\label{final-page}

\newpage
\fancyhead[C]{\textbf{Solutions première partie}}


\begin{enumerate}[label=\textbf{\arabic*.}]
	\item a)
	\item b)
	\item c)
	\item d)
	\item a)
	\item b)
	\item b)
	\item c)
	\item d)
	\item a)
	\item d)
	\item a)
\end{enumerate}

\newpage
\fancyhead[C]{\textbf{Solutions deuxième partie}}

\subsection*{Exercice 1}

\shipoutAnswer

\newpage

\subsection*{Exercice 2}


\begin{enumerate}
    %\item On a choisi un modèle continue car on utilise une fonction et non pas une suite pour modéliser la distance.
    %\item Ce modèle est à croissance linéaire car la fonction utilisée est affine.
    \item D'après l'énoncé, $f(2) = 10$ et $f(4)=25$.
    \item $m$ est le coefficient directeur de $f$, donc d'après la formule du cours :
    $$m = \dfrac{f(5)-f(2)}{5-2}$$
    $$ = \dfrac{25-10}{3}$$
    $$ = 5$$
    \item En 3 secondes, la voiture a parcouru 15 mètres. Donc, sa vitesse est :
    $$v = \dfrac{\text{distance}}{\text{temps}}$$
    $$ = \dfrac{15}{3}$$
    $$ = 5\text{m/s}$$
    Pour convertir en km/h, on multiplie par 3600 et on divise par 1000. On trouve $18$ km/h.
    \item On sait que $f(2)=10$, donc $5 \times 2 +p = 10$, donc $p=0$.
    \item 
    \begin{enumerate}
        \item La fonction en pointillés est $g$ car son ordonnée à l'origine est 4, ce qui représente les 4m d'avance qu'elle au démarrage. On peut aussi dire que son coefficient directeur est inférieur à celui de la droite en trait plein, or c'est bien la voiture B qui va moins vite que la voiture A.
        
		\begin{center}
		\begin{tikzpicture}[>=stealth, scale=1.2]
		\begin{axis}[xmin = -.5, xmax=3.9, ymin=-.5, ymax=17, axis x line=middle, axis y line=middle, axis line style=->, xlabel={}, ylabel={}, grid=both, grid style = {opacity=.5}, ytick distance = 5, xtick distance = 1, extra x ticks = {0}, extra y ticks = {0}, clip=true, xlabel = temps $t$ (secondes), ylabel = distance (mètres), ylabel near ticks, xlabel near ticks]
		
			\addplot[BLUE_E, very thick, domain =-1:4, samples=2, dashed] {4+3*x}  node[pos = .8, below right] {$\C_g$};	
			\addplot[PURPLE_E, very thick, domain =-1:4, samples=2] {5*x}  node[pos = .8, above left] {$\C_f$};			
		\end{axis}
		\end{tikzpicture}
		\end{center}
    \item Le point d'intersection des deux droites a pour abscisse 2, donc $S = \{2 \}$.
    \item D'après la question précédente, on en déduit qu'au bout de 2 secondes, les voitures auront parcouru la même distance (en prenant en compte les 4m d'avance de la voiture B), donc la voiture A aura rattrapé la B au bout de deux secondes.
    \end{enumerate}
    
    
\end{enumerate}

\end{document}
