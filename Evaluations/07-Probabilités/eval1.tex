\documentclass[a4paper, 12pt]{extarticle}

\usepackage[utf8x]{inputenc}
%fonts
\usepackage{libertinus,libertinust1math}
\usepackage{amsmath,amsthm,amssymb,mathtools}

% SOLUTION SWITCH

\ifsolutions
	\newcommand{\exe}[2]{
		\begin{ex} #1  \end{ex}
		\begin{sol} #2 \end{sol}
	}
\else
	\newcommand{\exe}[2]{
		\begin{ex} #1  \end{ex}
	}
	
\fi


\usepackage[french]{babel}
\usepackage[
a4paper,
margin=2cm,
nomarginpar,% We don't want any margin paragraphs
]{geometry}

% HEADER, ARRAY, ENUM, MULTIOCL
\usepackage{fancyhdr}
\usepackage{array}
\usepackage{multicol, enumitem}
\newcolumntype{P}[1]{>{\centering\arraybackslash}p{#1}}
\usepackage{stackengine}
\newcommand\xrowht[2][0]{\addstackgap[.5\dimexpr#2\relax]{\vphantom{#1}}}

% theorems

\theoremstyle{theorem}
\newtheorem{thm}{Théorème}
\theoremstyle{plain}
\newtheorem*{sol}{Solution}
\theoremstyle{definition}
\newtheorem{ex}{Exercice}
\newtheorem{dfn}{Définition}
\newtheorem*{dfn*}{Définition}


%couleurs
\usepackage{tcolorbox}
\definecolor{myg}{RGB}{56, 140, 70}
\definecolor{myb}{RGB}{45, 111, 177}
\definecolor{myr}{RGB}{199, 68, 64}
\definecolor{mygr}{HTML}{2C3338}


\tcbuselibrary{theorems,skins,hooks}
\newcounter{commonbox}
\makeatletter
\newtcbtheorem[use counter=commonbox]{theorem}{Théorème }%
{
	enhanced,
	colback=white,
	colframe=mygr,
	attach boxed title to top left={yshift*=-\tcboxedtitleheight},
	fonttitle=\bfseries,
	title={#2},
	boxed title size=title,
	boxed title style={%
			sharp corners,
			rounded corners=northwest,
			colback=tcbcolframe,
			boxrule=0pt,
		},
	underlay boxed title={%
			\path[fill=tcbcolframe] (title.south west)--(title.south east)
			to[out=0, in=180] ([xshift=5mm]title.east)--
			(title.center-|frame.east)
			[rounded corners=\kvtcb@arc] |-
			(frame.north) -| cycle;
		},
	#1
}{th}
\newtcbtheorem[use counter=commonbox]{rappel}{Rappel }%
{
	enhanced,
	colback=white,
	colframe=mygr,
	attach boxed title to top left={yshift*=-\tcboxedtitleheight},
	fonttitle=\bfseries,
	title={#2},
	boxed title size=title,
	boxed title style={%
			sharp corners,
			rounded corners=northwest,
			colback=tcbcolframe,
			boxrule=0pt,
		},
	underlay boxed title={%
			\path[fill=tcbcolframe] (title.south west)--(title.south east)
			to[out=0, in=180] ([xshift=5mm]title.east)--
			(title.center-|frame.east)
			[rounded corners=\kvtcb@arc] |-
			(frame.north) -| cycle;
		},
	#1
}{th}
\newtcbtheorem[use counter=commonbox]{strategie}{Stratégie }%
{
	enhanced,
	colback=white,
	colframe=mygr,
	attach boxed title to top left={yshift*=-\tcboxedtitleheight},
	fonttitle=\bfseries,
	title={#2},
	boxed title size=title,
	boxed title style={%
			sharp corners,
			rounded corners=northwest,
			colback=tcbcolframe,
			boxrule=0pt,
		},
	underlay boxed title={%
			\path[fill=tcbcolframe] (title.south west)--(title.south east)
			to[out=0, in=180] ([xshift=5mm]title.east)--
			(title.center-|frame.east)
			[rounded corners=\kvtcb@arc] |-
			(frame.north) -| cycle;
		},
	#1
}{th}
\newtcbtheorem[use counter=commonbox]{outil}{Outil }%
{
	enhanced,
	colback=white,
	colframe=mygr,
	attach boxed title to top left={yshift*=-\tcboxedtitleheight},
	fonttitle=\bfseries,
	title={#2},
	boxed title size=title,
	boxed title style={%
			sharp corners,
			rounded corners=northwest,
			colback=tcbcolframe,
			boxrule=0pt,
		},
	underlay boxed title={%
			\path[fill=tcbcolframe] (title.south west)--(title.south east)
			to[out=0, in=180] ([xshift=5mm]title.east)--
			(title.center-|frame.east)
			[rounded corners=\kvtcb@arc] |-
			(frame.north) -| cycle;
		},
	#1
}{th}
\newtcbtheorem[use counter=commonbox]{but}{Buts du chapitre }%
{
	enhanced,
	colback=white,
	colframe=mygr,
	attach boxed title to top left={yshift*=-\tcboxedtitleheight},
	fonttitle=\bfseries,
	title={#2},
	boxed title size=title,
	boxed title style={%
			sharp corners,
			rounded corners=northwest,
			colback=tcbcolframe,
			boxrule=0pt,
		},
	underlay boxed title={%
			\path[fill=tcbcolframe] (title.south west)--(title.south east)
			to[out=0, in=180] ([xshift=5mm]title.east)--
			(title.center-|frame.east)
			[rounded corners=\kvtcb@arc] |-
			(frame.north) -| cycle;
		},
	#1
}{th}
\newtcbtheorem[use counter=commonbox]{propriete}{Propriété }%
{
	enhanced,
	colback=white,
	colframe=mygr,
	attach boxed title to top left={yshift*=-\tcboxedtitleheight},
	fonttitle=\bfseries,
	title={#2},
	boxed title size=title,
	boxed title style={%
			sharp corners,
			rounded corners=northwest,
			colback=tcbcolframe,
			boxrule=0pt,
		},
	underlay boxed title={%
			\path[fill=tcbcolframe] (title.south west)--(title.south east)
			to[out=0, in=180] ([xshift=5mm]title.east)--
			(title.center-|frame.east)
			[rounded corners=\kvtcb@arc] |-
			(frame.north) -| cycle;
		},
	#1
}{th}
\newtcbtheorem[number within=commonbox]{definition}{Définition }%
{
	enhanced,
	colback=white,
	colframe=mygr,
	attach boxed title to top left={yshift*=-\tcboxedtitleheight},
	fonttitle=\bfseries,
	title={#2},
	boxed title size=title,
	boxed title style={%
			sharp corners,
			rounded corners=northwest,
			colback=tcbcolframe,
			boxrule=0pt,
		},
	underlay boxed title={%
			\path[fill=tcbcolframe] (title.south west)--(title.south east)
			to[out=0, in=180] ([xshift=5mm]title.east)--
			(title.center-|frame.east)
			[rounded corners=\kvtcb@arc] |-
			(frame.north) -| cycle;
		},
	#1
}{th}
\newtcbtheorem[number within=commonbox]{exemples}{Exemples }%
{
	enhanced,
	colback=white,
	colframe=mygr,
	attach boxed title to top left={yshift*=-\tcboxedtitleheight},
	fonttitle=\bfseries,
	title={#2},
	boxed title size=title,
	boxed title style={%
			sharp corners,
			rounded corners=northwest,
			colback=tcbcolframe,
			boxrule=0pt,
		},
	underlay boxed title={%
			\path[fill=tcbcolframe] (title.south west)--(title.south east)
			to[out=0, in=180] ([xshift=5mm]title.east)--
			(title.center-|frame.east)
			[rounded corners=\kvtcb@arc] |-
			(frame.north) -| cycle;
		},
	#1
}{th}
\newtcbtheorem[number within=commonbox]{exemple}{Exemple }%
{
	enhanced,
	colback=white,
	colframe=mygr,
	attach boxed title to top left={yshift*=-\tcboxedtitleheight},
	fonttitle=\bfseries,
	title={#2},
	boxed title size=title,
	boxed title style={%
			sharp corners,
			rounded corners=northwest,
			colback=tcbcolframe,
			boxrule=0pt,
		},
	underlay boxed title={%
			\path[fill=tcbcolframe] (title.south west)--(title.south east)
			to[out=0, in=180] ([xshift=5mm]title.east)--
			(title.center-|frame.east)
			[rounded corners=\kvtcb@arc] |-
			(frame.north) -| cycle;
		},
	#1
}{th}
\newtcbtheorem[number within=commonbox]{questions}{Questions guidantes }%
{
	enhanced,
	colback=white,
	colframe=mygr,
	attach boxed title to top left={yshift*=-\tcboxedtitleheight},
	fonttitle=\bfseries,
	title={#2},
	boxed title size=title,
	boxed title style={%
			sharp corners,
			rounded corners=northwest,
			colback=tcbcolframe,
			boxrule=0pt,
		},
	underlay boxed title={%
			\path[fill=tcbcolframe] (title.south west)--(title.south east)
			to[out=0, in=180] ([xshift=5mm]title.east)--
			(title.center-|frame.east)
			[rounded corners=\kvtcb@arc] |-
			(frame.north) -| cycle;
		},
	#1
}{th}
\makeatother

% corps
\newcommand{\R}{\mathbb{R}}
\newcommand{\Rnn}{\mathbb{R}^{2n}}
\newcommand{\Z}{\mathbb{Z}}
\newcommand{\N}{\mathbb{N}}
\newcommand{\Q}{\mathbb{Q}}

% domain
\newcommand{\D}{\mathcal{D}}
% for calligraphic C
\usepackage{calrsfs}
\newcommand{\C}{\mathcal{C}}

% date
\usepackage{advdate}

% ensembles tq. 
\newcommand{\xRtq}[1]{
	$\left\{ x \in \R \text{ tq. } #1 \right\}$
}

% vabs
\newcommand{\vabs}[1]{
	\left| #1 \right|
}

%pinfty minfty
\newcommand{\pinfty}{{+}\infty}
\newcommand{\minfty}{{-}\infty}

% plots
\usepackage{pgfplots}

%virgules
\usepackage{icomma}
\pgfplotsset{/pgf/number format/use comma}

%subfigures
\usepackage{subcaption}

%hyperlink footnote
\usepackage{hyperref}

%wider tabulars
\def\arraystretch{2}
\setlength\tabcolsep{15pt}

% tableaux var, signe
\usepackage{tkz-tab}

\SetDate[13/02/2026]
\reversemarginpar

\setlength{\marginparsep}{.5cm}

\renewcommand{\ExerciseHeader}{
	\textbf{Exercice \theExercise.}
	\ifnum\ExerciseDifficulty=0
	\else
		(\theExerciseDifficulty)
	\fi
}

\begin{document}
\pagestyle{fancy}
\fancyhead[L]{Première}
\fancyhead[C]{\textbf{Évaluation — Probabilités conditionnelles}}
\fancyhead[R]{\today}

\null\vspace{-30pt}
Consignes particulières : 
\begin{itemize}[label=$\bullet$]
	\item 
	La calculatrice est {interdite}.
	\item
	Les probabilités peuvent être exprimées sous forme de fractions qu'il n'est pas nécessaire de simplifier.
	\item
	L'évaluation fait 2 pages.
\end{itemize}

\marginpar{[pts]}
\hrule



\exe{10}{
	Chaque individu fait partie d'un unique groupe sanguin (O, A, B ou AB), de Rhésus soit positif, soit négatif.
	Connaître son groupe sanguin ainsi que son Rhésus permet de déterminer quelles transfusions sanguines sont possibles.
	Par exemple, le groupe AB+ est receveur universel, et le groupe O- donneur universel.
	
	On sonde 500 personnes au sujet de leur groupe sanguin ainsi que de leur Rhésus.
	Les résultats du sondage sont décrits dans le tableau d'effectifs suivant.

	\begin{center}
	\begin{tabular}{|c|c|c|c|c|} \hline
	Groupe sanguin & O & A & B & AB \\ \hline
	Rhésus positif & 180 & 190 & 40 & 15 \\ \hline
	Rhésus négatif & 30 & 35 & 5 & 5 \\ \hline
	\end{tabular}
	\end{center}

	On choisit au hasard, de manière équiprobable, une personne sondée.
	On considère les événements $A$ et $P$ suivants.
	
		A : « La personne choisie appartient au groupe sanguin A. » 
		
		P : « La personne choisie a un Rhésus positif. ».

	\begin{enumerate}
		\item
		Décrire l'événement $A\cap P$ avec des mots.
		\item
		Déterminer la probabilité de l'événement $A\cap P$.
		\item
		Déterminer la probabilité que la personne choisie appartienne au groupe sanguin A sachant que son Rhésus est positif.
		\item
		Rappeler la définition d'indépendance de deux événements. Les événements $A$ et $P$ sont-ils indépendants ? 
		\item
		Un personne sait qu'elle appartient au groupe sanguin AB mais ne connaît pas son Rhésus.
		En lisant le tableau, celle-ci affirme : « il est trois fois plus probable que j'aie un Rhésus positif qu'un Rhésus négatif ! »
		
		A-t-elle raison ? Justifier.
	\end{enumerate}

}{exe:tableau}{
}

\newpage

\exe{10}{
	On lance un dé à $6$ faces équilibré puis on regarde le numéro de la face du dessus (un entier de 1 à 6).
	\begin{enumerate}[label=$\bullet$]
		\item Si le numéro obtenu est 1 ou 2, on extrait au hasard une boule dans l'urne 1 qui contient 3 boules noires, 4 boules blanches et 3 boules rouges
		\item Sinon, on extrait une boule dans l'urne 2 qui contient 3 boules noires et 2 boules blanches.
	\end{enumerate}
	
	\begin{enumerate}
		\item
		Déduire de l'énoncé les probabilités suivantes.
		\begin{enumerate}
			\item
			La probabilité de tirer dans l'urne 1.
			\item
			La probabilité de tirer une boule blanche sachant qu'on tire dans l'urne 1.
			\item
			La probabilité de tirer une boule rouge sachant qu'on tire dans l'urne 1.
			\item
			La probabilité de tirer une boule noire sachant qu'on tire dans l'urne 2.
		\end{enumerate}
		\item
		Représenter la situation par un arbre de probabilité.
		\item
		\begin{enumerate}
			\item
			Déterminer la probabilité qu'on tire dans l'urne 1 et qu'on y tire une boule noire.
			\item
			Vérifier que la probabilité de tirer une boule noire est égale à 50\%.
		\end{enumerate}
	\end{enumerate}
	Une première personne lance le dé et une deuxième personne tire dans l'urne.
	Cette deuxième personne tire une boule noire sans connaître l'urne dans laquelle elle tire.
	Elle affirme : \\ « Il est très probable que j'aie tiré dans l'urne 2 ! »
	\begin{enumerate}[resume]
		\item
		Calculer la probabilité d'avoir tiré dans l'urne 1 sachant qu'on a tiré une boule noire à l'aide de la formule
			\[ P(A \sct B) = \dfrac{P(A \cap B)}{P(B)} \]
		et décider si l'affirmation est vraie.
	\end{enumerate}
}{exe:arbre}{
	On ajoute les probabilités de chacun des événements sur les branches associées.
	Par exemple, de la racine, pour atteindre l'embranchement \og multiple de $3$ \fg, il y a $1$ chance sur $3$ (car seuls $3$ et $6$ sont les issues favorables).
	On ajoute donc un $\frac13$ sur la branche qui mène à cet événement, est un $\frac23$ sur l'autre.
	
	En sachant qu'on ait obtenu un nombre qui n'est pas un multiple de $3$, on tire une boule dans la deuxième urne  qui contient 3 boules noires et 2 boules blanches.
	La probabilité d'obtenir une boule blanche est donc $\frac25$, et celle d'obtenir une boule noire est $\frac35$.
	En notant toutes les probabilités ainsi calculées, on obtient l'arbre de probabilité suivant.
	
	\begin{center}
	\begin{tikzpicture}
		% depth 1
		\foreach \i in {-3, 3}
		\draw[-, thick, black] (0,0) node {$\bullet$} -- (\i,-2);
		% depth 2
		\foreach \i in {3} \foreach \j in {-2, 0, 2}
			\draw[-, thick, black] (\i,-2) node {$\bullet$} -- (\i+\j,-4) node {$\bullet$};
			
		\foreach \i in {-3} \foreach \j in {-1, 1}
			\draw[-, thick, black] (\i,-2) node {$\bullet$} -- (\i+\j,-4) node {$\bullet$};
			
		\draw (0,0) node[above] {Départ};
		\draw (3,-2) node[right] {1 ou 2};
		\draw (-3,-2) node[left] {3 ou 4 ou 5 ou 6};
		\draw (-4,-4) node[below] {B};
		\draw (-2,-4) node[below] {N};
		\draw (1,-4) node[below] {B};
		\draw (3,-4) node[below] {N};
		\draw (5,-4) node[below] {R};
		
		\draw (-1.5, -1) node[above left] {$\dfrac23$};
		\draw (1.5, -1) node[above right] {$\dfrac13$};
		
		\draw (-3.5, -3.25) node[above left] {$\frac25$};
		\draw (-2.5, -3.25) node[above right] {$\frac35$};
		
		\draw (2, -3.25) node[above left] {$\frac25$};
		\draw (3, -3.25) node[right] {$\frac3{10}$};
		\draw (4, -3.25) node[above right] {$\frac3{10}$};
	\end{tikzpicture}
	\end{center}
	La probabilité d'un chemin racine-feuille est le produit des probabilités rencontrées en le parcourant.
	En outre, deux tels chemins correspondent à des événements disjoints.
	On calcule par exemple
		\[ P(N) = \dfrac23 \cdot \dfrac35 + \dfrac13 \cdot \dfrac3{10} = \dfrac12. \]
}

%%%%%%%%%%%%

\newpage
\fancyhead[C]{\textbf{Solutions}}
\shipoutAnswer

\end{document}
