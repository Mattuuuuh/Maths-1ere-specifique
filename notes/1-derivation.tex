%!TEX encoding = UTF8
%!TEX root = 0-notes.tex

\chapter{Dérivation}

\section{Introduction : calcul de vitesse}

Supposons qu'une voiture roule sur une route droite, qu'on modélise par un point évoluant sur une axe.
En tout temps $t$, on peut connaitre sa position, qu'on appelle $f(t)$.
La voiture n'avance pas à vitesse constante ;
elle peut même s'arrêter pendant un certain temps, ou reculer.

\qs{}{
	Comment calculer la vitesse $V(t)$ de la voiture en tout temps $t$ ?
}{}

Il semble qu'on ait assez d'information pour calculer la vitesse de la voiture, mais que la définition même de la vitesse n'est pas claire.
En physique, on calcule une vitesse moyenne grâce à deux positions en deux temps différents.

	\begin{align}
		v = \dfrac{f(t_2) - f(t_1)}{t_2-t_1}. \label{eq:3.1}
	\end{align}

C'est faisable ici bien sûr, mais cela ne répond pas à la question : la vitesse $V(t)$ qu'on souhaite calculer est la vitesse en un temps $t$ donné, et un seul.
C'est la vitesse \emph{instantanée} de la voiture --- celle qu'on lit sur le tachymètre d'une voiture.


Avant de commencer l'étude, dessinons d'abord quelques situations qui peuvent subvenir.
D'abord, la notation $t$ et $f(t)$ n'est pas anodine : la position de la voiture est une fonction de temps (car à chaque temps est associé une unique position), et on peut la représenter dans un graphique comme ci-dessous.
\begin{center}
\includegraphics[page=1]{figures/fig-derivation.pdf}
\end{center}

De l'expression de la vitesse de l'équation \eqref{eq:3.1}, et en prenant $t_1 < t_2$, on remarque trois cas de figure.
\begin{enumerate}
	\item Lorsque la position $f(t)$ augmente, la vitesse $V(t)$ est positive. 
	C'est le cas où la voiture avance.
	\item Lorsque la position $f(t)$ est constante, la vitesse $V(t)$ est nulle.
	C'est le cas où la voiture est immobile. 
	\item Lorsque la position $f(t)$ diminue, la vitesse $V(t)$ est négative.
	C'est le cas où la voiture recule.
\end{enumerate}
Le signe de la vitesse de la voiture est donc liée aux variations de sa position $f(t)$.
Un tableau permet de regrouper les informations obtenues jusqu'ici.
	
\includegraphics[page=2]{figures/fig-derivation.pdf}

Le but de ce chapitre est de traiter de la partie \og Variation instantanée, variation globale \fg~du bulletin officiel.

Les capacités attendues sont les suivantes.
\begin{itemize}
	\item Interpréter le nombre dérivé dans le cadre d'un modèle d'évolution.
	\item Interpréter géométriquement le nombre dérivé comme coefficient directeur de la tangente.
	\item Décrire les variations d'un phénomène en mobilisant la dérivée d'une fonction.
	\item Déterminer le sens de variation d'une fonction polynomiale de degré inférieur ou égal à troie (la forme factorisée de la dérivée pourra être donnée).
	\item Prévoir l'évolution d'un phénomène grâce à l'étude de la dérivée d'une fonction.
\end{itemize}

Comme la vitesse est, par définition, une distance sur un temps, une façon de calculer la vitesse de la voiture en $t_1=4$ est de considérer la position de la voiture légèrement après $4$.
Le choix du pas de temps, qu'on notera $h$, semble cependant important. Posons $t_2 = 4+h$.
Plus le pas est petit, plus la valeur calculée semble intuitivement s'approcher de la vitesse instantanée au temps $t=4$.

La formule de la vitesse de l'équation \eqref{eq:3.1} donne les ratios suivants à calculer avec l'expression $f(t) = \sin(90t)$.
\begin{align*}
	v_{h=1} &= \dfrac{f(5) - f(4)}{5-4} = 1 & v_{h=0,9} &=  \dfrac{f(4,9) - f(4)}{4,9-4} = 1,097 \\
	v_{h=0,8} &=  \dfrac{f(4,8) - f(4)}{4,8-4} = 1,188 & v_{h=0,7} &=  \dfrac{f(4,7) - f(4)}{4,7-4} = 1,272 \\
	v_{h=0,6} &=  \dfrac{f(4,6) - f(4)}{4,6-4} = 1,348 & v_{h=0,5} &=  \dfrac{f(4,5) - f(4)}{4,5-4} = 1,414 \\
	&\dots & &\dots
\end{align*}
On pourra comparer à l'aide de la calculatrice la valeur de $v_{h}$ lorsque $h$ est de plus en plus petit (p.ex. $0,1$ puis $0,01$) à $\frac\pi2$ pour se douter qu'une certaine converge occurre...

\qs{}{
	La vitesse moyenne de la voiture entre $t=4$ et $4+h$ approche-t-elle une certaine valeur qu'on peut déterminer lorsque $h$ tend vers $0$ ?
}{}

Pour répondre à cette question, il s'agit d'interpréter les vitesses calculées graphiquement et de voir ce qu'on obtient lorsque le pas devient nul.

Considérons les points $A$ et $B$ d'abscisses respectivement $t_1$ et $t_2$ sur $\C_f$.
La propriété fondamentale nous dit que l'ordonnée des points est l'image de leur abscisse.
C'est-à-dire que $y_A = f(x_A) = f(t_1)$, et $y_B = f(t_2)$.

L'équation \eqref{eq:3.1} est donc équivalente à
	\[ \dfrac{y_B - y_A}{x_B - x_A}, \]
qui est exactement la formule du coefficient directeur de la fonction affine passant par $A$ et $B$.
Rappelons que le coefficient directeur mesure la pente d'une fonction affine, et que le signe de celui-ci donne les variations de celle-ci.
On répète en fait la même chose en langage affine : si le coefficient directeur est positif, c'est que la vitesse moyenne est positive, et donc que la fonction augmente.

	\begin{center}
	\begin{multicols}{2}
	\includegraphics[page=3]{figures/fig-derivation.pdf}
	\includegraphics[page=4]{figures/fig-derivation.pdf}
	
	\hfill
	\includegraphics[page=5]{figures/fig-derivation.pdf}
	\includegraphics[page=6]{figures/fig-derivation.pdf}
	 \end{multicols}
	 \end{center}

\dfn{dérivation ponctuelle}{
	Soit $f$ une fonction position suffisamment lisse.
	On calcule, au temps $t$, la vitesse instantanée en considérant
		\[ v_h = \dfrac{f(t+h)-f(t)}{h}, \]
	et en laissant $h$ tendre vers $0$.

	On appelle la valeur limite de $v_h$ la \emph{dérivée ponctuelle} de $f$ au temps $t$.
}{}

\section{Dérivation}

\dfn{fonction dérivée}{
	Soit $f$ une fonction sur $\R$ suffisamment lisse.
	On définit $f'$, la \emph{fonction dérivée} de $f$, telle que $f'(t)$ soit la dérivée ponctuelle de $f$ en $t$.
}{}

\thm{linéarité de la dérivée}{
	Soient $f, g$ deux fonctions dérivables sur $\R$, et $c\in\R$ un nombre réel quelconque. 
	Alors
		\begin{multicols}{2}
		\begin{enumerate}
			\item $(f+g)' = f' + g'$ ; et
			\item $(c\times f)' = c\times f'$.
		\end{enumerate}
		\end{multicols}
}{thm:lin-der}

\mprop{dérivée des monômes usuels}{
	Pour chaque monôme suivant, on donne sa dérivée formelle.
	\begin{center}
	\begin{tabular}{|c|c|c|c|c|}\hline
		Monôme & 1 & $x$ & $x^2$ & $x^3$ \\ \hline
		Dérivée & 0 & 1 & $2x$ & $3x^2$ \\ \hline
	\end{tabular}
	\end{center}
}{prop:der-mono}

\pf{}{
	Prenons $f(x) = x^2$.
	Alors $f'(x)$ est la limite de $\dfrac{f(x+h)-f(x)}h$ lorsque $h$ tend vers 0.
	Calculons donc
		\[ \dfrac{f(x+h)-f(x)}h = \dfrac{(x+h)^2 - x^2}h = \dfrac{x^2 + h^2 + 2xh - x^2}h = 2x + h, \]
	qui approche bien $2x$ lorsque $h$ est très proche de $0$ (on peut ici prendre $h=0$).
	
	Les autres monômes se dérivent de la même façon.
}

\ex{}{
	La dérivée du polynôme de degré 3 $f(x) = 2x^3 - 6x^2 - 48x + 30$ est donnée par
		\begin{align*}
		 f'(x) &= \bigl(f(x)\bigr)' =  \bigl( 2x^3 - 6x^2 - 48x + 30\bigr)' , \\
		 	&= \bigl( 2x^3\bigr)' + \bigl(- 6x^2\bigr)' + \bigl( - 48x \bigr)' + \bigl( 30\bigr)', \\
		 	&= 2\bigl( x^3\bigr)' -6 \bigl(x^2\bigr)' -48 \bigl(x \bigr)' + 30\bigl( 1\bigr)', \\
		 	&= 2\bigl( 3x^2\bigr) -6 \bigl(2x\bigr) -48 \bigl(1 \bigr)' + 30\bigl( 0\bigr), \\
		 	&= 6x^2-12x -48.
		\end{align*}
}{ex:cube1}


\exe{}{
	Montrer qu'une fonction constante $f(x) = c$ est de dérivée identiquement nulle : $f'(x) = 0$.
}{}

\nt{
	D'après l'exercice précédent, on se permettra d'ignorer instantanément les monômes constants lors de la dérivation.
}

\thm{signes et variations}{
	Soit $f$ une fonction dérivable sur $\R$ et $I$ un intervalle quelconque.
	Alors
		\begin{center}
			$f$ est croissante sur $I$ \qquad $\iff$ \qquad $f'$ est positive sur $I$. 
		\end{center}
}{}


\ex{}{
	Considérons la fonction affine $f(x) = 3x - 61$.
	Sa dérivée est donnée par
		\[ f'(x) = \bigl(f(x)\bigr)' = \bigl( 3x - 61 \bigr)' = \bigl( 3x \bigr)' + \bigl( - 61 \bigr)' = 3 \bigl( x \bigr)' + 0 = 3. \]
	$f'$ vaut donc constamment 3, qui est positif, et donc $f$ est une fonction croissante.
}{}

\ex{}{
	En reprenant l'étude de $f(x) = 2x^3 - 6x^2 - 48x + 30$ entamée à l'exercice \ref{ex:cube1}, on démontre que $f'(x) = 6(x+2)(x-4)$ en développant la forme factorisée.
		\[ 6(x+2)(x-4) = 6\bigl[ (x+2)(x-4) \bigr] = 6\bigl[ x^2 -2x - 8 \bigr] = 6x^2 - 12x - 48 = f'(x). \]
	Un tableau de signes permet l'étude d'un produit de fonctions affines.
	Ici les affines $x+2$ et $x-4$ sont toutes les deux croissantes et donc sont négatives, nulles, puis positives.
	On ajoute une ligne pour les variations de $f$ données par le signe de $f'$.
	
	\begin{center}
	\includegraphics[page=7]{figures/fig-derivation.pdf}
	 \end{center}
}{ex:cube2}

\exe{}{
	Soit $f(x) = ax+b$ une fonction affine où $a$ et $b$ sont des réels fixés.
	Montrer que $f'(x) = a$ et en déduire les variations de $f$ selon le signe de son coefficient directeur.	
}{exe:affineprime}{
	TODO
}


\exe{, difficulty=1}{
	Spinoza, déterministe notoire, jette verticalement un objet de masse $m$ au temps $t=0$ et souhaite connaître sa hauteur $h(t)$ en tout temps $t>0$ à l'aide uniquement de ses conditions initiales : la hauteur initiale $h(0)$, la vitesse initiale $h'(0)$.
	
	La deuxième loi de Newton lui permet d'écrire, en négligeant toute force qui n'est pas la gravité, que l'accélération $h''$ est donnée par $h''(t) = -\dfrac{g}m$, où $m$ est la masse de l'objet et $g$ l'accélération de la pesanteur.
	
	\begin{enumerate}
		\item 
		Montrer, en dérivant deux fois $h$, qu'elle est de la forme
			\[ h(t) = -\dfrac{g}{2m} t^2 + b t + c, \]
		où $b$ et $c$ sont des constantes qu'on ne connaît pas encore.
	
		\item Montrer que $c$ est la hauteur de l'objet au temps initial $t=0$.
		\item Montrer que $b$ est la vitesse de l'objet au temps initial $t=0$.
		\item Écrire la hauteur de l'objet en fonction du temps avec les données $g=9,8 ; m=1 ; h(0) = 3 ; h'(0) = 2$. Quand est-ce que l'objet touche le sol, approximativement ? On cherche le $t$ positif tel que $h(t) = 0$.
	\end{enumerate}
}{exe:spinoza}{
	TODO
}


\exe{, difficulty=1}{
	Soit $f$ une fonction réelle et $c\in\R$ une constante quelconque.
	
	\begin{enumerate}
		\item Montrer que $f(x)$ et $g(x) = f(x)+c$ admettent les mêmes variations. Comment dessiner $\C_g$ en connaissant $\C_f$ ? Comparer par exemple les courbe représentatives de $x^2$ et $x^2 + 2$.
		\item Montrer que $f(x)$ et $c\cdot f(x)$ admettent
			\begin{enumerate}[label=\roman*)]
				\item les mêmes variations si $c > 0$ ; et
				\item des variations opposées si $c < 0$.
			\end{enumerate}
		 Comment dessiner $\C_g$ en connaissant $\C_f$ ? Comparer par exemple les courbe représentatives de $x^2, -x^2,$ et $3x^2$.
	\end{enumerate}
}{exe:fonctions-parentes}{
	TODO
}

\exe{}{
	Donner une fonction $f$ telle que sa dérivée $f'$ soit
		\[ f'(x) =  x^2 - 4. \]
	En quels $x$ la fonction $f$ change-t-elle de variations ? Ce sont les antécédents des extrema locaux.
}{exe:intégration}{
	TODO
}

\exe{, difficulty=1}{
	Donner une fonction $f$ telle que
		\[ f'(x) = (2-x)(x-1). \]
	En quels $x$ la fonction $f$ change-t-elle de variations ? Ce sont les antécédents des extrema locaux.
}{exe:intégration2}{
	TODO
}

\exe{, difficulty=2}{
	Soient $a_1 < a_2 < \dots < a_n$ des nombres réels ordonnés.
	On pose $f$ comme le produit des facteurs affines $x-a_i$ :
		\[ f(x) = (x-a_1) \cdot (x-a_2) \cdots (x-a_n). \]
	\begin{enumerate}
		\item Montrer que $f$ s'annule exactement en $x=a_1, a_2, \dots, a_n$.
		\item Montrer que $f(x)$ est positif pour $x<a_1$ si $n$ est pair, et négatif sinon.
		\item Montrer que le signe de $f(x)$ change $n$ fois en $x=a_1, a_2, \dots, a_n$.
	\end{enumerate}
}{exe:signe-décomposition}{
	TODO
}

\exe{, difficulty=1}{
	Montrer que, pour $x, h \in \R$,
		\[ (x+h)^4 = x^4 + 4x^3 h +6x^2h^2 + 4xh^3 + h^4. \]
	En déduire que
		\[ \dfrac{(x+h)^4 - x^4}h \]
	tend vers $4x^3$ lorsque $h$ devient nul.
	Conclure que $(x^4)' = 4x^3$.
}{exe:x4prime}{
	TODO
}

\exe{, difficulty=1}{
	Montrer que, pour $x, h \in \R$ tels que $x\neq0$ et $h$ suffisamment petit pour que $x+h\neq0$,
		\[ \dfrac1{x+h} - \dfrac1x = \dfrac{-h}{x(x+h)}. \]
	Conclure que
		\[ \left(\dfrac1x\right)' = \dfrac{-1}{x^2}. \]
}{exe:invprime}{
	TODO
}

\exe{, difficulty=1}{
	À l'aide du résultat de l'exercice \ref{exe:invprime} qui montre que
		\[ \left(\dfrac1x\right)' = \dfrac{-1}{x^2}, \]
	montrer que la fonction inverse $f(x) = \dfrac1x$ est strictement décroissante sur son domaine de définition $\R^*$, la droite réelle perforée en $0$.
}{exe:invprime2}{
	TODO
}

\exe{, difficulty=2}{
	Soit $n\in\N_{\geq2}$ un entier naturel supérieur ou égal à $2$. On souhaite montrer que
		\[ (x+h)^n = x^n + nhx^{n-1} + h^2 g(x, h), \]
	où $g$ est une fonction polynomiale en $x$ et $h$.
	
	\begin{enumerate}
		\item Montrer, en développant $(x+h)^2$, que cette égalité est vraie pour $n=2$.
		\item En utilisant que 
			\[ (x+h)^{n+1} = (x+h) \cdot (x+h)^{n} = x(x+h)^n + h(x+h)^n, \]
		montrer que si cette égalité est vraie pour un entier $n$, alors elle est aussi vraie pour $n+1$.
		 \item En déduire que l'égalité est vraie pour tout les entiers naturels $n\in\N_{\geq2}$.
	\end{enumerate}
}{exe:xnprime}{
	TODO
}

\exe{, difficulty=2}{
	Soit $n\in\N_{\geq2}$ un entier naturel supérieur ou égal à $2$. L'exercice \ref{exe:xnprime} donne
		\[ (x+h)^n = x^n + nhx^{n-1} + h^2 g(x, h), \]
	où $g$ est une fonction polynomiale en $x$ et $h$.
	
	En déduire que $(x^n)' = nx^{n-1}$.
}{exe:xnprime2}{
	TODO
}
