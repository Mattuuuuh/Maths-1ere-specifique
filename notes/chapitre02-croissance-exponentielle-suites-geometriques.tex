%!TEX encoding = UTF8
%!TEX root =notes.tex

\chapter{Croissance exponentielle}

\section{Suites géométriques}

\subsection{Définition et exemples}

\dfn{Suite}{
	Une suite $u$ est une fonction qui à tout entier naturel $n \in \N$ associe un réel
		\[ u(n) \in \R. \]
	Le \emph{terme initial} de la suite est donné par $u(0)$, et son terme de \emph{rang} $n$ est $u(n)$.
}{}

\ex{}{
	Les fonctions suivantes sont des suites données par leur rang $n$ : on connait donc leur valeur pour tout les entiers naturels.
	\begin{multicols}{2}
	\begin{enumerate}
		\item $u(n) = n+1$
		\item $v(n) = 15 + n$
		\item $\xi(n) = 3n + 1$
		\item $a(n) = n^2$
	\end{enumerate}
	\end{multicols}
	Une suite n'a pas forcément de formule générale pour tout $n$, c'est une simplement \emph{suite} de nombres réels.
}{}

\dfn{Suite géométrique}{
	On dit d'une suite $u$ qu'elle est arithmétique dès qu'elle vérifie, pour tout $n\in\N$,
		\begin{align}\label{eq:suite-geom}
			u(n+1) = q \cdot u(n),
		\end{align}
	où $q \in \R$ est un réel fixé qui ne dépend pas de $n$.
	
	On appelle $q$ la \emph{raison} de la suite.
}{}

\nt{
	Comme $n$ et $n+1$ représentent un entier quelconque et son suivant, la relation \eqref{eq:suite-geom} se lit de la façon suivante :
		\begin{center}
		\og pour passer d'un terme au suivant, on multiplie par $q$ \fg.
		\end{center}
}{}

\thm{Propriétés des puissances}{
	Soient $a, b, c \in \Z$. On a les relations suivantes.
		\begin{gather*}
			a^{b+c} = a^{b} \times a^{c} \\
			\left(a^b\right)^c = a^{b \times c} \\
			a^{c} \times b^c = \left( a \times b \right)^c
		\end{gather*}
	En particulier, si $a \neq 0$, on a
		\begin{align*}
			a^0 = 1, &
			&a^1 = a, &
			&a^{-1} = \dfrac1a. 
		\end{align*}
}{}

\ex{}{
	La suite $u$ définie par
		\[ u(n) = 5 \times 3^n \]
	est géométrique. 
	En effet, pour tout $n\in\N$, on calcule
		\begin{align*}
			u(n+1) &= 5 \times 3^{n+1} \\
					&= 5 \times 3^n \times 3 \\
					&= 3 \times \left( 5 \times 3^n \right) \\
					&= 3 u(n).
		\end{align*}
	Comme $u(0) = 5 \times 3^0 = 5 \times 1 = 5$, on en déduit que $u$ est géométrique de raison $3$ et de terme initial $5$.
}{}

\exe{}{
	Pour chacune des suites données algébriquement pour tout $n\in\N$, donner leur raison et leur terme initial.
	\begin{multicols}{2}
	\begin{enumerate}
		\item $u(n) = 2 \times 3^n$
		\item $v(n) = 7 \times \left(\dfrac12 \right)^n$
		\item $\xi(n) = - 6^n$
		\item $a(n) = 11 \times 5^{2n}$
		\item $a(n) = 3 \times 5^{2n+3}$
		\item $a(n) = 10^{-n}$
	\end{enumerate}
	\end{multicols}
}{}

\subsection{Conséquences : expression algébrique et variations}

\ex{}{
	Considérons une suite $G$, géométrique de raison $4$ et de terme initial $3$.
	Alors, par définition,
		\[ G(0) = 3. \]
	En spécialisant l'équation \eqref{eq:suite-geom} pour $n=0$, on trouve ensuite
		\[ G(1) = G(0+1) = 4 \times G(0) = 4 \times 3 = 12. \]
	On peut déduire la suite des termes de façon analogue.
		\begin{align*}
			G(0) &= 3 \\
			G(1) &= 4 \times G(0) = 12 \\
			G(2) &= 4 \times G(1) = 48 \\
			G(3) &= 4 \times G(2) = 196 \\
			\vdots &\, \qquad \vdots
		\end{align*}
	On souhaite désormais connaître une formule valable pour tous les entiers naturels $n\in\N$ pour qu'on ait pas à reconstruire la suite depuis le début à chaque étude.
	Pour cela, on choisit de réécrire les termes de la suite de la façon suivante, en utilisant les propriétés de la puissance.
		\begin{align*}
			G(0) &= 3 = 3 \times 4^{0} \\
			G(1) &= 4 \times G(0) = 3 \times 4^{1} \\
			G(2) &= 4 \times G(1) = 3 \times 4^{2} \\
			G(3) &= 4 \times G(2) = 3 \times 4^{3} \\
			\vdots &\, \qquad \vdots \\
			G(n) &= 3 \times 4^n
		\end{align*}
}{}

\thm{de Maya}{
	Soit $v$ une suite géométrique de raison $q \in \R$ et de terme initial $v(0) \in \R$.
	Alors
		\[ v(n) = v(0) \times q^n. \]
}{thm:maya}

\mprop{Variations}{
	Soit $u$ une suite géométrique de raison $q > 0$ et de terme initial $u(0) > 0$.
	On distingue alors les trois cas suivants.
		\begin{enumerate}[label=---]
			\item Si $0 < q < 1$, alors $u$ est décroissante et tend vers $0$ exponentiellement.
			\item Si $q=1$, alors $u$ est constante : $u(n) = u(0)$ pour tout $n\in\N$.
			\item Si $q>1$, alors $u$ est croissante et diverge exponentiellement vers $+ \infty$.
		\end{enumerate}
}{}

\subsection{Problèmes de seuil}

\str{
	On décrit ici une stratégie pour résoudre un problème de seuil courant qui est facile à résoudre pour les suites arithmétiques mais très dur à résoudre pour les suites géométriques.
	Étant donné une suite géométrique $v$ croissante et un seuil $M > 0$, on souhaite trouver le plus petit entier naturel $N\in\N$ vérifiant
		\[ v(N) > M. \]
	Considérons l'exemple suivant.
		\begin{align*}
			v(n) = 3 \times 5^n && \text{et} && M = 100~000.
		\end{align*}
	La stratégie est de considérer un intervalle dans lequel le $N$ recherché doit nécessairement appartenir, et de scinder cet intervalle à chaque étape.
	Pour commencer, on prend
		\[ I = [0 ; 10], \]
	car $v(0) = 3 < 100~000 < v(10) \approx 2,9 \times 10^{7}$.
	La borne supérieure $10$ a été choisie au hasard et suffisamment grande telle que $v(10)$ dépasse le seuil.
	
	On divise l'intervalle en deux parts égales en considérant son milieu, 
		\[ m = \dfrac{0+10}2 = 5. \]
	On évalue $v$ en $5$ pour décider si $N$ est plus grand ou plus petit que $5$.
	En l'occurrence, 
		\[ v(5) = 9375 < 100~000. \]
	Le seuil recherché est donc nécessairement supérieur à $5$, et on peut définir
		\[ I = [5 ; 10], \]
	comme nouvel intervalle dans lequel $N$ appartient.
	
	On répète l'opération en calculant le milieu $m = \dfrac{5+10}2 = 7,5$, et 
		\[ v(7,5) \approx 5,2 \times 10^5 > 100~000. \]
	D'où $I = [5 ; 7,5]$.
	Ici, on peut soit tester les $3$ valeurs qui restent ($5 ; 6 ; 7$) ou continuer pour trouver
		\[ I = [ 6,25 ; 7,5 ], \]
	dans lequel $N = 7$ est le seul entier possible.
	
	On vérifiera bien sûr que $v(6) < 100~000 < v(7)$, comme voulu.
}{}

\IncMargin{1em}
\begin{algorithm}
\SetKwInput{KwRes}{retourner}%
\SetKwIF{Si}{SinonSi}{Sinon}{si}{alors}{sinon si}{sinon}{fin si}%
\SetKwFor{Tq}{tant que}{faire}{fin tq}%
\SetKwInOut{Input}{entrée}\SetKwInOut{Output}{sortie}
	\Input{Suite géométrique $G$ de raison $q > 1$ tel que $G(0) > 0$. Seuil $M$. Intervalle $I=[a ; b]$ tel que $G(a) < M < G(b)$.}
	\Output{Le plus petit entier naturel $N\in\N$ tel que $G(N) \geq M$.}
	\BlankLine
	\emph{L'intervalle $I=[a;b]$ de départ doit vérifier $G(a) < M < G(b)$, de telle sorte que l'entier $N$ recherché lui appartienne nécessairement car $G$ est croissante. On pourrait prendre $a=0$ et $b$ extrêmement grand en cas de doute.}\\
	\Tq{ la longueur de l'intervalle $I=[a;b]$ est supérieure  ou égale à $1$}{
		$m = \dfrac{a+b}2$ \emph{, le milieu de l'intervalle $I$}\\
		\Si{ $G(m) \geq M$}{
			$I = [a ; m]$
		}
		\Sinon{
			$I = [m ; b]$
		}
	}
	\KwRes{L'unique entier appartenant à l'intervalle $I$.}
\caption{Problème de seuil.}\label{algo:seuil-geom}
\end{algorithm}\DecMargin{1em} 

\exe{}{
	Trouver le plus petit entier naturel $n\in\N$ vérifiant les inéquations suivantes.
	\begin{multicols}{2}
	\begin{enumerate}
		\item $2 \times 3^n > 100~000$
		\item $7 \times \left(\dfrac32 \right)^n > 50~000$
		\item $3 \times \left( \dfrac43 \right)^n > 1~000$
		\item $3 \times \left( \dfrac43 \right)^n > 10~000$
	\end{enumerate}
	\end{multicols}
}{}




\nt{
	C'est au début du 17è siècle que John Neper crée les premières tables de valeurs de la fonction $\ln$ qui porte son nom : le logarithme népérien.
	La notation $\log$ désigne une fonction qui transforme la multiplication en une addition, cette dernière opération étant beaucoup plus facile lorsqu'on manipule des grands nombres.
	Une bonne table de valeurs permet alors, pour multiplier deux grands nombres $A$ et $B$, d'additionner leur logarithmes grâce à la relation
		\[ \log(A \times B) = \log(A) + \log(B). \]
	Pour retrouver le produit, on utilise alors la table de valeurs dans le sens inverse.
	
	Plus généralement, l'application du $\log$ convertit une équation exponentielle en une équation linéaire simple.
	L'étude du logarithme est au programme du cours de mathématiques complémentaires, option de Terminale.
}{}

\subsection{Lecture graphique : échelle logarithmique}

