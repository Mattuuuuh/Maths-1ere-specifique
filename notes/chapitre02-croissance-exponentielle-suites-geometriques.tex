%!TEX encoding = UTF8
%!TEX root =notes.tex

\chapter{Croissance exponentielle}

\section{Suites géométriques}


\dfn{Suite}{
	Une suite $u$ est une fonction qui à tout entier naturel $n \in \N$ associe un réel
		\[ u(n) \in \R. \]
	Le \emph{terme initial} de la suite est donné par $u(0)$, et son terme de \emph{rang} $n$ est $u(n)$.
}{}

\ex{}{
	Les fonctions suivantes sont des suites données par leur rang $n$ : on connait donc leur valeur pour tout les entiers naturels.
	\begin{multicols}{2}
	\begin{enumerate}
		\item $u(n) = n+1$
		\item $v(n) = 15 + n$
		\item $\xi(n) = 3n + 1$
		\item $a(n) = n^2$
	\end{enumerate}
	\end{multicols}
	Une suite n'a pas forcément de formule générale pour tout $n$, c'est une simplement \emph{suite} de nombres réels.
}{}

\dfn{Suite géométrique}{
	On dit d'une suite $u$ qu'elle est arithmétique dès qu'elle vérifie, pour tout $n\in\N$,
		\begin{align}\label{eq:suite-geom}
			u(n+1) = q \cdot u(n),
		\end{align}
	où $q \in \R$ est un réel fixé qui ne dépend pas de $n$.
	
	On appelle $q$ la \emph{raison} de la suite.
}{}

\nt{
	Comme $n$ et $n+1$ représentent un entier quelconque et son suivant, la relation \eqref{eq:suite-geom} se lit de la façon suivante :
		\begin{center}
		\og pour passer d'un terme au suivant, on multiplie par $q$ \fg.
		\end{center}
}{}

\thm{Propriétés des puissances}{
	Soient $a, b, c \in \Z$. On a les relations suivantes.
		\begin{gather*}
			a^{b+c} = a^{b} \times a^{c} \\
			\left(a^b\right)^c = a^{b \times c} \\
			a^{c} \times b^c = \left( a \times b \right)^c
		\end{gather*}
	En outre, si $a \neq 0$, on a
		\begin{gather*}
			a^0 = 1 \\
			a^1 = a \\
			a^{-1} = \dfrac1a
		\end{gather*}
}{}

\ex{}{
	La suite $u$ définie par
		\[ u(n) = 5 \times 3^n \]
	est géométrique. 
	En effet, pour tout $n\in\N$, on calcule
		\begin{align*}
			u(n+1) &= 5 \times 3^{n+1} \\
					&= 5 \times 3^n \times 3 \\
					&= 3 \times \left( 5 \times 3^n \right) \\
					&= 3 u(n).
		\end{align*}
	Comme $u(0) = 5 \times 3^0 = 5 \times 1 = 5$, on en déduit que $u$ est géométrique de raison $3$ et de terme initial $5$.
}{}

\exe{}{
	Pour chacune des suites données algébriquement pour tout $n\in\N$, donner leur raison et leur terme initial.
	\begin{multicols}{2}
	\begin{enumerate}
		\item $u(n) = 2 \times 3^n$
		\item $v(n) = 7 \times \left(\dfrac12 \right)^n$
		\item $\xi(n) = - 6^n$
		\item $a(n) = 11 \times 5^{2n}$
		\item $a(n) = 3 \times 5^{2n+3}$
		\item $a(n) = 10^{-n}$
	\end{enumerate}
	\end{multicols}
}{}

\ex{}{
	Considérons une suite $G$, géométrique de raison $4$ et de terme initial $3$.
	Alors, par définition,
		\[ G(0) = 3. \]
	En spécialisant l'équation \eqref{eq:suite-geom} pour $n=0$, on trouve ensuite
		\[ G(1) = G(0+1) = 4 \times G(0) = 4 \times 3 = 12. \]
	On peut déduire la suite des termes de façon analogue.
		\begin{align*}
			G(0) &= 3 \\
			G(1) &= 4 \times G(0) = 12 \\
			G(2) &= 4 \times G(1) = 48 \\
			G(3) &= 4 \times G(2) = 196 \\
			\vdots &\, \qquad \vdots
		\end{align*}
	On souhaite désormais connaître une formule valable pour tous les entiers naturels $n\in\N$ pour qu'on ait pas à reconstruire la suite depuis le début à chaque étude.
	Pour cela, on choisit de réécrire les termes de la suite de la façon suivante, en utilisant les propriétés de la puissance.
		\begin{align*}
			G(0) &= 3 = 3 \times 4^{0} \\
			G(1) &= 4 \times G(0) = 3 \times 4^{1} \\
			G(2) &= 4 \times G(1) = 3 \times 4^{2} \\
			G(3) &= 4 \times G(2) = 3 \times 4^{3} \\
			\vdots &\, \qquad \vdots \\
			G(n) &= 3 \times 4^n
		\end{align*}
}{}

\thm{de Maya}{
	Soit $v$ une suite géométrique de raison $q \in \R$ et de terme initial $v(0) \in \R$.
	Alors
		\[ v(n) = v(0) \times q^n. \]
}{thm:maya}

\IncMargin{1em}
\begin{algorithm}
\SetKwData{Left}{left}\SetKwData{This}{this}\SetKwData{Up}{up}
\SetKwFunction{Union}{Union}\SetKwFunction{FindCompress}{FindCompress}
\SetKwInOut{Input}{entrée}\SetKwInOut{Output}{sortie}
\Input{Suite géométrique $G$ de raison $q > 1$. Seuil $M$.}
\Output{Entier naturel $N\in\N$ tel que $G(N) \geq M$, et $G(N-1) < M$.}
\BlankLine
\emph{}\;
\caption{Problème de seuil.}\label{algo:seuil-geom}
\end{algorithm}\DecMargin{1em} 

