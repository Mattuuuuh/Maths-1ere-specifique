% SOLUTION SWITCH
\newif\ifsolutions
				\solutionstrue
				\solutionsfalse
				
\documentclass[a4paper, 12pt]{extarticle}

\usepackage[utf8x]{inputenc}
%fonts
\usepackage{libertinus,libertinust1math}
\usepackage{amsmath,amsthm,amssymb,mathtools}

% SOLUTION SWITCH

\ifsolutions
	\newcommand{\exe}[2]{
		\begin{ex} #1  \end{ex}
		\begin{sol} #2 \end{sol}
	}
\else
	\newcommand{\exe}[2]{
		\begin{ex} #1  \end{ex}
	}
	
\fi


\usepackage[french]{babel}
\usepackage[
a4paper,
margin=2cm,
nomarginpar,% We don't want any margin paragraphs
]{geometry}

% HEADER, ARRAY, ENUM, MULTIOCL
\usepackage{fancyhdr}
\usepackage{array}
\usepackage{multicol, enumitem}
\newcolumntype{P}[1]{>{\centering\arraybackslash}p{#1}}
\usepackage{stackengine}
\newcommand\xrowht[2][0]{\addstackgap[.5\dimexpr#2\relax]{\vphantom{#1}}}

% theorems

\theoremstyle{theorem}
\newtheorem{thm}{Théorème}
\theoremstyle{plain}
\newtheorem*{sol}{Solution}
\theoremstyle{definition}
\newtheorem{ex}{Exercice}
\newtheorem{dfn}{Définition}
\newtheorem*{dfn*}{Définition}


%couleurs
\usepackage{tcolorbox}
\definecolor{myg}{RGB}{56, 140, 70}
\definecolor{myb}{RGB}{45, 111, 177}
\definecolor{myr}{RGB}{199, 68, 64}
\definecolor{mygr}{HTML}{2C3338}


\tcbuselibrary{theorems,skins,hooks}
\newcounter{commonbox}
\makeatletter
\newtcbtheorem[use counter=commonbox]{theorem}{Théorème }%
{
	enhanced,
	colback=white,
	colframe=mygr,
	attach boxed title to top left={yshift*=-\tcboxedtitleheight},
	fonttitle=\bfseries,
	title={#2},
	boxed title size=title,
	boxed title style={%
			sharp corners,
			rounded corners=northwest,
			colback=tcbcolframe,
			boxrule=0pt,
		},
	underlay boxed title={%
			\path[fill=tcbcolframe] (title.south west)--(title.south east)
			to[out=0, in=180] ([xshift=5mm]title.east)--
			(title.center-|frame.east)
			[rounded corners=\kvtcb@arc] |-
			(frame.north) -| cycle;
		},
	#1
}{th}
\newtcbtheorem[use counter=commonbox]{rappel}{Rappel }%
{
	enhanced,
	colback=white,
	colframe=mygr,
	attach boxed title to top left={yshift*=-\tcboxedtitleheight},
	fonttitle=\bfseries,
	title={#2},
	boxed title size=title,
	boxed title style={%
			sharp corners,
			rounded corners=northwest,
			colback=tcbcolframe,
			boxrule=0pt,
		},
	underlay boxed title={%
			\path[fill=tcbcolframe] (title.south west)--(title.south east)
			to[out=0, in=180] ([xshift=5mm]title.east)--
			(title.center-|frame.east)
			[rounded corners=\kvtcb@arc] |-
			(frame.north) -| cycle;
		},
	#1
}{th}
\newtcbtheorem[use counter=commonbox]{strategie}{Stratégie }%
{
	enhanced,
	colback=white,
	colframe=mygr,
	attach boxed title to top left={yshift*=-\tcboxedtitleheight},
	fonttitle=\bfseries,
	title={#2},
	boxed title size=title,
	boxed title style={%
			sharp corners,
			rounded corners=northwest,
			colback=tcbcolframe,
			boxrule=0pt,
		},
	underlay boxed title={%
			\path[fill=tcbcolframe] (title.south west)--(title.south east)
			to[out=0, in=180] ([xshift=5mm]title.east)--
			(title.center-|frame.east)
			[rounded corners=\kvtcb@arc] |-
			(frame.north) -| cycle;
		},
	#1
}{th}
\newtcbtheorem[use counter=commonbox]{outil}{Outil }%
{
	enhanced,
	colback=white,
	colframe=mygr,
	attach boxed title to top left={yshift*=-\tcboxedtitleheight},
	fonttitle=\bfseries,
	title={#2},
	boxed title size=title,
	boxed title style={%
			sharp corners,
			rounded corners=northwest,
			colback=tcbcolframe,
			boxrule=0pt,
		},
	underlay boxed title={%
			\path[fill=tcbcolframe] (title.south west)--(title.south east)
			to[out=0, in=180] ([xshift=5mm]title.east)--
			(title.center-|frame.east)
			[rounded corners=\kvtcb@arc] |-
			(frame.north) -| cycle;
		},
	#1
}{th}
\newtcbtheorem[use counter=commonbox]{but}{Buts du chapitre }%
{
	enhanced,
	colback=white,
	colframe=mygr,
	attach boxed title to top left={yshift*=-\tcboxedtitleheight},
	fonttitle=\bfseries,
	title={#2},
	boxed title size=title,
	boxed title style={%
			sharp corners,
			rounded corners=northwest,
			colback=tcbcolframe,
			boxrule=0pt,
		},
	underlay boxed title={%
			\path[fill=tcbcolframe] (title.south west)--(title.south east)
			to[out=0, in=180] ([xshift=5mm]title.east)--
			(title.center-|frame.east)
			[rounded corners=\kvtcb@arc] |-
			(frame.north) -| cycle;
		},
	#1
}{th}
\newtcbtheorem[use counter=commonbox]{propriete}{Propriété }%
{
	enhanced,
	colback=white,
	colframe=mygr,
	attach boxed title to top left={yshift*=-\tcboxedtitleheight},
	fonttitle=\bfseries,
	title={#2},
	boxed title size=title,
	boxed title style={%
			sharp corners,
			rounded corners=northwest,
			colback=tcbcolframe,
			boxrule=0pt,
		},
	underlay boxed title={%
			\path[fill=tcbcolframe] (title.south west)--(title.south east)
			to[out=0, in=180] ([xshift=5mm]title.east)--
			(title.center-|frame.east)
			[rounded corners=\kvtcb@arc] |-
			(frame.north) -| cycle;
		},
	#1
}{th}
\newtcbtheorem[number within=commonbox]{definition}{Définition }%
{
	enhanced,
	colback=white,
	colframe=mygr,
	attach boxed title to top left={yshift*=-\tcboxedtitleheight},
	fonttitle=\bfseries,
	title={#2},
	boxed title size=title,
	boxed title style={%
			sharp corners,
			rounded corners=northwest,
			colback=tcbcolframe,
			boxrule=0pt,
		},
	underlay boxed title={%
			\path[fill=tcbcolframe] (title.south west)--(title.south east)
			to[out=0, in=180] ([xshift=5mm]title.east)--
			(title.center-|frame.east)
			[rounded corners=\kvtcb@arc] |-
			(frame.north) -| cycle;
		},
	#1
}{th}
\newtcbtheorem[number within=commonbox]{exemples}{Exemples }%
{
	enhanced,
	colback=white,
	colframe=mygr,
	attach boxed title to top left={yshift*=-\tcboxedtitleheight},
	fonttitle=\bfseries,
	title={#2},
	boxed title size=title,
	boxed title style={%
			sharp corners,
			rounded corners=northwest,
			colback=tcbcolframe,
			boxrule=0pt,
		},
	underlay boxed title={%
			\path[fill=tcbcolframe] (title.south west)--(title.south east)
			to[out=0, in=180] ([xshift=5mm]title.east)--
			(title.center-|frame.east)
			[rounded corners=\kvtcb@arc] |-
			(frame.north) -| cycle;
		},
	#1
}{th}
\newtcbtheorem[number within=commonbox]{exemple}{Exemple }%
{
	enhanced,
	colback=white,
	colframe=mygr,
	attach boxed title to top left={yshift*=-\tcboxedtitleheight},
	fonttitle=\bfseries,
	title={#2},
	boxed title size=title,
	boxed title style={%
			sharp corners,
			rounded corners=northwest,
			colback=tcbcolframe,
			boxrule=0pt,
		},
	underlay boxed title={%
			\path[fill=tcbcolframe] (title.south west)--(title.south east)
			to[out=0, in=180] ([xshift=5mm]title.east)--
			(title.center-|frame.east)
			[rounded corners=\kvtcb@arc] |-
			(frame.north) -| cycle;
		},
	#1
}{th}
\newtcbtheorem[number within=commonbox]{questions}{Questions guidantes }%
{
	enhanced,
	colback=white,
	colframe=mygr,
	attach boxed title to top left={yshift*=-\tcboxedtitleheight},
	fonttitle=\bfseries,
	title={#2},
	boxed title size=title,
	boxed title style={%
			sharp corners,
			rounded corners=northwest,
			colback=tcbcolframe,
			boxrule=0pt,
		},
	underlay boxed title={%
			\path[fill=tcbcolframe] (title.south west)--(title.south east)
			to[out=0, in=180] ([xshift=5mm]title.east)--
			(title.center-|frame.east)
			[rounded corners=\kvtcb@arc] |-
			(frame.north) -| cycle;
		},
	#1
}{th}
\makeatother

% corps
\newcommand{\R}{\mathbb{R}}
\newcommand{\Rnn}{\mathbb{R}^{2n}}
\newcommand{\Z}{\mathbb{Z}}
\newcommand{\N}{\mathbb{N}}
\newcommand{\Q}{\mathbb{Q}}

% domain
\newcommand{\D}{\mathcal{D}}
% for calligraphic C
\usepackage{calrsfs}
\newcommand{\C}{\mathcal{C}}

% date
\usepackage{advdate}

% ensembles tq. 
\newcommand{\xRtq}[1]{
	$\left\{ x \in \R \text{ tq. } #1 \right\}$
}

% vabs
\newcommand{\vabs}[1]{
	\left| #1 \right|
}

%pinfty minfty
\newcommand{\pinfty}{{+}\infty}
\newcommand{\minfty}{{-}\infty}

% plots
\usepackage{pgfplots}

%virgules
\usepackage{icomma}
\pgfplotsset{/pgf/number format/use comma}

%subfigures
\usepackage{subcaption}

%hyperlink footnote
\usepackage{hyperref}

%wider tabulars
\def\arraystretch{2}
\setlength\tabcolsep{15pt}

% tableaux var, signe
\usepackage{tkz-tab}

\def\arraystretch{1.6}
\setlength\tabcolsep{15pt}

\AdvanceDate[1]

\begin{document}
\pagestyle{fancy}
\fancyhead[L]{Première}
\fancyhead[C]{\textbf{Nombre dérivé : fonction carré \ifsolutions -- Solutions \fi}}
\fancyhead[R]{\today}


\exe{
	Le but de l'exercice est de calculer, à l'aide de la définition, le nombre dérivé de la fonction carré 
		\[ f(x) = x^2 \]
	en certains points, puis d'en déduire une loi générale.
	
	\begin{enumerate}
		\item On étudie d'abord la fonction $f$.
			\begin{enumerate}[label=\roman*)]
				\item Remplir le tableau de valeurs figure \ref{fig:f}.
				\item Esquisser la courbe de $f$ figure \ref{fig:Cf}.
				\item Remplir le tableau de variations de $f$ figure \ref{fig:var-signe} (on remplira le signe de $f'(x)$ plus tard).
			\end{enumerate}
		\item On souhaite calculer le nombre dérivé de $f$ en $x=1$.
		Pour cela, on considère les accroissements autour de $1$, pour $h$ de plus en plus petit
			\[ v_h = \dfrac{f(1+h)-f(1)}h. \]
			\begin{enumerate}[label=\roman*)]
				\item Remplir le tableau de valeurs figure \ref{fig:1a}.
				\item Vers quelle valeur $v_h$ tend-il quand $h$ se rapproche de $0$ ? On appelle cette valeur $f'(1)$, qu'on ajoutera dans la première colonne du tableau figure \ref{fig:f'}.
			\end{enumerate}
		\item Calculer les accroissements autour de $x=-1; -0,5 ; 0; \dots$ et compléter les tableaux figure \ref{fig:vh}. Utiliser les résultats pour remplir le tableau de valeurs de $f'$ figure \ref{fig:f'}.
		\item En déduire une formule algébrique pour obtenir $f'(x)$ comme fonction de $x$.
		\item Remplir le tableau de signes de $f'$ figure \ref{fig:var-signe}.
	\end{enumerate}
}{}

\begin{figure}[h]
	\centering
	\begin{tabular}{|c|c|c|c|c|c|c|c|c|c|}\hline
		$x$ & -4 & -3 & -2 & -1 & 0 & 1 & 2 & 3 & 4 \\ \hline
		$f(x)$ &&&&&&&&& \\ \hline
	\end{tabular}
	\caption{Tableau de valeurs de $f$.}
	\label{fig:f}
\end{figure}

\begin{figure}[h]
	\centering
	\begin{tabular}{|c|c|c|c|c|c|c|c|c|}\hline
		$x$ & 1 & -1 & -0,5 & 0 & 1,5 & 0,4 & -2 & 3  \\ \hline
		$f'(x)$ &&&&&&&& \\ \hline
	\end{tabular}
	\caption{Tableau de valeurs de $f'$.}
	\label{fig:f'}
\end{figure}


\begin{figure}[h]
	\centering	
	\begin{tikzpicture}
		\tkzTabInit
		 %[lgt=3,espcl=1.5]
	       		{$x$ / 1 , Variation de $f(x)$ / 2, Signe de $f'(x)$ / 1}
	       		{-4,,,4}
	\end{tikzpicture}
	\caption{Tableau de variations de $f$ et de signes de $f'$.}
	\label{fig:var-signe}
\end{figure}

\newpage

\begin{figure}[h]
	\centering
	\begin{tikzpicture}[scale=1.2]
	\begin{axis}[
	xmin = -4, xmax=4, ymin=0, ymax=16, 
	grid = none,  xlabel={$x$}, ylabel={$f(x)$},
	xtick = {-4, ..., 4},
	ytick = {0, 2, ..., 16},
	x=15pt,
	y=10pt,
	grid=both,
	]
	\end{axis}
	\end{tikzpicture}
	\caption{Graphe de $\C_f$ sur $[-4;4]$.}
	\label{fig:Cf}
\end{figure}


\begin{figure}[h!]
	\centering
	\begin{subfigure}{0.2\textwidth}
	\begin{tabular}{|c|c|}\hline
		$h$ & $v_h$ \\ \hline
		1 & \hspace{1cm} \\ \hline
		0,1 & \\ \hline
		0,001 & \\ \hline
		-1 & \\ \hline
		-0,1 & \\ \hline
		-0,001 & \\ \hline
	\end{tabular}
	\caption{Accroissements autour de $x=1$.}
	\label{fig:1a}
	\end{subfigure}
	\hfill
	\begin{subfigure}{0.2\textwidth}
	\begin{tabular}{|c|c|}\hline
		$h$ & $v_h$ \\ \hline
		1 & \hspace{1cm} \\ \hline
		0,1 & \\ \hline
		0,001 & \\ \hline
		-1 & \\ \hline
		-0,1 & \\ \hline
		-0,001 & \\ \hline
	\end{tabular}
	\caption{Accroissements autour de $x=-1$.}
	\label{fig:1b}
	\end{subfigure}
	\hfill
	\begin{subfigure}{0.2\textwidth}
	\begin{tabular}{|c|c|}\hline
		$h$ & $v_h$ \\ \hline
		1 & \hspace{1cm} \\ \hline
		0,1 & \\ \hline
		0,001 & \\ \hline
		-1 & \\ \hline
		-0,1 & \\ \hline
		-0,001 & \\ \hline
	\end{tabular}
	\caption{Accroissements autour de $x=-0,5$.}
	\label{fig:1c}
	\end{subfigure}
	\hfill
	\begin{subfigure}{0.2\textwidth}
	\begin{tabular}{|c|c|}\hline
		$h$ & $v_h$ \\ \hline
		1 & \hspace{1cm} \\ \hline
		0,1 & \\ \hline
		0,001 & \\ \hline
		-1 & \\ \hline
		-0,1 & \\ \hline
		-0,001 & \\ \hline
	\end{tabular}
	\caption{Accroissements autour de $x=0$.}
	\label{fig:1d}
	\end{subfigure}
	
	\begin{subfigure}{0.2\textwidth}
	\begin{tabular}{|c|c|}\hline
		$h$ & $v_h$ \\ \hline
		1 & \hspace{1cm} \\ \hline
		0,1 & \\ \hline
		0,001 & \\ \hline
		-1 & \\ \hline
		-0,1 & \\ \hline
		-0,001 & \\ \hline
	\end{tabular}
	\caption{Accroissements autour de $x=1,5$.}
	\label{fig:1e}
	\end{subfigure}
	\hfill
	\begin{subfigure}{0.2\textwidth}
	\begin{tabular}{|c|c|}\hline
		$h$ & $v_h$ \\ \hline
		1 & \hspace{1cm} \\ \hline
		0,1 & \\ \hline
		0,001 & \\ \hline
		-1 & \\ \hline
		-0,1 & \\ \hline
		-0,001 & \\ \hline
	\end{tabular}
	\caption{Accroissements autour de $x=0,4$.}
	\label{fig:1f}
	\end{subfigure}
	\hfill
	\begin{subfigure}{0.2\textwidth}
	\begin{tabular}{|c|c|}\hline
		$h$ & $v_h$ \\ \hline
		1 & \hspace{1cm} \\ \hline
		0,1 & \\ \hline
		0,001 & \\ \hline
		-1 & \\ \hline
		-0,1 & \\ \hline
		-0,001 & \\ \hline
	\end{tabular}
	\caption{Accroissements autour de $x=-2$.}
	\label{fig:1g}
	\end{subfigure}
	\hfill
	\begin{subfigure}{0.2\textwidth}
	\begin{tabular}{|c|c|}\hline
		$h$ & $v_h$ \\ \hline
		1 & \hspace{1cm} \\ \hline
		0,1 & \\ \hline
		0,001 & \\ \hline
		-1 & \\ \hline
		-0,1 & \\ \hline
		-0,001 & \\ \hline
	\end{tabular}
	\caption{Accroissements autour de $x=3$.}
	\label{fig:h}
	\end{subfigure}
	\caption{Tableaux de valeurs d'accroissements $v_h$ autour de différents $x$.}
	\label{fig:vh}
\end{figure}


\end{document}