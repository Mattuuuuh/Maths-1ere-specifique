% SOLUTION SWITCH
\newif\ifsolutions
				\solutionstrue
				\solutionsfalse
				
\documentclass[a4paper, 12pt]{extarticle}

\usepackage[utf8x]{inputenc}
%fonts
\usepackage{libertinus,libertinust1math}
\usepackage{amsmath,amsthm,amssymb,mathtools}

% SOLUTION SWITCH

\ifsolutions
	\newcommand{\exe}[2]{
		\begin{ex} #1  \end{ex}
		\begin{sol} #2 \end{sol}
	}
\else
	\newcommand{\exe}[2]{
		\begin{ex} #1  \end{ex}
	}
	
\fi


\usepackage[french]{babel}
\usepackage[
a4paper,
margin=2cm,
nomarginpar,% We don't want any margin paragraphs
]{geometry}

% HEADER, ARRAY, ENUM, MULTIOCL
\usepackage{fancyhdr}
\usepackage{array}
\usepackage{multicol, enumitem}
\newcolumntype{P}[1]{>{\centering\arraybackslash}p{#1}}
\usepackage{stackengine}
\newcommand\xrowht[2][0]{\addstackgap[.5\dimexpr#2\relax]{\vphantom{#1}}}

% theorems

\theoremstyle{theorem}
\newtheorem{thm}{Théorème}
\theoremstyle{plain}
\newtheorem*{sol}{Solution}
\theoremstyle{definition}
\newtheorem{ex}{Exercice}
\newtheorem{dfn}{Définition}
\newtheorem*{dfn*}{Définition}


%couleurs
\usepackage{tcolorbox}
\definecolor{myg}{RGB}{56, 140, 70}
\definecolor{myb}{RGB}{45, 111, 177}
\definecolor{myr}{RGB}{199, 68, 64}
\definecolor{mygr}{HTML}{2C3338}


\tcbuselibrary{theorems,skins,hooks}
\newcounter{commonbox}
\makeatletter
\newtcbtheorem[use counter=commonbox]{theorem}{Théorème }%
{
	enhanced,
	colback=white,
	colframe=mygr,
	attach boxed title to top left={yshift*=-\tcboxedtitleheight},
	fonttitle=\bfseries,
	title={#2},
	boxed title size=title,
	boxed title style={%
			sharp corners,
			rounded corners=northwest,
			colback=tcbcolframe,
			boxrule=0pt,
		},
	underlay boxed title={%
			\path[fill=tcbcolframe] (title.south west)--(title.south east)
			to[out=0, in=180] ([xshift=5mm]title.east)--
			(title.center-|frame.east)
			[rounded corners=\kvtcb@arc] |-
			(frame.north) -| cycle;
		},
	#1
}{th}
\newtcbtheorem[use counter=commonbox]{rappel}{Rappel }%
{
	enhanced,
	colback=white,
	colframe=mygr,
	attach boxed title to top left={yshift*=-\tcboxedtitleheight},
	fonttitle=\bfseries,
	title={#2},
	boxed title size=title,
	boxed title style={%
			sharp corners,
			rounded corners=northwest,
			colback=tcbcolframe,
			boxrule=0pt,
		},
	underlay boxed title={%
			\path[fill=tcbcolframe] (title.south west)--(title.south east)
			to[out=0, in=180] ([xshift=5mm]title.east)--
			(title.center-|frame.east)
			[rounded corners=\kvtcb@arc] |-
			(frame.north) -| cycle;
		},
	#1
}{th}
\newtcbtheorem[use counter=commonbox]{strategie}{Stratégie }%
{
	enhanced,
	colback=white,
	colframe=mygr,
	attach boxed title to top left={yshift*=-\tcboxedtitleheight},
	fonttitle=\bfseries,
	title={#2},
	boxed title size=title,
	boxed title style={%
			sharp corners,
			rounded corners=northwest,
			colback=tcbcolframe,
			boxrule=0pt,
		},
	underlay boxed title={%
			\path[fill=tcbcolframe] (title.south west)--(title.south east)
			to[out=0, in=180] ([xshift=5mm]title.east)--
			(title.center-|frame.east)
			[rounded corners=\kvtcb@arc] |-
			(frame.north) -| cycle;
		},
	#1
}{th}
\newtcbtheorem[use counter=commonbox]{outil}{Outil }%
{
	enhanced,
	colback=white,
	colframe=mygr,
	attach boxed title to top left={yshift*=-\tcboxedtitleheight},
	fonttitle=\bfseries,
	title={#2},
	boxed title size=title,
	boxed title style={%
			sharp corners,
			rounded corners=northwest,
			colback=tcbcolframe,
			boxrule=0pt,
		},
	underlay boxed title={%
			\path[fill=tcbcolframe] (title.south west)--(title.south east)
			to[out=0, in=180] ([xshift=5mm]title.east)--
			(title.center-|frame.east)
			[rounded corners=\kvtcb@arc] |-
			(frame.north) -| cycle;
		},
	#1
}{th}
\newtcbtheorem[use counter=commonbox]{but}{Buts du chapitre }%
{
	enhanced,
	colback=white,
	colframe=mygr,
	attach boxed title to top left={yshift*=-\tcboxedtitleheight},
	fonttitle=\bfseries,
	title={#2},
	boxed title size=title,
	boxed title style={%
			sharp corners,
			rounded corners=northwest,
			colback=tcbcolframe,
			boxrule=0pt,
		},
	underlay boxed title={%
			\path[fill=tcbcolframe] (title.south west)--(title.south east)
			to[out=0, in=180] ([xshift=5mm]title.east)--
			(title.center-|frame.east)
			[rounded corners=\kvtcb@arc] |-
			(frame.north) -| cycle;
		},
	#1
}{th}
\newtcbtheorem[use counter=commonbox]{propriete}{Propriété }%
{
	enhanced,
	colback=white,
	colframe=mygr,
	attach boxed title to top left={yshift*=-\tcboxedtitleheight},
	fonttitle=\bfseries,
	title={#2},
	boxed title size=title,
	boxed title style={%
			sharp corners,
			rounded corners=northwest,
			colback=tcbcolframe,
			boxrule=0pt,
		},
	underlay boxed title={%
			\path[fill=tcbcolframe] (title.south west)--(title.south east)
			to[out=0, in=180] ([xshift=5mm]title.east)--
			(title.center-|frame.east)
			[rounded corners=\kvtcb@arc] |-
			(frame.north) -| cycle;
		},
	#1
}{th}
\newtcbtheorem[number within=commonbox]{definition}{Définition }%
{
	enhanced,
	colback=white,
	colframe=mygr,
	attach boxed title to top left={yshift*=-\tcboxedtitleheight},
	fonttitle=\bfseries,
	title={#2},
	boxed title size=title,
	boxed title style={%
			sharp corners,
			rounded corners=northwest,
			colback=tcbcolframe,
			boxrule=0pt,
		},
	underlay boxed title={%
			\path[fill=tcbcolframe] (title.south west)--(title.south east)
			to[out=0, in=180] ([xshift=5mm]title.east)--
			(title.center-|frame.east)
			[rounded corners=\kvtcb@arc] |-
			(frame.north) -| cycle;
		},
	#1
}{th}
\newtcbtheorem[number within=commonbox]{exemples}{Exemples }%
{
	enhanced,
	colback=white,
	colframe=mygr,
	attach boxed title to top left={yshift*=-\tcboxedtitleheight},
	fonttitle=\bfseries,
	title={#2},
	boxed title size=title,
	boxed title style={%
			sharp corners,
			rounded corners=northwest,
			colback=tcbcolframe,
			boxrule=0pt,
		},
	underlay boxed title={%
			\path[fill=tcbcolframe] (title.south west)--(title.south east)
			to[out=0, in=180] ([xshift=5mm]title.east)--
			(title.center-|frame.east)
			[rounded corners=\kvtcb@arc] |-
			(frame.north) -| cycle;
		},
	#1
}{th}
\newtcbtheorem[number within=commonbox]{exemple}{Exemple }%
{
	enhanced,
	colback=white,
	colframe=mygr,
	attach boxed title to top left={yshift*=-\tcboxedtitleheight},
	fonttitle=\bfseries,
	title={#2},
	boxed title size=title,
	boxed title style={%
			sharp corners,
			rounded corners=northwest,
			colback=tcbcolframe,
			boxrule=0pt,
		},
	underlay boxed title={%
			\path[fill=tcbcolframe] (title.south west)--(title.south east)
			to[out=0, in=180] ([xshift=5mm]title.east)--
			(title.center-|frame.east)
			[rounded corners=\kvtcb@arc] |-
			(frame.north) -| cycle;
		},
	#1
}{th}
\newtcbtheorem[number within=commonbox]{questions}{Questions guidantes }%
{
	enhanced,
	colback=white,
	colframe=mygr,
	attach boxed title to top left={yshift*=-\tcboxedtitleheight},
	fonttitle=\bfseries,
	title={#2},
	boxed title size=title,
	boxed title style={%
			sharp corners,
			rounded corners=northwest,
			colback=tcbcolframe,
			boxrule=0pt,
		},
	underlay boxed title={%
			\path[fill=tcbcolframe] (title.south west)--(title.south east)
			to[out=0, in=180] ([xshift=5mm]title.east)--
			(title.center-|frame.east)
			[rounded corners=\kvtcb@arc] |-
			(frame.north) -| cycle;
		},
	#1
}{th}
\makeatother

% corps
\newcommand{\R}{\mathbb{R}}
\newcommand{\Rnn}{\mathbb{R}^{2n}}
\newcommand{\Z}{\mathbb{Z}}
\newcommand{\N}{\mathbb{N}}
\newcommand{\Q}{\mathbb{Q}}

% domain
\newcommand{\D}{\mathcal{D}}
% for calligraphic C
\usepackage{calrsfs}
\newcommand{\C}{\mathcal{C}}

% date
\usepackage{advdate}

% ensembles tq. 
\newcommand{\xRtq}[1]{
	$\left\{ x \in \R \text{ tq. } #1 \right\}$
}

% vabs
\newcommand{\vabs}[1]{
	\left| #1 \right|
}

%pinfty minfty
\newcommand{\pinfty}{{+}\infty}
\newcommand{\minfty}{{-}\infty}

% plots
\usepackage{pgfplots}

%virgules
\usepackage{icomma}
\pgfplotsset{/pgf/number format/use comma}

%subfigures
\usepackage{subcaption}

%hyperlink footnote
\usepackage{hyperref}

%wider tabulars
\def\arraystretch{2}
\setlength\tabcolsep{15pt}

% tableaux var, signe
\usepackage{tkz-tab}


\AdvanceDate[1]

\begin{document}
\pagestyle{fancy}
\fancyhead[L]{Première}
\fancyhead[C]{\textbf{Chapitre 3 --- Dérivation \ifsolutions -- Solutions \fi}}
\fancyhead[R]{\today}


\begin{questions*}{}{}
	
	 \begin{enumerate}[label=\roman*)]
	 	\item Quelles sont les variations de $x^2 - 2x + 3$ sur $\R$ ?
	 	\item Quel est le maximum, le minimum de $x^3 - 6x + 9x - 5$ sur un intervalle donné ?
		\item Pour quel $x \in [0;2]$ la valeur $x \sqrt{4-x^2}$ est-elle maximale ?
	\end{enumerate}
\end{questions*}


\begin{but*}{}{}
	\begin{enumerate}
		\item Comprendre le sens de variations d'une fonction par analyse locale.
		\item En déduire les extrema (maximum, minimum) d'une fonction par analyse globale.
	\end{enumerate}
\end{but*}

\begin{rappel*}{fonctions affines}{}
	
	Une fonction $f$ est \emph{affine} si elle s'écrit
		\[ f(x) = \qquad\qquad\qquad, \]
	où $a$ est le
	
	La courbe représentative de $f$ est une % droite de pente $a$.
\end{rappel*}

\begin{figure}[h]
	\centering
	\begin{tabular}{|c|c|c|c|c|c|c|c|}\hline
		$x$ & $-3$ & $-2$ & $-1$ & $0$ &  1 & 2 & 3 \\ \hline
		$f(x) = -2x+1$ &&&&&&& \\ \hline
	\end{tabular}
	\caption{Tableau de valeurs de $f(x) = -2x+1$.}
	\label{fig:f(t)}
\end{figure}

\begin{figure}[h]
	\centering
	\begin{tikzpicture}[scale=1]
	\begin{axis}[
	xmin = -3, xmax=3, ymin=-5, ymax=7, 
	grid = both,  xlabel={$x$}, ylabel={$f(x) = -2x+1$},
	xtick = {-3, ..., 3},
	ytick = {-5, ..., 7},
	]
	\end{axis}
	\end{tikzpicture}
	\caption{Graphe de $\C_f$ où $f(x) = -2x+1$.}
	\label{fig:Cf}
\end{figure}

\begin{propriete}{}{}
	Soit $f$ une fonction affine et $a$ son coefficient directeur.
		\begin{enumerate}	
			\item si $a > 0$, alors $f$ est
			\item si $a < 0$, alors $f$ est
		\end{enumerate}
\end{propriete}

\begin{outil*}{théorème du coefficient directeur}{}
	Soient $x, y\in\R$ deux nombres distincts, et $f$ une fonction affine de coefficient directeur $a$.
	Alors
	
\begin{multicols}{2}
		%\[ a = \dfrac{f(y) - f(x)}{y-x}. \]
		\[ a = \qquad\qquad\qquad. \]
		\vfill
		\,
		
	\begin{tikzpicture}[scale=.8]
	\begin{axis}[xmin = -2, xmax=2, ymin=-5, ymax=3, grid = none,  xlabel={$x$}, ylabel={$f(x)$}]
		\addplot[domain=-2:2, samples=2, myb, thick] {2*x-1};
		
		\addplot[black, thick, mark=*, mark size = 1] (-1,-3) node[right=5pt] {$A(x; f(x))$};
		\addplot[black, thick, mark=*, mark size = 1] (1,1) node[above left] {$B(y; f(y))$};
	\end{axis}
	\end{tikzpicture}
\end{multicols}

\end{outil*}

\begin{exemple*}{étude de $\mathbf{f(x) = \dfrac12(x^3 + 2x^2 - x - 2)}$ autour de $\mathbf{x=-0,5}$}{}

\begin{center}
\begin{tikzpicture}[scale=.8]
\begin{axis}[xmin = -3, xmax=2, ymin=-5, ymax=6, grid = none,  xlabel={$x$}, ylabel={$f(x)$}]
	\addplot[domain=-3:2, samples=500, myb, thick] {.5*(x+2)*(x+1)*(x-1)};
	
	% zoom square A to B
	\newcommand\xA{-2}
	\newcommand\yA{-2}
	\newcommand\xB{1}
	\newcommand\yB{2}
	\draw[black, thick] (axis cs:\xA,\yA) -- (axis cs:\xA,\yB);
	\draw[black, thick] (axis cs:\xB,\yA) -- (axis cs:\xB, \yB);
	\draw[black, thick] (axis cs:\xA,\yB) -- (axis cs:\xB,\yB);
	\draw[black, thick] (axis cs:\xA,\yA) -- (axis cs:\xB,\yA);
\end{axis}
\end{tikzpicture}
\begin{tikzpicture}[scale=.8]
\begin{axis}[xmin = -2, xmax=1, ymin=-2, ymax=2, grid = none,  xlabel={$x$}, ylabel={$f(x)$}]
	\addplot[domain=-2:1, samples=500, myb, thick] {.5*(x+2)*(x+1)*(x-1)};
	
	% zoom square A to B
	\newcommand\xA{-1}
	\newcommand\yA{-1.5}
	\newcommand\xB{0}
	\newcommand\yB{0.5}
	\draw[black, thick] (axis cs:\xA,\yA) -- (axis cs:\xA,\yB);
	\draw[black, thick] (axis cs:\xB,\yA) -- (axis cs:\xB, \yB);
	\draw[black, thick] (axis cs:\xA,\yB) -- (axis cs:\xB,\yB);
	\draw[black, thick] (axis cs:\xA,\yA) -- (axis cs:\xB,\yA);
\end{axis}
\end{tikzpicture}

\begin{tikzpicture}[scale=.8]
\begin{axis}[xmin = -1, xmax=0, ymin=-1.5, ymax=.5, grid = none,  xlabel={$x$}, ylabel={$f(x)$}]
	\addplot[domain=-1:0, samples=500, myb, thick] {.5*(x+2)*(x+1)*(x-1)};
	
	% zoom square A to B
	\newcommand\xA{-.75}
	\newcommand\yA{-1}
	\newcommand\xB{-.25}
	\newcommand\yB{0}
	\draw[black, thick] (axis cs:\xA,\yA) -- (axis cs:\xA,\yB);
	\draw[black, thick] (axis cs:\xB,\yA) -- (axis cs:\xB, \yB);
	\draw[black, thick] (axis cs:\xA,\yB) -- (axis cs:\xB,\yB);
	\draw[black, thick] (axis cs:\xA,\yA) -- (axis cs:\xB,\yA);
\end{axis}
\end{tikzpicture}
\begin{tikzpicture}[scale=.8]
\begin{axis}[xmin = -.75, xmax=-.25, ymin=-1, ymax=0, grid = none,  xlabel={$x$}, ylabel={$f(x)$}]
	\addplot[domain=-1:0, samples=500, myb, thick] {.5*(x+2)*(x+1)*(x-1)};
	
	\addplot[black, thick, mark=*, mark size = 1] (-.5,-.5625) node[right=5pt] {$A(x; f(x))$};
	\addplot[black, thick, mark=*, mark size = 1] (-.6,-.448) ;
	\addplot[black] (-.5,-.388) node{$B(x-h; f(x-h))$};
	\addplot[black, thick, mark=*, mark size = 1] (-.4,-.672);
	\addplot[black] (-.4,-.75) node{$C(x+h; f(x+h))$};
\end{axis}
\end{tikzpicture}
\end{center}

La pente de la droite $(AC)$ est
	\[ a_h = \qquad\qquad\qquad\qquad\qquad\qquad. \]
La pente de la droite $(AB)$ est
	\[ a_h = \qquad\qquad\qquad\qquad\qquad\qquad. \]

\end{exemple*}


\begin{definition*}{nombre dérivé}{}
	On considère les coefficients directeurs
		\begin{align}
			a_h = \dfrac{f(x+h) - f(x)}h, \label{eq:ah}
		\end{align}
	où $h$ est de plus en plus petit.

	Si $f$ est suffisamment lisse, on admet que $a_h$ converge vers une valeur qu'on écrit
		\[ f'(x). \]
	C'est le \emph{nombre dérivé} de $f$ en $x$.
\end{definition*}

\begin{definition*}{tangente à un courbe}{}
	La droite approximant $\C_f$ localement autour de $x$ est la \emph{tangente} à $\C_f$ en $x$.
	
	Son coefficient directeur est $f'(x)$ et elle passe par $(x; f(x))$.
	
\end{definition*}

\begin{exemple*}{tangente à $\mathbf{f(x) = \dfrac12(x^3 + 2x^2 - x - 2)}$ en $\mathbf{x=-0,5}$}{}

On pose $x=-0,5$ et $h=0,001$ dans l'équation \eqref{eq:ah} qui devient
	\[ a_{0,001} = \dfrac{f(-0,499) - f(-0,5)}{0,001} = \dfrac{-0,563625 - (-0,5625)}{0,001} = -1,125. \]
	
On trace la droite de pente $-1,125$ passant par $(-0,5 ; f(-0,5))$ ci-dessous.

\begin{center}
\begin{tikzpicture}[scale=.8]
\begin{axis}[xmin = -3, xmax=2, ymin=-5, ymax=6, grid = none,  xlabel={$x$}, ylabel={$f(x)$}]
	\addplot[domain=-3:2, samples=500, myb, thick] {.5*(x+2)*(x+1)*(x-1)};
	
	% tangente
	\addplot[domain=-2:1, samples=2, myr, thick, <->] {-1.125*(x+.5)-.5625};
\end{axis}
\end{tikzpicture}
\end{center}

\end{exemple*}

%\begin{propriete}{}{}
%	$f'(x)$ est le coefficient directeur de la tangente à $f$ en $x$.
%\end{propriete}

\begin{propriete}{}{}
	Le signe de $f'(x)$ donne la variation de $f$ en $x$ et inversement :
		\begin{enumerate}
			\item si $f'(x) > 0$, alors 
			\item si $f$ est croissante autour de $x$, alors
			\item si $f'(x) < 0$, alors 
			\item si $f$ est décroissante autour de $x$, alors
		\end{enumerate}
\end{propriete}

\begin{exemple*}{variations de $\mathbf{f(x) = \dfrac12(x^3 + 2x^2 - x - 2)}$, signe de $\mathbf{f'(x)}$}{}
	\begin{center}
	\begin{tikzpicture}
		\tkzTabInit
		 %[lgt=3,espcl=1.5]
	       		{$x$ / 1 , Variation de $f(x)$ / 2, Signe de $f'(x)$ / 1}
	       		{-3,,,,2}
	\end{tikzpicture}
	\end{center}
\end{exemple*}

\begin{strategie*}{déterminer la variation de $\mathbf{f}$ autour de $\mathbf{x}$}{}
	On considère une courbe $\C_f$ lisse quelconque.
	Par zooms successifs, on remarque que $\C_f$ devient presque droite.
	L'erreur d'approximation devient de plus en plus petite en zoomant.
	
	\begin{enumerate}
		\item On suppose $h$ assez petit tel que $\C_f$ soit droite sur $[x-h ; x+h]$.
		\item On calcule le coefficient directeur de la droite à l'aide de l'équation \eqref{eq:ah}. Il tend vers $f'(x)$.
		\item On regarde le signe de $f'(x)$ qui donne la variation de $f$ autour de $x$.
	\end{enumerate}
\end{strategie*}

\begin{rappel*}{}{}
	Un exetremum local (maximum ou minimum) de $f$ survient lorsque $f$ change de variations.
\end{rappel*}

\begin{propriete}{}{}
		\begin{center}
			Un exetremum local de $f$ survient lorsque $f'$ change de \qquad\qquad\qquad\qquad
		\end{center}
		\begin{center}
			Si un exetremum local de $f$ survient en $x$, on a nécessairement \qquad\qquad\qquad\qquad
		\end{center}
\end{propriete}
%\begin{strategie*}{déterminer les extrema de $f$}{}
%	\begin{enumerate}
%		\item On calcule $f'(x)$ comme précédemment.
%		\item On trouve les $x$ vérifiant $f'(x) = 0$.
%		\item On regarde le signe de $f'$ autour de $x$ : s'il change, c'est un extremum.
%	\end{enumerate}
%\end{strategie*}


\end{document}