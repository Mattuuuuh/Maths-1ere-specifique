%!TEX encoding = UTF8
%!TEX root =notes.tex

\section{Fonctions exponentielles}

Le but de cette partie est d'étendre la notion de suite géométrique
	\[ S(n) = S(0) \times q^n, \]
\emph{a priori} définie uniquement pour $n\in\N$ entiers naturels, à une fonction sur $\R$ tout entier.

On pourra alors écrire
	\[ g(x) = g(0) \times q^x, \]
sans prendre de précautions, mais lors du calcul d'image, on se heurte aux problèmes suivants :
	\begin{itemize}
		\item On a défini $q^0 = 1$, mais pourquoi ?
		\item Que signifie $q^{-1}$ ?
		\item Que signifie $q^{0,5}$ ?
		\item Que signifie $q^{-1/3}$ ?
		\item Que signifie $q^{\pi}$ ?
	\end{itemize}
Une première étape est donc de comprendre la signification de l'exponentiation lorsque l'exposant n'est pas entier.

\subsection{Généralisation de l'exponentiation}

\dfn{Vocabulaire d'exponentiation : puissance, base, exposant}{
	On parle d'\emph{exponentiation} lorsqu'on prend une valeur à la \emph{puissance} d'une autre.
	
	On dit \og $a$ puissance $b$ \fg ou \og $a$ exposant $b$ \fg pour signifier 
		\[ a^b, \]
	et on appelle alors $a$ la \emph{base}, et $b$ l'\emph{exposant}.
}{}

\nt{
	Dans la suite, $q\in\R^*_+$ est un nombre réel strictement positif, base des différentes exponentiations.
}


\dfn{Puissance sur $\N^*$}{
	Soit $m \in \N^*$ un entier naturel non nul.
	Alors \og $q$ puissance $m$ \fg~est égal à
		\[ q^{m} = \underbrace{q \times q \times \cdots \times q}_{\text{$m$ fois}}. \]
	%En particulier, $q^1 = q$.
}{}

\nt{
	Afin d'étendre la définition de $q^m$ aux $m$ nuls et négatifs entiers, il faut partir d'une relation algébrique qu'on souhaite universellement vraie, et d'en étudier ses conséquences.
	
	Pour $m\in\N^*$, on a 
		\[ q^{m+1} = \underbrace{q \times q \times \cdots \times q}_{\text{$m+1$ fois}} = q \times q^{m} \]
	On se demande à présent : que signifie $q^0$ ?
	Si on devrait assigner une valeur à cette quantité, elle devrait respecter la relation algébrique ci-dessous.
	On étend alors cette relation à tous les $m\in\Z$ et on en déduit que, en prenant $m=0$,
		\[ q^1 = q \times q^{0}, \]
	et donc que $q^0 = 1$.
	
	En prenant $m=-1$ pour donner une valeur à $q^{-1}$, on obtient
		\[ 1 = q^{0} = q \times q^{-1}, \]
	et donc $q^{-1} = \dfrac1q$.
	
	On continue avec $m=-2$, pour trouver
		\[ \dfrac1q = q^{-1} = q \times q^{-2}, \]
	et donc $q^{-2} = \dfrac1{q^2}$.
}{}

\dfn{Extension de la puissance à $\Z$}{
	On étend la notation $q^m$ à $m \in \Z$ pour obtenir les valeurs suivantes.
		\begin{center}
		\begin{tabular}{|c|c|c|c|c|c|c|}\hline
			$m$ & $-3$ & $-2$ & $-1$ & $0$ & $1$ & $2$ \\ \hline
			$q^m$ & $\frac1{q^3}$ & $\frac1{q^2}$ & $\frac1q$ & $1$ & $q$ & $q^2$ \\ \hline
		\end{tabular}
		\end{center}
	En général, on trouve
		\begin{align*}
			q^{0} = 1, && q^{-1} = \dfrac1q, && \text{ et } && q^{-m} = \dfrac1{q^m}.
		\end{align*}
}{}

\exe{}{
	Lorsque l'exposant est négatif, il suffit donc de prendre l'inverse multiplicatif.
	Ainsi,
		\begin{align*}
			3^4 = 81, && 3^{-4} = \dfrac{1}{81}, && 2^{-7} = \dfrac1{2^7} = \dfrac1{128}, && \dfrac{1}{3^{-2}} = 3^{2} = 9.
		\end{align*}
}{}

\nt{
	Afin d'étendre la définition de $q^m$ aux $m$ rationnels (c'est-à-dire fractions de deux entiers), il faut à nouveau partir d'une relation algébrique qu'on souhaite universellement vraie, et d'en étudier ses conséquences.
	
	Pour $m, n \in \N^*$, on a 
		\[ \left( q^m \right)^n = \underbrace{q^m \times q^m \times \cdots \times q^m}_{\text{$n$ fois}} = q^{m \times n} \]
	On se demande à présent : que signifie $q^{0,5}$ ?
	Si on devrait assigner une valeur à cette quantité, elle devrait respecter la relation algébrique ci-dessous.
	On étend alors cette relation à tous les $m, n\in\Q$ et on en déduit que, en prenant $m=0,5$ et $n=2$,
		\[ \left( q^{0,5} \right)^2 = q^{0,5 \times 2} = q^{1} = q, \]
	et donc que 
		\[ q^{0,5} = \sqrt{q}. \]
		
	En prenant $m=1/3$, on a nécessairement que 
		\[ \left( q^{1/3} \right)^3 = q, \]
	et que $q^{1/3}$ est la racine cubique de $q$.
}{}

\dfn{Extension de la puissance à $\Q$}{
	On étend la relation $q^m$ à $m\in\Q$, pour obtenir
		\[ \left( q^{1/m} \right)^{m} = q. \]
	Le nombre $q^{1/m}$ est donc l'unique nombre réel $x\in\R^*_+$ strictement positif tel que
		\[ x^m = q. \]
	C'est la racine $m$-ième de $q$ et, pour $m=2$, on retrouve la racine carré.
	
	On étend alors la puissance à tout nombre rationnel $\frac{n}{m} \in \Q$ par
		\[ q^{n/m} = q^{n \times 1/m} = \left( q^{1/m} \right)^n. \]
}{}

\nt{
	Remarquons qu'on a toujours pas étendu l'exponentiation à tout les nombres réels, car la plupart d'entre eux ne sont pas rationnels (c'est le cas de $\sqrt{2}, \pi$, et en fait n'importe quel réel pris au hasard --- en supposant que ça soit possible).
	
	Cependant tout réel est approximable par un nombre rationnel d'aussi près que désiré : il suffit de couper son expansion décimale à partir d'un certain point $k$, et on obtient une approximation d'ordre $10^{-k}$.
	La valeur de $2^\pi$ n'est donc pas connue exactement, mais peut être approximée à autant de chiffres après la virgule que désiré. C'est comme si on connaissait sa valeur, en pratique !
}{}

\ex{}{
	On a la relation $3^2 = 9$. En mettant à la puissance $\frac12$ des deux côtés, on trouve
		\[ \sqrt{9} = 3 = \left( 3^2 \right)^{1/2} =  9^{1/2}. \]
	La puissance $\frac12$ est donc bien la racine carré.
	
	De la relation $7^3 = 343$, on trouve que
		\[ 343^{1/3} = 7. \]
	En mettant au carré, on a alors
		\[ 343^{2/3} = 7^2 = 49. \]
		
	On a en outre la relation $2^6 = 64$. Il suit alors que 
		\begin{align*}
			64^{-1/6} = \dfrac12, && 64^{1/3} = \left( 64^{1/6} \right)^2 = 2^2 = 4, && 64^{1/2} =  \left( 64^{1/6} \right)^3 = 2^3 = 8.
		\end{align*}
	On vérifiera qu'on ait bien $4^3 = 8^2 = 64$ pour s'assurer de la cohérence des opérations.
}{}

\subsection{Application : évolution moyenne}

\dfn{Pourcentage}{
	Soient $P, x \in\R$ deux nombres réels positifs ou nuls.
	Alors
	  	\begin{center}
	  	\og $P \%$ de $x$ \fg~= $\dfrac{P}100 \times x$.
	  	\end{center}
	En particulier,
	  	\begin{center}
	  	\og $A \%$ de $B$ \fg~= \og $B \%$ de $A$ \fg.
	  	\end{center}
}{}

\nt{
	Lorsqu'on lit \og $P \%$ de $x$ \fg, on convertit $P\% = \dfrac{P}{100}$, et on remplace \og de \fg par le signe de multiplication $\times$.
}{}

\ex{}{
	Comme $50\% = \dfrac{50}{100} = 0,5 = \dfrac12$, on a $50\%$ de $340 = \dfrac12 \times 340 = 170$.
	
	De plus, $30\%$ de $50 = \dfrac{30}{100} \times 50 = 0,3 \times 50 = 15$.
	On aurait aussi pû dire que c'est égal à $50\%$ de $30$, qui vaut $15$ sans hésiter.
}{}

\thm{Évolutions successives}{
	L'évolution d'une valeur correspond à sa multiplication par un coefficient multiplicateur $m$.
	
	\begin{enumerate}
		\item Si $m>1$, $m=1+p$, et l'évolution est une augmentation de $100p \%$.
		\item Si $m<1$, $m=1-p$, et l'évolution est une diminution de $100p \%$.
		\item Si $m=1$, la valeur ne change pas.
	\end{enumerate}
	
	Lorsque deux évolutions successives ont lieu, les coefficients sont multipliés entre eux pour obtenir un coefficient multiplicateur global.

	\begin{center}
	\begin{tikzpicture}
		% nodes
		\draw (0,0) ellipse (2cm and .5cm) node {Valeur initiale};
		
		\draw (5,0) ellipse (2cm and .5cm) node {Nouvelle valeur};
		
		\draw (10,0) ellipse (2cm and .5cm) node {Valeur finale};
		
		% vertices
		\draw[->, thick, myg] (1cm,.6cm) arc (105:75:7) node[midway, above] {$\times m_1$};
		\draw[->, thick, myg] (6cm,.6cm) arc (105:75:7) node[midway, above] {$\times m_2$};
		
		\draw[->, thick, myr] (1cm,-.5cm) arc (-105:-75:15) node[midway, below=5pt] {$\times (m_1 \times m_2)$};
	\end{tikzpicture}
	\end{center}
}{thm:ev-succ}

\thm{Évolution réciproque}{
	L'évolution réciproque est l'évolution qui permet de revenir à une valeur initale.
	
	\begin{center}
	\begin{tikzpicture}
		% nodes
		\draw (0,0) ellipse (2cm and .5cm) node {Valeur initiale};
		
		\draw (5,0) ellipse (2cm and .5cm) node {Nouvelle valeur};
		
		\draw (10,0) ellipse (2cm and .5cm) node {Valeur initiale};
		
		% vertices
		\draw[->, thick, myg] (1cm,.6cm) arc (105:75:7) node[midway, above] {$\times m_1$};
		\draw[->, thick, myg] (6cm,.6cm) arc (105:75:7) node[midway, above] {$\times \dfrac1{m_1}$};
		
		\draw[->, thick, myr] (1cm,-.5cm) arc (-105:-75:15) node[midway, below=5pt] {$\times 1$};
	\end{tikzpicture}
	\end{center}
}{}

\thm{Évolution moyenne}{
	Si une valeur de départ subit $n$ évolutions successives, alors on appelle \emph{coefficient multiplicateur moyen} le nombre
		%\[ M_{\text{Moyen}} = \left( M_{\text{Global}} \right)^{1/n}. \]
		\[ M_{\text{Moyen}} = M_{\text{Global}}^{1/n}. \]

	\begin{center}
	\begin{tikzpicture}[scale=.8]
		% nodes
		\draw (0,0) ellipse (2cm and .5cm) node {Valeur $1$};
		
		\draw (5,0) ellipse (2cm and .5cm) node {Valeur $2$};
		
		\draw (10,0) ellipse (2cm and .5cm) node {Valeur $3$};
		
		\draw (15,0) ellipse (2cm and .5cm) node {Valeur $4$};
		
		% vertices
		\draw[->, thick, myg] (1cm,.6cm) arc (105:75:7) node[midway, above] {$\times m_1$};
		\draw[->, thick, myg] (6cm,.6cm) arc (105:75:7) node[midway, above] {$\times m_2$};
		\draw[->, thick, myg] (11cm,.6cm) arc (105:75:7) node[midway, above] {$\times m_3$};
		
		\draw[->, thick, myr] (1cm,-.5cm) arc (-105:-75:25) node[midway, below] {$\times M_{\text{Global}}$};
	\end{tikzpicture}
	\end{center}
}{}

\subsection{Fonction exponentielle}

\dfn{Fonction exponentielle de base $q$}{
	On dit qu'une fonction $g$ est exponentielle de base $q\in\R_+^*$ si elle est de la forme 
		\[ g(x) = g(0) \times q^{x}. \]
}{}

\nt{
	Une fonction exponentielle $g$ est donc donnée par deux paramètres : sa base $q$ et sa valeur en $0$, $g(0)$.
	Il suffit de deux points de $\C_g$ pour connaître $g$ tout entière, car si 
		\begin{align*}
			g(x_1) = y_1, && g(x_2) = y_2,
		\end{align*}
	alors $\dfrac{g(x_1)}{g(x_2)} = q^{x_1 - x_2}$, et donc 
		\[ q = \left( \dfrac{y_1}{y_2} \right)^{1/(x_1 - x_2)}. \]
	L'équation $g(x_1) = y_1$ permet alors de trouver $g(0)$.
	
	Tout ceci est à mettre en parallèle aux méthodes permettant de trouver les paramètres d'une fonction affine.
	Au lieu de faire un ratio de différences $\left(\dfrac{y_2 - y_1}{x_2-x_1}\right)$, on a fait une exponentiation de ratios.
	
	C'est peu surprenant car on a défini les suites arithmétiques par une addition d'une raison et les suites géométriques par une multiplication.
}{}

TODO: variations

