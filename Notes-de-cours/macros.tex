%!TEX encoding = UTF8
%!TEX root = 0-notes.tex

%%%%%%%%%%%%%%%%%%%%%%%%%%%%%%
% SELF MADE COMMANDS
%%%%%%%%%%%%%%%%%%%%%%%%%%%%%%


%%
% tcolor environments VS clean environments
%%

\newcommand{\thm}[3]{\begin{theorem}[#1]\label{#3}#2\end{theorem}}
\newcommand{\cor}[3]{\begin{corollaire}[#1]\label{#3}#2\end{corollaire}}
\newcommand{\lem}[3]{\begin{lemme}[#1]\label{#3}#2\end{lemme}}
\newcommand{\mprop}[3]{\begin{proposition}[#1]\label{#3}#2\end{proposition}}
\newcommand{\ex}[3]{\begin{exemple}[#1]\label{#3}#2\end{exemple}}
\newcommand{\dfn}[3]{\begin{definition}[#1]\label{#3}#2\end{definition}}
\newcommand{\qs}[2]{\begin{question}[#1]#2\end{question}}
\newcommand{\pf}[2]{\begin{preuve}[#1]#2\end{preuve}}
\newcommand{\nt}[1]{\begin{remarque}#1\end{remarque}}
\newcommand{\str}[1]{\begin{strategie}#1\end{strategie}}
\newcommand{\mth}[1]{\begin{methode}#1\end{methode}}
\newcommand{\ax}[3]{\begin{axiome}[#1]\label{#3}#2\end{axiome}}
\newcommand{\notations}[1]{\begin{notation}#1 \end{notation}}
\newcommand{\nomen}[1]{\begin{nomenclature}#1 \end{nomenclature}}
\newcommand{\heur}[1]{\begin{heuristique}#1\end{heuristique}}

\newcommand{\exe}[4]{
	\begin{Exercise}[title=#1, label=#3]
		\marginpar{\mbox{\scriptsize(solution p.\pageref{\ExerciseLabel-Answer})}}
		#2
	\end{Exercise}
	\begin{Answer}[ref=#3]
		#4
	\end{Answer}
}

\newcommand{\exemulticols}[5]{
	\begin{multicols}{2}
	\begin{Exercise}[title=#1, label=#4]
		\marginnote{\mbox{\scriptsize(solution p.\pageref{\ExerciseLabel-Answer})}}
		#2
	\end{Exercise}
	\columnbreak
		#3
	\end{multicols}
	\begin{Answer}[ref=#4]
		#5
	\end{Answer}
}

%%

% deliminators
\DeclarePairedDelimiter{\abs}{\lvert}{\rvert}
%\DeclarePairedDelimiter{\norm}{\lVert}{\rVert}

\DeclarePairedDelimiter{\ceil}{\lceil}{\rceil}
\DeclarePairedDelimiter{\floor}{\lfloor}{\rfloor}
\DeclarePairedDelimiter{\round}{\lfloor}{\rceil}

\newsavebox\diffdbox
\newcommand{\slantedromand}{{\mathpalette\makesl{d}}}
\newcommand{\makesl}[2]{%
\begingroup
\sbox{\diffdbox}{$\mathsurround=0pt#1\mathrm{#2}$}%
\pdfsave
\pdfsetmatrix{1 0 0.2 1}%
\rlap{\usebox{\diffdbox}}%
\pdfrestore
\hskip\wd\diffdbox
\endgroup
}
\newcommand{\dd}[1][]{\ensuremath{\mathop{}\!\ifstrempty{#1}{%
\slantedromand\@ifnextchar^{\hspace{0.2ex}}{\hspace{0.1ex}}}%
{\slantedromand\hspace{0.2ex}^{#1}}}}
\ProvideDocumentCommand\dv{o m g}{%
  \ensuremath{%
    \IfValueTF{#3}{%
      \IfNoValueTF{#1}{%
        \frac{\dd #2}{\dd #3}%
      }{%
        \frac{\dd^{#1} #2}{\dd #3^{#1}}%
      }%
    }{%
      \IfNoValueTF{#1}{%
        \frac{\dd}{\dd #2}%
      }{%
        \frac{\dd^{#1}}{\dd #2^{#1}}%
      }%
    }%
  }%
}
\providecommand*{\pdv}[3][]{\frac{\partial^{#1}#2}{\partial#3^{#1}}}
%  - others
\DeclareMathOperator{\Lap}{\mathcal{L}}
\DeclareMathOperator{\Var}{Var} % variance
\DeclareMathOperator{\Cov}{Cov} % covariance

% Since the amsthm package isn't loaded

% I prefer the slanted \leq
\let\oldleq\leq % save them in case they're every wanted
\let\oldgeq\geq
\renewcommand{\leq}{\leqslant}
\renewcommand{\geq}{\geqslant}

% tel que
\newcommand{\tqs}{\text{ tels que }}
\newcommand{\tq}{\text{ tq. }}
\newcommand{\et}{\text{ et }}
\newcommand{\ou}{\text{ ou }}
\newcommand{\pourtout}{\text{ pour tout }}
\newcommand{\sct}{\text{ sachant }}

% Lois
\newcommand{\Bern}{\text{Bern}}
\newcommand{\Binom}{\text{Binom}}

% ensemble avec bigl et bigr
\newcommand{\bigset}[1]{\bigl\{ #1 \bigr\}}
\newcommand{\Bigset}[1]{\Bigl\{ #1 \Bigr\}}
\newcommand{\bigpar}[1]{\bigl( #1 \bigr)}
\newcommand{\Bigpar}[1]{\Bigl( #1 \Bigr)}

% PLUS INFTY AND MINUS INFTY WITH NO SPACE
\newcommand{\pinfty}{{+}\infty}
\newcommand{\minfty}{{-}\infty}

% vecteur flèche
\renewcommand{\vec}[1]{\overrightarrow{#1}}

% vecteur pmatrix
\newcommand{\pvec}[2]{\begin{pmatrix} #1 \\ #2 \end{pmatrix}}

% vecteur norme
\newcommand{\norm}[1]{\left\Vert #1 \right\Vert}

% point plan
\newcommand{\point}[3]{
	#1\left(#2 ; #3 \right)
}

% \smash avant \underline pour coller la ligne au mot
\let\oldunderline\underline
\renewcommand{\underline}[1]{\oldunderline{\smash{#1}}}

% emph + index
\newcommand{\emphindex}[1]{\emph{#1}\index{#1}}

% tableau croisé
\newcommand{\tableaucroise}[4]{
\begin{tabular}{|c|c|c|c|}
	\cline{2-4}
	\multicolumn{1}{c|}{} & #1 \\ \hline
	#2 \\ \hline
	#3  \\ \hline
	#4  \\ \hline
\end{tabular}
}

% python minted
\newcommand{\python}[1]{
\inputminted[
		linenos,
		gobble=0,
		breaklines=true, % otherwise it breaks for no apparent reason?
		breakafter=,,
		fontsize=\small,
		numbersep=8pt,
		tabsize=4, % tab ident = 4 spaces
		fontfamily=courier, %important pour les signes <, >
]{python}{python/#1.py}
}
