%!TEX encoding = UTF8
%!TEX root = 0-notes.tex

\chapter{Lecture graphique et fonctions affines}


%%%%%%%
%%%%%%%

% possibilité de faire comme suit : définir uniquement l'identité y=x et s'assurer que c'est une droite.
% ensuite y = ax est une fonction parente, ce qui donne l'aspect selon le signe de a d'ailleurs
% puis y = ax+b est un shift d'ordonnées. ce qui donne l'ordonnée à l'origine d'ailleurs

%%%%%%%
%%%%%%%
%%%%%%%

\section{Lecture graphique}

\subsection{Courbe représentative}

\notations{
	Pour une fonction $f$ quelconque, 
	on note $\D\subseteq\R$ un \emphindex{domaine} sur laquelle elle est bien définie et étudiée :
	$f$ admet une image $f(x)$ pour chaque $x\in\D$ du domaine d'étude.
}

Pour chaque $x\in\D\subset\R$, élément du domaine de $f$, il faut pouvoir facilement lire son image $f(x)$ par $f$.
À cette fin et pour chaque $x\in\D$, on crée un point d'abscisse $x$ et d'ordonnée $f(x)$.
L'ensemble de ces points est la courbe représentative de $f$.
Pour lire l'image de $x$ par $f$, il suffit alors de trouver l'ordonnée de l'unique point de la courbe d'abscisse $x$.

\dfn{courbe représentative}{
	Considérons $f : \D \rightarrow \R$ une fonction.
	
	La \emphindex{courbe représentative} de $f$, notée $\C_f$, est donnée par l'ensemble de points
		\[ \C_f = \Bigset{ \bigpar{x ; f(x)} \text{ où $x$ parcourt } \D }. \]
	On lit \og la courbe représentative de $f$ est l'ensemble des points $\bigl(x,f(x)\bigr)$ du plan où $x$ parcourt le domaine de $f$ \fg.
}{dfn:Cf}	

\nt{
	Connaître $f$ sur son domaine $\D$ c'est connaître $\C_f$, et vice versa.
	Cependant, il n'est possible de dessiner $\C_f$ si le domaine $\D$ est infini, ou si $f$ prend des valeurs toujours plus grandes.
}

\cor{propriété fondamentale}{
	On a donc la propriété fondamentale suivante valable pour tout $x, y\in\R$.
		\begin{align*}
			(x;y) \in \C_f && \iff && y = f(x).
		\end{align*}
		
		\begin{center}
		 \includegraphics[page=12, scale=1.5]{figures/fig-affines.pdf}
		\end{center}
	%\emph{Note : $f(x) = \dfrac1{200}(x+3,5)(x+2,5)(x-3,5)(x-4,5)(x-5,5) + 2$, polynôme de degré $5$ dans l'exemple ci-dessus.}
}{cor:prop-fond}

\nomen{
	On appelle \emphindex{graphe} la représentation graphique d'une fonction.
}

\ex{}{
	Soit la fonction $f$ donnée algébriquement par
		\[ f(x) = 3x^3 -x - 3, \]
	et considérons les points $P(0;-3), Q(1; 1), R(-1; 1),$ et $S(2; 19)$.
	On se demande si les points appartiennent à la courbe représentative de $f$ ou non.
	
	D'après la propriété fondamentale \ref{cor:prop-fond}, un point $(x;y)$ appartient à la courbe de $f$ si et seulement si l'équation $y=f(x)$ est vérifiée.
	On a donc les (non) appartenances suivantes.
	\begin{multicols}{2}
		\begin{enumerate}
			\item $f(0) = -3$, et donc $P \in \C_f$.
			\item $f(1) = -1 \neq 1$, donc $Q \not\in \C_f$.
			\item $f(-1) = -5 \neq 1$, donc $R \not\in\C_f$.
			\item $f(2) = 19$, et donc $S\in\C_f$.
		\end{enumerate}
		\vfill
		
		\begin{center}
		 \includegraphics[page=13, scale=1.1]{figures/fig-affines.pdf}
		\end{center}
	\end{multicols}
}{}


\exe{}{
	Considérons la fonction $f$ sur $\R$ donnée algébriquement par
		\[ f(x) = \dfrac17-x. \]
	Pour chaque point suivant, déterminer s'il appartient à $\C_f$ ou non.
	
	\begin{multicols}{3}
	\begin{enumerate}[label=\roman*)]
		\item $\left(0; \dfrac17\right)$
		\item $\left(\dfrac17 ; 0\right)$
		\item $\left(\dfrac27 ; \dfrac37\right)$
		\item $\left(-\dfrac{13}7 ; 2\right)$
		\item $\left(\dfrac67 ; 1\right)$
		\item $\left(\dfrac27 ; -\dfrac17\right)$
	\end{enumerate}
	\end{multicols}

}{exe:Cf}{
	TODO
}

\exe{}{
	Considérons deux fonctions $f, g$ sur $\D = ]-3 ; 3[$ données algébriquement par
		\begin{align*}
			f(x) = x^2 - 2x && g(x) = (x-1)^2
		\end{align*}
	
	\begin{enumerate}
		\item Esquisser les représentations graphiques de $f$ et de $g$ dans un même repère.
		\item Démontrer que $g(x) = f(x) + 1$ pour tout $x$ du domaine.
	\end{enumerate}
}{exe:id-rem-graph}{
	TODO
}

\exe{}{
	Esquisser la courbe de la fonction $f$ sur $\D=[-2; 4]$ donnée algébriquement par
		\[ f(x) = 3. \]
	Que dire de $f$ et de $\C_f$ ?
}{exe:graph-const}{
	TODO
}

\exe{}{
	Esquisser la courbe de la fonction $f$ sur $\D=[-5;3]$ donnée algébriquement par
		\[ f(x) = 1-x. \]
	Que dire $\C_f$ ?
}{exe:graph-droite}{
	TODO
}

\exe{}{
	Esquisser la courbe de la fonction $f$ sur $\D=[3;10]$ donnée algébriquement par
		\[ f(x) = \dfrac3x + 1. \]
}{exe:graph-droite2}{
	TODO
}


\exe{}{
	Considérons la représentation graphique suivante d'une fonction $f$ définie sur $\D = ]{-}3,4 ; 2,3[$.
	
	\begin{center}
	\includegraphics[page=14]{figures/fig-affines.pdf}
	\end{center}
	\begin{enumerate}
		\item Donner approximativement les images de $-1,5$ et de $-\dfrac{20}7$ par $f$.
		\item Énumérer approximativement les antécédents de $-2$ et de $2$ par $f$.
		\item Donner approximativement un réel qui admet exactement deux antécédents par $f$.
		\item Si $f$ était définie sur $\R$ tout entier, serait-il toujours possible de connaître l'image de $-2$ ? Et tous les antécédents de $-2$ ?
	\end{enumerate}
	Supposons désormais que $f(x) = 3-2\cdot x +\dfrac13 \cdot x^3$ pour tout $x\in\D$ du domaine.
	\begin{enumerate}
		\item[5.] Vérifier à la calculatrice les réponses aux deux premières questions.
		\item[6.] Montrer sans calculatrice que l'image par $f$ de $-3$ est $0$ et que l'image par $f$ de $0$ est $3$.
	\end{enumerate}
}{exe:deg3}{
	TODO
}
\exemulticols{}{
	Un fonction $f$ admet une représentation graphique ci-contre.
	Parmis les expressions algébriques suivantes, lequelles ne peuvent pas correspondre à $f(x)$ ?
		\begin{multicols}{2}
		\begin{enumerate}[label=\roman*)]
			\item $1-x$
			\item $\dfrac{-1-x}3$
			\item $\left(x+\dfrac13\right)^2$
			\item $-2x - \dfrac23$
		\end{enumerate}
		\end{multicols}
}{
	\begin{center}
	\includegraphics[page=15]{figures/fig-affines.pdf}
	\end{center}
}{exe:expr-from-graph}{
	TODO
}

\exe{}{
	Comparer les représentations graphiques des fonctions suivantes données algébriquement.
		\begin{align*}
			f(x) = x^2 && g(x) = x^2 - 3 && h(x) = (x+4)^2.
		\end{align*}
}{exe:y-shift}{
	TODO
}

\exe{, difficulty=2}{
	Donner une fonction réelle et un domaine telle qu'une de ses images admet
		\begin{multicols}{2}
		\begin{enumerate}
			\item Exactement un antécédent
			\item Exactement deux antécédents
			\item Exactement trois antécédents
			\item Une infinité d'antécédents
		\end{enumerate}	
		\end{multicols}
}{exe:nb-antécédents}{
	TODO
}

\exe{, difficulty=2}{
	Donner graphiquement une fonction sur $\R$ non constante telle que toutes les images de $f$ admettent un nombre infini d'antécédents.
}{exe:infinité-antécédents}{
	TODO
}

\newpage

\subsection{Résoudre graphiquement $f(x) = k$}


On cherche à résoudre graphiquement une équation du type $f(x)=k$ pour un $k\in\R$ donné, c'est-à-dire trouver l'ensemble des $x\in\R$ pour lesquels l'équation est vérifiée.
En reformulant, il s'agit de trouver l'ensemble des antécédents de $k$ par $f$. 

En reprenant la solution de l'exercice \ref{exe:deg3}, on peut généraliser la stratégie à employer.
Supposons qu'on souhaite résoudre $f(x) = 2$ où $x\in]{-}3,4 ; 2,3[$ et $\C_f$ est donnée graphiquement.
Comme chaque point de $\C_f$ est de la forme $\bigl(x; f(x)\bigr)$, on souhaite trouver les points de la forme $(x;2)$ pour ensuite lire leur abscisse et enfin obtenir les différents $x$ du domaine.
Pour trouver tous les points d'ordonnée $2$, on peut s'aider en créant une droite horizontale d'ordonnée $2$, comme ci-dessous.
	\begin{center}
	\includegraphics[page=16, scale=1.25]{figures/fig-affines.pdf}
	\end{center}
Les coordonnées des points $A,B,C$ trouvés en intersectant la droite horizontale avec $\C_f$ sont
	\begin{align*}
		A \approx (-2,67 ; 2) && B \approx (0,52 ; 2) && C \approx (2,14 ; 2)
	\end{align*}
L'ensemble de solutions de l'équation $f(x)=2$ est donc approximativement donné par 
	\[ \bigset{ -2,67 ; 0,52 ; 2,14 }. \]

\thm{}{
	Soit $f:\D \rightarrow \R$ une fonction réelle définie sur son domaine $\D$.
	Pour tout réel $k$, les solutions dans $\D$ de l'équation $f(x)=k$ sont les abscisses des points d'intersection de $\C_f$ et de la droite d'équation $y=k$.
}{}

\exemulticols{}{
	Donner l'ensemble des $x$ vérifiant $f(x) = -2$ à l'aide du graphe de $f$ ci-contre.
}{
	TODO graph
}{exe:fx=k}{
	TODO
}

\newpage

\section{Fonctions affines}\label{sec:aff-1}


\subsection{Lecture graphique et définitions}\label{sec:aff-1}

\dfn{}{
	Un \emphindex{fonction affine} est une fonction $f$ vérifiant pour tout $x\in\R$
		\begin{align}
			f(x) = ax + b, \label{eq:affine}
		\end{align}
	où $a$ et $b$ sont deux paramètres : le \emphindex{coefficient directeur} et l'\emphindex{ordonnée à l'origine}.
}{def:affine}

\nomen{
	Une \emphindex{fonction linéaire} est une fonction affine avec $b=0$.
}

\exe{, difficulty=1}{
	Montrer que deux fonctions affines sont égales si et seulement si les coefficients directeurs sont égaux et les ordonnées à l'origines sont égales.
}{exe:f=g-affines}{
	TODO
}

\mprop{}{
	La courbe représentative d'une fonction affine est une droite non verticale.
}{prop:Cf-affine}


\ex{}{
	Considérons la fonction $id$, dite \emphindex{fonction identité}, donnée par
		\[ id(x) = x \qquad \text{pour tout $x \in \R$.} \]
	Cette fonction est affine, car c'est un cas spécial de la forme \ref{eq:affine} où $a=1$ et $b=0$.
	La droite $(d) = \C_{id}$ vérifie donc
		\begin{align*}
			(x;y) \in (d) && \iff && y = x.
		\end{align*}
		
	\begin{multicols}{2}
	Un point appartient donc à $(d)$ si et seulement si sont abscisse est égale à son ordonnée.
	
	Ainsi, $(3;3) \in (d)$ et $\bigl(-\sqrt{2}; -\sqrt{2}\bigr) \in (d)$, mais $(3;4) \notin (d)$.
	La droite restreinte à \mbox{$\D = [-4 ; 4]$} est donc l'ensemble des points en bleu ci-contre.
	
	\vfill
	
	\begin{center}
	\includegraphics[page=4]{figures/fig-affines.pdf}
	\end{center}
	
	\end{multicols}
}{}

%\ex{}{
%	Considérons la fonction $f$ donnée par
%		\[ f(x) = x+6 \qquad \text{pour tout $x \in \R$.} \]
%	Cette fonction est affine, car c'est un cas spécial de la forme \ref{eq:affine} où $a=1$ et $b=6$.
%	La droite $(d) = \C_{f}$ vérifie donc
%		\begin{align*}
%			(x;y) \in (d) && \iff && y = x+6.
%		\end{align*}
%	Ainsi, $(3;9) \in (d)$ et $\bigl(-\sqrt{2}; -\sqrt{2}+6\bigr) \in (d)$, mais $(6;0) \notin (d)$.
%	La droite est donc l'ensemble des points en bleu ci-dessous.
%	
%		\begin{center}
%	\includegraphics[page=5]{figures/fig-affines.pdf}
%	\end{center}
%}{ex:droite}
%
%\ex{}{
%	Considérons la fonction $f$ donnée par
%		\[ f(x) = 2,2 \qquad \text{pour tout $x \in \R$.} \]
%	Cette fonction est affine, car c'est un cas spécial de la forme \ref{eq:affine} où $a=0$ et $b=2,2$.
%	La droite $(d) = \C_{f}$ vérifie donc
%		\begin{align*}
%			(x;y) \in (d) && \iff && y = 2,2.
%		\end{align*}
%	Ainsi, $(3;2,2) \in (d)$ et $\bigl(-\sqrt{2}; 2,2\bigr) \in (d)$, mais $(0;0) \notin (d)$.
%	La droite est donc l'ensemble des points en bleu ci-dessous.
%	
%	\begin{center}
%	\includegraphics[page=6]{figures/fig-affines.pdf}
%	\end{center}
%}{}
%
%\ex{}{
%	Considérons la fonction $f$ donnée par
%		\[ f(x) = -2x + 1 \qquad \text{pour tout $x \in \R$.} \]
%	Cette fonction est affine, car c'est un cas spécial de la forme \ref{eq:affine} où $a=-2$ et $b=1$.
%	La droite $(d) = \C_{f}$ vérifie donc
%		\begin{align*}
%			(x;y) \in (d) && \iff && y = -2x+1.
%		\end{align*}
%	Ainsi, $(-1;3) \in (d)$ et $(-\sqrt{2}; 2\sqrt{2}+1) \in (d)$, mais $(1;1) \notin (d)$.
%	La droite est donc l'ensemble des points en bleu ci-dessous.
%	
%	\begin{center}
%	\includegraphics[page=7]{figures/fig-affines.pdf}
%	\end{center}
%}{}

\nt{
	On peut réécrire la courbe représentative de $f$ de la façon suivante.
		\[ \C_f = \bigset{ \bigl(x ; f(x) \bigr) \text{ où $x$ parcourt $\R$} }
		= \bigset{ (x ; y ) \text{ où $x$ parcourt $\R$ et où $y = f(x)$} } \]
}{}

% à mettre plus haut non ?
\dfn{équation de droite}{
	Pour une fonction affine $f(x) = ax+b$, on parle de $\C_f$ comme la « \emphindex{droite d'équation $y=ax+b$} ».
}{dfn:eq-droite}

\notations{
	Pour se défaire de la fonction $f$, on peut noter
		\[ (d) : y = ax+b. \]
}

\exe{1}{
	Donner deux points distincts appartenant à la droite d'équation $y = 2x-1$.
	Représenter la droite dans un repère.
}{exe:droite}{
	TODO
}


\nomen{
	On dit d'un point $(x, y)$ du plan qu'il est à \emphindex{coordonnées entières} dès que $x, y \in \Z$ sont deux entiers relatifs.
}

\exe{, difficulty=1}{
	On considère la fonction affine $f(x) = \sqrt2 x$.
	Donner tous les points $(x,y)$ à coordonnées entières de $\C_f$.
}{exe:sqrt2x-integer}{
	Tout d'abord, $(x,y) \in \C_f \iff y = \sqrt2 x$.
	On distingue donc deux cas : si $x=0$, on trouve le couple $(0 ; 0) \in \C_f$ à coordonnées entières.
	Sinon, $\sqrt2 = \dfrac{y}x$ pour deux entiers relatifs $x, y\in\Z$.
	Comme $\sqrt2$ est irrationnel, aucun couple ne peut vérifier cette égalité.
	
	Le seul couple est donc $(0 ; 0)$.
}

\exe{, difficulty=1}{
	On considère la fonction affine $f(x) = \frac12 x$.
	Donner tous les points $(x,y)$ à coordonnées entières de $\C_f$.
}{exe:12x-integer}{
	Tout d'abord, $(x,y) \in \C_f \iff y = \dfrac12  x \iff x = 2y$.
	On retrouve là la définition d'un nombre pair : $x$ est multiple de 2.
	
	L'ensemble recherché est donc $\bigset{ (2k, k) \text{ où } k\in\Z}$.
}

\exe{}{
	On considère la fonction affine $f(x) = \frac13 -\frac23 x$.
	Montrer que $(x,y) \in \C_f$ si et seulement si $2x + 3y = 1$.
	Montrer ensuite que tous les points de la forme $(x, y) = (-3k -1 ; 2k+1)$ où $k\in\Z$ appartiennent à $\C_f$.
}{exe:2x+3y=1-integer}{
	$(x,y) \in \C_f \iff y = f(x) = \dfrac13 -\dfrac23 x \iff 3y = 1 - 2x \iff 2x + 3y = 1$, comme requis.
	
	Pour $x=-3k-1$ et $y=2k+1$, on a bien
		\[ 2x+3y = 2(-3k-1) + 3(2k+1) = -2 + 3 = 1. \]
}

\nt{
	L'équation $2x  - 3y= 3$ étant équivalente à $y = \frac23 x - 1$, c'est aussi une équation de droite.
	La première forme est appelée \emphindex{équation cartésienne}, et la deuxième \emphindex{équation réduite}.
}

\exe{}{
	On considère l'ensemble des points $(x,y)$ vérifiants $5x - 9y = 1$.
	Exprimer cet ensemble comme la courbe représentative d'une fonction affine dont on explicitera l'expression algébrique.
	Donner un point de la droite à coordonnées entières.
}{exe:isoler-y}{
	$5x - 9y = 1 \iff 5x = 1 + 9y \iff 9y = 5x -1 \iff y = \dfrac59x - \dfrac19$.
	L'ensemble des points vérifiants cette équation est donc la courbe de $f(x) = \dfrac59 x - \dfrac19$.
	
	Soit $(x, y)$ à coordonnées entières. Alors $9y = 5x -1$.
	Ainsi $9y$ doit se terminer en $9$ ou en $4$. C'est le cas de $9$ assez immédiatement ($y=1$ et donc $x=2$), mais aussi de $54$ ($y = 6$ et donc $x=11$), par exemple.
	On a trouvé deux points à coordonnées entières : $(2 ; 1)$ et $(11 ; 6)$.
}

\exe{, difficulty=2}{
	Montrer que si $(x_0,y_0)$ est solution de $5x-9y=1$, alors $(x_0+9, y_0+5)$ aussi.
	En déduire qu'il existe une infinité de points à coordonnée entière $(x,y)$ vérifiant $5x-9y=1$, et en donner 3 distincts.
}{exe:dioph59}{
	$5(x_0+9) - 9(y_0 + 5) = 5x_0 - 9y_0 + 5\times9 - 9\times5 = 5x_0 - 9y_0 = 1$.
	La solution entière $(x_0, y_0) = (2 ; 1)$ trouvée à l'exercice \ref{exe:isoler-y} permet de conclure, car ajouter 9 et ajouter 5 ne change pas le caractère entier d'un nombre.
	
	Du couple $(2 ; 1)$ on déduit le couple $(2+9 ; 1+5) = (11 ; 6)$, puis $(11+9 ; 6 + 5 ) = (20 ; 11)$.
	On vérifiera que $20 \times 5 - 11 \times 9 = 100 - 99 = 1$ pour se rassurer.
}

\subsection{Ordonnée à l'origine $b$}

\lem{ordonnée à l'origine}{
	Soit $f(x) = a x + b$ affine.
	Alors
		\[ f(0) = b, \]
	et donc
		\[ (0 ; b) \in \C_f. \]
}{lem:affine-b}

\begin{multicols}{2}
\ex{}{
	Considérons la droite $(d)$ ci-contre, d'équation $y=ax+b$.
	
	Le point de $(d)$ d'abscisse nulle admet pour coordonnées $(0 ; 2)$.
	Ainsi
		\begin{align*}
			(0 ; 2) \in (d) && \iff && 2 = a\cdot0 + b,
		\end{align*}
	Et on en déduit que le paramètre $b$ vaut $2$.
	\vfill
}{}
	\begin{center}
	\includegraphics[page=8, scale=.9]{figures/fig-affines.pdf}
	\end{center}
	
\end{multicols}

\ex{}{
	Considérons une droite $(d)$ qui contient les points
		\begin{align*}
			(0;3) \in (d) && \text{ et } && (1;8) \in (d).
		\end{align*}
	On souhaite trouver la fonction affine $f$ telle que $(d)$ soit la courbe représentative de $f$.
	La propriété fondamentale \ref{cor:prop-fond} donne donc les deux équations suivantes.
		\begin{align*}
			f(0) = 3, && \text{ et } && f(1) = 8.
		\end{align*}
	Comme $f$ est affine, on sait qu'elle s'écrit $f(x) = ax+b$ pour tout $x\in\R$ et pour certains paramètres $a,b\in\R$.
	On réécrit les équations 
		\begin{align*}
			a \times 0 + b = 3 && \text{ et } && a\times1 + b = 8, \\
			b = 3 && \text{ et } && a + b = 8.
		\end{align*}
	Par suite, $b=3$, et $a= 8-b = 8-3 = 5$. Par conséquent,
		\[ f(x) = 5x + 3 \qquad \text{ pour tout } x\in\R. \]
}{}

\subsection{Coefficient directeur $a$}

\lem{coefficient directeur}{
	Soit $f(x) = a x + b$ affine.
	Alors, pour tout $x\in\R$ réel,
		\[ f(x+1) - f(x) = a. \]
	Lorsqu'on augmente l'abscisse $x$ de 1, l'ordonnée $y$ augmente de $a$.
}{lem:coeff-dir}

%\pf{Preuve du lemme \ref{lem:coeff-dir}}{
\pf{}{
	Pour calculer $f(x+1)$, posons la variable intermédiaire $t = x+1$.
	On a alors
		\[ f(x+1) = f(t) = a\cdot t + b = a\cdot (x+1) + b = a \cdot x + a + b. \]
	En soustrayant $f(x) = ax+b$, on trouve bien le résultat recherché.
		\begin{align*}
			f(x+1) - f(x) &= ax + a + b - (ax + b) \\ &= ax + a + b -ax -b \\ &= a
		\end{align*}
}

\cor{}{
	On a, en particulier,
		$ a = f(1) - f(0).$
}{cor:affine-a}

\exe{}{
	Montrer le corollaire \ref{cor:affine-a}.
}{exe:affine-a}{
	TODO
}

\exe{}{
	Pour chaque fonction affine sur $\R$ suivante, déterminer son coefficient directeur $a$ et son ordonnée à l'origine $b$ à l'aide du lemme \ref{lem:affine-b} et du corollaire \ref{cor:affine-a}.
	\begin{multicols}{2}
	\begin{enumerate}
		\item $f(x) = 2x + 1$
		\item $f(x) = 1 + 2x$
		\item $f(x) = - x$
		\item $f(x) = -42$
		\item $f(x) = 10x + 2$
		\item $f(x) = 2 + 10x$
		\item $f(x) = 1 - x$
		\item $f(x) = 0$
	\end{enumerate}
	\end{multicols}
}{exe:paramètres-affines}{
	TODO
}

\begin{multicols}{2}
		
\ex{}{
	On considère la droite ci-contre, courbe représentative $\C_f$ d'une certaine fonction affine $f(x)=ax+b$.
	
	On regarde deux points de $\C_f$ dont les abscisses sont espacées de $1$ et on regarde comme leur ordonnée évolue.
	Par exemple, $(1;0)$ et $(2,-2)$. 
	Le lemme \ref{lem:coeff-dir} implique que, en prenant $x=1$,
		\begin{align*}
			f(2) - f(1) = a.
		\end{align*}
	On en déduit que 
		\[ a = (-2) - (0) = -2. \]
	En effet, on a augmenté $x$ de $1$, et l'image a augmenté de $-2$ (ou diminué de $2$).
	
	Remarquons qu'on aurait pû prendre d'autres points, par exemple $(0;2)$ et $(1;0)$, ou $(2;-2)$ et $(3;-4)$.
	La différence des ordonnées vaut bien $-2$, le coefficient directeur.	
}{}
	\vfill\null
	\begin{center}
	\includegraphics[page=9]{figures/fig-affines.pdf}
	\end{center}
	\vfill\null
\end{multicols}

\exe{}{
	Vérifier que les points $(1;2)$ et $(4;-4)$ appartiennent bien à la droite d'équation $y = -2x+4$.
}{exe:point-sur-droite}{
	TODO
}

\exe{}{
	Décrire entièrement la droite $(d)$ contenant les points $(3;-4)$ et $(-2;14)$.
}{exe:interpolation-affine}{
	TODO
}


\nt{
	Du lemme \ref{lem:coeff-dir}, on déduit que lorsqu'on augmente l'abscisse $x$ de 1, l'ordonnée $y$ augmente de $a$.
	
	Il suit naturellement que, lorsqu'on augmente $x$ de 2, alors $y$ augmente de $a+a = 2a$.
	De plus, lorsqu'on augmente $x$ de $-1$, alors $y$ augmente de $-a$.
	
	Plus généralement, on montre facilement que $f(x+c) - f(x) = ca$ pour n'importe que $c\in\R$.
	Si on note $x' = x+c$ on a donc, en supposant $c \neq 0$ non nul,
		\[ a = \dfrac{f(x+c)-f(x)}{c} = \dfrac{f(x') - f(x)}{x' - x}. \]
	Ceci démontre la première partie du théorème \ref{thm:param-affine}.
}

\ex{}{
	Soient $(1;2)$ et $(4;-4)$ deux points du plan qui appartiennent à une droite d'équation $y=ax+b$.
	Le but est de connaître $a$ et $b$ afin de connaître entièrement la droite passant par ces deux points.
	
	En augmentant l'abscisse de 3 unités, l'ordonnée augmente de $-6$.
	D'après la remarque ci-dessus, $a=\frac{-6}3 = -2$.
	
	L'équation de la droite est donc $y=-2x + b$.
	Comme $(1 ; 2)$ lui appartient, la propriété fondamentale donne $2 = -2\cdot1 + b = -2 + b$.
	On conclut que $b=4$.
}{ex:systeme-affine}

\exe{}{
	Montrer qu'une droite ne peut pas contenir les trois points $(0;1), (1;0),$ et $(2;1)$ simultanément.
}{exe:alignés}{
	TODO
}

\exe{,difficulty=2}{
	Décrire l'ensemble $\C_f$ contenant les points $(0;1), (1;0),$ et $(2;1)$ où $f(x) = ax^2 + bx + c$.
	Grapher la parabole, par exemple en écrivant \texttt{y=x² -2x + 1} sur GeoGebra.\footnote{\href{https://www.geogebra.org/calculator}{https://www.geogebra.org/calculator}}
	
}{exe:interpolation-quadratique}{
	TODO
}

\thm{interpolation linéaire}{
	Si deux points $(x_A, y_A)$ et $(x_B, y_B)$ vérifient $x_A\neq x_B$ et appartiennent à la droite d'équation $y=ax+b$, alors
		\begin{align*}
			a = \dfrac{y_A - y_B}{x_A - x_B} && \text{et} && b = \dfrac{x_A y_B - x_B y_A}{x_A-x_B}. 
		\end{align*}
}{thm:param-affine}

\nt{
	Un moyen mnémotechnique pour retenir la formule de $a$ est
		\[ a = \dfrac{\text{différence des $y$}}{\text{différence des $x$}} = \dfrac{y-y'}{x-x'} = \dfrac{dy}{dx}. \]
	L'ordre des différences ($A$ moins $B$ ou $B$ moins $A$) doit être le même au numérateur et au dénominateur mais, même s'il ne l'est pas, on aura seulement commis une erreur de signe.
	Il est donc possible d'ignorer l'ordre des points et de décider du signe après grâce au caractère croissant ou décroissant de la fonction.
	
	La formule pour $b$ n'a pas a être retenue car, une fois $a$ connu, une appartenance d'un point suffit à donner une équation pour trouver $b$.
}

\begin{multicols}{2}
\ex{}{
	On considère la droite \mbox{$y=ax+b$} ci-contre.
	Le théorème \ref{thm:param-affine} affirme que
		\begin{align*}
			a = \dfrac{1,5 - (-3,5)}{0,5 - (-0,5)}  = \dfrac{5}{1} = 5. 
		\end{align*}
	On aurait aussi pû appliquer le lemme \ref{lem:coeff-dir} : lorsque $x$ augmente de 1 en passant de $-0,5$ à $+0,5$, l'ordonnée augmente de $5$ en passant de $-1,5$ à $3,5$.
}{}

	\begin{center}
	\includegraphics[page=10]{figures/fig-affines.pdf}
	\end{center}
\end{multicols}

\exe{}{
	Calculer $a$ et $b$ de l'exemple \ref{ex:systeme-affine} à l'aide du théorème \ref{thm:param-affine} et comparer avec les valeurs obtenues.
}{exe:a-b-affine}{
	TODO
}

\exe{}{
	Donner l'équation de la droite contenant les points $(-1;-10)$ et $(1;30)$.
	Le point $(3 ;  60)$ appartient-il à la droite ?
}{exe:a-b-affine2}{
	TODO
}

\exe{, difficulty=1}{
	Les points $(-3 ; -6)$, $(1 ;1)$, et $(3 ; 4)$ sont-ils alignés ?
}{exe:affine-alignement}{
	TODO
}

\exe{, difficulty=2}{
	Considérons une fonction quadratique 
		\[ f(x) = ax^2 + bx + c, \]
	où $a, b, c\in\R$ sont trois paramètres réels.
	Supposons de surcroît qu'on connaisse deux zéros distincts de $f$, c'est-à-dire qu'on connaisse $\alpha, \beta\in\R$ tels que $\alpha\neq\beta$ et
		\[ f(\alpha) = f(\beta) = 0. \]
	\begin{enumerate}
		\item Montrer que la fonction $g$ donnée par
			\[ g(x) = f(x) - a (x-\alpha)(x-\beta) \qquad \text{ pour tout } x\in\R \]
		est affine.
		\item Montrer que $g$ admet deux zéros distincts.
		\item En déduire, par interpolation linéaire, que $g$ est constamment nulle et donc que
			\[ f(x) = a (x-\alpha)(x-\beta)  \qquad \text{ pour tout } x\in\R.  \]
	\end{enumerate}
}{exe:thm-fond-alg-2}{
	TODO
}



\section{Variations et signe du coefficient directeur}


On distingue trois cas différents de droites parmis les exemples de la section \ref{sec:aff-1} : les deux premières sont croissantes, la suivante est constante, et la dernière est décroissante.

\dfn{variations}{
	Soit $f : \D \rightarrow \R$ une fonction $f$ quelconque sur un intervalle $I\subseteq\R$.
	Alors
		\begin{itemize}
			\item On dit que $f$ est \emphindex{strictement croissante} si, pour tous les $x,y\in I$ du domaine,
				\begin{align*}
					x < y && \iff && f(x) < f(y).
				\end{align*}	
			On interprète l'implication ainsi :
			\begin{center}
				\og lorsqu'on augmente l'abscisse $x$, l'ordonnée $f(x)$ augmente \fg.
			\end{center}
				
			\item On dit que $f$ est \emphindex{strictement décroissante} si, pour tous les $x,y\in I$ du domaine,
				\begin{align*}
					x < y && \iff && f(x) > f(y).
				\end{align*}
			On interprète l'implication ainsi :
			\begin{center}
				\og lorsqu'on augmente l'abscisse $x$, l'ordonnée $f(x)$ diminue \fg.
			\end{center}
				
			\item On dit que $f$ est \emphindex{constante} si, pour tous les $x\in I$ du domaine, et pour une certaine constante $K\in\R$,
				\begin{align*}
					f(x) = K.
				\end{align*}
		\end{itemize}
}{}

\thm{variations}{
	Soit $f$ une fonction affine où $a, b \in\R$ sont ses deux paramètres réels.
		\begin{align*}
			f(x) = a x + b && (x\in\R)
		\end{align*}
	On distingue trois cas de figure.
		\begin{itemize}
			\item Si $a < 0$, alors $f$ est \emphindex{strictement décroissante}.
			\item Si $a=0$, alors $f$ est \emphindex{constante}.
			\item Si $a>0$, alors $f$ est \emphindex{strictement croissante}.
		\end{itemize}
}{thm:affine-var}

%\pf{Démonstration du théorème \ref{thm:affine-var}}{
%	Si $a=0$, $f$ est clairement constante car
%		\begin{align*}
%			f(x) = b \qquad \text{ pour tout $x\in\R$}.
%		\end{align*}
%	
%	Sinon, on utilise les règles de manipulation des inégalités du théorème \ref{thm:ineg} vues au chapitre \ref{chap:3}.
%	
%	Si $a>0$, alors, pour tous les $x,y \in \R$ tels que $x < y$, on a
%		\begin{align*}
%			x &< y \\
%			a \cdot x &< a \cdot y \\
%			a \cdot x + b &< a \cdot y + b \\
%			f(x) &< f(y),
%		\end{align*}
%	et $f$ est donc strictement croissante.
%	
%	Si $a<0$, alors, pour tous les $x,y \in \R$ tels que $x < y$, on a
%		\begin{align*}
%			x &< y \\
%			a \cdot x &> a \cdot y \\
%			a \cdot x + b &> a \cdot y + b \\
%			f(x) &> f(y),
%		\end{align*}
%	et $f$ est donc strictement décroissante.
%}{}

