\documentclass[a4paper, 12pt]{extarticle}

\usepackage[utf8x]{inputenc}
%fonts
\usepackage{libertinus,libertinust1math}
\usepackage{amsmath,amsthm,amssymb,mathtools}

% SOLUTION SWITCH

\ifsolutions
	\newcommand{\exe}[2]{
		\begin{ex} #1  \end{ex}
		\begin{sol} #2 \end{sol}
	}
\else
	\newcommand{\exe}[2]{
		\begin{ex} #1  \end{ex}
	}
	
\fi


\usepackage[french]{babel}
\usepackage[
a4paper,
margin=2cm,
nomarginpar,% We don't want any margin paragraphs
]{geometry}

% HEADER, ARRAY, ENUM, MULTIOCL
\usepackage{fancyhdr}
\usepackage{array}
\usepackage{multicol, enumitem}
\newcolumntype{P}[1]{>{\centering\arraybackslash}p{#1}}
\usepackage{stackengine}
\newcommand\xrowht[2][0]{\addstackgap[.5\dimexpr#2\relax]{\vphantom{#1}}}

% theorems

\theoremstyle{theorem}
\newtheorem{thm}{Théorème}
\theoremstyle{plain}
\newtheorem*{sol}{Solution}
\theoremstyle{definition}
\newtheorem{ex}{Exercice}
\newtheorem{dfn}{Définition}
\newtheorem*{dfn*}{Définition}


%couleurs
\usepackage{tcolorbox}
\definecolor{myg}{RGB}{56, 140, 70}
\definecolor{myb}{RGB}{45, 111, 177}
\definecolor{myr}{RGB}{199, 68, 64}
\definecolor{mygr}{HTML}{2C3338}


\tcbuselibrary{theorems,skins,hooks}
\newcounter{commonbox}
\makeatletter
\newtcbtheorem[use counter=commonbox]{theorem}{Théorème }%
{
	enhanced,
	colback=white,
	colframe=mygr,
	attach boxed title to top left={yshift*=-\tcboxedtitleheight},
	fonttitle=\bfseries,
	title={#2},
	boxed title size=title,
	boxed title style={%
			sharp corners,
			rounded corners=northwest,
			colback=tcbcolframe,
			boxrule=0pt,
		},
	underlay boxed title={%
			\path[fill=tcbcolframe] (title.south west)--(title.south east)
			to[out=0, in=180] ([xshift=5mm]title.east)--
			(title.center-|frame.east)
			[rounded corners=\kvtcb@arc] |-
			(frame.north) -| cycle;
		},
	#1
}{th}
\newtcbtheorem[use counter=commonbox]{rappel}{Rappel }%
{
	enhanced,
	colback=white,
	colframe=mygr,
	attach boxed title to top left={yshift*=-\tcboxedtitleheight},
	fonttitle=\bfseries,
	title={#2},
	boxed title size=title,
	boxed title style={%
			sharp corners,
			rounded corners=northwest,
			colback=tcbcolframe,
			boxrule=0pt,
		},
	underlay boxed title={%
			\path[fill=tcbcolframe] (title.south west)--(title.south east)
			to[out=0, in=180] ([xshift=5mm]title.east)--
			(title.center-|frame.east)
			[rounded corners=\kvtcb@arc] |-
			(frame.north) -| cycle;
		},
	#1
}{th}
\newtcbtheorem[use counter=commonbox]{strategie}{Stratégie }%
{
	enhanced,
	colback=white,
	colframe=mygr,
	attach boxed title to top left={yshift*=-\tcboxedtitleheight},
	fonttitle=\bfseries,
	title={#2},
	boxed title size=title,
	boxed title style={%
			sharp corners,
			rounded corners=northwest,
			colback=tcbcolframe,
			boxrule=0pt,
		},
	underlay boxed title={%
			\path[fill=tcbcolframe] (title.south west)--(title.south east)
			to[out=0, in=180] ([xshift=5mm]title.east)--
			(title.center-|frame.east)
			[rounded corners=\kvtcb@arc] |-
			(frame.north) -| cycle;
		},
	#1
}{th}
\newtcbtheorem[use counter=commonbox]{outil}{Outil }%
{
	enhanced,
	colback=white,
	colframe=mygr,
	attach boxed title to top left={yshift*=-\tcboxedtitleheight},
	fonttitle=\bfseries,
	title={#2},
	boxed title size=title,
	boxed title style={%
			sharp corners,
			rounded corners=northwest,
			colback=tcbcolframe,
			boxrule=0pt,
		},
	underlay boxed title={%
			\path[fill=tcbcolframe] (title.south west)--(title.south east)
			to[out=0, in=180] ([xshift=5mm]title.east)--
			(title.center-|frame.east)
			[rounded corners=\kvtcb@arc] |-
			(frame.north) -| cycle;
		},
	#1
}{th}
\newtcbtheorem[use counter=commonbox]{but}{Buts du chapitre }%
{
	enhanced,
	colback=white,
	colframe=mygr,
	attach boxed title to top left={yshift*=-\tcboxedtitleheight},
	fonttitle=\bfseries,
	title={#2},
	boxed title size=title,
	boxed title style={%
			sharp corners,
			rounded corners=northwest,
			colback=tcbcolframe,
			boxrule=0pt,
		},
	underlay boxed title={%
			\path[fill=tcbcolframe] (title.south west)--(title.south east)
			to[out=0, in=180] ([xshift=5mm]title.east)--
			(title.center-|frame.east)
			[rounded corners=\kvtcb@arc] |-
			(frame.north) -| cycle;
		},
	#1
}{th}
\newtcbtheorem[use counter=commonbox]{propriete}{Propriété }%
{
	enhanced,
	colback=white,
	colframe=mygr,
	attach boxed title to top left={yshift*=-\tcboxedtitleheight},
	fonttitle=\bfseries,
	title={#2},
	boxed title size=title,
	boxed title style={%
			sharp corners,
			rounded corners=northwest,
			colback=tcbcolframe,
			boxrule=0pt,
		},
	underlay boxed title={%
			\path[fill=tcbcolframe] (title.south west)--(title.south east)
			to[out=0, in=180] ([xshift=5mm]title.east)--
			(title.center-|frame.east)
			[rounded corners=\kvtcb@arc] |-
			(frame.north) -| cycle;
		},
	#1
}{th}
\newtcbtheorem[number within=commonbox]{definition}{Définition }%
{
	enhanced,
	colback=white,
	colframe=mygr,
	attach boxed title to top left={yshift*=-\tcboxedtitleheight},
	fonttitle=\bfseries,
	title={#2},
	boxed title size=title,
	boxed title style={%
			sharp corners,
			rounded corners=northwest,
			colback=tcbcolframe,
			boxrule=0pt,
		},
	underlay boxed title={%
			\path[fill=tcbcolframe] (title.south west)--(title.south east)
			to[out=0, in=180] ([xshift=5mm]title.east)--
			(title.center-|frame.east)
			[rounded corners=\kvtcb@arc] |-
			(frame.north) -| cycle;
		},
	#1
}{th}
\newtcbtheorem[number within=commonbox]{exemples}{Exemples }%
{
	enhanced,
	colback=white,
	colframe=mygr,
	attach boxed title to top left={yshift*=-\tcboxedtitleheight},
	fonttitle=\bfseries,
	title={#2},
	boxed title size=title,
	boxed title style={%
			sharp corners,
			rounded corners=northwest,
			colback=tcbcolframe,
			boxrule=0pt,
		},
	underlay boxed title={%
			\path[fill=tcbcolframe] (title.south west)--(title.south east)
			to[out=0, in=180] ([xshift=5mm]title.east)--
			(title.center-|frame.east)
			[rounded corners=\kvtcb@arc] |-
			(frame.north) -| cycle;
		},
	#1
}{th}
\newtcbtheorem[number within=commonbox]{exemple}{Exemple }%
{
	enhanced,
	colback=white,
	colframe=mygr,
	attach boxed title to top left={yshift*=-\tcboxedtitleheight},
	fonttitle=\bfseries,
	title={#2},
	boxed title size=title,
	boxed title style={%
			sharp corners,
			rounded corners=northwest,
			colback=tcbcolframe,
			boxrule=0pt,
		},
	underlay boxed title={%
			\path[fill=tcbcolframe] (title.south west)--(title.south east)
			to[out=0, in=180] ([xshift=5mm]title.east)--
			(title.center-|frame.east)
			[rounded corners=\kvtcb@arc] |-
			(frame.north) -| cycle;
		},
	#1
}{th}
\newtcbtheorem[number within=commonbox]{questions}{Questions guidantes }%
{
	enhanced,
	colback=white,
	colframe=mygr,
	attach boxed title to top left={yshift*=-\tcboxedtitleheight},
	fonttitle=\bfseries,
	title={#2},
	boxed title size=title,
	boxed title style={%
			sharp corners,
			rounded corners=northwest,
			colback=tcbcolframe,
			boxrule=0pt,
		},
	underlay boxed title={%
			\path[fill=tcbcolframe] (title.south west)--(title.south east)
			to[out=0, in=180] ([xshift=5mm]title.east)--
			(title.center-|frame.east)
			[rounded corners=\kvtcb@arc] |-
			(frame.north) -| cycle;
		},
	#1
}{th}
\makeatother

% corps
\newcommand{\R}{\mathbb{R}}
\newcommand{\Rnn}{\mathbb{R}^{2n}}
\newcommand{\Z}{\mathbb{Z}}
\newcommand{\N}{\mathbb{N}}
\newcommand{\Q}{\mathbb{Q}}

% domain
\newcommand{\D}{\mathcal{D}}
% for calligraphic C
\usepackage{calrsfs}
\newcommand{\C}{\mathcal{C}}

% date
\usepackage{advdate}

% ensembles tq. 
\newcommand{\xRtq}[1]{
	$\left\{ x \in \R \text{ tq. } #1 \right\}$
}

% vabs
\newcommand{\vabs}[1]{
	\left| #1 \right|
}

%pinfty minfty
\newcommand{\pinfty}{{+}\infty}
\newcommand{\minfty}{{-}\infty}

% plots
\usepackage{pgfplots}

%virgules
\usepackage{icomma}
\pgfplotsset{/pgf/number format/use comma}

%subfigures
\usepackage{subcaption}

%hyperlink footnote
\usepackage{hyperref}

%wider tabulars
\def\arraystretch{2}
\setlength\tabcolsep{15pt}

% tableaux var, signe
\usepackage{tkz-tab}

\SetDate[17/12/2025]

\begin{document}
\pagestyle{fancy}
\fancyhead[L]{Première spécifique}
\fancyhead[C]{\textbf{Révisions chapitres 1-5}}
\fancyhead[R]{\today}



\exe{}{
	Sans calculatrice, exprimer les nombres suivants sous la forme $q^n$, où $q \in \N$ et $n\in\Z$ sont des entiers.
	\begin{multicols}{3}
	\begin{enumerate}[label=\roman*)]
		\item $10^3 \times 10^5$
		\item $\left(4^5\right)^2$
		\item $\dfrac{5^3}{5^3}$
		\item $1$
		\item $\dfrac{2^4}{2^7}$
		\item $\left(2^{-1}\right)^3$
		\item $\left(2^{3}\right)^{-1}$
		\item $\left(\dfrac{1}{7^{2,5}}\right)^6$
		\item $\dfrac{10^{12}}{10^{-12}}$
		\item $\dfrac{10^{-5,2}}{10^{6,8}}$
	\end{enumerate}
	\end{multicols}
}{exe:puissances}{
	\begin{multicols}{3}
	\begin{enumerate}[label=\roman*)]
		\item $10^8$
		\item $4^{10}$
		\item $5^0 = 1^1 = 4^0 = q^0$ pour n'importe quel $q\neq0$
		\item $1^1 = 2^0 = 100^0 = q^0$ pour n'importe quel $q\neq0$
		\item $2^{-3}$
		\item $2^{-3}$
		\item $2^{-3}$
		\item $7^{-15}$
		\item $10^{24}$
		\item $10^{-12}$
	\end{enumerate}
	\end{multicols}
}

\exe{}{
	Tracer les courbes représentatives des fonctions $f(x) = x + 2$ et $g(x) = 5-x$ pour $x \in [-2 ; 3]$.
	
	En déduire la solution de $f(x) = g(x)$ graphiquement puis résoudre $x+2 = 5-x$ et comparer.
}{exe:cfcg}{
	todo
}

\exe{}{
	Soit $u$ une suite arithmétique de raison $-2$ et de terme initial $u(0) = 10$.
	Calculer les premiers termes de $u$. À partir de quel rang $u$ est-elle égale à 2 ?
	
	Exprimer $u(n)$ en fonction de $n$ et trouver $n$ vérifiant $u(n) = 2$. Comparer avec le résultat obtenu précédemment.
}{exe:uartihm}{
	todo
}

\exe{}{
	Soit $v$ une suite géométrique de raison $1,4$ et de terme initial $v(0) = 12$.
	Exprimer $v(n)$ en fonction de $n$.
	
	Calculer les premiers termes de $v$ et placer le nuage de points dans un repère.
}{exe:vgeom-rep}{
	todo
}


\exe{}{
	Le nombre de nuitées de touristes étrangers au land de Thuringe (Allemagne) au cours des années $2010$ à $2013$ sont donnés ci-dessous. \footnote{Source : \href{https://statistik.thueringen.de/analysen/Aufsatz-06a-2021.pdf}{https://statistik.thueringen.de/analysen/Aufsatz-06a-2021.pdf}, Abbildung 16, page 47.}
	
	\begin{center}
	\vspace{.5cm}
	\begin{tikzpicture}[scale=.8]
		% nodes
		\draw (0,0) ellipse (2cm and .5cm) node {567 826};
		
		\draw (5,0) ellipse (2cm and .5cm) node {565 645};
		
		\draw (10,0) ellipse (2cm and .5cm) node {593 444 };
		
		\draw (15,0) ellipse (2cm and .5cm) node {563 236};
		
		% vertices
		\draw[->, thick, myg] (1cm,.6cm) arc (105:75:7);
		\draw[->, thick, myg] (6cm,.6cm) arc (105:75:7);
		\draw[->, thick, myg] (11cm,.6cm) arc (105:75:7);
		
		\draw[->, thick, myr] (1cm,-.5cm) arc (-105:-75:25);
	\end{tikzpicture}
	\vspace{.5cm}
	\end{center}

	\begin{enumerate}
		\item Compléter le schéma en ajoutant les coefficients multiplicateurs.
		\item Calculer le coefficient multiplicateur moyen.
		\item En déduire le taux d'évolution annuel moyen.
	\end{enumerate}

}{exe:3}{
	todo
}


\exe{, difficulty=1}{
	Le \emph{nombre de reproduction de base} d'un virus est le nombre moyen de personnes saines qu'un personne contagieuse infecte. Il permet de décrire l'évolution d'une maladie à ses débuts (avant que l'immunité collective fasse effet).
	
	On estime que le nombre de reproduction de base d'un virus est de $3$ : chaque semaine, le nombre de malades est multiplié par $3$.
	On compte, au temps $0$, $M(0) = 18\ 347$ malades.

	\begin{enumerate}
		\item Écrire $M(n)$, le nombre de malades après $n$ semaines, où $n\in\N$ est un entier naturel.
		\item Écrire $M(t)$, le nombre de malades au temps $t$ (en semaines). $t$ peut être négatif.
		\item Combien de malades y aura-t-il après $2,5$ semaines ? Arrondir à $10^{-1}$.
		\item Combien de malades y avait-t-il il y a $3$ semaines ? Arrondir à $10^{-1}$.
		\item À partir de combien de semaines le nombre de malades dépassera $70$ millions ? Arrondir à  $10^{-1}$.
		\item En combien de semaines le premier malade a-t-il infecté $18\ 347$ personnes ? On cherche le plus petit $t$ tel que $M(t) \geq 1$. Arrondir à  $10^{-1}$.
	\end{enumerate}

}{exe:4}{
	todo
}


%%%%%%%%%%%%

\newpage
\fancyhead[C]{\textbf{Solutions}}
\shipoutAnswer


\end{document}
