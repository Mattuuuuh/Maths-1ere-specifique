\documentclass[a4paper, 12pt]{extarticle}

\usepackage[utf8x]{inputenc}
%fonts
\usepackage{libertinus,libertinust1math}
\usepackage{amsmath,amsthm,amssymb,mathtools}

% SOLUTION SWITCH

\ifsolutions
	\newcommand{\exe}[2]{
		\begin{ex} #1  \end{ex}
		\begin{sol} #2 \end{sol}
	}
\else
	\newcommand{\exe}[2]{
		\begin{ex} #1  \end{ex}
	}
	
\fi


\usepackage[french]{babel}
\usepackage[
a4paper,
margin=2cm,
nomarginpar,% We don't want any margin paragraphs
]{geometry}

% HEADER, ARRAY, ENUM, MULTIOCL
\usepackage{fancyhdr}
\usepackage{array}
\usepackage{multicol, enumitem}
\newcolumntype{P}[1]{>{\centering\arraybackslash}p{#1}}
\usepackage{stackengine}
\newcommand\xrowht[2][0]{\addstackgap[.5\dimexpr#2\relax]{\vphantom{#1}}}

% theorems

\theoremstyle{theorem}
\newtheorem{thm}{Théorème}
\theoremstyle{plain}
\newtheorem*{sol}{Solution}
\theoremstyle{definition}
\newtheorem{ex}{Exercice}
\newtheorem{dfn}{Définition}
\newtheorem*{dfn*}{Définition}


%couleurs
\usepackage{tcolorbox}
\definecolor{myg}{RGB}{56, 140, 70}
\definecolor{myb}{RGB}{45, 111, 177}
\definecolor{myr}{RGB}{199, 68, 64}
\definecolor{mygr}{HTML}{2C3338}


\tcbuselibrary{theorems,skins,hooks}
\newcounter{commonbox}
\makeatletter
\newtcbtheorem[use counter=commonbox]{theorem}{Théorème }%
{
	enhanced,
	colback=white,
	colframe=mygr,
	attach boxed title to top left={yshift*=-\tcboxedtitleheight},
	fonttitle=\bfseries,
	title={#2},
	boxed title size=title,
	boxed title style={%
			sharp corners,
			rounded corners=northwest,
			colback=tcbcolframe,
			boxrule=0pt,
		},
	underlay boxed title={%
			\path[fill=tcbcolframe] (title.south west)--(title.south east)
			to[out=0, in=180] ([xshift=5mm]title.east)--
			(title.center-|frame.east)
			[rounded corners=\kvtcb@arc] |-
			(frame.north) -| cycle;
		},
	#1
}{th}
\newtcbtheorem[use counter=commonbox]{rappel}{Rappel }%
{
	enhanced,
	colback=white,
	colframe=mygr,
	attach boxed title to top left={yshift*=-\tcboxedtitleheight},
	fonttitle=\bfseries,
	title={#2},
	boxed title size=title,
	boxed title style={%
			sharp corners,
			rounded corners=northwest,
			colback=tcbcolframe,
			boxrule=0pt,
		},
	underlay boxed title={%
			\path[fill=tcbcolframe] (title.south west)--(title.south east)
			to[out=0, in=180] ([xshift=5mm]title.east)--
			(title.center-|frame.east)
			[rounded corners=\kvtcb@arc] |-
			(frame.north) -| cycle;
		},
	#1
}{th}
\newtcbtheorem[use counter=commonbox]{strategie}{Stratégie }%
{
	enhanced,
	colback=white,
	colframe=mygr,
	attach boxed title to top left={yshift*=-\tcboxedtitleheight},
	fonttitle=\bfseries,
	title={#2},
	boxed title size=title,
	boxed title style={%
			sharp corners,
			rounded corners=northwest,
			colback=tcbcolframe,
			boxrule=0pt,
		},
	underlay boxed title={%
			\path[fill=tcbcolframe] (title.south west)--(title.south east)
			to[out=0, in=180] ([xshift=5mm]title.east)--
			(title.center-|frame.east)
			[rounded corners=\kvtcb@arc] |-
			(frame.north) -| cycle;
		},
	#1
}{th}
\newtcbtheorem[use counter=commonbox]{outil}{Outil }%
{
	enhanced,
	colback=white,
	colframe=mygr,
	attach boxed title to top left={yshift*=-\tcboxedtitleheight},
	fonttitle=\bfseries,
	title={#2},
	boxed title size=title,
	boxed title style={%
			sharp corners,
			rounded corners=northwest,
			colback=tcbcolframe,
			boxrule=0pt,
		},
	underlay boxed title={%
			\path[fill=tcbcolframe] (title.south west)--(title.south east)
			to[out=0, in=180] ([xshift=5mm]title.east)--
			(title.center-|frame.east)
			[rounded corners=\kvtcb@arc] |-
			(frame.north) -| cycle;
		},
	#1
}{th}
\newtcbtheorem[use counter=commonbox]{but}{Buts du chapitre }%
{
	enhanced,
	colback=white,
	colframe=mygr,
	attach boxed title to top left={yshift*=-\tcboxedtitleheight},
	fonttitle=\bfseries,
	title={#2},
	boxed title size=title,
	boxed title style={%
			sharp corners,
			rounded corners=northwest,
			colback=tcbcolframe,
			boxrule=0pt,
		},
	underlay boxed title={%
			\path[fill=tcbcolframe] (title.south west)--(title.south east)
			to[out=0, in=180] ([xshift=5mm]title.east)--
			(title.center-|frame.east)
			[rounded corners=\kvtcb@arc] |-
			(frame.north) -| cycle;
		},
	#1
}{th}
\newtcbtheorem[use counter=commonbox]{propriete}{Propriété }%
{
	enhanced,
	colback=white,
	colframe=mygr,
	attach boxed title to top left={yshift*=-\tcboxedtitleheight},
	fonttitle=\bfseries,
	title={#2},
	boxed title size=title,
	boxed title style={%
			sharp corners,
			rounded corners=northwest,
			colback=tcbcolframe,
			boxrule=0pt,
		},
	underlay boxed title={%
			\path[fill=tcbcolframe] (title.south west)--(title.south east)
			to[out=0, in=180] ([xshift=5mm]title.east)--
			(title.center-|frame.east)
			[rounded corners=\kvtcb@arc] |-
			(frame.north) -| cycle;
		},
	#1
}{th}
\newtcbtheorem[number within=commonbox]{definition}{Définition }%
{
	enhanced,
	colback=white,
	colframe=mygr,
	attach boxed title to top left={yshift*=-\tcboxedtitleheight},
	fonttitle=\bfseries,
	title={#2},
	boxed title size=title,
	boxed title style={%
			sharp corners,
			rounded corners=northwest,
			colback=tcbcolframe,
			boxrule=0pt,
		},
	underlay boxed title={%
			\path[fill=tcbcolframe] (title.south west)--(title.south east)
			to[out=0, in=180] ([xshift=5mm]title.east)--
			(title.center-|frame.east)
			[rounded corners=\kvtcb@arc] |-
			(frame.north) -| cycle;
		},
	#1
}{th}
\newtcbtheorem[number within=commonbox]{exemples}{Exemples }%
{
	enhanced,
	colback=white,
	colframe=mygr,
	attach boxed title to top left={yshift*=-\tcboxedtitleheight},
	fonttitle=\bfseries,
	title={#2},
	boxed title size=title,
	boxed title style={%
			sharp corners,
			rounded corners=northwest,
			colback=tcbcolframe,
			boxrule=0pt,
		},
	underlay boxed title={%
			\path[fill=tcbcolframe] (title.south west)--(title.south east)
			to[out=0, in=180] ([xshift=5mm]title.east)--
			(title.center-|frame.east)
			[rounded corners=\kvtcb@arc] |-
			(frame.north) -| cycle;
		},
	#1
}{th}
\newtcbtheorem[number within=commonbox]{exemple}{Exemple }%
{
	enhanced,
	colback=white,
	colframe=mygr,
	attach boxed title to top left={yshift*=-\tcboxedtitleheight},
	fonttitle=\bfseries,
	title={#2},
	boxed title size=title,
	boxed title style={%
			sharp corners,
			rounded corners=northwest,
			colback=tcbcolframe,
			boxrule=0pt,
		},
	underlay boxed title={%
			\path[fill=tcbcolframe] (title.south west)--(title.south east)
			to[out=0, in=180] ([xshift=5mm]title.east)--
			(title.center-|frame.east)
			[rounded corners=\kvtcb@arc] |-
			(frame.north) -| cycle;
		},
	#1
}{th}
\newtcbtheorem[number within=commonbox]{questions}{Questions guidantes }%
{
	enhanced,
	colback=white,
	colframe=mygr,
	attach boxed title to top left={yshift*=-\tcboxedtitleheight},
	fonttitle=\bfseries,
	title={#2},
	boxed title size=title,
	boxed title style={%
			sharp corners,
			rounded corners=northwest,
			colback=tcbcolframe,
			boxrule=0pt,
		},
	underlay boxed title={%
			\path[fill=tcbcolframe] (title.south west)--(title.south east)
			to[out=0, in=180] ([xshift=5mm]title.east)--
			(title.center-|frame.east)
			[rounded corners=\kvtcb@arc] |-
			(frame.north) -| cycle;
		},
	#1
}{th}
\makeatother

% corps
\newcommand{\R}{\mathbb{R}}
\newcommand{\Rnn}{\mathbb{R}^{2n}}
\newcommand{\Z}{\mathbb{Z}}
\newcommand{\N}{\mathbb{N}}
\newcommand{\Q}{\mathbb{Q}}

% domain
\newcommand{\D}{\mathcal{D}}
% for calligraphic C
\usepackage{calrsfs}
\newcommand{\C}{\mathcal{C}}

% date
\usepackage{advdate}

% ensembles tq. 
\newcommand{\xRtq}[1]{
	$\left\{ x \in \R \text{ tq. } #1 \right\}$
}

% vabs
\newcommand{\vabs}[1]{
	\left| #1 \right|
}

%pinfty minfty
\newcommand{\pinfty}{{+}\infty}
\newcommand{\minfty}{{-}\infty}

% plots
\usepackage{pgfplots}

%virgules
\usepackage{icomma}
\pgfplotsset{/pgf/number format/use comma}

%subfigures
\usepackage{subcaption}

%hyperlink footnote
\usepackage{hyperref}

%wider tabulars
\def\arraystretch{2}
\setlength\tabcolsep{15pt}

% tableaux var, signe
\usepackage{tkz-tab}

\AdvanceDate[1]

\begin{document}
\pagestyle{fancy}
\fancyhead[L]{Première spécifique}
\fancyhead[C]{\textbf{Fonctions polynomiales — approfondissements}}
\fancyhead[R]{\today}


\exe{}{
	Montrer que $x$ et $x^2$ sont non miscibles, c'est-à-dire montrer que l'identité
		\[ x^2 = x \]
	ne peut pas être vraie pour tout $x\in\R$.
	
	\textbf{Conclusion} : la notation $x^2 + 3x + 1$ n'est pas simplifiable davantage car les puissances de $x$ sont indépendantes.
}{exe:x-x2-nonmiscibles}{
	TODO
}

\nomen{
	On appelle \emphindex{polynôme} une expression de la forme
		\[ a_d x^d + a_{d-1}x^{d-1} + \cdots + a_1 x + a_0, \]
	où chaque $a_i\in\R$ est un coefficient fixé.
	La plus haute puissance, $d$, est appelée le \emphindex{degré} du polynôme.
}

\exe{, difficulty=1}{
    Soient $f(x) = 3x^2 + 17x - 11$ et $g(x) = 2x^2 + 17x - 10$ pour tout $x \in \R$ deux fonctions quadratiques.

    Déterminer entièrement $\C_f \cap \C_g$.
}{exe:2}{
	On pose $f(x) = g(x)$, qu'on simplifie en $x^2 = 1$.
	On trouve donc deux solution, $x=1$ et $x=-1$.
	
	Comme $f(1) = 9$ et $f(-1) = -25$, les deux points d'intersections sont donc $( 1 ; 9)$ et $(-1 ; -25)$.
}

\exe{, difficulty=2}{
    Soient $f(x) = x$ et $g(x) = x^3 - 3x^2 + 4x - 1$ deux fonctions polynomiales sur $\R$.

    \begin{enumerate}
        \item 
        Montrer que $(x-1)^3 = x^3 - 3x^2 + 3x - 1$ pour tout $x\in\R$.
        \item
        Déterminer entièrement $\C_f \cap \C_g$.
        \item
        Créer une fonction polynomiale $h$ de degré 4 telle que $\C_f \cap \C_h = \bigset{ (1;1) }.$
    \end{enumerate}
    
}{exe:3}{
    \begin{enumerate}
        \item 
        On utilise que $(x-1)^3 = (x-1)(x-1)(x-1) = (x^2 - 2x + 1)(x-1)$ et on développe tranquillement.
        \item
        Après avoir posé $f(x) = g(x)$, on obtient $(x-1)^3 = 0$, d'après la première question.
        C'est un produit nul (trois fois le même terme), donc $x-1=0$ et $x=1$.
        Le point d'intersection est donc $(1 ; 1)$.
        \item
        On choisit $h(x) = (x-1)^4 + x$
        En développant, on trouve $h(x) = x^4 - 4x^3 + 6x^2 - 3x + 1$.
    \end{enumerate}
}

\exe{, difficulty=1}{
    Soient $f(x) = -2x^2 + 7x + 2$ et $g(x) = -3x^2 + 2x +16$ pour tout $x \in \R$ deux fonctions quadratiques.
    
    \begin{enumerate}
        \item 
        Déterminer l'autre solution de $f(x)-g(x)=0$, sachant que $x=2$ en est une et donc que $f(x)-g(x) = (x-2)h(x)$, où $h$ est affine.
        \item
        Déterminer entièrement $\C_f \cap \C_g$.
        \item
        Créer une fonction polynomiale $F$ de degré $2$ telle que $\C_f \cap \C_F = \bigset{ (2;8), (-1; -7) }.$
    \end{enumerate}
}{exe:4}{
    \begin{enumerate}
        \item 
        $f(x) - g(x) = x^2 + 5x - 14 = (x-2)h(x)$
        Comme $h$ est affine, $h(x) = ax+b$, et $x^2 + 5x - 14 = (x-2)(ax+b) = ax^2 + (b-2a)x - 2b$.
        
        En identifiant les coefficients devant les $x^2$, on trouve $1=a$.
        Les coefficients constants donnent $-14 = -2b$, et $b=7$.
        Donc $f(x)-g(x) = (x-2)(x+7)$.
        \item
        On cherche les $x$ vérifiant $(x-2)(x+7)=0$. 
        C'est un produit nul, et donc un des deux termes est nul.
        Il suit que $x=2$ et $x=-7$ sont les deux solutions.
        
        Ainsi, $(2 ; 8)$ et $(-7 ; -145)$ sont les deux points d'intersection.
        \item
        On vérifiera bien sûr que $f(-1)=-7$ avant de tenter quoi que ce soit...
        
        Par analogie au raisonnement précédent, on souhaite désormais créer $F$ de telle sorte que $f(x) - F(x) = (x-2)(x+1)$.
        Ainsi $F(x) = f(x) - (x-2)(x+1) = -3x^2 +8x +4$.
    \end{enumerate}
}

\nomen{
	Soit $p$ un polynôme de degré 2.
	\begin{itemize}
		\item
		La forme $p(x) = ax^2 + bx + c$ est la \emphindex{forme développée réduite}.
		\item
		La forme $p(x) = a(x-r)(x-s)$ est la \emphindex{forme factorisée}.
		\item
		La forme $p(x) = a(x-\alpha)^2 + \beta$ est \emphindex{forme canonique}.
	\end{itemize}
}

\exe{, difficulty=1}{
	Considérons, pour tout $x\in\R$, la fonction quadratique suivante.
		\[ f(x) = x^2 -9x + 20 \]
	\begin{enumerate}
		\item Montrer que si $f(x) = (x-r) (x-s)$ pour certains nombres réels $r, s\in\R$, alors on a forcément $f(r) = f(s) = 0$.
	\end{enumerate}
	Supposons désormais que $f(x) = (x-r) (x-s)$.
	\begin{enumerate}[resume]
		\item Développer l'expression de droite, identifier les coeffients en $1, x, x^2$, et déduire que
			\[\begin{cases} r + s = 9, \\ r s = 20. \end{cases} \]
		\item Trouver $r$ et $s$ en cherchant deux nombres dont la somme vaut 9, et le produit 20.
	\end{enumerate}
}{exe:8}{
	\begin{enumerate}
		\item Évaluer en $x=r$ ou $x=s$ annule un des termes du produit, et donc le produit entier.
		C'est pour cela que la forme factorisée donne directement les racines d'un polynôme.
		\item 
			\[ f(x) = x^2 - 9x + 20 = (x-r)(x-s) = x^2 - (r + s)x + rs. \]
		En $x^2$, on trouve $1=1$, en $x$, on trouve $r+s=9$, et en $1$, on trouve $rs=20$.
		\item 
		On cherche deux nombres donc le produit vaut 20 et la somme 9.
		$(r, s) = (5 ; 4)$ fonctionne. À noter qu'on pourrait échanger les valeurs de $r$ et $s$, par symétrie du problème.
			\[ x^2 - 9x + 20 = (x-5)(x-4). \]
	\end{enumerate}
}

\nomen{
	Soit $p$ un polynôme quelconque.
	Si $p(r) = 0$ pour un certain $r\in\R$, on appelle $r$ une \emph{racine} de $p$.
}

\exe{, difficulty=1}{
	Considérons $f(x) = x^2 -4x - 5$ une fonction quadratique.
	
	\begin{enumerate}
		\item
		Quels $\alpha, \beta$ choisir pour exprimer $f$ sous forme canonique ? c'est-à-dire pour que
			\[ f(x) = (x-\alpha)^2 + \beta. \]
		\item 
		La question précédente donne $f(x) = (x-2)^2 - 9$.
		En gardant cette forme, donner les deux $x$ vérifiant $f(x) = 0$.
		\item
		La question précédente donne $f(5) = f(-1) = 0$.
		Vérifier que $f(x) = (x-5)(x+1)$.
	\end{enumerate}
}{exe:canon-ex}{
	todo
}

\exe{, difficulty=2}{
	Soit $f(x) = ax^2 + bx + c$ une fonction quadratique ($a\neq0$).
	On souhaite d'abord exprimer $f$ sous forme canonique pour ensuite en déduire ses racines, afin de généraliser l'exercice \ref{exe:canon-ex}.
	Supposons donc que $f(x) = a'(x-\alpha)^2 - \beta$ pour certains $a', \alpha, \beta$.
	
	\begin{enumerate}
		\item 
		Montrer que $a'=a$, puis que $\alpha = \dfrac{-b}{2a}$, et enfin que $\beta = \dfrac{b^2 - 4ac}{4a}$.
	\end{enumerate}
	
	Posons $\Delta = b^2 - 4ac$ (lu « Delta ») le \emph{discriminant} de $f$.
	
	\begin{enumerate}[resume]
		\item 
		Montrer que la recherche de racine $f(x)=0$ est équivalente à la résolution de l'équation 
			\[ (x - \alpha)^2 = \dfrac{\Delta}{4a^2}. \]
		\item
		Démontrer le théorème \ref{thm:1}.
	\end{enumerate}
}{exe:canon-gen}{
	todo
}

\thm{}{
	Soit $f(x) = ax^2 + bx + c$ une fonction quadratique ($a\neq0$).
	Posons $\Delta = b^2 - 4ac$ son discriminant.
	On distingue trois cas concernant les racines de $f$.
		\begin{enumerate}
			\item Si $\Delta < 0$, $f$ n'a aucune racine réelle.
			\item Si $\Delta =0$, la seule racine réelle est $x = -\frac{b}{2a}$.
			\item Si $\Delta > 0$, $f$ a deux racines réelles, 
				\begin{align*}
					x_1 = \dfrac{-b + \sqrt{\Delta}}{2a}, && \et && x_2 = \dfrac{-b - \sqrt{\Delta}}{2a}.
				\end{align*}
		\end{enumerate}
}{thm:1}

\exe{}{
	À l'aide du théorème \ref{thm:1}, trouver les racines des fonctions quadratiques suivantes.	
	\begin{multicols}{2}
	\begin{enumerate}[label=\alph*)]
		\item $x^2 - 4x -5$
		\item $2x^2 + 16x + 14$
		\item $-x^2 - 20x - 96$
		\item $3x^2 - 18x + 27$
		\item $-x^2 +2x - 1$
		\item $3x^2 + 18x + 30$
		\item $-3x^2 - 18x - 15$
		\item $12x^2 + 72x + 110$
	\end{enumerate}
	\end{multicols}
}{exe:quadratic-roots}{
	todo
}

\exe{}{
	Montrer que $f(x) = (x-31)^2 + 2$ est toujours strictement positive : $f(x) > 0$ pour tout $x\in\R$.
}{exe:quadratic-pos-canon}{
	todo
}

\exe{, difficulty=1}{
	Montrer que $3x^2 - 72x + 433 > 0$ pour tout $x\in\R$.
}{exe:quadratic-pos-dev}{
	todo
}

% trop dur, inutile
%\thm{}{
%	Si $f$ polynomiale admet une racine $r$ (c'est-à-dire que $f(r) = 0$), alors $f(x) = (x-r)g(x)$, avec $g$ polynomiale.
%}{thm:1}
%
%\exe{, difficulty=1}{
%	Soit $f(x) = x^3 - 3x^2 - 10x + 24$ une fonction cubique.
%	
%	\begin{enumerate}
%		\item Montrer que $f(2) = 0$.
%		\item Par division polynomiale, trouver $g$ quadratique telle que $f(x) = (x-2)g(x)$.
%		\item Montrer que $g(4) = 0$, et factoriser complètement $f$ comme produit de facteurs linéaires.
%	\end{enumerate}
%}{exe:9}{
%	\begin{enumerate}
%		\item Trivial.
%		\item La division polynômiale fonctionne ainsi : à chaque étape, le diviseur ($x-2$ ici) annule la plus grande puissance de $x$ du divisé.
%		Ainsi, $f(x) - x^2(x-2) = -x^2 - 10x + 24$ annule le terme en $x^3$.
%		On continue : 
%			\[ f(x) - x^2(x-2) + x(x-2) = -12x + 24, \]
%		annule le terme en $x^2$.
%		Puis, 
%			\[ f(x) - x^2(x-2) + x(x-2) + 12(x-2) = 0, \]
%		qui permet de conclure que
%			\[ f(x) = (x^2 - x - 12)(x-2) = g(x)(x-2). \]
%		\item 
%		On diviser $x^2 - x - 12$ par $x-4$ pour trouver
%			\[ x^2 - x - 12 = (x-4)(x+3). \]
%		On conclut que $x^3 - 3x^2 - 10x + 24 = (x-2)(x-4)(x+3)$.
%	\end{enumerate}
%}
%
%\exe{, difficulty=3}{
%	Soit $f, g$ deux polynômes.
%	Montrer qu'il existe deux polynômes, $r$, et $q$, tels que
%		\[ f = gq + r, \]
%	avec le degré de $r$ strictement inférieur à celui de $q$.
%	
%	En déduire le théorème \ref{thm:1}.
%}{exe:10}{
%	La division polynomiale comme décrite à l'exercice \ref{exe:9} permet de se convaincre qu'on peut enlever un multiple de $g$ à $f$ pour réduire son degré, jusqu'à obtenir un reste $r$ de degré strictement inférieur.
%	
%	Le théorème \ref{thm:1} se démontre donc ainsi : si $f$ est polynomiale et admet une racine $r$, alors la division $f(x) = q(x) (x-r) + r(x)$ implique que $r(x) = c$ est une constante (car son degré est strictement inférieur à celui de $x-r$).
%	Or, en évaluant en $x=r$, on trouve que $0 = 0 + c$, et donc que $c=0$.
%	D'où $f(x) = q(x) (x-r)$, comme voulu.
%	
%	Remarquons qu'en continuant, $f$ se scinde en produit de facteurs linéaires du type $(x-r)\dots(x-r')$, où $r, \dots, r'$ sont les racines réelles de $f$.
%	
%	En outre, certains polynômes ne se scindent pas car ils n'admettent pas de racine réelle. C'est le cas, par exemple, de $x^2 + 1$.
%}

% prochaine série
%\exe{, difficulty=2}{
%	Supposons que le polynôme du second degré $p(x) = ax^2 + bx + c$ s'écrive de la forme $p(x) = a'(x-\alpha)^2 - \beta$.
%	On appelle cette forme, la \emph{forme canonique} de $f$.
%	
%	Montrer, par identification des coefficients en $x^2, x, 1$, que $a'=a$, puis que $\alpha = \dfrac{-b}{2a}$, et enfin que $\beta = \dfrac{b^2 - 4ac}{4a}$.
%}{exe:déterminant-2nddeg}{
%	Sol11.
%}

%%%%%%%%%%%%

\newpage
\fancyhead[C]{\textbf{Solutions}}
\shipoutAnswer

\end{document}
