\documentclass[a4paper, 12pt]{extarticle}

\usepackage[utf8x]{inputenc}
%fonts
\usepackage{libertinus,libertinust1math}
\usepackage{amsmath,amsthm,amssymb,mathtools}

% SOLUTION SWITCH

\ifsolutions
	\newcommand{\exe}[2]{
		\begin{ex} #1  \end{ex}
		\begin{sol} #2 \end{sol}
	}
\else
	\newcommand{\exe}[2]{
		\begin{ex} #1  \end{ex}
	}
	
\fi


\usepackage[french]{babel}
\usepackage[
a4paper,
margin=2cm,
nomarginpar,% We don't want any margin paragraphs
]{geometry}

% HEADER, ARRAY, ENUM, MULTIOCL
\usepackage{fancyhdr}
\usepackage{array}
\usepackage{multicol, enumitem}
\newcolumntype{P}[1]{>{\centering\arraybackslash}p{#1}}
\usepackage{stackengine}
\newcommand\xrowht[2][0]{\addstackgap[.5\dimexpr#2\relax]{\vphantom{#1}}}

% theorems

\theoremstyle{theorem}
\newtheorem{thm}{Théorème}
\theoremstyle{plain}
\newtheorem*{sol}{Solution}
\theoremstyle{definition}
\newtheorem{ex}{Exercice}
\newtheorem{dfn}{Définition}
\newtheorem*{dfn*}{Définition}


%couleurs
\usepackage{tcolorbox}
\definecolor{myg}{RGB}{56, 140, 70}
\definecolor{myb}{RGB}{45, 111, 177}
\definecolor{myr}{RGB}{199, 68, 64}
\definecolor{mygr}{HTML}{2C3338}


\tcbuselibrary{theorems,skins,hooks}
\newcounter{commonbox}
\makeatletter
\newtcbtheorem[use counter=commonbox]{theorem}{Théorème }%
{
	enhanced,
	colback=white,
	colframe=mygr,
	attach boxed title to top left={yshift*=-\tcboxedtitleheight},
	fonttitle=\bfseries,
	title={#2},
	boxed title size=title,
	boxed title style={%
			sharp corners,
			rounded corners=northwest,
			colback=tcbcolframe,
			boxrule=0pt,
		},
	underlay boxed title={%
			\path[fill=tcbcolframe] (title.south west)--(title.south east)
			to[out=0, in=180] ([xshift=5mm]title.east)--
			(title.center-|frame.east)
			[rounded corners=\kvtcb@arc] |-
			(frame.north) -| cycle;
		},
	#1
}{th}
\newtcbtheorem[use counter=commonbox]{rappel}{Rappel }%
{
	enhanced,
	colback=white,
	colframe=mygr,
	attach boxed title to top left={yshift*=-\tcboxedtitleheight},
	fonttitle=\bfseries,
	title={#2},
	boxed title size=title,
	boxed title style={%
			sharp corners,
			rounded corners=northwest,
			colback=tcbcolframe,
			boxrule=0pt,
		},
	underlay boxed title={%
			\path[fill=tcbcolframe] (title.south west)--(title.south east)
			to[out=0, in=180] ([xshift=5mm]title.east)--
			(title.center-|frame.east)
			[rounded corners=\kvtcb@arc] |-
			(frame.north) -| cycle;
		},
	#1
}{th}
\newtcbtheorem[use counter=commonbox]{strategie}{Stratégie }%
{
	enhanced,
	colback=white,
	colframe=mygr,
	attach boxed title to top left={yshift*=-\tcboxedtitleheight},
	fonttitle=\bfseries,
	title={#2},
	boxed title size=title,
	boxed title style={%
			sharp corners,
			rounded corners=northwest,
			colback=tcbcolframe,
			boxrule=0pt,
		},
	underlay boxed title={%
			\path[fill=tcbcolframe] (title.south west)--(title.south east)
			to[out=0, in=180] ([xshift=5mm]title.east)--
			(title.center-|frame.east)
			[rounded corners=\kvtcb@arc] |-
			(frame.north) -| cycle;
		},
	#1
}{th}
\newtcbtheorem[use counter=commonbox]{outil}{Outil }%
{
	enhanced,
	colback=white,
	colframe=mygr,
	attach boxed title to top left={yshift*=-\tcboxedtitleheight},
	fonttitle=\bfseries,
	title={#2},
	boxed title size=title,
	boxed title style={%
			sharp corners,
			rounded corners=northwest,
			colback=tcbcolframe,
			boxrule=0pt,
		},
	underlay boxed title={%
			\path[fill=tcbcolframe] (title.south west)--(title.south east)
			to[out=0, in=180] ([xshift=5mm]title.east)--
			(title.center-|frame.east)
			[rounded corners=\kvtcb@arc] |-
			(frame.north) -| cycle;
		},
	#1
}{th}
\newtcbtheorem[use counter=commonbox]{but}{Buts du chapitre }%
{
	enhanced,
	colback=white,
	colframe=mygr,
	attach boxed title to top left={yshift*=-\tcboxedtitleheight},
	fonttitle=\bfseries,
	title={#2},
	boxed title size=title,
	boxed title style={%
			sharp corners,
			rounded corners=northwest,
			colback=tcbcolframe,
			boxrule=0pt,
		},
	underlay boxed title={%
			\path[fill=tcbcolframe] (title.south west)--(title.south east)
			to[out=0, in=180] ([xshift=5mm]title.east)--
			(title.center-|frame.east)
			[rounded corners=\kvtcb@arc] |-
			(frame.north) -| cycle;
		},
	#1
}{th}
\newtcbtheorem[use counter=commonbox]{propriete}{Propriété }%
{
	enhanced,
	colback=white,
	colframe=mygr,
	attach boxed title to top left={yshift*=-\tcboxedtitleheight},
	fonttitle=\bfseries,
	title={#2},
	boxed title size=title,
	boxed title style={%
			sharp corners,
			rounded corners=northwest,
			colback=tcbcolframe,
			boxrule=0pt,
		},
	underlay boxed title={%
			\path[fill=tcbcolframe] (title.south west)--(title.south east)
			to[out=0, in=180] ([xshift=5mm]title.east)--
			(title.center-|frame.east)
			[rounded corners=\kvtcb@arc] |-
			(frame.north) -| cycle;
		},
	#1
}{th}
\newtcbtheorem[number within=commonbox]{definition}{Définition }%
{
	enhanced,
	colback=white,
	colframe=mygr,
	attach boxed title to top left={yshift*=-\tcboxedtitleheight},
	fonttitle=\bfseries,
	title={#2},
	boxed title size=title,
	boxed title style={%
			sharp corners,
			rounded corners=northwest,
			colback=tcbcolframe,
			boxrule=0pt,
		},
	underlay boxed title={%
			\path[fill=tcbcolframe] (title.south west)--(title.south east)
			to[out=0, in=180] ([xshift=5mm]title.east)--
			(title.center-|frame.east)
			[rounded corners=\kvtcb@arc] |-
			(frame.north) -| cycle;
		},
	#1
}{th}
\newtcbtheorem[number within=commonbox]{exemples}{Exemples }%
{
	enhanced,
	colback=white,
	colframe=mygr,
	attach boxed title to top left={yshift*=-\tcboxedtitleheight},
	fonttitle=\bfseries,
	title={#2},
	boxed title size=title,
	boxed title style={%
			sharp corners,
			rounded corners=northwest,
			colback=tcbcolframe,
			boxrule=0pt,
		},
	underlay boxed title={%
			\path[fill=tcbcolframe] (title.south west)--(title.south east)
			to[out=0, in=180] ([xshift=5mm]title.east)--
			(title.center-|frame.east)
			[rounded corners=\kvtcb@arc] |-
			(frame.north) -| cycle;
		},
	#1
}{th}
\newtcbtheorem[number within=commonbox]{exemple}{Exemple }%
{
	enhanced,
	colback=white,
	colframe=mygr,
	attach boxed title to top left={yshift*=-\tcboxedtitleheight},
	fonttitle=\bfseries,
	title={#2},
	boxed title size=title,
	boxed title style={%
			sharp corners,
			rounded corners=northwest,
			colback=tcbcolframe,
			boxrule=0pt,
		},
	underlay boxed title={%
			\path[fill=tcbcolframe] (title.south west)--(title.south east)
			to[out=0, in=180] ([xshift=5mm]title.east)--
			(title.center-|frame.east)
			[rounded corners=\kvtcb@arc] |-
			(frame.north) -| cycle;
		},
	#1
}{th}
\newtcbtheorem[number within=commonbox]{questions}{Questions guidantes }%
{
	enhanced,
	colback=white,
	colframe=mygr,
	attach boxed title to top left={yshift*=-\tcboxedtitleheight},
	fonttitle=\bfseries,
	title={#2},
	boxed title size=title,
	boxed title style={%
			sharp corners,
			rounded corners=northwest,
			colback=tcbcolframe,
			boxrule=0pt,
		},
	underlay boxed title={%
			\path[fill=tcbcolframe] (title.south west)--(title.south east)
			to[out=0, in=180] ([xshift=5mm]title.east)--
			(title.center-|frame.east)
			[rounded corners=\kvtcb@arc] |-
			(frame.north) -| cycle;
		},
	#1
}{th}
\makeatother

% corps
\newcommand{\R}{\mathbb{R}}
\newcommand{\Rnn}{\mathbb{R}^{2n}}
\newcommand{\Z}{\mathbb{Z}}
\newcommand{\N}{\mathbb{N}}
\newcommand{\Q}{\mathbb{Q}}

% domain
\newcommand{\D}{\mathcal{D}}
% for calligraphic C
\usepackage{calrsfs}
\newcommand{\C}{\mathcal{C}}

% date
\usepackage{advdate}

% ensembles tq. 
\newcommand{\xRtq}[1]{
	$\left\{ x \in \R \text{ tq. } #1 \right\}$
}

% vabs
\newcommand{\vabs}[1]{
	\left| #1 \right|
}

%pinfty minfty
\newcommand{\pinfty}{{+}\infty}
\newcommand{\minfty}{{-}\infty}

% plots
\usepackage{pgfplots}

%virgules
\usepackage{icomma}
\pgfplotsset{/pgf/number format/use comma}

%subfigures
\usepackage{subcaption}

%hyperlink footnote
\usepackage{hyperref}

%wider tabulars
\def\arraystretch{2}
\setlength\tabcolsep{15pt}

% tableaux var, signe
\usepackage{tkz-tab}

\AdvanceDate[1]

\begin{document}
\pagestyle{fancy}
\fancyhead[L]{Première spécifique}
\fancyhead[C]{\textbf{Fonctions affines — approfondissements}}
\fancyhead[R]{\today}

\exe{}{
	Pour chacune des paires de fonctions affines $f, g$, calculer l'intersection des droites $\C_f \cap \C_g$.
	
	\begin{multicols}{2}
	\begin{enumerate}
		\item $f(x) = 2x + 1, g(x) = -x+1$.
		\item $f(x) = -x + 1, g(x) = 2x + 1$.
		\item $f(x) = 3+7x, g(x) = 2$.
		\item $f(x) = 9-2x, g(x) = 17-x$.
		\item $f(x) = 2x+1, g(x) = 2x+1$.
		\item $f(x) = 2x+1, g(x) = 2x+2$.
	\end{enumerate}
	\end{multicols}
}{exe:intersections-affines}{

	\begin{enumerate}
		\item $f(x) = 2x + 1, g(x) = -x+1$ .
			On cherche un point $P(x_P, y_P) \in \R^2$ vérifiant les deux équations
				\begin{align*}
					y_P = f(x_P) && \text{ et } && y_P = g(x_P)
				\end{align*}
			Comme bien sûr $y_P = y_P$, on peut résoudre l'équation
				\[ f(x_P) = g(x_P) \]
			pour trouver $x_P$.
			Ici, on pose calmement
				\begin{align*}
					f(x_P) &= g(x_P) \\
					2x_P + 1 &= -x_P + 1 \\
					3x_P &= 0 \\
					x_P &= 0,
				\end{align*}
			ce qui nous fournit $x_P = 0$.
			Reste plus qu'à utiliser que $y_P = f(x_P) = g(x_P)$ pour calculer $y_P$.
			
			D'une part, 
				\[ f(x_P) = f(0) = 2\cdot0 + 1 = 1. \]
			D'autre part, on aurait pû calculer
				\[ g(x_P) = g(0) = -0 + 1 = 1. \]
			Rien de surprenant à ce qu'on trouve la même valeur $y_P = 1$, car $x_P$ a été trouvé vérifiant $f(x_P) = g(x_P)$.
			
			En conclusion, $\C_f \cap \C_g = \{ (0;1) \}$.
			
		\item 
			Similairement,
			\begin{align*}
				f(x_P) &= g(x_P) \\
				-x_P + 1 & = 2x_P + 1 \\
				x_P &= 0,
			\end{align*}
			et $y_P = f(0) = 1$.
			Sans surprise, $(0;1)$ est à nouveau le point d'intersection de $\C_f$ et de $\C_g$ car se sont les mêmes fonctions que ci-dessus mais échangées.
		
		\item 
			Similairement,
			\begin{align*}
				f(x_P) &= g(x_P) \\
				3 + 7x_P & = 2 \\
				7x_P &= -1 \\
				x_P &= \dfrac{-1}7,
			\end{align*}
			\[
				y_P = f(x_P) = 3+7\cdot \dfrac{-1}7  = 2.
			\]
			Il aurait été encore plus facile de calculer $g(x_P) = 2$ à la place !
			
			En conclusion, $\C_f \cap \C_g = \{ \left(-\dfrac72 ; 2\right) \}$.
		
		\item
			Similairement,
			\begin{align*}
				f(x_P) &= g(x_P) \\
				9-2x_P & = 17-x_P \\
				-8 &= x_P,
			\end{align*}
			\[
				y_P = f(x_P) = 9-2 \cdot(-8) = 25.
			\]
		
			En conclusion, $\C_f \cap \C_g = \{ (-8;25) \}$.
		
		\item 			
		Similairement,
			\begin{align*}
				f(x_P) &= g(x_P) \\
				2x_P + 1 & = 2x_P + 1 \\
				0 &= 0,
			\end{align*}
			cette équation étant évidemment vraie, tous les $x_P \in \R$ vérifient l'équation.
			\[
				y_P = f(x_P) = 2x_P + 1.
			\]
			N'importe quel $(x_P ; 2x_P+1) \in \R$ appartient donc à l'intersection des droites.
			
			En conclusion, $\C_f \cap \C_g = \{ (x_P ; 2x_P+1) \text{ tq. } x_P \in \R \} = \{ (x_P ; y_P) \in \R^2 \text{ tq. } y_P = 2x_P + 1 \} = \C_f = \C_g$.
			Il n'est pas surprenant qu'on trouve que toute la droite appartienne à l'intersection, car $f$ et $g$ sont identiques : tous les points de la droite $\C_f$ sont des points d'intersection.
		
		
		\item 
			Similairement,
			\begin{align*}
				f(x_P) &= g(x_P) \\
				2x_P + 1 & = 2x_P + 2 \\
				1 &= 2,
			\end{align*}
			cette équation étant évidemment fausse, aucun $x_P \in \R$ ne vérifie l'équation.
			
			En conclusion, $\C_f \cap \C_g = \emptyset$, l'ensemble vide.
			En fait, les droites $\C_f$ et $\C_g$ sont parallèles car elles ont le même coefficient directeur $2$.
			Or les droites ne sont pas égales car leurs ordonnées à l'origine sont différentes, donc il n'y a aucun point d'intersection.
	\end{enumerate}

}

\exe{, difficulty=2}{
    Soit $f(x) = ax+b$ et $g(x) = a'x + b'$ pour tout $x \in \R$ deux fonctions affines de paramètres $a, a', b, b' \in\R$.
	
	Montrer que si $a=a'$, alors $\C_f$ et $\C_g$ sont parallèles : on a soit $\C_f \cap \C_g = \emptyset$, soit $\C_f = \C_g$.

    Montrer que si $a\neq a'$, alors
        \[ \C_f \cap \C_g = \left\{ \left(\ \dfrac{b'-b}{a-a'} ; \dfrac{ab' - ba'}{a-a'} \right) \right\}. \]
}{exe:5}{
	Soit $P(x_P ; y_P) \in \C_f \cap \C_g$.
	Alors $f(x_P) = y_P = g(x_P)$.
	Il suit que 
		\[ f(x_P) = g(x_P) \iff ax_P + b = a' x_P + b' \iff x_P = \dfrac{b'-b}{a-a'}. \]
	
	Ensuite, 
		\[ y_P = f(x_P) = a\dfrac{b'-b}{a-a'} + b = \dfrac{ab' -ab + ba -ba'}{a-a'} = \dfrac{ab' -ba'}{a-a'}. \]
}

\exe{, difficulty=1}{
    Soit $f(x) = ax + b$ une fonction affine sur $\R$ de paramètres $a, b\in\R$ et $P(x_P;y_P)$ un point du plan.

    \begin{enumerate}
        \item 
        Montrer que la fonction affine $g$ donnée par
            \[ g(x) = a(x-x_P) + y_P, \]
        pour tout $x\in\R$ vérifie que $\C_f$ est parallèle à $\C_g$ et que $P \in \C_g$.
        \item
        Montrer que si $P\in\C_f$, alors $f=g$ ($f$ et $g$ admettent le même coefficient directeur et la même ordonnée à l'origine).
    \end{enumerate}

}{exe:1}{
    \begin{enumerate}
        \item 
        Sont coefficient directeur est $a$, le même que celui de $f$.
        Sa courbe est donc parallèle à celle de $f$.
        
        De plus, $g(x_P) = a(x_P - x_P) + y_P = y_P$, donc $P$ appartient à la courbe de $g$.
        \item
        Les coefficients directeurs sont déjà égaux.
        L'ordonnée à l'origine de $g$ est $y_P - ax_P$.
        Or, comme $P$ appartient à la courbe de $f$, on a $f(x_P) = ax_P + b = y_P$.
        Il suit que $y_P - ax_P = b$, ce qui conclut.
    \end{enumerate}
}

\exe{, difficulty=1}{
	Soit $f(x) = ax+b$ une fonction affine sur $\R$ à paramètre $a, b\in\R$.
	
	Montrer que si $f(r)=0$ pour un certain $r\in\R$ on a alors, pour tout $x\in\R$,
		\[ f(x) = a(x-r). \]
}{exe:6}{
	L'exercice \ref{exe:1} conclut, grâce au point $P(r ; 0)$.
}

\exe{, difficulty=2}{
	Considérons une fonction quadratique 
		\[ f(x) = ax^2 + bx + c, \]
	où $a, b, c\in\R$ sont trois paramètres réels.
	On appelle cette forme, la \emph{forme développée} de $f$.
	
	Supposons de surcroît qu'on connaisse deux zéros distincts de $f$, c'est-à-dire qu'on connaisse $\alpha, \beta\in\R$ tels que $\alpha\neq\beta$ et
		\[ f(\alpha) = f(\beta) = 0. \]
	\begin{enumerate}
		\item Montrer que la fonction $g$ donnée par
			\[ g(x) = f(x) - a (x-\alpha)(x-\beta) \qquad \text{ pour tout } x\in\R \]
		est affine.
		\item Montrer que $g(\alpha) = g(\beta) = 0$.
		\item En déduire que $g$ est constamment nulle et donc que
			\[ f(x) = a (x-\alpha)(x-\beta)  \qquad \text{ pour tout } x\in\R.  \]
	\end{enumerate}
	
	\textbf{Conclusion} : une fonction quadratique $f$ se factorise comme produit de facteurs linéaires $x-r$ où $r$ est un zéro de $f$ (c'est-à-dire que $f(r) = 0$).
	C'est la \emph{forme factorisée} du polynôme.
	
	%Ce résultat est généralisé par le théorème énoncé ci-dessous.
	Ce résultat est vrai pour les autres polynômes. Par exemple, $x^3 -x^2-16x+16 = (x-1)(x+4)(x-4)$.
}{exe:7}{
	\begin{enumerate}
		\item 
			En développant l'expression de $g$, le terme quadratique (en $x^2$) s'annule.
			Ne subsiste qu'une expression de la forme $ax+b$, expression d'une fonction affine.
		\item 
			$g(\alpha) = f(\alpha) - a(\alpha-\alpha)(\alpha-\beta) = 0$, car $f(\alpha)=0$, et $\alpha-\alpha=0$.
			Idem pour $\beta$.
		\item 
			$g$ est affine et passe par $(\alpha ;0)$, et $(\beta ; 0)$, deux points distincts.
			$g$ est donc la fonction nulle, $g(x) = 0$.
			Or comme $g(x) = f(x) - a (x-\alpha)(x-\beta) =0$, on déduit que $f(x) = a (x-\alpha)(x-\beta)$.
	\end{enumerate}
}

%\thm{}{
%	Si $f$ polynomiale admet un zéro $r$, alors $f(x) = (x-r)g(x)$, avec $g$ polynomiale.
%}{thm:1}
%
%\exe{, difficulty=1}{
%	Soit $f(x) = x^3 - 3x^2 - 10x + 24$ une fonction cubique.
%	
%	\begin{enumerate}
%		\item Montrer que $f(2) = 0$.
%		\item Par division polynomiale, trouver $g$ quadratique telle que $f(x) = (x-2)g(x)$.
%		\item Montrer que $g(4) = 0$, et factoriser complètement $f$ comme produit de facteurs linéaires.
%	\end{enumerate}
%}{exe:9}{
%	\begin{enumerate}
%		\item Trivial.
%		\item La division polynômiale fonctionne ainsi : à chaque étape, le diviseur ($x-2$ ici) annule la plus grande puissance de $x$ du divisé.
%		Ainsi, $f(x) - x^2(x-2) = -x^2 - 10x + 24$ annule le terme en $x^3$.
%		On continue : 
%			\[ f(x) - x^2(x-2) + x(x-2) = -12x + 24, \]
%		annule le terme en $x^2$.
%		Puis, 
%			\[ f(x) - x^2(x-2) + x(x-2) + 12(x-2) = 0, \]
%		qui permet de conclure que
%			\[ f(x) = (x^2 - x - 12)(x-2) = g(x)(x-2). \]
%		\item 
%		On diviser $x^2 - x - 12$ par $x-4$ pour trouver
%			\[ x^2 - x - 12 = (x-4)(x+3). \]
%		On conclut que $x^3 - 3x^2 - 10x + 24 = (x-2)(x-4)(x+3)$.
%	\end{enumerate}
%}

%\exe{, difficulty=3}{
%	Soit $f, g$ deux polynômes.
%	Montrer qu'il existe deux polynômes, $r$, et $q$, tels que
%		\[ f = gq + r, \]
%	avec le degré de $r$ strictement inférieur à celui de $q$.
%	C'est l'équivalent de la division euclidienne pour les polynômes.
%	
%	En déduire le théorème \ref{thm:1}.
%}{exe:10}{
%	La division polynomiale comme décrite à l'exercice \ref{exe:9} permet de se convaincre qu'on peut enlever un multiple de $g$ à $f$ pour réduire son degré, jusqu'à obtenir un reste $r$ de degré strictement inférieur.
%	
%	Le théorème \ref{thm:1} se démontre donc ainsi : si $f$ est polynomiale et admet une racine $r$, alors la division $f(x) = q(x) (x-r) + r(x)$ implique que $r(x) = c$ est une constante (car son degré est strictement inférieur à celui de $x-r$).
%	Or, en évaluant en $x=r$, on trouve que $0 = 0 + c$, et donc que $c=0$.
%	D'où $f(x) = q(x) (x-r)$, comme voulu.
%	
%	Remarquons qu'en continuant, $f$ se scinde en produit de facteurs linéaires du type $(x-r)\dots(x-r')$, où $r, \dots, r'$ sont les racines réelles de $f$.
%	
%	En outre, certains polynômes ne se scindent pas car ils n'admettent pas de racine réelle. C'est le cas, par exemple, de $x^2 + 1$.
%}

% prochaine série
%\exe{, difficulty=2}{
%	Supposons que le polynôme du second degré $p(x) = ax^2 + bx + c$ s'écrive de la forme $p(x) = a'(x-\alpha)^2 - \beta$.
%	On appelle cette forme, la \emph{forme canonique} de $f$.
%	
%	Montrer, par identification des coefficients en $x^2, x, 1$, que $a'=a$, puis que $\alpha = \dfrac{-b}{2a}$, et enfin que $\beta = \dfrac{b^2 - 4ac}{4a}$.
%}{exe:déterminant-2nddeg}{
%	Sol11.
%}

%%%%%%%%%%%%

\newpage
\fancyhead[C]{\textbf{Solutions}}
\shipoutAnswer

\end{document}
