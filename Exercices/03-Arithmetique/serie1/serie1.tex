\documentclass[a4paper, 12pt]{extarticle}

\usepackage[utf8x]{inputenc}
%fonts
\usepackage{libertinus,libertinust1math}
\usepackage{amsmath,amsthm,amssymb,mathtools}

% SOLUTION SWITCH

\ifsolutions
	\newcommand{\exe}[2]{
		\begin{ex} #1  \end{ex}
		\begin{sol} #2 \end{sol}
	}
\else
	\newcommand{\exe}[2]{
		\begin{ex} #1  \end{ex}
	}
	
\fi


\usepackage[french]{babel}
\usepackage[
a4paper,
margin=2cm,
nomarginpar,% We don't want any margin paragraphs
]{geometry}

% HEADER, ARRAY, ENUM, MULTIOCL
\usepackage{fancyhdr}
\usepackage{array}
\usepackage{multicol, enumitem}
\newcolumntype{P}[1]{>{\centering\arraybackslash}p{#1}}
\usepackage{stackengine}
\newcommand\xrowht[2][0]{\addstackgap[.5\dimexpr#2\relax]{\vphantom{#1}}}

% theorems

\theoremstyle{theorem}
\newtheorem{thm}{Théorème}
\theoremstyle{plain}
\newtheorem*{sol}{Solution}
\theoremstyle{definition}
\newtheorem{ex}{Exercice}
\newtheorem{dfn}{Définition}
\newtheorem*{dfn*}{Définition}


%couleurs
\usepackage{tcolorbox}
\definecolor{myg}{RGB}{56, 140, 70}
\definecolor{myb}{RGB}{45, 111, 177}
\definecolor{myr}{RGB}{199, 68, 64}
\definecolor{mygr}{HTML}{2C3338}


\tcbuselibrary{theorems,skins,hooks}
\newcounter{commonbox}
\makeatletter
\newtcbtheorem[use counter=commonbox]{theorem}{Théorème }%
{
	enhanced,
	colback=white,
	colframe=mygr,
	attach boxed title to top left={yshift*=-\tcboxedtitleheight},
	fonttitle=\bfseries,
	title={#2},
	boxed title size=title,
	boxed title style={%
			sharp corners,
			rounded corners=northwest,
			colback=tcbcolframe,
			boxrule=0pt,
		},
	underlay boxed title={%
			\path[fill=tcbcolframe] (title.south west)--(title.south east)
			to[out=0, in=180] ([xshift=5mm]title.east)--
			(title.center-|frame.east)
			[rounded corners=\kvtcb@arc] |-
			(frame.north) -| cycle;
		},
	#1
}{th}
\newtcbtheorem[use counter=commonbox]{rappel}{Rappel }%
{
	enhanced,
	colback=white,
	colframe=mygr,
	attach boxed title to top left={yshift*=-\tcboxedtitleheight},
	fonttitle=\bfseries,
	title={#2},
	boxed title size=title,
	boxed title style={%
			sharp corners,
			rounded corners=northwest,
			colback=tcbcolframe,
			boxrule=0pt,
		},
	underlay boxed title={%
			\path[fill=tcbcolframe] (title.south west)--(title.south east)
			to[out=0, in=180] ([xshift=5mm]title.east)--
			(title.center-|frame.east)
			[rounded corners=\kvtcb@arc] |-
			(frame.north) -| cycle;
		},
	#1
}{th}
\newtcbtheorem[use counter=commonbox]{strategie}{Stratégie }%
{
	enhanced,
	colback=white,
	colframe=mygr,
	attach boxed title to top left={yshift*=-\tcboxedtitleheight},
	fonttitle=\bfseries,
	title={#2},
	boxed title size=title,
	boxed title style={%
			sharp corners,
			rounded corners=northwest,
			colback=tcbcolframe,
			boxrule=0pt,
		},
	underlay boxed title={%
			\path[fill=tcbcolframe] (title.south west)--(title.south east)
			to[out=0, in=180] ([xshift=5mm]title.east)--
			(title.center-|frame.east)
			[rounded corners=\kvtcb@arc] |-
			(frame.north) -| cycle;
		},
	#1
}{th}
\newtcbtheorem[use counter=commonbox]{outil}{Outil }%
{
	enhanced,
	colback=white,
	colframe=mygr,
	attach boxed title to top left={yshift*=-\tcboxedtitleheight},
	fonttitle=\bfseries,
	title={#2},
	boxed title size=title,
	boxed title style={%
			sharp corners,
			rounded corners=northwest,
			colback=tcbcolframe,
			boxrule=0pt,
		},
	underlay boxed title={%
			\path[fill=tcbcolframe] (title.south west)--(title.south east)
			to[out=0, in=180] ([xshift=5mm]title.east)--
			(title.center-|frame.east)
			[rounded corners=\kvtcb@arc] |-
			(frame.north) -| cycle;
		},
	#1
}{th}
\newtcbtheorem[use counter=commonbox]{but}{Buts du chapitre }%
{
	enhanced,
	colback=white,
	colframe=mygr,
	attach boxed title to top left={yshift*=-\tcboxedtitleheight},
	fonttitle=\bfseries,
	title={#2},
	boxed title size=title,
	boxed title style={%
			sharp corners,
			rounded corners=northwest,
			colback=tcbcolframe,
			boxrule=0pt,
		},
	underlay boxed title={%
			\path[fill=tcbcolframe] (title.south west)--(title.south east)
			to[out=0, in=180] ([xshift=5mm]title.east)--
			(title.center-|frame.east)
			[rounded corners=\kvtcb@arc] |-
			(frame.north) -| cycle;
		},
	#1
}{th}
\newtcbtheorem[use counter=commonbox]{propriete}{Propriété }%
{
	enhanced,
	colback=white,
	colframe=mygr,
	attach boxed title to top left={yshift*=-\tcboxedtitleheight},
	fonttitle=\bfseries,
	title={#2},
	boxed title size=title,
	boxed title style={%
			sharp corners,
			rounded corners=northwest,
			colback=tcbcolframe,
			boxrule=0pt,
		},
	underlay boxed title={%
			\path[fill=tcbcolframe] (title.south west)--(title.south east)
			to[out=0, in=180] ([xshift=5mm]title.east)--
			(title.center-|frame.east)
			[rounded corners=\kvtcb@arc] |-
			(frame.north) -| cycle;
		},
	#1
}{th}
\newtcbtheorem[number within=commonbox]{definition}{Définition }%
{
	enhanced,
	colback=white,
	colframe=mygr,
	attach boxed title to top left={yshift*=-\tcboxedtitleheight},
	fonttitle=\bfseries,
	title={#2},
	boxed title size=title,
	boxed title style={%
			sharp corners,
			rounded corners=northwest,
			colback=tcbcolframe,
			boxrule=0pt,
		},
	underlay boxed title={%
			\path[fill=tcbcolframe] (title.south west)--(title.south east)
			to[out=0, in=180] ([xshift=5mm]title.east)--
			(title.center-|frame.east)
			[rounded corners=\kvtcb@arc] |-
			(frame.north) -| cycle;
		},
	#1
}{th}
\newtcbtheorem[number within=commonbox]{exemples}{Exemples }%
{
	enhanced,
	colback=white,
	colframe=mygr,
	attach boxed title to top left={yshift*=-\tcboxedtitleheight},
	fonttitle=\bfseries,
	title={#2},
	boxed title size=title,
	boxed title style={%
			sharp corners,
			rounded corners=northwest,
			colback=tcbcolframe,
			boxrule=0pt,
		},
	underlay boxed title={%
			\path[fill=tcbcolframe] (title.south west)--(title.south east)
			to[out=0, in=180] ([xshift=5mm]title.east)--
			(title.center-|frame.east)
			[rounded corners=\kvtcb@arc] |-
			(frame.north) -| cycle;
		},
	#1
}{th}
\newtcbtheorem[number within=commonbox]{exemple}{Exemple }%
{
	enhanced,
	colback=white,
	colframe=mygr,
	attach boxed title to top left={yshift*=-\tcboxedtitleheight},
	fonttitle=\bfseries,
	title={#2},
	boxed title size=title,
	boxed title style={%
			sharp corners,
			rounded corners=northwest,
			colback=tcbcolframe,
			boxrule=0pt,
		},
	underlay boxed title={%
			\path[fill=tcbcolframe] (title.south west)--(title.south east)
			to[out=0, in=180] ([xshift=5mm]title.east)--
			(title.center-|frame.east)
			[rounded corners=\kvtcb@arc] |-
			(frame.north) -| cycle;
		},
	#1
}{th}
\newtcbtheorem[number within=commonbox]{questions}{Questions guidantes }%
{
	enhanced,
	colback=white,
	colframe=mygr,
	attach boxed title to top left={yshift*=-\tcboxedtitleheight},
	fonttitle=\bfseries,
	title={#2},
	boxed title size=title,
	boxed title style={%
			sharp corners,
			rounded corners=northwest,
			colback=tcbcolframe,
			boxrule=0pt,
		},
	underlay boxed title={%
			\path[fill=tcbcolframe] (title.south west)--(title.south east)
			to[out=0, in=180] ([xshift=5mm]title.east)--
			(title.center-|frame.east)
			[rounded corners=\kvtcb@arc] |-
			(frame.north) -| cycle;
		},
	#1
}{th}
\makeatother

% corps
\newcommand{\R}{\mathbb{R}}
\newcommand{\Rnn}{\mathbb{R}^{2n}}
\newcommand{\Z}{\mathbb{Z}}
\newcommand{\N}{\mathbb{N}}
\newcommand{\Q}{\mathbb{Q}}

% domain
\newcommand{\D}{\mathcal{D}}
% for calligraphic C
\usepackage{calrsfs}
\newcommand{\C}{\mathcal{C}}

% date
\usepackage{advdate}

% ensembles tq. 
\newcommand{\xRtq}[1]{
	$\left\{ x \in \R \text{ tq. } #1 \right\}$
}

% vabs
\newcommand{\vabs}[1]{
	\left| #1 \right|
}

%pinfty minfty
\newcommand{\pinfty}{{+}\infty}
\newcommand{\minfty}{{-}\infty}

% plots
\usepackage{pgfplots}

%virgules
\usepackage{icomma}
\pgfplotsset{/pgf/number format/use comma}

%subfigures
\usepackage{subcaption}

%hyperlink footnote
\usepackage{hyperref}

%wider tabulars
\def\arraystretch{2}
\setlength\tabcolsep{15pt}

% tableaux var, signe
\usepackage{tkz-tab}

\AdvanceDate[1]

\begin{document}
\pagestyle{fancy}
\fancyhead[L]{Première spécifique}
\fancyhead[C]{\textbf{Fonctions affines}}
\fancyhead[R]{\today}

\exe{}{
	On considère trois suites arithmétiques.
	Compléter le nuage de points et déterminer leur expression algébrique.
	\begin{center}
	\begin{tikzpicture}[>=stealth, scale=1.1]
		\begin{axis}[xmin = 0, xmax=5.5, ymin=-2.8, ymax=4.8, axis x line=middle, axis y line=middle, axis line style=->, xlabel={rang $n$}, x label style={anchor=north}, grid=both, ytick distance=1, xtick distance = 1, x=2cm]
			\addplot[black, thick, only marks, mark=*] coordinates {(0,1) (4,3)}
				node[pos=0, left=25pt] {$v$};
			
			\addplot[black, thick, only marks, mark=star] coordinates {(0,4.5) (2,2.5)}
				node[pos=0, left=25pt] {$u$};
			
			\addplot[black, thick, only marks, mark=square] coordinates {(0,-1.5) (1,-1.5)}
				node[pos=0, left=25pt] {$w$};
		\end{axis}
	\end{tikzpicture}
	\end{center}
	
	\begin{align*}
		u(n) = &&&& v(n) = &&&& w(n) = 
	\end{align*}
}{exe:graph}{
	\begin{center}
	\begin{tikzpicture}[>=stealth, scale=1.1]
		\begin{axis}[xmin = 0, xmax=5.5, ymin=-2.8, ymax=4.8, axis x line=middle, axis y line=middle, axis line style=->, xlabel={rang $n$}, x label style={anchor=north}, grid=both, ytick distance=1, xtick distance = 1, x=2cm]
			\addplot[black, thick, only marks, mark=*] coordinates {(0,1) (1,1.5) (2,2) (3,2.5) (4,3) (5,3.5)}
				node[pos=0, left=25pt] {$v$};
			
			\addplot[black, thick, only marks, mark=star] coordinates {(0,4.5) (1, 3.5) (2,2.5) (3,1.5) (4,0.5)}
				node[pos=0, left=25pt] {$u$};
			
			\addplot[black, thick, only marks, mark=square] coordinates {(0,-1.5) (1,-1.5) (2,-1.5) (3,-1.5) (4,-1.5) (5,-1.5)}
				node[pos=0, left=25pt] {$w$};
		\end{axis}
	\end{tikzpicture}
	\end{center}
	
	\begin{align*}
		u(n) = 4,5 - n  &&&& v(n) = 1 + 0,5n &&&& w(n) = -1,5
	\end{align*}
}

\exe{}{
	Un professeur hésite entre louer un appartement ou l'acheter.
	\begin{itemize}
		\item
		D'une part, le loyer mensuel de l'appartement est de $700$€, avec un coût initial (la caution) fixé à deux loyers, soit $1\ 400$€.
		\item
		D'autre part, à l'achat, cet appartement coûte $140~000$€.
	\end{itemize}
	
	Dans la première alternative, lorsque l'appartement est loué, on dénote par $u(n)$ le montant total payé en euros par le locataire après $n$ mois (et ceci pour tout $n\in\N$).
	Ainsi, par exemple, après $1$ mois, un seul loyer a été payé en plus de la caution initiale. D'où $u(1) = 2~100$.
	\begin{enumerate}
		\item Donner le terme initial $u(0)$.
		\item Donner $u(2)$, le montant payé par le locataire après $2$ mois.
		\item Justifier du caractère arithmétique de la suite $u$. Quelle est sa raison ?
		\item Pour tout $n\in\N$, donner $u(n)$ en fonction de $n$ sans justifier.
		\item À partir de combien de mois le locataire aura-t-il payé plus que le coût de l'appartement à l'achat ? Convertir le résultat en années.
	\end{enumerate}
}{exe:loyer-achat}{
	\begin{enumerate}
		\item $u(0) = 1~400$.
		\item $u(2) = u(1) + 700 = 2~100 + 700 = 2~800$.
		\item D'un mois à l'autre, on ajoute 700 au montant total payé.
		La raison est donc de 700.
		\item D'après le cours, $u(n) = 1400 + 700n = 700(2+n)$.
		\item 
		On pose et on résoud $u(n) = 140~000$.
			\begin{align*}
				u(n) &= 140~000 \\
				1400 + 700n &= 140~000 \\
				700(2+n) &= 140~000 \\
				2+n &=200 \\
				n &= 198
			\end{align*}
		Par conséquent, au bout de 198 mois, le locataire aura payé le coût de l'appartement à l'achat.
		En divisant par 12, on convertit le résultat en $\frac{198}{12} = \frac{99}{6} = \frac{33}2 = 16,5$ années.
	\end{enumerate}
}

\exe{}{
	Le mathématicien français Abraham de Moivre, alors âgé de plus que 80 ans, étudia dans les années 1750 la durée de son sommeil ; il remarqua que celle-ci augmentait de façon inquiétante.
	Il prit donc note chaque jour de son heure de coucher et de réveil et compara avec une valeur initiale : la nuit $0$, il dormit $8$ heures.
	
	Au fil des jours, il compta que son temps de sommeil augmentait chaque nuit de $15$ minutes, soit $\frac14$ d'heure.
	En notant $v(n)$ la durée de son sommeil en heures lors de sa $n$-ième nuit, il put alors calculer la nuit de la mort : celle où il dormirait $24$ heures.
	\begin{enumerate}
		\item Donner le terme initial $v(0)$.
		\item Donner $v(4)$, la durée de son sommeil en heures lors de la $4$-ème nuit.
		\item Pourquoi la suite $v$ est-elle arithmétique ? Quelle est sa raison ?
		\item Pour tout $n\in\N$, donner $v(n)$ en fonction de $n$ sans justifier.
		\item Quelle nuit Abraham de Moivre mourut-t-il ?
	\end{enumerate}
}{exe:Moivre}{
	\begin{enumerate}
		\item $v(0) = 8$, car on compte en heures.
		\item $v(4) = 8 + \frac14 \times 4 = 9$.
		\item D'une nuit à l'autre, de Moivre dort $\frac14$ d'heure en plus.
		La suite est donc arithmétique de raison $\frac14$.
		\item D'après le cours, $v(n) = 8 + \frac14 n$.
		\item 
		On pose et on résoud $v(n) = 24$.
			\begin{align*}
				v(n) &= 24 \\
				8 + \frac14 n &= 24 \\
				\frac14n &= 16 \\
				n &= 4\times16 = 48
			\end{align*}
		On conclut que de Moivre meurt la 48-ème nuit après le début de son étude.
	\end{enumerate}
}



%%%%%%%%%%%%

\newpage
\fancyhead[C]{\textbf{Solutions}}
\shipoutAnswer

\end{document}
