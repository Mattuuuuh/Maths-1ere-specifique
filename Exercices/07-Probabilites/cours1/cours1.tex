\documentclass[a4paper, 12pt]{extarticle}

\usepackage[utf8x]{inputenc}
%fonts
\usepackage{libertinus,libertinust1math}
\usepackage{amsmath,amsthm,amssymb,mathtools}

% SOLUTION SWITCH

\ifsolutions
	\newcommand{\exe}[2]{
		\begin{ex} #1  \end{ex}
		\begin{sol} #2 \end{sol}
	}
\else
	\newcommand{\exe}[2]{
		\begin{ex} #1  \end{ex}
	}
	
\fi


\usepackage[french]{babel}
\usepackage[
a4paper,
margin=2cm,
nomarginpar,% We don't want any margin paragraphs
]{geometry}

% HEADER, ARRAY, ENUM, MULTIOCL
\usepackage{fancyhdr}
\usepackage{array}
\usepackage{multicol, enumitem}
\newcolumntype{P}[1]{>{\centering\arraybackslash}p{#1}}
\usepackage{stackengine}
\newcommand\xrowht[2][0]{\addstackgap[.5\dimexpr#2\relax]{\vphantom{#1}}}

% theorems

\theoremstyle{theorem}
\newtheorem{thm}{Théorème}
\theoremstyle{plain}
\newtheorem*{sol}{Solution}
\theoremstyle{definition}
\newtheorem{ex}{Exercice}
\newtheorem{dfn}{Définition}
\newtheorem*{dfn*}{Définition}


%couleurs
\usepackage{tcolorbox}
\definecolor{myg}{RGB}{56, 140, 70}
\definecolor{myb}{RGB}{45, 111, 177}
\definecolor{myr}{RGB}{199, 68, 64}
\definecolor{mygr}{HTML}{2C3338}


\tcbuselibrary{theorems,skins,hooks}
\newcounter{commonbox}
\makeatletter
\newtcbtheorem[use counter=commonbox]{theorem}{Théorème }%
{
	enhanced,
	colback=white,
	colframe=mygr,
	attach boxed title to top left={yshift*=-\tcboxedtitleheight},
	fonttitle=\bfseries,
	title={#2},
	boxed title size=title,
	boxed title style={%
			sharp corners,
			rounded corners=northwest,
			colback=tcbcolframe,
			boxrule=0pt,
		},
	underlay boxed title={%
			\path[fill=tcbcolframe] (title.south west)--(title.south east)
			to[out=0, in=180] ([xshift=5mm]title.east)--
			(title.center-|frame.east)
			[rounded corners=\kvtcb@arc] |-
			(frame.north) -| cycle;
		},
	#1
}{th}
\newtcbtheorem[use counter=commonbox]{rappel}{Rappel }%
{
	enhanced,
	colback=white,
	colframe=mygr,
	attach boxed title to top left={yshift*=-\tcboxedtitleheight},
	fonttitle=\bfseries,
	title={#2},
	boxed title size=title,
	boxed title style={%
			sharp corners,
			rounded corners=northwest,
			colback=tcbcolframe,
			boxrule=0pt,
		},
	underlay boxed title={%
			\path[fill=tcbcolframe] (title.south west)--(title.south east)
			to[out=0, in=180] ([xshift=5mm]title.east)--
			(title.center-|frame.east)
			[rounded corners=\kvtcb@arc] |-
			(frame.north) -| cycle;
		},
	#1
}{th}
\newtcbtheorem[use counter=commonbox]{strategie}{Stratégie }%
{
	enhanced,
	colback=white,
	colframe=mygr,
	attach boxed title to top left={yshift*=-\tcboxedtitleheight},
	fonttitle=\bfseries,
	title={#2},
	boxed title size=title,
	boxed title style={%
			sharp corners,
			rounded corners=northwest,
			colback=tcbcolframe,
			boxrule=0pt,
		},
	underlay boxed title={%
			\path[fill=tcbcolframe] (title.south west)--(title.south east)
			to[out=0, in=180] ([xshift=5mm]title.east)--
			(title.center-|frame.east)
			[rounded corners=\kvtcb@arc] |-
			(frame.north) -| cycle;
		},
	#1
}{th}
\newtcbtheorem[use counter=commonbox]{outil}{Outil }%
{
	enhanced,
	colback=white,
	colframe=mygr,
	attach boxed title to top left={yshift*=-\tcboxedtitleheight},
	fonttitle=\bfseries,
	title={#2},
	boxed title size=title,
	boxed title style={%
			sharp corners,
			rounded corners=northwest,
			colback=tcbcolframe,
			boxrule=0pt,
		},
	underlay boxed title={%
			\path[fill=tcbcolframe] (title.south west)--(title.south east)
			to[out=0, in=180] ([xshift=5mm]title.east)--
			(title.center-|frame.east)
			[rounded corners=\kvtcb@arc] |-
			(frame.north) -| cycle;
		},
	#1
}{th}
\newtcbtheorem[use counter=commonbox]{but}{Buts du chapitre }%
{
	enhanced,
	colback=white,
	colframe=mygr,
	attach boxed title to top left={yshift*=-\tcboxedtitleheight},
	fonttitle=\bfseries,
	title={#2},
	boxed title size=title,
	boxed title style={%
			sharp corners,
			rounded corners=northwest,
			colback=tcbcolframe,
			boxrule=0pt,
		},
	underlay boxed title={%
			\path[fill=tcbcolframe] (title.south west)--(title.south east)
			to[out=0, in=180] ([xshift=5mm]title.east)--
			(title.center-|frame.east)
			[rounded corners=\kvtcb@arc] |-
			(frame.north) -| cycle;
		},
	#1
}{th}
\newtcbtheorem[use counter=commonbox]{propriete}{Propriété }%
{
	enhanced,
	colback=white,
	colframe=mygr,
	attach boxed title to top left={yshift*=-\tcboxedtitleheight},
	fonttitle=\bfseries,
	title={#2},
	boxed title size=title,
	boxed title style={%
			sharp corners,
			rounded corners=northwest,
			colback=tcbcolframe,
			boxrule=0pt,
		},
	underlay boxed title={%
			\path[fill=tcbcolframe] (title.south west)--(title.south east)
			to[out=0, in=180] ([xshift=5mm]title.east)--
			(title.center-|frame.east)
			[rounded corners=\kvtcb@arc] |-
			(frame.north) -| cycle;
		},
	#1
}{th}
\newtcbtheorem[number within=commonbox]{definition}{Définition }%
{
	enhanced,
	colback=white,
	colframe=mygr,
	attach boxed title to top left={yshift*=-\tcboxedtitleheight},
	fonttitle=\bfseries,
	title={#2},
	boxed title size=title,
	boxed title style={%
			sharp corners,
			rounded corners=northwest,
			colback=tcbcolframe,
			boxrule=0pt,
		},
	underlay boxed title={%
			\path[fill=tcbcolframe] (title.south west)--(title.south east)
			to[out=0, in=180] ([xshift=5mm]title.east)--
			(title.center-|frame.east)
			[rounded corners=\kvtcb@arc] |-
			(frame.north) -| cycle;
		},
	#1
}{th}
\newtcbtheorem[number within=commonbox]{exemples}{Exemples }%
{
	enhanced,
	colback=white,
	colframe=mygr,
	attach boxed title to top left={yshift*=-\tcboxedtitleheight},
	fonttitle=\bfseries,
	title={#2},
	boxed title size=title,
	boxed title style={%
			sharp corners,
			rounded corners=northwest,
			colback=tcbcolframe,
			boxrule=0pt,
		},
	underlay boxed title={%
			\path[fill=tcbcolframe] (title.south west)--(title.south east)
			to[out=0, in=180] ([xshift=5mm]title.east)--
			(title.center-|frame.east)
			[rounded corners=\kvtcb@arc] |-
			(frame.north) -| cycle;
		},
	#1
}{th}
\newtcbtheorem[number within=commonbox]{exemple}{Exemple }%
{
	enhanced,
	colback=white,
	colframe=mygr,
	attach boxed title to top left={yshift*=-\tcboxedtitleheight},
	fonttitle=\bfseries,
	title={#2},
	boxed title size=title,
	boxed title style={%
			sharp corners,
			rounded corners=northwest,
			colback=tcbcolframe,
			boxrule=0pt,
		},
	underlay boxed title={%
			\path[fill=tcbcolframe] (title.south west)--(title.south east)
			to[out=0, in=180] ([xshift=5mm]title.east)--
			(title.center-|frame.east)
			[rounded corners=\kvtcb@arc] |-
			(frame.north) -| cycle;
		},
	#1
}{th}
\newtcbtheorem[number within=commonbox]{questions}{Questions guidantes }%
{
	enhanced,
	colback=white,
	colframe=mygr,
	attach boxed title to top left={yshift*=-\tcboxedtitleheight},
	fonttitle=\bfseries,
	title={#2},
	boxed title size=title,
	boxed title style={%
			sharp corners,
			rounded corners=northwest,
			colback=tcbcolframe,
			boxrule=0pt,
		},
	underlay boxed title={%
			\path[fill=tcbcolframe] (title.south west)--(title.south east)
			to[out=0, in=180] ([xshift=5mm]title.east)--
			(title.center-|frame.east)
			[rounded corners=\kvtcb@arc] |-
			(frame.north) -| cycle;
		},
	#1
}{th}
\makeatother

% corps
\newcommand{\R}{\mathbb{R}}
\newcommand{\Rnn}{\mathbb{R}^{2n}}
\newcommand{\Z}{\mathbb{Z}}
\newcommand{\N}{\mathbb{N}}
\newcommand{\Q}{\mathbb{Q}}

% domain
\newcommand{\D}{\mathcal{D}}
% for calligraphic C
\usepackage{calrsfs}
\newcommand{\C}{\mathcal{C}}

% date
\usepackage{advdate}

% ensembles tq. 
\newcommand{\xRtq}[1]{
	$\left\{ x \in \R \text{ tq. } #1 \right\}$
}

% vabs
\newcommand{\vabs}[1]{
	\left| #1 \right|
}

%pinfty minfty
\newcommand{\pinfty}{{+}\infty}
\newcommand{\minfty}{{-}\infty}

% plots
\usepackage{pgfplots}

%virgules
\usepackage{icomma}
\pgfplotsset{/pgf/number format/use comma}

%subfigures
\usepackage{subcaption}

%hyperlink footnote
\usepackage{hyperref}

%wider tabulars
\def\arraystretch{2}
\setlength\tabcolsep{15pt}

% tableaux var, signe
\usepackage{tkz-tab}

\SetDate[07/01/2026]

\begin{document}
\pagestyle{fancy}
\fancyhead[L]{Première spécifique}
\fancyhead[C]{\textbf{Chapitre 6 — Probabilités conditionnelles}}
\fancyhead[R]{\today}

\newcommand{\hide}[1]{\quad\phantom{#1}\quad}
% (un)comment below for completion
% cannot renew phantom itself otherwise multicols environment shows a "p" in the right margin idk why
\renewcommand{\hide}[1]{{\color{RED_E!80!BLACK}#1}}



%\section{Événements conditionnels}

\begin{definition}[événement conditionnel]
	Soient $A, B$ deux événements. L'événement \mbox{« $B$ sachant $A$ »}, aussi noté $B|A$
	correspond à l'événement associé à $B$ dans le nouvel univers où \hide{\mbox{on sait que $A$ se réalise.}}
\end{definition}



\subsection*{Tableaux croisés}

\begin{exemple}\label{ex:2}
	On interroge $500$ personnes pour savoir si elles sont allées chez le dentiste en 2025. 
	
	\begin{center}
	\def\arraystretch{1.6}
	\setlength\tabcolsep{25pt}
	\tableaucroise{Dentiste & Pas dentiste & Total}{Enfant & 140 & 60 & 200}{Adulte & 260 & \hide{40} & 300}{Total & \hide{400} & 100 & 500}
	\end{center}
	On choisit une personne uniformément au hasard
	et on considère les événements $D$ : « la personne choisie est allée chez le dentiste » et $E$ : « la personne choisie est un enfant ».
	\begin{align*}
		P(D) = \hide{\dfrac{400}{500} = 0,8} && && &&  P\left(\overline{D}\right) &= \hide{ 1 - P(D) = 1 - 0,8 = 0,2} && 
		\\
		P(E) = \hide{\dfrac{200}{500} = 0,4} && && &&  P(D \cap E) &= \hide{\dfrac{140}{500} = 0,28} && 
		\\
		P(D \sct E) = \hide{\dfrac{140}{200} = 0,7} && && && P(E \sct D) &= \hide{\dfrac{140}{400} = 0,35} &&
	\end{align*}
\end{exemple}

\begin{proprietes}\,

	\begin{enumerate}[label=$\bullet$]
		\item On voit la fréquence d'un événement comme sa probabilité.
		\item Pour lire la fréquence conditionnelle, \hide{on se restreint à une colonne ou une ligne.}
	\end{enumerate}
\end{proprietes}


\subsection*{Indépendance et corrélation}

% même pas important lol
% 
%\begin{definition}[probabilité conditionnelle]
%		%\[ P(B \sct A) = \dfrac{P(A \cap B)}{P(A)} \]
%		\begin{align*}
%			P(A \cap B) = P(B \sct A) \times P(A) && \iff && P(B \sct A) = \dfrac{P(A \cap B)}{P(A)} 
%		\end{align*}
%\end{definition}



%\begin{theorem}[de Bayes]
%	Soient $A, B$ deux événements. Alors
%		\[ P(A \sct B) = \dfrac{P(A)}{P(B)} \times P(B \sct A). \]
%\end{theorem}


\begin{definition}%[événements indépendants]
	Les événements $A$ et $B$ sont \emph{indépendants} si
		\begin{align*}
			P(B \sct A) = P(B) && \iff && \hide{P(A \sct B) = P(A)} && \iff && \hide{P(A\cap B) = P(A)\times P(B)}
		\end{align*}
	On dira aussi que $A$ et $B$ sont \hide{\emph{décorrélés}, ou de \emph{corrélation nulle}.}
	
\end{definition}

\begin{remarque}[corrélation] \,

	\begin{enumerate}[label=$\bullet$]
		\item Si $P(B \sct A) > P(B)$, savoir que $A$ se réalise augmente la probabilité que $B$ se réalise. Les événements sont \hide{positivement corrélés.}
		\item Si $P(B \sct A) < P(B)$, savoir que $A$ se réalise diminue la probabilité que $B$ se réalise. Les événements sont \hide{négativement corrélés.}·
	\end{enumerate}
\end{remarque}



\begin{exemple}
	Les événements $D$ et $E$ de l'exemple \ref{ex:2} sont \hide{négativement corrélés car}
		\[ \hide{0,7} = P(D \sct E) \hide{\quad<\quad} P(D) = \hide{0,8} \]
	%ou car
	%	\[ 0,35 = P(E \sct D) < P(E) = 0,4. \]
	Interprétations : 
		\begin{itemize}
			\item \hide{Savoir qu'on a choisit un enfant diminue la probabilité que la peronne choisie soit allée chez le dentiste.}
			\item \hide{Savoir que la personne choisie est allée chez le dentiste diminue la probabilité qu'elle soit un enfant.}
		\end{itemize}
\end{exemple}

\newpage


\subsection*{Arbres de probabilité}


\begin{proprietes}\,

	\begin{enumerate}[label=$\bullet$]
		\item Les probabilités de l'arbre sont conditionnées par le chemin déjà parcouru.
		\item Pour chaque nœud, la somme des probabilités des sous-branches vaut $1$.
		\item Chaque feuille est une issue (et vice versa). Pour connaître la probabilité d'une issue, il faut multiplier les probabilités le long du chemin racine-feuille.
		\item Un événement est un ensemble d'issues. Pour connaître la probabilité d'un événement, il faut sommer la probabilité de chaque feuille lui correspondant.
	\end{enumerate}
\end{proprietes}



\begin{exemple}
	%On considère un arbre modélisant une expérience aléatoire à deux épreuves.
	%$A$ et $\overline{A}$ sont les deux issues de la première épreuve. Chaque feuille correspond à une issue finale.
	L'arbre suivant modélise une expérience à deux épreuves. On sait que
	\begin{multicols}{3}
	\begin{enumerate}[label=$\bullet$]
		\item $P(A) = 0,4$
		\item $P(B \sct A) = 0,75$
		\item $P(B \sct \overline{A}) = 0,7$
	\end{enumerate}
	\end{multicols}
	\begin{center}
	\begin{tikzpicture}
		% depth 1
		\foreach \i in {-3, 3}
		\draw[-, thick, black] (0,0) node {$\bullet$} -- (\i,-2);
		
		\draw (0,0);% node[above] {racine};
		\draw (-3,-2) node[above left] {$A$};
		\draw (3,-2) node[above right] {$\overline{A}$};
		% depth 2
		\foreach \i in {-3, 3} \foreach \j in {-1, 1}
			\draw[-, thick, black] (\i,-2) node {$\bullet$} -- (\i+\j,-4) node {$\bullet$};
			
		\draw (-4,-4) node[below] {$A\cap B$};
		\draw (-2,-4) node[below] {$A\cap \overline{B}$};
		\draw (2,-4) node[below] {$\overline{A}\cap B$};
		\draw (4,-4) node[below] {$\overline{A}\cap \overline{B}$};
		
		% sols
		\draw (1.5,-.5) node {\hide{$0,6$}};
		\draw (-1.5,-.5) node {\hide{$0,4$}};
		
		\draw (4,-3) node {\hide{$0,3$}};
		\draw (2,-3) node {\hide{$0,7$}};
		
		\draw (-2,-3) node {\hide{$0,25$}};
		\draw (-4,-3) node {\hide{$0,75$}};
		
		
		\draw (5,-3) node[right] {\thead{probabilités \\ conditionnelles \\ sachant $\overline{A}$}};
		\draw (-7.5,-3) node[right] {\thead{probabilités \\ conditionnelles \\ sachant $A$}};
		\draw[->, thick] (5,-3) -- (4.5,-3);
		\draw[dashed] (1.55,-3.35) rectangle (4.45,-2.65);
		\draw[->, thick] (-5,-3) -- (-4.5,-3);
		\draw[dashed] (-4.45,-3.35) rectangle (-1.55,-2.65);
	\end{tikzpicture}
	\end{center}
\end{exemple}

\begin{exemple}

	Un professeur décide de lancer une pièce bien équilibrée $8$ fois de suite :
	« Si j'obtiens pile au moins une fois, alors Clémentine change de place ! »
	
	Les lancers sont supposés indépendants : un résultat n'influe pas les autres.
	
	\begin{enumerate}
		\item Décrire l'événement complémentaire avec des mots.
		\item Esquisser un arbre de probabilité (pas forcément complet) et donner la probabilité de chacune des issues de l'expérience.
		\item En calculant d'abord la probabilité de l'événement complémentaire, donner la probabilité que Clémentine change de place.
%		\item Calculer la probabilité de l'événement
%		%
%			\begin{center} « obtenir pile exactement 1 fois » \end{center}
%			
%		en comptant le nombre de chemins racine-feuille qui y correspondent.
%		
%		\item En déduire la probabilité de l'événement
%			%
%			\begin{center} « obtenir pile au moins 2 fois » \end{center}
	\end{enumerate}
	
	\hide{
	\begin{enumerate}
		\item 
		L'événement complémentaire (ou contraire) est donné par : « ne jamais obtenir pile en 8 lancers. »
		\item 
		Il n'est utile que de considérer le chemin de l'arbre correspondant à 8 « face » d'affilé.
		\item 
		En notant $E$ l'événement « obtenir pile au moins une fois », alors
			\[ P\bigl(\overline{E}\bigr) = \left(\dfrac12\right)^8, \]
		et donc
			\[ P(E) = 1 - \left(\dfrac12\right)^8. \]
	\end{enumerate}
	}
\end{exemple}


\end{document}
