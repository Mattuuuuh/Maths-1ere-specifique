\documentclass[a4paper, 12pt]{extarticle}

\usepackage[utf8x]{inputenc}
%fonts
\usepackage{libertinus,libertinust1math}
\usepackage{amsmath,amsthm,amssymb,mathtools}

% SOLUTION SWITCH

\ifsolutions
	\newcommand{\exe}[2]{
		\begin{ex} #1  \end{ex}
		\begin{sol} #2 \end{sol}
	}
\else
	\newcommand{\exe}[2]{
		\begin{ex} #1  \end{ex}
	}
	
\fi


\usepackage[french]{babel}
\usepackage[
a4paper,
margin=2cm,
nomarginpar,% We don't want any margin paragraphs
]{geometry}

% HEADER, ARRAY, ENUM, MULTIOCL
\usepackage{fancyhdr}
\usepackage{array}
\usepackage{multicol, enumitem}
\newcolumntype{P}[1]{>{\centering\arraybackslash}p{#1}}
\usepackage{stackengine}
\newcommand\xrowht[2][0]{\addstackgap[.5\dimexpr#2\relax]{\vphantom{#1}}}

% theorems

\theoremstyle{theorem}
\newtheorem{thm}{Théorème}
\theoremstyle{plain}
\newtheorem*{sol}{Solution}
\theoremstyle{definition}
\newtheorem{ex}{Exercice}
\newtheorem{dfn}{Définition}
\newtheorem*{dfn*}{Définition}


%couleurs
\usepackage{tcolorbox}
\definecolor{myg}{RGB}{56, 140, 70}
\definecolor{myb}{RGB}{45, 111, 177}
\definecolor{myr}{RGB}{199, 68, 64}
\definecolor{mygr}{HTML}{2C3338}


\tcbuselibrary{theorems,skins,hooks}
\newcounter{commonbox}
\makeatletter
\newtcbtheorem[use counter=commonbox]{theorem}{Théorème }%
{
	enhanced,
	colback=white,
	colframe=mygr,
	attach boxed title to top left={yshift*=-\tcboxedtitleheight},
	fonttitle=\bfseries,
	title={#2},
	boxed title size=title,
	boxed title style={%
			sharp corners,
			rounded corners=northwest,
			colback=tcbcolframe,
			boxrule=0pt,
		},
	underlay boxed title={%
			\path[fill=tcbcolframe] (title.south west)--(title.south east)
			to[out=0, in=180] ([xshift=5mm]title.east)--
			(title.center-|frame.east)
			[rounded corners=\kvtcb@arc] |-
			(frame.north) -| cycle;
		},
	#1
}{th}
\newtcbtheorem[use counter=commonbox]{rappel}{Rappel }%
{
	enhanced,
	colback=white,
	colframe=mygr,
	attach boxed title to top left={yshift*=-\tcboxedtitleheight},
	fonttitle=\bfseries,
	title={#2},
	boxed title size=title,
	boxed title style={%
			sharp corners,
			rounded corners=northwest,
			colback=tcbcolframe,
			boxrule=0pt,
		},
	underlay boxed title={%
			\path[fill=tcbcolframe] (title.south west)--(title.south east)
			to[out=0, in=180] ([xshift=5mm]title.east)--
			(title.center-|frame.east)
			[rounded corners=\kvtcb@arc] |-
			(frame.north) -| cycle;
		},
	#1
}{th}
\newtcbtheorem[use counter=commonbox]{strategie}{Stratégie }%
{
	enhanced,
	colback=white,
	colframe=mygr,
	attach boxed title to top left={yshift*=-\tcboxedtitleheight},
	fonttitle=\bfseries,
	title={#2},
	boxed title size=title,
	boxed title style={%
			sharp corners,
			rounded corners=northwest,
			colback=tcbcolframe,
			boxrule=0pt,
		},
	underlay boxed title={%
			\path[fill=tcbcolframe] (title.south west)--(title.south east)
			to[out=0, in=180] ([xshift=5mm]title.east)--
			(title.center-|frame.east)
			[rounded corners=\kvtcb@arc] |-
			(frame.north) -| cycle;
		},
	#1
}{th}
\newtcbtheorem[use counter=commonbox]{outil}{Outil }%
{
	enhanced,
	colback=white,
	colframe=mygr,
	attach boxed title to top left={yshift*=-\tcboxedtitleheight},
	fonttitle=\bfseries,
	title={#2},
	boxed title size=title,
	boxed title style={%
			sharp corners,
			rounded corners=northwest,
			colback=tcbcolframe,
			boxrule=0pt,
		},
	underlay boxed title={%
			\path[fill=tcbcolframe] (title.south west)--(title.south east)
			to[out=0, in=180] ([xshift=5mm]title.east)--
			(title.center-|frame.east)
			[rounded corners=\kvtcb@arc] |-
			(frame.north) -| cycle;
		},
	#1
}{th}
\newtcbtheorem[use counter=commonbox]{but}{Buts du chapitre }%
{
	enhanced,
	colback=white,
	colframe=mygr,
	attach boxed title to top left={yshift*=-\tcboxedtitleheight},
	fonttitle=\bfseries,
	title={#2},
	boxed title size=title,
	boxed title style={%
			sharp corners,
			rounded corners=northwest,
			colback=tcbcolframe,
			boxrule=0pt,
		},
	underlay boxed title={%
			\path[fill=tcbcolframe] (title.south west)--(title.south east)
			to[out=0, in=180] ([xshift=5mm]title.east)--
			(title.center-|frame.east)
			[rounded corners=\kvtcb@arc] |-
			(frame.north) -| cycle;
		},
	#1
}{th}
\newtcbtheorem[use counter=commonbox]{propriete}{Propriété }%
{
	enhanced,
	colback=white,
	colframe=mygr,
	attach boxed title to top left={yshift*=-\tcboxedtitleheight},
	fonttitle=\bfseries,
	title={#2},
	boxed title size=title,
	boxed title style={%
			sharp corners,
			rounded corners=northwest,
			colback=tcbcolframe,
			boxrule=0pt,
		},
	underlay boxed title={%
			\path[fill=tcbcolframe] (title.south west)--(title.south east)
			to[out=0, in=180] ([xshift=5mm]title.east)--
			(title.center-|frame.east)
			[rounded corners=\kvtcb@arc] |-
			(frame.north) -| cycle;
		},
	#1
}{th}
\newtcbtheorem[number within=commonbox]{definition}{Définition }%
{
	enhanced,
	colback=white,
	colframe=mygr,
	attach boxed title to top left={yshift*=-\tcboxedtitleheight},
	fonttitle=\bfseries,
	title={#2},
	boxed title size=title,
	boxed title style={%
			sharp corners,
			rounded corners=northwest,
			colback=tcbcolframe,
			boxrule=0pt,
		},
	underlay boxed title={%
			\path[fill=tcbcolframe] (title.south west)--(title.south east)
			to[out=0, in=180] ([xshift=5mm]title.east)--
			(title.center-|frame.east)
			[rounded corners=\kvtcb@arc] |-
			(frame.north) -| cycle;
		},
	#1
}{th}
\newtcbtheorem[number within=commonbox]{exemples}{Exemples }%
{
	enhanced,
	colback=white,
	colframe=mygr,
	attach boxed title to top left={yshift*=-\tcboxedtitleheight},
	fonttitle=\bfseries,
	title={#2},
	boxed title size=title,
	boxed title style={%
			sharp corners,
			rounded corners=northwest,
			colback=tcbcolframe,
			boxrule=0pt,
		},
	underlay boxed title={%
			\path[fill=tcbcolframe] (title.south west)--(title.south east)
			to[out=0, in=180] ([xshift=5mm]title.east)--
			(title.center-|frame.east)
			[rounded corners=\kvtcb@arc] |-
			(frame.north) -| cycle;
		},
	#1
}{th}
\newtcbtheorem[number within=commonbox]{exemple}{Exemple }%
{
	enhanced,
	colback=white,
	colframe=mygr,
	attach boxed title to top left={yshift*=-\tcboxedtitleheight},
	fonttitle=\bfseries,
	title={#2},
	boxed title size=title,
	boxed title style={%
			sharp corners,
			rounded corners=northwest,
			colback=tcbcolframe,
			boxrule=0pt,
		},
	underlay boxed title={%
			\path[fill=tcbcolframe] (title.south west)--(title.south east)
			to[out=0, in=180] ([xshift=5mm]title.east)--
			(title.center-|frame.east)
			[rounded corners=\kvtcb@arc] |-
			(frame.north) -| cycle;
		},
	#1
}{th}
\newtcbtheorem[number within=commonbox]{questions}{Questions guidantes }%
{
	enhanced,
	colback=white,
	colframe=mygr,
	attach boxed title to top left={yshift*=-\tcboxedtitleheight},
	fonttitle=\bfseries,
	title={#2},
	boxed title size=title,
	boxed title style={%
			sharp corners,
			rounded corners=northwest,
			colback=tcbcolframe,
			boxrule=0pt,
		},
	underlay boxed title={%
			\path[fill=tcbcolframe] (title.south west)--(title.south east)
			to[out=0, in=180] ([xshift=5mm]title.east)--
			(title.center-|frame.east)
			[rounded corners=\kvtcb@arc] |-
			(frame.north) -| cycle;
		},
	#1
}{th}
\makeatother

% corps
\newcommand{\R}{\mathbb{R}}
\newcommand{\Rnn}{\mathbb{R}^{2n}}
\newcommand{\Z}{\mathbb{Z}}
\newcommand{\N}{\mathbb{N}}
\newcommand{\Q}{\mathbb{Q}}

% domain
\newcommand{\D}{\mathcal{D}}
% for calligraphic C
\usepackage{calrsfs}
\newcommand{\C}{\mathcal{C}}

% date
\usepackage{advdate}

% ensembles tq. 
\newcommand{\xRtq}[1]{
	$\left\{ x \in \R \text{ tq. } #1 \right\}$
}

% vabs
\newcommand{\vabs}[1]{
	\left| #1 \right|
}

%pinfty minfty
\newcommand{\pinfty}{{+}\infty}
\newcommand{\minfty}{{-}\infty}

% plots
\usepackage{pgfplots}

%virgules
\usepackage{icomma}
\pgfplotsset{/pgf/number format/use comma}

%subfigures
\usepackage{subcaption}

%hyperlink footnote
\usepackage{hyperref}

%wider tabulars
\def\arraystretch{2}
\setlength\tabcolsep{15pt}

% tableaux var, signe
\usepackage{tkz-tab}

\SetDate[07/01/2026]

\begin{document}
\pagestyle{fancy}
\fancyhead[L]{Première spécifique}
\fancyhead[C]{\textbf{Probabilités 2 : arbres de probabilité}}
\fancyhead[R]{\today}

\exe{}{
	Compléter l'arbre sachant que 
		\begin{multicols}{3}
		\begin{enumerate}[label=$\bullet$]
			\item $P(A) = 0,3$
			\item $P(B \sct A) = 0,6$
			\item $P(B \sct \overline{A}) = 0,25$
		\end{enumerate}
		\end{multicols}
	\begin{center}
	\begin{tikzpicture}
		% depth 1
		\foreach \i in {-3, 3}
		\draw[-, thick, black] (0,0) node {$\bullet$} -- (\i,-1.5);
		% depth 2
		\foreach \i in {-3, 3} \foreach \j in {-1, 1}
			\draw[-, thick, black] (\i,-1.5) node {$\bullet$} -- (\i+\j,-3) node {$\bullet$};
			
		\draw (-3,-1.5) node[above left] {$A$};
		\draw (3,-1.5) node[above right] {$\overline{A}$};
			
		\draw (-4,-3) node[below] {$B\cap A$};
		\draw (2,-3) node[below] {$B\cap\overline{A}$};
		\draw (-2,-3) node[below] {$\overline{B}\cap A$};
		\draw (4,-3) node[below] {$\overline{B}\cap\overline{A}$};
	\end{tikzpicture}
	\end{center}
	Calculer $P(B\cap A)$ et $P(\overline{B}\cap\overline{A})$ en multipliant les probabilités des chemins racine-feuille correspondant.
}{exe:proba1}{
	\begin{center}
	\begin{tikzpicture}
		% depth 1
		\draw[-, thick, black] (0,0) node {$\bullet$} -- (-3,-1.5) node[pos=.5, above left] {$0,3$};
		\draw[-, thick, black] (0,0) node {$\bullet$} -- (3,-1.5) node[pos=.5, above right] {$0,7$};
		% depth 2		
		\draw[-, thick, black] (-3,-1.5) node {$\bullet$} -- (-4,-3) node[pos=.5, left] {$0,6$};
		\draw[-, thick, black] (-3,-1.5) node {$\bullet$} -- (-2,-3) node[pos=.5, right] {$0,4$};
		
		\draw[-, thick, black] (3,-1.5) node {$\bullet$} -- (2,-3) node[pos=.5, left] {$0,25$};
		\draw[-, thick, black] (3,-1.5) node {$\bullet$} -- (4,-3) node[pos=.5, right] {$0,75$};
		
		\draw (-3,-1.5) node[above left] {$A$};
		\draw (3,-1.5) node[above right] {$\overline{A}$};
			
		\draw (-4,-3) node[below] {$B\cap A$};
		\draw (2,-3) node[below] {$B\cap\overline{A}$};
		\draw (-2,-3) node[below] {$\overline{B}\cap A$};
		\draw (4,-3) node[below] {$\overline{B}\cap\overline{A}$};
	\end{tikzpicture}
	\end{center}
	
	\begin{align*}
		P(B\cap A) &= 0,3 \times 0,6 = 0,18 \\
		P(\overline{B}\cap\overline{A}) &= 0,7 \times 0,75 = 0,525
	\end{align*}
}


%\exe{}{
%	Compléter l'arbre correspondant à une expérience aléatoire à deux épreuves d'issues $\{A ; B ; C ; D\}$ et répondre aux questions suivantes.
%	\begin{center}
%	\begin{tikzpicture}[scale=.9]
%		% depth 1
%		\draw[-, thick, black] (0,0) node {$\bullet$} -- (3,-2) node[midway, above right] {};
%		\draw[-, thick, black] (0,0) node {$\bullet$} -- (-3,-2) node[midway, above left] {$0,7$};
%		% depth 2
%		\draw[-, thick, black] (-3,-2) node {$\bullet$} -- (-1,-4) node[midway, above right] {$\frac49$};
%		\draw[-, thick, black] (-3,-2) node {$\bullet$} -- (-3,-4);
%		\draw[-, thick, black] (-3,-2) node {$\bullet$} -- (-5,-4) node[midway, above left] {$\frac13$};
%		
%		\draw[-, thick, black] (3,-2) node {$\bullet$} -- (1,-4) node[midway, above left] {$\frac16$};
%		\draw[-, thick, black] (3,-2) node {$\bullet$} -- (3,-4) node[pos=.6, left] {$\frac12$};
%		\draw[-, thick, black] (3,-2) node {$\bullet$} -- (5,-4);
%		
%		\draw (1,-4) node {$\bullet$};
%		\draw (3,-4) node {$\bullet$};
%		\draw (5,-4) node {$\bullet$};
%		\draw (1,-4) node[below] {$D$};
%		\draw (3,-4) node[below] {$B$};
%		\draw (5,-4) node[below] {$C$};
%		
%		\draw (-1,-4) node {$\bullet$};
%		\draw (-3,-4) node {$\bullet$};
%		\draw (-5,-4) node {$\bullet$};
%		\draw (-1,-4) node[below] {$C$};
%		\draw (-3,-4) node[below] {$B$};
%		\draw (-5,-4) node[below] {$A$};
%	\end{tikzpicture}
%	\end{center}
%	
%	\begin{multicols}{2}
%	\begin{enumerate}
%		\item Calculer $P(D)$.
%		\item Calculer $P(B)$.
%		\item Calculer $P(D \cup B)$.
%		\item Calculer $P(A\cup C)$.
%	\end{enumerate}
%	\end{multicols}
%}{exe:proba3}{
%	La somme des probabilités de chaque sous-branche est toujours $1$. 
%	On complète donc l'arbre comme ci-dessous.
%	
%	\begin{center}
%	\begin{tikzpicture}
%		% depth 1
%		\draw[-, thick, black] (0,0) node {$\bullet$} -- (3,-2) node[midway, above right] {$0,3$};
%		\draw[-, thick, black] (0,0) node {$\bullet$} -- (-3,-2) node[midway, above left] {$0,7$};
%		% depth 2
%		\draw[-, thick, black] (-3,-2) node {$\bullet$} -- (-1,-4) node[midway, above right] {$\frac49$};
%		\draw[-, thick, black] (-3,-2) node {$\bullet$} -- (-3,-4) node[pos=.6, left] {$\frac29$};
%		\draw[-, thick, black] (-3,-2) node {$\bullet$} -- (-5,-4) node[midway, above left] {$\frac13$};
%		
%		\draw[-, thick, black] (3,-2) node {$\bullet$} -- (1,-4) node[midway, above left] {$\frac16$};
%		\draw[-, thick, black] (3,-2) node {$\bullet$} -- (3,-4) node[pos=.6, left] {$\frac12$};
%		\draw[-, thick, black] (3,-2) node {$\bullet$} -- (5,-4) node[midway, above right] {$\frac13$};
%		
%		\draw (1,-4) node {$\bullet$};
%		\draw (3,-4) node {$\bullet$};
%		\draw (5,-4) node {$\bullet$};
%		\draw (1,-4) node[below] {$D$};
%		\draw (3,-4) node[below] {$B$};
%		\draw (5,-4) node[below] {$C$};
%		
%		\draw (-1,-4) node {$\bullet$};
%		\draw (-3,-4) node {$\bullet$};
%		\draw (-5,-4) node {$\bullet$};
%		\draw (-1,-4) node[below] {$C$};
%		\draw (-3,-4) node[below] {$B$};
%		\draw (-5,-4) node[below] {$A$};
%	\end{tikzpicture}
%	\end{center}
%	
%	\begin{enumerate}
%		\item 
%			\begin{align*}
%				P(D) &= 0,3 \times \dfrac16 \\ &= \dfrac{0,3}{6} \\ &= \dfrac{0,1}{2} = 0,05.
%			\end{align*}
%		\item 
%			\begin{align*}
%				P(B) &= 0,7 \times \dfrac29 + 0,3 \times \dfrac12 \approx 0,31.
%			\end{align*}
%		\item 
%			Comme $D$ et $B$ sont deux issues distinctes de l'univers, on a
%			\begin{align*}
%				P(D \cup B) &= P(D) + P(B) \\ &\approx 0,05 + 0,31 = 0,36.
%			\end{align*}
%		\item On peut soit procéder comme ci-dessus, ou alors utiliser le fait que
%			\[ P(A\cup C) = P(A) + P(C) = 1 - \left( P(D) + P(B) \right) = 1 - P(D \cup B), \]
%		et donc
%			\[ P(A\cup C) \approx 0,64. \]
%	\end{enumerate}
%}

\exe{}{
	On dispose d'une pièce de monnaie truquée pour laquelle la probabilité d'obtenir pile lors d'un lancer est égale à $\frac14$.
	\begin{enumerate}
		\item
		Déterminer la probabilités d'obtenir face.
		\item
		On lance trois fois de suite cette pièce de monnaie, les trois lancers étant indépendants, et on note pour chaque lancer le résultat (pile ou face) obtenu.
		\begin{enumerate}[label=\alph*)]
			\item
			Représenter la situation par un arbre de probabilités.
			\item
			Quelle est la probabilité d'obtenir exactement une fois pile lors de ces trois lancers ?
			\item
			Quelle est la probabilité de ne jamais obtenir pile ?
		\end{enumerate}
	\end{enumerate}
}{exe:sujet2-0}{
	todo
}



\exe{}{
	On tire une boule dans une urne contenant $2$ boules rouges et $4$ boules vertes.
	\begin{enumerate}[label=---]
		\item Si la boule tirée est verte, on la met de côté et on retire une nouvelle boule
		\item Si la boule tirée est rouge, on la remet dans l'urne et on retire une nouvelle boule
	\end{enumerate}
	On considère les quatre événements suivants :
		\begin{multicols}{2}
		\begin{enumerate}[label=]
			\item v : \og la première boule tirée est verte \fg
			\item r : \og la première boule tirée est rouge \fg
			\item V : \og la deuxième boule tirée est verte \fg
			\item R : \og la deuxième boule tirée est rouge \fg
		\end{enumerate}
		\end{multicols}

	\begin{multicols}{2}
	\begin{enumerate}
		\item Donner $P(\text{v})$ et $P(\text{r})$.
		\item Donner $P(\text{V sachant v})$ et $P(\text{V sachant r})$.
		\item Calculer $P(\text{V $\cap$ v})$ et $P(\text{V $\cap$ r})$.
		\item Calculer $P(\text{V})$ puis $P(\text{R})$.
	\end{enumerate}
	\end{multicols}

	\begin{center}
	\begin{tikzpicture}[scale=1.2]
		% depth 1
		\foreach \i in {-3, 3}
		\draw[-, thick, black] (0,0) node {$\bullet$} -- (\i,-1.5);
		% depth 2
		\foreach \i in {-3, 3} \foreach \j in {-1, 1}
			\draw[-, thick, black] (\i,-1.5) node {$\bullet$} -- (\i+\j,-3) node {$\bullet$};
			
		\draw (-3,-1.5) node[above left] {v};
		\draw (3,-1.5) node[above right] {r};
			
		\draw (-4,-3) node[below] {V$\cap$v};
		\draw (2,-3) node[below] {V$\cap$r};
		\draw (-2,-3) node[below] {R$\cap$v};
		\draw (4,-3) node[below] {R$\cap$r};
	\end{tikzpicture}
	\end{center}
}{exe:proba2}{
	\, \\
	\begin{center}
	\begin{tikzpicture}
		% depth 1
		\foreach \i in {-3, 3}
		\draw[-, thick, black] (0,0) node {$\bullet$} -- (\i,-2);
		% depth 2
		\foreach \i in {-3, 3} \foreach \j in {-1, 1}
			\draw[-, thick, black] (\i,-2) node {$\bullet$} -- (\i+\j,-4) node {$\bullet$};
			
		\draw (-3,-2) node[above left] {v};
		\draw (3,-2) node[above right] {r};
			
		\draw (-4,-4) node[below] {V};
		\draw (2,-4) node[below] {V};
		\draw (-2,-4) node[below] {R};
		\draw (4,-4) node[below] {R};
		
		% sols
		\draw (1.5,0) node {$\frac13$};
		\draw (-1.5,0) node {$\frac23$};
		
		\draw (4,-2) node {$\frac13$};
		\draw (2,-2) node {$\frac23$};
		
		\draw (-2,-2) node {$\frac25$};
		\draw (-4,-2) node {$\frac35$};
	\end{tikzpicture}
	\end{center}
	La probabilité d'une issue est la somme des probabilités de chacun des chemins racine-feuille.
	Pour obtenir la probabilité d'un chemin racine-feuille, on multiplie les probabilités rencontrées en le parcourant.
	
		\begin{align*}
			P(V) &= \dfrac23 \times \dfrac35 + \dfrac13 \times \dfrac23 \\
				&= \dfrac25 + \dfrac29 \\
				&= \dfrac{18 + 10}{5 \times 9} = \dfrac{28}{45}
		\end{align*}
	
		\begin{align*}
			P(R) &= \dfrac23 \times \dfrac25 + \dfrac13 \times \dfrac13 \\
				&= \dfrac4{15} + \dfrac19 \\
				&= \dfrac{36 + 15}{15 \times 9} = \dfrac{51}{135} = \dfrac{17}{45}
		\end{align*}
		
		On aurait aussi pû utiliser le fait que $P(R) = 1 - P(V)$ pour ne pas augmenter la probabilité de faire une erreur de calcul.
}

%%%%%%%%%%%%

\newpage
\fancyhead[C]{\textbf{Solutions}}
\shipoutAnswer

\end{document}
