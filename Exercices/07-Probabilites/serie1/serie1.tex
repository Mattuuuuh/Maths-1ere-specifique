\documentclass[a4paper, 12pt]{extarticle}

\usepackage[utf8x]{inputenc}
%fonts
\usepackage{libertinus,libertinust1math}
\usepackage{amsmath,amsthm,amssymb,mathtools}

% SOLUTION SWITCH

\ifsolutions
	\newcommand{\exe}[2]{
		\begin{ex} #1  \end{ex}
		\begin{sol} #2 \end{sol}
	}
\else
	\newcommand{\exe}[2]{
		\begin{ex} #1  \end{ex}
	}
	
\fi


\usepackage[french]{babel}
\usepackage[
a4paper,
margin=2cm,
nomarginpar,% We don't want any margin paragraphs
]{geometry}

% HEADER, ARRAY, ENUM, MULTIOCL
\usepackage{fancyhdr}
\usepackage{array}
\usepackage{multicol, enumitem}
\newcolumntype{P}[1]{>{\centering\arraybackslash}p{#1}}
\usepackage{stackengine}
\newcommand\xrowht[2][0]{\addstackgap[.5\dimexpr#2\relax]{\vphantom{#1}}}

% theorems

\theoremstyle{theorem}
\newtheorem{thm}{Théorème}
\theoremstyle{plain}
\newtheorem*{sol}{Solution}
\theoremstyle{definition}
\newtheorem{ex}{Exercice}
\newtheorem{dfn}{Définition}
\newtheorem*{dfn*}{Définition}


%couleurs
\usepackage{tcolorbox}
\definecolor{myg}{RGB}{56, 140, 70}
\definecolor{myb}{RGB}{45, 111, 177}
\definecolor{myr}{RGB}{199, 68, 64}
\definecolor{mygr}{HTML}{2C3338}


\tcbuselibrary{theorems,skins,hooks}
\newcounter{commonbox}
\makeatletter
\newtcbtheorem[use counter=commonbox]{theorem}{Théorème }%
{
	enhanced,
	colback=white,
	colframe=mygr,
	attach boxed title to top left={yshift*=-\tcboxedtitleheight},
	fonttitle=\bfseries,
	title={#2},
	boxed title size=title,
	boxed title style={%
			sharp corners,
			rounded corners=northwest,
			colback=tcbcolframe,
			boxrule=0pt,
		},
	underlay boxed title={%
			\path[fill=tcbcolframe] (title.south west)--(title.south east)
			to[out=0, in=180] ([xshift=5mm]title.east)--
			(title.center-|frame.east)
			[rounded corners=\kvtcb@arc] |-
			(frame.north) -| cycle;
		},
	#1
}{th}
\newtcbtheorem[use counter=commonbox]{rappel}{Rappel }%
{
	enhanced,
	colback=white,
	colframe=mygr,
	attach boxed title to top left={yshift*=-\tcboxedtitleheight},
	fonttitle=\bfseries,
	title={#2},
	boxed title size=title,
	boxed title style={%
			sharp corners,
			rounded corners=northwest,
			colback=tcbcolframe,
			boxrule=0pt,
		},
	underlay boxed title={%
			\path[fill=tcbcolframe] (title.south west)--(title.south east)
			to[out=0, in=180] ([xshift=5mm]title.east)--
			(title.center-|frame.east)
			[rounded corners=\kvtcb@arc] |-
			(frame.north) -| cycle;
		},
	#1
}{th}
\newtcbtheorem[use counter=commonbox]{strategie}{Stratégie }%
{
	enhanced,
	colback=white,
	colframe=mygr,
	attach boxed title to top left={yshift*=-\tcboxedtitleheight},
	fonttitle=\bfseries,
	title={#2},
	boxed title size=title,
	boxed title style={%
			sharp corners,
			rounded corners=northwest,
			colback=tcbcolframe,
			boxrule=0pt,
		},
	underlay boxed title={%
			\path[fill=tcbcolframe] (title.south west)--(title.south east)
			to[out=0, in=180] ([xshift=5mm]title.east)--
			(title.center-|frame.east)
			[rounded corners=\kvtcb@arc] |-
			(frame.north) -| cycle;
		},
	#1
}{th}
\newtcbtheorem[use counter=commonbox]{outil}{Outil }%
{
	enhanced,
	colback=white,
	colframe=mygr,
	attach boxed title to top left={yshift*=-\tcboxedtitleheight},
	fonttitle=\bfseries,
	title={#2},
	boxed title size=title,
	boxed title style={%
			sharp corners,
			rounded corners=northwest,
			colback=tcbcolframe,
			boxrule=0pt,
		},
	underlay boxed title={%
			\path[fill=tcbcolframe] (title.south west)--(title.south east)
			to[out=0, in=180] ([xshift=5mm]title.east)--
			(title.center-|frame.east)
			[rounded corners=\kvtcb@arc] |-
			(frame.north) -| cycle;
		},
	#1
}{th}
\newtcbtheorem[use counter=commonbox]{but}{Buts du chapitre }%
{
	enhanced,
	colback=white,
	colframe=mygr,
	attach boxed title to top left={yshift*=-\tcboxedtitleheight},
	fonttitle=\bfseries,
	title={#2},
	boxed title size=title,
	boxed title style={%
			sharp corners,
			rounded corners=northwest,
			colback=tcbcolframe,
			boxrule=0pt,
		},
	underlay boxed title={%
			\path[fill=tcbcolframe] (title.south west)--(title.south east)
			to[out=0, in=180] ([xshift=5mm]title.east)--
			(title.center-|frame.east)
			[rounded corners=\kvtcb@arc] |-
			(frame.north) -| cycle;
		},
	#1
}{th}
\newtcbtheorem[use counter=commonbox]{propriete}{Propriété }%
{
	enhanced,
	colback=white,
	colframe=mygr,
	attach boxed title to top left={yshift*=-\tcboxedtitleheight},
	fonttitle=\bfseries,
	title={#2},
	boxed title size=title,
	boxed title style={%
			sharp corners,
			rounded corners=northwest,
			colback=tcbcolframe,
			boxrule=0pt,
		},
	underlay boxed title={%
			\path[fill=tcbcolframe] (title.south west)--(title.south east)
			to[out=0, in=180] ([xshift=5mm]title.east)--
			(title.center-|frame.east)
			[rounded corners=\kvtcb@arc] |-
			(frame.north) -| cycle;
		},
	#1
}{th}
\newtcbtheorem[number within=commonbox]{definition}{Définition }%
{
	enhanced,
	colback=white,
	colframe=mygr,
	attach boxed title to top left={yshift*=-\tcboxedtitleheight},
	fonttitle=\bfseries,
	title={#2},
	boxed title size=title,
	boxed title style={%
			sharp corners,
			rounded corners=northwest,
			colback=tcbcolframe,
			boxrule=0pt,
		},
	underlay boxed title={%
			\path[fill=tcbcolframe] (title.south west)--(title.south east)
			to[out=0, in=180] ([xshift=5mm]title.east)--
			(title.center-|frame.east)
			[rounded corners=\kvtcb@arc] |-
			(frame.north) -| cycle;
		},
	#1
}{th}
\newtcbtheorem[number within=commonbox]{exemples}{Exemples }%
{
	enhanced,
	colback=white,
	colframe=mygr,
	attach boxed title to top left={yshift*=-\tcboxedtitleheight},
	fonttitle=\bfseries,
	title={#2},
	boxed title size=title,
	boxed title style={%
			sharp corners,
			rounded corners=northwest,
			colback=tcbcolframe,
			boxrule=0pt,
		},
	underlay boxed title={%
			\path[fill=tcbcolframe] (title.south west)--(title.south east)
			to[out=0, in=180] ([xshift=5mm]title.east)--
			(title.center-|frame.east)
			[rounded corners=\kvtcb@arc] |-
			(frame.north) -| cycle;
		},
	#1
}{th}
\newtcbtheorem[number within=commonbox]{exemple}{Exemple }%
{
	enhanced,
	colback=white,
	colframe=mygr,
	attach boxed title to top left={yshift*=-\tcboxedtitleheight},
	fonttitle=\bfseries,
	title={#2},
	boxed title size=title,
	boxed title style={%
			sharp corners,
			rounded corners=northwest,
			colback=tcbcolframe,
			boxrule=0pt,
		},
	underlay boxed title={%
			\path[fill=tcbcolframe] (title.south west)--(title.south east)
			to[out=0, in=180] ([xshift=5mm]title.east)--
			(title.center-|frame.east)
			[rounded corners=\kvtcb@arc] |-
			(frame.north) -| cycle;
		},
	#1
}{th}
\newtcbtheorem[number within=commonbox]{questions}{Questions guidantes }%
{
	enhanced,
	colback=white,
	colframe=mygr,
	attach boxed title to top left={yshift*=-\tcboxedtitleheight},
	fonttitle=\bfseries,
	title={#2},
	boxed title size=title,
	boxed title style={%
			sharp corners,
			rounded corners=northwest,
			colback=tcbcolframe,
			boxrule=0pt,
		},
	underlay boxed title={%
			\path[fill=tcbcolframe] (title.south west)--(title.south east)
			to[out=0, in=180] ([xshift=5mm]title.east)--
			(title.center-|frame.east)
			[rounded corners=\kvtcb@arc] |-
			(frame.north) -| cycle;
		},
	#1
}{th}
\makeatother

% corps
\newcommand{\R}{\mathbb{R}}
\newcommand{\Rnn}{\mathbb{R}^{2n}}
\newcommand{\Z}{\mathbb{Z}}
\newcommand{\N}{\mathbb{N}}
\newcommand{\Q}{\mathbb{Q}}

% domain
\newcommand{\D}{\mathcal{D}}
% for calligraphic C
\usepackage{calrsfs}
\newcommand{\C}{\mathcal{C}}

% date
\usepackage{advdate}

% ensembles tq. 
\newcommand{\xRtq}[1]{
	$\left\{ x \in \R \text{ tq. } #1 \right\}$
}

% vabs
\newcommand{\vabs}[1]{
	\left| #1 \right|
}

%pinfty minfty
\newcommand{\pinfty}{{+}\infty}
\newcommand{\minfty}{{-}\infty}

% plots
\usepackage{pgfplots}

%virgules
\usepackage{icomma}
\pgfplotsset{/pgf/number format/use comma}

%subfigures
\usepackage{subcaption}

%hyperlink footnote
\usepackage{hyperref}

%wider tabulars
\def\arraystretch{2}
\setlength\tabcolsep{15pt}

% tableaux var, signe
\usepackage{tkz-tab}

\SetDate[07/01/2026]

\begin{document}
\pagestyle{fancy}
\fancyhead[L]{Première spécifique}
\fancyhead[C]{\textbf{Probabilités 1 : tableaux croisés}}
\fancyhead[R]{\today}

\exe{}{
	Un vendeur de voitures possède un stock de 1 000 voitures dont les caractéristiques sont résumées dans le tableau ci-dessous.
	
	\begin{center}
	\begin{tabular}{|c|c|c|c|c|}
	\cline{2-5}
	\multicolumn{1}{c|}{}	&	Blanche	&	Noire	&	Rouge	&	TOTAL \\ \hline
	Française	&	150	&	$x$	&	400	&	750 \\ \hline
	Étrangère	&	100	&	50	&	100	&	250 \\ \hline
	TOTAL	&	250	&	250	&	500	&	1000 \\\hline
	\end{tabular}
	\end{center}
	
	\begin{enumerate}
		\item Indiquer ce que représente $x$ et déterminer sa valeur.
		\item Quel est le pourcentage de voitures noires parmi les voitures du stock ?
		\item Quel est le pourcentage de voitures noires étrangères parmi les voitures du stock ?
		\item Quel est le pourcentage de voitures blanches parmi les voitures françaises ?
		\item Quel est le pourcentage de voitures françaises parmi les voitures blanches ?
		\item Alice et Benoît jouent au jeu suivant.
			\begin{enumerate}[label=--]
				\item Alice choisit au harsard une voiture parmi les voitures françaises.
				Elle remporte 1 euro si ce n'est pas une voiture rouge.
				\item Benoît choisit au hasard une voiture parmi les voitures blanches.
				Il remporte 1 euro si c'est une voiture étrangère.
			\end{enumerate}
		Lequel des deux a le plus de chance de remporter 1 euro ?
	\end{enumerate}
}{exe:tableau0-1}{
	\begin{enumerate}
		\item
		$x$ est le nombre de voitures françaises et noires. 
		Comme $150+x+400 = 750$, on a $x = 200$.
		\item 
		Le pourcentage est obtenu en divisant le nombre de voitures noires par le nombre de voitures au total.
			\[ \dfrac{250}{1000} = 0,25 = 25\%. \]
		\item Quel est le pourcentage de voitures noires étrangères parmi les voitures du stock ?
		Le pourcentage est obtenu en divisant le nombre de voitures noires et étrangères par le nombre de voitures au total.
			\[ \dfrac{50}{1000} = 0,05 = 5\%. \]
		\item Quel est le pourcentage de voitures blanches parmi les voitures françaises ?
		Le pourcentage est obtenu en divisant le nombre de voitures blanches et françaises par le nombre de voitures françaises.
			\[ \dfrac{150}{750} = 0,2 = 20\%. \]
		\item Quel est le pourcentage de voitures françaises parmi les voitures blanches ?
		Le pourcentage est obtenu en divisant le nombre de voitures françaises et blanches par le nombre de voitures blanches.
			\[ \dfrac{150}{250} = 0,6 = 60\%. \]
		\item
		Lorsqu'on choisit uniformément au hasard, la probabilité est égale à la proportion.
		On calcule donc, pour Alice, sa probabilité de gagner 1 euro :
			\[ \dfrac{150+200}{750} = \dfrac{350}{750} = \dfrac{35}{75} = \dfrac{7}{15}, \]
		et, pour Benoît,
			\[ \dfrac{100}{250} = \dfrac{10}{25} = \dfrac{2}{5}. \]
		Il s'agit désormais de comparer les fractions $\frac{7}{15}$ et $\frac25$.
		Pour cela, on met la deuxième au même dénominateur que la première :
			\[ \dfrac25 = \dfrac{2\times3}{5\times3} = \dfrac{6}{15} < \dfrac{7}{15}. \]
		Alice a donc plus de chance de remporter 1 euro.
	\end{enumerate}
}


\exe{}{
	Une urne contient $49$ billes numérotées de $1$ à $49$.
	La moitié des billes paires sont bleues, les $\frac25$ des billes impaires sont jaunes.
	\begin{center}
	\setlength\tabcolsep{20pt}
	\tableaucroise{Paire & Impaire & Total}{Bleue & & &}{Jaune & & &}{Total &&&}
	\end{center}
	
	On choisit une bille uniformément au hasard et on dénote
		\begin{center}
			$I$ : \og La bille a un numéro impair. \fg
			\hspace{3cm}
			$B$ : \og La bille est bleue. \fg
		\end{center}
	\begin{enumerate}
		\item Compléter le tableau croisé d'effectifs.
		\item Calculer $P(I)$ et $P(I \cap B)$ à l'aide du tableau. 
		\item Décrire l'événement $\overline{I}$ avec des mots et calculer $P\left(\overline{I}\right)$.
		\item Calculer $P(I \sct B)$ à l'aide du tableau.
		\item Les événements $I$ et $B$ sont-ils indépendants ? corrélés positivement ? négativement ? Justifier.
	\end{enumerate}
}{exe:proba6}{
	\begin{center}
	\setlength\tabcolsep{20pt}
	\tableaucroise{Paire & Impaire & Total}{Bleue & 12 & 15 & 27}{Jaune & 12 & 10 & 22}{Total & 24 & 25 & 49}
	\end{center}
	\begin{enumerate}[start=2]
		\item 
		$P(I) = \dfrac{25}{49}$ et $P(I \cap B) = \dfrac{15}{49}$.
		\item
		L'événement $\overline{I}$ est l'événement complémentaire (ou contraire) à $I$.
		C'est donc $\overline{I}$ : « La bille n'a pas un numéro impair. » ou « La bille a un numéro pair. »
		
		Comme $P\left(\overline{I}\right)+ P(I) = 1$, par propriété des événements complémentaires, 
			\[ P\left(\overline{I}\right) = 1 - P(I) = 1 - \dfrac{25}{49} = \dfrac{49-25}{49} = \dfrac{24}{49}. \]
		\item 
		$P(I \sct B) = \dfrac{15}{27}$.
		\item 
		Il s'agit ici de comparer $P(I)$ et $P(I \sct B)$ à l'aide des valeurs calculées.
		Le dénominateur commun des deux fractions est $49\times27$, qu'il est inutile de calculer.
		D'une part,
			\[ P(I) = \dfrac{25}{49} = \dfrac{25\times27}{49\times27} = \dfrac{675}{49\times27}, \]
		où on a calculé $25\times27 = 20\times27 + 5\times27 = 540 + 135 = 675$.
		D'autre part,
			\[ P(I \sct B) = \dfrac{15}{27} =\dfrac{15\times49}{27\times49} = \dfrac{735}{27\times49}, \]
		où on a calculé $15\times49 = 15\times50 - 15 = 750 - 15 = 735$.
		
		Finalement, comme $P(I\sct B) > P(I)$, les événements $I$ et $B$ sont positivement corrélés.
	\end{enumerate}
}


%%%%%%%%%%%%

\newpage
\fancyhead[C]{\textbf{Solutions}}
\shipoutAnswer

\end{document}
