\documentclass[a4paper, 12pt]{extarticle}

\usepackage[utf8x]{inputenc}
%fonts
\usepackage{libertinus,libertinust1math}
\usepackage{amsmath,amsthm,amssymb,mathtools}

% SOLUTION SWITCH

\ifsolutions
	\newcommand{\exe}[2]{
		\begin{ex} #1  \end{ex}
		\begin{sol} #2 \end{sol}
	}
\else
	\newcommand{\exe}[2]{
		\begin{ex} #1  \end{ex}
	}
	
\fi


\usepackage[french]{babel}
\usepackage[
a4paper,
margin=2cm,
nomarginpar,% We don't want any margin paragraphs
]{geometry}

% HEADER, ARRAY, ENUM, MULTIOCL
\usepackage{fancyhdr}
\usepackage{array}
\usepackage{multicol, enumitem}
\newcolumntype{P}[1]{>{\centering\arraybackslash}p{#1}}
\usepackage{stackengine}
\newcommand\xrowht[2][0]{\addstackgap[.5\dimexpr#2\relax]{\vphantom{#1}}}

% theorems

\theoremstyle{theorem}
\newtheorem{thm}{Théorème}
\theoremstyle{plain}
\newtheorem*{sol}{Solution}
\theoremstyle{definition}
\newtheorem{ex}{Exercice}
\newtheorem{dfn}{Définition}
\newtheorem*{dfn*}{Définition}


%couleurs
\usepackage{tcolorbox}
\definecolor{myg}{RGB}{56, 140, 70}
\definecolor{myb}{RGB}{45, 111, 177}
\definecolor{myr}{RGB}{199, 68, 64}
\definecolor{mygr}{HTML}{2C3338}


\tcbuselibrary{theorems,skins,hooks}
\newcounter{commonbox}
\makeatletter
\newtcbtheorem[use counter=commonbox]{theorem}{Théorème }%
{
	enhanced,
	colback=white,
	colframe=mygr,
	attach boxed title to top left={yshift*=-\tcboxedtitleheight},
	fonttitle=\bfseries,
	title={#2},
	boxed title size=title,
	boxed title style={%
			sharp corners,
			rounded corners=northwest,
			colback=tcbcolframe,
			boxrule=0pt,
		},
	underlay boxed title={%
			\path[fill=tcbcolframe] (title.south west)--(title.south east)
			to[out=0, in=180] ([xshift=5mm]title.east)--
			(title.center-|frame.east)
			[rounded corners=\kvtcb@arc] |-
			(frame.north) -| cycle;
		},
	#1
}{th}
\newtcbtheorem[use counter=commonbox]{rappel}{Rappel }%
{
	enhanced,
	colback=white,
	colframe=mygr,
	attach boxed title to top left={yshift*=-\tcboxedtitleheight},
	fonttitle=\bfseries,
	title={#2},
	boxed title size=title,
	boxed title style={%
			sharp corners,
			rounded corners=northwest,
			colback=tcbcolframe,
			boxrule=0pt,
		},
	underlay boxed title={%
			\path[fill=tcbcolframe] (title.south west)--(title.south east)
			to[out=0, in=180] ([xshift=5mm]title.east)--
			(title.center-|frame.east)
			[rounded corners=\kvtcb@arc] |-
			(frame.north) -| cycle;
		},
	#1
}{th}
\newtcbtheorem[use counter=commonbox]{strategie}{Stratégie }%
{
	enhanced,
	colback=white,
	colframe=mygr,
	attach boxed title to top left={yshift*=-\tcboxedtitleheight},
	fonttitle=\bfseries,
	title={#2},
	boxed title size=title,
	boxed title style={%
			sharp corners,
			rounded corners=northwest,
			colback=tcbcolframe,
			boxrule=0pt,
		},
	underlay boxed title={%
			\path[fill=tcbcolframe] (title.south west)--(title.south east)
			to[out=0, in=180] ([xshift=5mm]title.east)--
			(title.center-|frame.east)
			[rounded corners=\kvtcb@arc] |-
			(frame.north) -| cycle;
		},
	#1
}{th}
\newtcbtheorem[use counter=commonbox]{outil}{Outil }%
{
	enhanced,
	colback=white,
	colframe=mygr,
	attach boxed title to top left={yshift*=-\tcboxedtitleheight},
	fonttitle=\bfseries,
	title={#2},
	boxed title size=title,
	boxed title style={%
			sharp corners,
			rounded corners=northwest,
			colback=tcbcolframe,
			boxrule=0pt,
		},
	underlay boxed title={%
			\path[fill=tcbcolframe] (title.south west)--(title.south east)
			to[out=0, in=180] ([xshift=5mm]title.east)--
			(title.center-|frame.east)
			[rounded corners=\kvtcb@arc] |-
			(frame.north) -| cycle;
		},
	#1
}{th}
\newtcbtheorem[use counter=commonbox]{but}{Buts du chapitre }%
{
	enhanced,
	colback=white,
	colframe=mygr,
	attach boxed title to top left={yshift*=-\tcboxedtitleheight},
	fonttitle=\bfseries,
	title={#2},
	boxed title size=title,
	boxed title style={%
			sharp corners,
			rounded corners=northwest,
			colback=tcbcolframe,
			boxrule=0pt,
		},
	underlay boxed title={%
			\path[fill=tcbcolframe] (title.south west)--(title.south east)
			to[out=0, in=180] ([xshift=5mm]title.east)--
			(title.center-|frame.east)
			[rounded corners=\kvtcb@arc] |-
			(frame.north) -| cycle;
		},
	#1
}{th}
\newtcbtheorem[use counter=commonbox]{propriete}{Propriété }%
{
	enhanced,
	colback=white,
	colframe=mygr,
	attach boxed title to top left={yshift*=-\tcboxedtitleheight},
	fonttitle=\bfseries,
	title={#2},
	boxed title size=title,
	boxed title style={%
			sharp corners,
			rounded corners=northwest,
			colback=tcbcolframe,
			boxrule=0pt,
		},
	underlay boxed title={%
			\path[fill=tcbcolframe] (title.south west)--(title.south east)
			to[out=0, in=180] ([xshift=5mm]title.east)--
			(title.center-|frame.east)
			[rounded corners=\kvtcb@arc] |-
			(frame.north) -| cycle;
		},
	#1
}{th}
\newtcbtheorem[number within=commonbox]{definition}{Définition }%
{
	enhanced,
	colback=white,
	colframe=mygr,
	attach boxed title to top left={yshift*=-\tcboxedtitleheight},
	fonttitle=\bfseries,
	title={#2},
	boxed title size=title,
	boxed title style={%
			sharp corners,
			rounded corners=northwest,
			colback=tcbcolframe,
			boxrule=0pt,
		},
	underlay boxed title={%
			\path[fill=tcbcolframe] (title.south west)--(title.south east)
			to[out=0, in=180] ([xshift=5mm]title.east)--
			(title.center-|frame.east)
			[rounded corners=\kvtcb@arc] |-
			(frame.north) -| cycle;
		},
	#1
}{th}
\newtcbtheorem[number within=commonbox]{exemples}{Exemples }%
{
	enhanced,
	colback=white,
	colframe=mygr,
	attach boxed title to top left={yshift*=-\tcboxedtitleheight},
	fonttitle=\bfseries,
	title={#2},
	boxed title size=title,
	boxed title style={%
			sharp corners,
			rounded corners=northwest,
			colback=tcbcolframe,
			boxrule=0pt,
		},
	underlay boxed title={%
			\path[fill=tcbcolframe] (title.south west)--(title.south east)
			to[out=0, in=180] ([xshift=5mm]title.east)--
			(title.center-|frame.east)
			[rounded corners=\kvtcb@arc] |-
			(frame.north) -| cycle;
		},
	#1
}{th}
\newtcbtheorem[number within=commonbox]{exemple}{Exemple }%
{
	enhanced,
	colback=white,
	colframe=mygr,
	attach boxed title to top left={yshift*=-\tcboxedtitleheight},
	fonttitle=\bfseries,
	title={#2},
	boxed title size=title,
	boxed title style={%
			sharp corners,
			rounded corners=northwest,
			colback=tcbcolframe,
			boxrule=0pt,
		},
	underlay boxed title={%
			\path[fill=tcbcolframe] (title.south west)--(title.south east)
			to[out=0, in=180] ([xshift=5mm]title.east)--
			(title.center-|frame.east)
			[rounded corners=\kvtcb@arc] |-
			(frame.north) -| cycle;
		},
	#1
}{th}
\newtcbtheorem[number within=commonbox]{questions}{Questions guidantes }%
{
	enhanced,
	colback=white,
	colframe=mygr,
	attach boxed title to top left={yshift*=-\tcboxedtitleheight},
	fonttitle=\bfseries,
	title={#2},
	boxed title size=title,
	boxed title style={%
			sharp corners,
			rounded corners=northwest,
			colback=tcbcolframe,
			boxrule=0pt,
		},
	underlay boxed title={%
			\path[fill=tcbcolframe] (title.south west)--(title.south east)
			to[out=0, in=180] ([xshift=5mm]title.east)--
			(title.center-|frame.east)
			[rounded corners=\kvtcb@arc] |-
			(frame.north) -| cycle;
		},
	#1
}{th}
\makeatother

% corps
\newcommand{\R}{\mathbb{R}}
\newcommand{\Rnn}{\mathbb{R}^{2n}}
\newcommand{\Z}{\mathbb{Z}}
\newcommand{\N}{\mathbb{N}}
\newcommand{\Q}{\mathbb{Q}}

% domain
\newcommand{\D}{\mathcal{D}}
% for calligraphic C
\usepackage{calrsfs}
\newcommand{\C}{\mathcal{C}}

% date
\usepackage{advdate}

% ensembles tq. 
\newcommand{\xRtq}[1]{
	$\left\{ x \in \R \text{ tq. } #1 \right\}$
}

% vabs
\newcommand{\vabs}[1]{
	\left| #1 \right|
}

%pinfty minfty
\newcommand{\pinfty}{{+}\infty}
\newcommand{\minfty}{{-}\infty}

% plots
\usepackage{pgfplots}

%virgules
\usepackage{icomma}
\pgfplotsset{/pgf/number format/use comma}

%subfigures
\usepackage{subcaption}

%hyperlink footnote
\usepackage{hyperref}

%wider tabulars
\def\arraystretch{2}
\setlength\tabcolsep{15pt}

% tableaux var, signe
\usepackage{tkz-tab}

\SetDate[28/01/2026]

\begin{document}
\pagestyle{fancy}
\fancyhead[L]{Première spécifique}
\fancyhead[C]{\textbf{Évaluation blanche — Probabilités}}
\fancyhead[R]{\today}

\exe{}{
	Dans cet exercice, exprimer les probabilités sous forme de fractions qu'il n'est pas nécessaire de simplifier.

	Dans un lycée comptant 2000 élèves, on donne la répartition des effectifs suivant le sexe et le choix de la LV1.
	
	\begin{center}
	\begin{tabular}{|c|c|c|}
	\cline{2-3}
	\multicolumn{1}{c|}{}	 & Fille & Garçon \\ \hline
	Anglais & 712 & 728 \\ \hline
	Autre LV1 & 288 & 272 \\ \hline
	\end{tabular}
	\end{center}
	
	\begin{enumerate}
		\item 
		Un élève affirme : « Dans ce lycée, il y a autant de filles que de garçons ».
		A-t-il raison ? Justifier.
	\end{enumerate}
	
	On choisit au hasard, de manière équiprobable, un élève dans ce lycée.
	On considère les événements suivants :
	\begin{center}
		F : « l'élève est une fille » ; et
		\hspace{2cm}
		A : « l'élève a choisi Anglais pour LV1 ». 
	\end{center}
	
	\begin{enumerate}[resume]
		\item
		Décrire l'événement $A\cap F$ avec des mots puis déterminer sa probabilité.
		\item
		Déterminer la probabilité de l'événement $A$ sachant que l'événement $F$ est réalisé.
		\item
		Les événements $A$ et $F$ sont-ils indépendants ? Justifier.
		\item
		On sait que l'élève choisi est un garçon.
		L'affirmation suivante est-elle vraie ? Justifier.
		
		« La probabilité qu'il ait choisi Anglais pour LV1 est plus de trois fois plus grande que la probabilité qu'il n'ait pas choisi Anglais pour LV1 »
	\end{enumerate}
}{exe:tableau0-2}{
	\begin{enumerate}
		\item
		Le nombre de filles et de garçons est calculé en sommant les effectifs des colonnes correspondantes.
		Il y a 1 000 filles et 1 000 garçons ; l'élève a donc raison.
		\item
		$A\cap F$ : « l'élève a choisi Anglais pour LV1 \textbf{et} est une fille ».
			\[ P(A\cap F) = \dfrac{712}{2000} \]
		\item
			\[ P(A \sct F) = \dfrac{712}{1000} \]
		\item
		On se demande si $P(A \sct F) = P(A)$, condition à remplir pour pouvoir affirmer que $A$ et $F$ sont indépendants.
		Or
			\[ P(A) = \dfrac{712 + 728}{2000} = \dfrac{1440}{2000}, \]
		et $P(A \sct F) = \frac{712}{1000} = \frac{1414}{2000} \neq P(A)$.
		
		Les événements $A$ et $F$ ne sont donc pas indépendants : le résultat de l'un influe le résultat de l'autre (même si légèrement).
		\item
		D'une part, la probabilité que l'élève ait choisi Anglais sachant que c'est un garçon est de
			\[ P\left(A \sct \overline{F}\right) = \dfrac{728}{1000}. \]
		D'autre part, la probabilité que l'élève n'ait pas choisi Anglais sachant que c'est un garçon est de
			\[ P\left(\overline{A} \sct \overline{F}\right) = \dfrac{272}{1000}. \]
		Il s'agit de décider si la première probabilité est plus de trois fois plus grande que la seconde.
		Calculons le triple de la seconde :
			\[ 3 \times \dfrac{272}{1000} = \dfrac{3\times272}{1000} = \dfrac{816}{1000}. \]
		L'affirmation est donc fausse.
	\end{enumerate}
}

\exe{}{
	Un employé reçoit des appels téléphoniques.
	On estime que la probabilité qu'un appel dure plus de cinq minutes est égale à $0,3$.
	On suppose que les durées des différents appels sont indépendantes.
	
	Ce matin, l'employé reçoit deux appels.
	\begin{enumerate}
		\item
		Quelle est la probabilité que le deuxième appel dure plus de cinq minutes sachant que le premier a duré moins de cinq minutes ? Justifier.
		\item
		Quelle est la probabilité que le deuxième appel dure moins de cinq minutes sachant que le premier a duré plus de cinq minutes ? Justifier.
		\item
		Calculer la probabilité que les deux appels durent tous les deux plus de cinq minutes.
		\item
		Calculer la probabilité qu'exactement un appel sur les deux dure plus de cinq minutes.
	\end{enumerate}
}{exe:arbre0-3}{
	L'exercice appelle un arbre de probabilités : une même expérience est répétée deux fois.
	La nouvelle expérience aléatoire est donc la somme des deux épreuves, l'une après l'autre.

	Notons $A$ l'événement « L'appel dure plus de 5 minutes. »
	Les durées des différents appels étant indépendantes, on remplit l'arbre comme suit.
	\begin{center}
	\begin{tikzpicture}
		% depth 1
		\foreach \i in {-3, 3}
		\draw[-, thick, black] (0,0) node {$\bullet$} -- (\i,-2);
		% depth 2
		\foreach \i in {-3, 3} \foreach \j in {-1, 1}
			\draw[-, thick, black] (\i,-2) node {$\bullet$} -- (\i+\j,-4) node {$\bullet$};
			
		\draw (-3,-2) node[above left] {$A$};
		\draw (3,-2) node[above right] {$\overline{A}$};
			
		\draw (-4,-4) node[below] {$A$};
		\draw (2,-4) node[below] {$A$};
		\draw (-2,-4) node[below] {$\overline{A}$};
		\draw (4,-4) node[below] {$\overline{A}$};
		
		% sols
		\draw (1.5,-.5) node {$0,7$};
		\draw (-1.5,-.5) node {$0,3$};
		
		\draw (4,-2.5) node {$0,7$};
		\draw (2,-2.5) node {$0,3$};
		
		\draw (-2,-2.5) node {$0,7$};
		\draw (-4,-2.5) node {$0,3$};
	\end{tikzpicture}
	\end{center}
	Il y a donc quatre issues à l'expérience, correspondant aux quatre feuilles de l'arbre. 
	La probabilité de chaque issue est calculée en faisant le produit des probabilités rencontrées sur le chemin racine-feuille correspondant.
	
	\begin{enumerate}
		\item
		Par indépendance, la probabilité que le deuxième appel dure plus de cinq minutes sachant que le premier a duré moins de cinq minutes est égale à la probabilité que le deuxième appel dure plus de cinq minutes sans condition, soit $0,3$.
		\item
		Par indépendance, la probabilité que le deuxième appel dure moins de cinq minutes sachant que le premier a duré plus de cinq minutes est égale à la probabilité que le deuxième appel dure moins de cinq minutes sans condition, soit $0,7$.
		\item
		L'événement correspond au chemin $A$ puis $A$ dans l'arbre, de probabilité
			\[ 0,3 \times 0,3 = \dfrac3{10}\times \dfrac3{10} = \dfrac{3\times3}{10\times10} = \dfrac{9}{100} = 0,09. \]
		\item
		L'événement correspond aux deux chemins distincts : $A$ puis $\overline{A}$, et $\overline{A}$ puis $A$.
		Sa probabilité est donc donnée par
			\[ 2\times0,3\times0,7 = 2\times\dfrac3{10}\times \dfrac7{10} =\dfrac{2\times3\times7}{10\times10} = \dfrac{42}{100} = 0,42. \]
	\end{enumerate}
}

\newpage

\exe{}{
	Un village propose aux participants de la fête du sport deux épreuves : une
	randonnée et un cross. Il n’est pas possible de s’inscrire aux deux épreuves à la fois.
	On dispose des informations suivantes :
	\begin{itemize}
		\item 90\% des participants ont choisi la randonnée et, parmi eux, 5\% sont licenciés dans un
club.
		\item 10\% des participants ont choisi le cross et, parmi eux, 40\% sont licenciés dans un
club.
	\end{itemize}
	Un journaliste interroge un participant au hasard.
	On considère les événements suivants :
	\begin{center}
	$R$ : « Le participant a choisi la randonnée » 
	\hspace{1cm}
	$L$ : « Le participant est licencié dans un club ».
	\end{center}
	
	\begin{enumerate}
		\item Par simple lecture de l’énoncé, indiquer :
		\begin{enumerate}
			\item
			La probabilité que le participant interrogé soit licencié dans un club sachant qu’il a choisi la randonnée.
			\item
			La probabilité que le participant interrogé soit licencié dans un club sachant qu’il a choisi le cross.
		\end{enumerate}
	\end{enumerate}
	En prenant connaissance de ces deux probabilités, le journaliste estime que s’il choisit un participant parmi ceux qui sont licenciés dans un club, la probabilité qu’il ait effectué le cross sera largement supérieure à 50\%. 
	L’objectif des questions suivantes est de vérifier si cette intuition est correcte.
	
	\begin{enumerate}[resume]
		\item 
		Représenter la situation par un arbre de probabilité.
		\item
		\begin{enumerate}
			\item Déterminer la probabilité que le participant interrogé ait choisi le cross et soit licencié dans un club.
			\item
			Vérifier que la probabilité que le participant interrogé soit licencié dans un club est égale à $\frac{850}{10~000}$, soit $8,5\%$.
		\end{enumerate}
		\item Le journaliste interroge un participant licencié dans un club. Déterminer la probabilité que ce participant ait choisi le cross.
		
		L’intuition du journaliste est-elle correcte ?
	\end{enumerate}
}{exe:proba0-3}{
	
	\begin{enumerate}
		\item Par simple lecture de l’énoncé, indiquer :
		\begin{enumerate}
			\item
			Elle est donnée par 5\%, soit $0,05$.
			\item
			Elle est donnée par 40\%, soit $0,4$.
		\end{enumerate}
	\end{enumerate}
	En prenant connaissance de ces deux probabilités, le journaliste estime que s’il choisit un participant parmi ceux qui sont licenciés dans un club, la probabilité qu’il ait effectué le cross sera largement supérieure à 50\%. 
	L’objectif des questions suivantes est de vérifier si cette intuition est correcte.
	
	\begin{enumerate}[resume]
		\item 
		Voir ci-après.
		\item
		\begin{enumerate}
			\item
			On multiplie les probabilités du chemin correspondant :
			$P\left(\overline{R} \cap L\right) = 0,1 \times 0,4 = \dfrac{4}{100} = 0,04$.
			\item
			On ajoute les probabilités des deux chemins correspondant :
				\begin{align*}
					P(L) &= P\left(\overline{R} \cap L\right) + P\left(R \cap L\right) \\
						&= 0,04 + 0,9\times0,05 \\
						&= 0,04 + \dfrac{45}{1000} \\
						&= 0,04 + 0,045 \\
						&= 0,085 = 8,5\%
				\end{align*}
		\end{enumerate}
		\item
		Il s'agit ici d'utiliser la formule
			\[ P\left( \overline{R} \sct L \right) = \dfrac{P\left( \overline{R} \cap L \right)}{P(L)}. \]
		Le numérateur est donné par $0,04$ et le dénominateur $0,085$ d'après les questions précédentes.
		Donc
			\[ P\left( \overline{R} \sct L \right) = \dfrac{0,04}{0,085} = \dfrac{40}{85}. \]
		Il s'agit désormais de comparer $\frac{40}{85}$ à 50\%, soit $0,5$ ou $\frac12$.
		Ici, le dénominateur est plus du double du numérateur, donc on s'attend à ce que la fraction soit inférieure à $\frac12$.
		C'est bien le cas :
			\[ P\left( \overline{R} \sct L \right) = \dfrac{40}{85} < \dfrac{42,5}{85} = \dfrac12 = 50\%. \]
		L'intuition du journaliste est donc erronée.
	\end{enumerate}
	
	
	
	\begin{center}
	\begin{tikzpicture}
		% depth 1
		\foreach \i in {-3, 3}
		\draw[-, thick, black] (0,0) node {$\bullet$} -- (\i,-2);
		% depth 2
		\foreach \i in {-3, 3} \foreach \j in {-1, 1}
			\draw[-, thick, black] (\i,-2) node {$\bullet$} -- (\i+\j,-4) node {$\bullet$};
			
		\draw (-3,-2) node[above left] {Randonnée};
		\draw (3,-2) node[above right] {Cross};
			
		\draw (-4,-4) node[below] {Licencé};
		\draw (2,-4) node[below] {Licencé};
		\draw (-2,-4) node[below] {Non licencé};
		\draw (4,-4) node[below] {Non licencé};
		
		% sols
		\draw (-1.5,-.5) node {$0,9$};
		\draw (1.5,-.5) node {$0,1$};
		
		\draw (-4,-3) node {$0,05$};
		\draw (-2,-3) node {$0,95$};
		
		\draw (2,-3) node {$0,4$};
		\draw (4,-3) node {$0,6$};
	\end{tikzpicture}
	\end{center}

}

%%%%%%%%%%%%

\newpage
\fancyhead[C]{\textbf{Solutions}}
\shipoutAnswer

\end{document}
