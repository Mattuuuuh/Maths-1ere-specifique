\documentclass[a4paper, 12pt]{extarticle}

\usepackage[utf8x]{inputenc}
%fonts
\usepackage{libertinus,libertinust1math}
\usepackage{amsmath,amsthm,amssymb,mathtools}

% SOLUTION SWITCH

\ifsolutions
	\newcommand{\exe}[2]{
		\begin{ex} #1  \end{ex}
		\begin{sol} #2 \end{sol}
	}
\else
	\newcommand{\exe}[2]{
		\begin{ex} #1  \end{ex}
	}
	
\fi


\usepackage[french]{babel}
\usepackage[
a4paper,
margin=2cm,
nomarginpar,% We don't want any margin paragraphs
]{geometry}

% HEADER, ARRAY, ENUM, MULTIOCL
\usepackage{fancyhdr}
\usepackage{array}
\usepackage{multicol, enumitem}
\newcolumntype{P}[1]{>{\centering\arraybackslash}p{#1}}
\usepackage{stackengine}
\newcommand\xrowht[2][0]{\addstackgap[.5\dimexpr#2\relax]{\vphantom{#1}}}

% theorems

\theoremstyle{theorem}
\newtheorem{thm}{Théorème}
\theoremstyle{plain}
\newtheorem*{sol}{Solution}
\theoremstyle{definition}
\newtheorem{ex}{Exercice}
\newtheorem{dfn}{Définition}
\newtheorem*{dfn*}{Définition}


%couleurs
\usepackage{tcolorbox}
\definecolor{myg}{RGB}{56, 140, 70}
\definecolor{myb}{RGB}{45, 111, 177}
\definecolor{myr}{RGB}{199, 68, 64}
\definecolor{mygr}{HTML}{2C3338}


\tcbuselibrary{theorems,skins,hooks}
\newcounter{commonbox}
\makeatletter
\newtcbtheorem[use counter=commonbox]{theorem}{Théorème }%
{
	enhanced,
	colback=white,
	colframe=mygr,
	attach boxed title to top left={yshift*=-\tcboxedtitleheight},
	fonttitle=\bfseries,
	title={#2},
	boxed title size=title,
	boxed title style={%
			sharp corners,
			rounded corners=northwest,
			colback=tcbcolframe,
			boxrule=0pt,
		},
	underlay boxed title={%
			\path[fill=tcbcolframe] (title.south west)--(title.south east)
			to[out=0, in=180] ([xshift=5mm]title.east)--
			(title.center-|frame.east)
			[rounded corners=\kvtcb@arc] |-
			(frame.north) -| cycle;
		},
	#1
}{th}
\newtcbtheorem[use counter=commonbox]{rappel}{Rappel }%
{
	enhanced,
	colback=white,
	colframe=mygr,
	attach boxed title to top left={yshift*=-\tcboxedtitleheight},
	fonttitle=\bfseries,
	title={#2},
	boxed title size=title,
	boxed title style={%
			sharp corners,
			rounded corners=northwest,
			colback=tcbcolframe,
			boxrule=0pt,
		},
	underlay boxed title={%
			\path[fill=tcbcolframe] (title.south west)--(title.south east)
			to[out=0, in=180] ([xshift=5mm]title.east)--
			(title.center-|frame.east)
			[rounded corners=\kvtcb@arc] |-
			(frame.north) -| cycle;
		},
	#1
}{th}
\newtcbtheorem[use counter=commonbox]{strategie}{Stratégie }%
{
	enhanced,
	colback=white,
	colframe=mygr,
	attach boxed title to top left={yshift*=-\tcboxedtitleheight},
	fonttitle=\bfseries,
	title={#2},
	boxed title size=title,
	boxed title style={%
			sharp corners,
			rounded corners=northwest,
			colback=tcbcolframe,
			boxrule=0pt,
		},
	underlay boxed title={%
			\path[fill=tcbcolframe] (title.south west)--(title.south east)
			to[out=0, in=180] ([xshift=5mm]title.east)--
			(title.center-|frame.east)
			[rounded corners=\kvtcb@arc] |-
			(frame.north) -| cycle;
		},
	#1
}{th}
\newtcbtheorem[use counter=commonbox]{outil}{Outil }%
{
	enhanced,
	colback=white,
	colframe=mygr,
	attach boxed title to top left={yshift*=-\tcboxedtitleheight},
	fonttitle=\bfseries,
	title={#2},
	boxed title size=title,
	boxed title style={%
			sharp corners,
			rounded corners=northwest,
			colback=tcbcolframe,
			boxrule=0pt,
		},
	underlay boxed title={%
			\path[fill=tcbcolframe] (title.south west)--(title.south east)
			to[out=0, in=180] ([xshift=5mm]title.east)--
			(title.center-|frame.east)
			[rounded corners=\kvtcb@arc] |-
			(frame.north) -| cycle;
		},
	#1
}{th}
\newtcbtheorem[use counter=commonbox]{but}{Buts du chapitre }%
{
	enhanced,
	colback=white,
	colframe=mygr,
	attach boxed title to top left={yshift*=-\tcboxedtitleheight},
	fonttitle=\bfseries,
	title={#2},
	boxed title size=title,
	boxed title style={%
			sharp corners,
			rounded corners=northwest,
			colback=tcbcolframe,
			boxrule=0pt,
		},
	underlay boxed title={%
			\path[fill=tcbcolframe] (title.south west)--(title.south east)
			to[out=0, in=180] ([xshift=5mm]title.east)--
			(title.center-|frame.east)
			[rounded corners=\kvtcb@arc] |-
			(frame.north) -| cycle;
		},
	#1
}{th}
\newtcbtheorem[use counter=commonbox]{propriete}{Propriété }%
{
	enhanced,
	colback=white,
	colframe=mygr,
	attach boxed title to top left={yshift*=-\tcboxedtitleheight},
	fonttitle=\bfseries,
	title={#2},
	boxed title size=title,
	boxed title style={%
			sharp corners,
			rounded corners=northwest,
			colback=tcbcolframe,
			boxrule=0pt,
		},
	underlay boxed title={%
			\path[fill=tcbcolframe] (title.south west)--(title.south east)
			to[out=0, in=180] ([xshift=5mm]title.east)--
			(title.center-|frame.east)
			[rounded corners=\kvtcb@arc] |-
			(frame.north) -| cycle;
		},
	#1
}{th}
\newtcbtheorem[number within=commonbox]{definition}{Définition }%
{
	enhanced,
	colback=white,
	colframe=mygr,
	attach boxed title to top left={yshift*=-\tcboxedtitleheight},
	fonttitle=\bfseries,
	title={#2},
	boxed title size=title,
	boxed title style={%
			sharp corners,
			rounded corners=northwest,
			colback=tcbcolframe,
			boxrule=0pt,
		},
	underlay boxed title={%
			\path[fill=tcbcolframe] (title.south west)--(title.south east)
			to[out=0, in=180] ([xshift=5mm]title.east)--
			(title.center-|frame.east)
			[rounded corners=\kvtcb@arc] |-
			(frame.north) -| cycle;
		},
	#1
}{th}
\newtcbtheorem[number within=commonbox]{exemples}{Exemples }%
{
	enhanced,
	colback=white,
	colframe=mygr,
	attach boxed title to top left={yshift*=-\tcboxedtitleheight},
	fonttitle=\bfseries,
	title={#2},
	boxed title size=title,
	boxed title style={%
			sharp corners,
			rounded corners=northwest,
			colback=tcbcolframe,
			boxrule=0pt,
		},
	underlay boxed title={%
			\path[fill=tcbcolframe] (title.south west)--(title.south east)
			to[out=0, in=180] ([xshift=5mm]title.east)--
			(title.center-|frame.east)
			[rounded corners=\kvtcb@arc] |-
			(frame.north) -| cycle;
		},
	#1
}{th}
\newtcbtheorem[number within=commonbox]{exemple}{Exemple }%
{
	enhanced,
	colback=white,
	colframe=mygr,
	attach boxed title to top left={yshift*=-\tcboxedtitleheight},
	fonttitle=\bfseries,
	title={#2},
	boxed title size=title,
	boxed title style={%
			sharp corners,
			rounded corners=northwest,
			colback=tcbcolframe,
			boxrule=0pt,
		},
	underlay boxed title={%
			\path[fill=tcbcolframe] (title.south west)--(title.south east)
			to[out=0, in=180] ([xshift=5mm]title.east)--
			(title.center-|frame.east)
			[rounded corners=\kvtcb@arc] |-
			(frame.north) -| cycle;
		},
	#1
}{th}
\newtcbtheorem[number within=commonbox]{questions}{Questions guidantes }%
{
	enhanced,
	colback=white,
	colframe=mygr,
	attach boxed title to top left={yshift*=-\tcboxedtitleheight},
	fonttitle=\bfseries,
	title={#2},
	boxed title size=title,
	boxed title style={%
			sharp corners,
			rounded corners=northwest,
			colback=tcbcolframe,
			boxrule=0pt,
		},
	underlay boxed title={%
			\path[fill=tcbcolframe] (title.south west)--(title.south east)
			to[out=0, in=180] ([xshift=5mm]title.east)--
			(title.center-|frame.east)
			[rounded corners=\kvtcb@arc] |-
			(frame.north) -| cycle;
		},
	#1
}{th}
\makeatother

% corps
\newcommand{\R}{\mathbb{R}}
\newcommand{\Rnn}{\mathbb{R}^{2n}}
\newcommand{\Z}{\mathbb{Z}}
\newcommand{\N}{\mathbb{N}}
\newcommand{\Q}{\mathbb{Q}}

% domain
\newcommand{\D}{\mathcal{D}}
% for calligraphic C
\usepackage{calrsfs}
\newcommand{\C}{\mathcal{C}}

% date
\usepackage{advdate}

% ensembles tq. 
\newcommand{\xRtq}[1]{
	$\left\{ x \in \R \text{ tq. } #1 \right\}$
}

% vabs
\newcommand{\vabs}[1]{
	\left| #1 \right|
}

%pinfty minfty
\newcommand{\pinfty}{{+}\infty}
\newcommand{\minfty}{{-}\infty}

% plots
\usepackage{pgfplots}

%virgules
\usepackage{icomma}
\pgfplotsset{/pgf/number format/use comma}

%subfigures
\usepackage{subcaption}

%hyperlink footnote
\usepackage{hyperref}

%wider tabulars
\def\arraystretch{2}
\setlength\tabcolsep{15pt}

% tableaux var, signe
\usepackage{tkz-tab}

\SetDate[08/10/2025]

\begin{document}
\pagestyle{fancy}
\fancyhead[L]{Première spécifique}
\fancyhead[C]{\textbf{Chapitre 3 — Suites géométriques}}
\fancyhead[R]{\today}

\newcommand{\hide}[1]{\phantom{#1}}
% (un)comment below for completion
% cannot renew phantom itself otherwise multicols environment shows a "p" in the right margin idk why
\renewcommand{\hide}[1]{#1}

\dfn{}{
	Soient $P, x \in\R$ deux nombres réels positifs ou nuls.
	Alors
	  	\begin{center}
	  	\og $P \%$ de $x$ \fg~= \hide{ $\dfrac{P}{100} \times x$.}
	  	\end{center}
	En particulier,\hspace{50pt}
	  	\hide{\og $A \%$ de $B$ \fg~= \og $B \%$ de $A$ \fg.}
}{def:pourcentage}

\thm{}{
	%L'évolution d'une valeur correspond à sa multiplication par un coefficient multiplicateur $m$.
	\,\\
	\begin{enumerate}
		\item Augmenter de $P \%$ revient à multiplier par \hide{$m = 1 + \frac{P}{100}$.}
		\item Diminuer de $P \%$ revient à multiplier par \hide{$m = 1 - \frac{P}{100}$.}
	\end{enumerate}
	
	Lorsque deux évolutions successives ont lieu, les coefficients sont multipliés entre eux pour obtenir un coefficient multiplicateur global.

	\begin{center}
	\begin{tikzpicture}
		% nodes
		\draw (0,0) ellipse (2cm and .5cm) node {Valeur initiale};
		
		\draw (5,0) ellipse (2cm and .5cm) node {Nouvelle valeur};
		
		\draw (10,0) ellipse (2cm and .5cm) node {Valeur finale};
		
		% vertices
		\draw[->, thick, BLUE_E] (1cm,.6cm) arc (105:75:7) node[midway, above] {$\times m_1$};
		\draw[->, thick, BLUE_E] (6cm,.6cm) arc (105:75:7) node[midway, above] {$\times m_2$};
		
		\draw[->, thick, RED_E] (1cm,-.5cm) arc (-105:-75:15) node[midway, below=5pt] {$\times \hide{(m_1 \times m_2)}$};
	\end{tikzpicture}
	\end{center}
}{thm:ev-succ}

\thm{}{
	L'évolution réciproque est l'évolution qui permet de revenir à une valeur initale.
	
	\begin{center}
	\begin{tikzpicture}
		% nodes
		\draw (0,0) ellipse (2cm and .5cm) node {Valeur initiale};
		
		\draw (5,0) ellipse (2cm and .5cm) node {Nouvelle valeur};
		
		\draw (10,0) ellipse (2cm and .5cm) node {Valeur initiale};
		
		% vertices
		\draw[->, thick, BLUE_E] (1cm,.6cm) arc (105:75:7) node[midway, above] {$\times m_1$};
		\draw[->, thick, BLUE_E] (6cm,.6cm) arc (105:75:7) node[midway, above] {$\times \hide{\frac1{m_1}}$};
		
		\draw[->, thick, RED_E] (1cm,-.5cm) arc (-105:-75:15) node[midway, below=5pt] {$\times \hide{1}$};
	\end{tikzpicture}
	\end{center}
}{thm:ev-rec}

% pas hyper intéressant en fait?
%\thm{}{
%	Si une valeur de départ subit $n$ évolutions successives, alors on appelle \emph{coefficient multiplicateur moyen} le nombre
%		%\[ M_{\text{Moyen}} = \left( M_{\text{Global}} \right)^{1/n}. \]
%		\[ M_{\text{Moyen}} = \dots\dots\dots\dots\dots\dots\dots \]
%
%	\begin{center}
%	\begin{tikzpicture}[scale=.8]
%		% nodes
%		\draw (0,0) ellipse (2cm and .5cm) node {Valeur $1$};
%		
%		\draw (5,0) ellipse (2cm and .5cm) node {Valeur $2$};
%		
%		\draw (10,0) ellipse (2cm and .5cm) node {Valeur $3$};
%		
%		\draw (15,0) ellipse (2cm and .5cm) node {Valeur $4$};
%		
%		% vertices
%		\draw[->, thick, myg] (1cm,.6cm) arc (105:75:7) node[midway, above] {$\times m_1$};
%		\draw[->, thick, myg] (6cm,.6cm) arc (105:75:7) node[midway, above] {$\times m_2$};
%		\draw[->, thick, myg] (11cm,.6cm) arc (105:75:7) node[midway, above] {$\times m_3$};
%		
%		\draw[->, thick, myr] (1cm,-.5cm) arc (-105:-75:25) node[midway, below] {$\times M_{\text{Global}}$};
%	\end{tikzpicture}
%	\end{center}
%}{thm:ev-rec}

\ex{}{
	\begin{enumerate}[itemsep=15pt]
		\item Une augmentation de 20\% correspond à \hide{multiplier par 1,2.}
		\item Une diminution de 30\% correspond à \hide{multiplier par 0,7.}
		\item Multiplier par 1,4 correspond à \hide{une augmentation de 40\%.}
		\item Diviser par 2 correspond à \hide{une diminution de 50\%.}
		\item L'évolution réciproque d'une augmentation de 25\% est \hide{une diminution de 20\%.}
	\end{enumerate}
}{ex:ev}

\dfn{}{
	Pour $n>0$ entier, $q>0$ réel, \og $q$ puissance $n$ \fg~est égal à
		\[ q^{n} = \underbrace{q \times q \times \cdots \times q}_{\hide{\text{$n$ fois}}}. \]
	De plus, \hspace{70pt} \hide{$q^0 = 1$} \hspace{70pt} et \hspace{70pt} \hide{$q^{-n} = \dfrac1{q^n}$.}
}{dfn:puissances}


%\ex{}{
%	\begin{multicols}{3}
%	\begin{enumerate}[label=]
%		\item $2^4 = 16$
%		\item $10^{-3} = 0,001$
%		\item $5^{-2} = 0,04$
%	\end{enumerate}
%	\end{multicols}
%}{ex:ev}

\end{document}
