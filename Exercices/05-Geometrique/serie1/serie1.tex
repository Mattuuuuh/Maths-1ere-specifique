\documentclass[a4paper, 12pt]{extarticle}

\usepackage[utf8x]{inputenc}
%fonts
\usepackage{libertinus,libertinust1math}
\usepackage{amsmath,amsthm,amssymb,mathtools}

% SOLUTION SWITCH

\ifsolutions
	\newcommand{\exe}[2]{
		\begin{ex} #1  \end{ex}
		\begin{sol} #2 \end{sol}
	}
\else
	\newcommand{\exe}[2]{
		\begin{ex} #1  \end{ex}
	}
	
\fi


\usepackage[french]{babel}
\usepackage[
a4paper,
margin=2cm,
nomarginpar,% We don't want any margin paragraphs
]{geometry}

% HEADER, ARRAY, ENUM, MULTIOCL
\usepackage{fancyhdr}
\usepackage{array}
\usepackage{multicol, enumitem}
\newcolumntype{P}[1]{>{\centering\arraybackslash}p{#1}}
\usepackage{stackengine}
\newcommand\xrowht[2][0]{\addstackgap[.5\dimexpr#2\relax]{\vphantom{#1}}}

% theorems

\theoremstyle{theorem}
\newtheorem{thm}{Théorème}
\theoremstyle{plain}
\newtheorem*{sol}{Solution}
\theoremstyle{definition}
\newtheorem{ex}{Exercice}
\newtheorem{dfn}{Définition}
\newtheorem*{dfn*}{Définition}


%couleurs
\usepackage{tcolorbox}
\definecolor{myg}{RGB}{56, 140, 70}
\definecolor{myb}{RGB}{45, 111, 177}
\definecolor{myr}{RGB}{199, 68, 64}
\definecolor{mygr}{HTML}{2C3338}


\tcbuselibrary{theorems,skins,hooks}
\newcounter{commonbox}
\makeatletter
\newtcbtheorem[use counter=commonbox]{theorem}{Théorème }%
{
	enhanced,
	colback=white,
	colframe=mygr,
	attach boxed title to top left={yshift*=-\tcboxedtitleheight},
	fonttitle=\bfseries,
	title={#2},
	boxed title size=title,
	boxed title style={%
			sharp corners,
			rounded corners=northwest,
			colback=tcbcolframe,
			boxrule=0pt,
		},
	underlay boxed title={%
			\path[fill=tcbcolframe] (title.south west)--(title.south east)
			to[out=0, in=180] ([xshift=5mm]title.east)--
			(title.center-|frame.east)
			[rounded corners=\kvtcb@arc] |-
			(frame.north) -| cycle;
		},
	#1
}{th}
\newtcbtheorem[use counter=commonbox]{rappel}{Rappel }%
{
	enhanced,
	colback=white,
	colframe=mygr,
	attach boxed title to top left={yshift*=-\tcboxedtitleheight},
	fonttitle=\bfseries,
	title={#2},
	boxed title size=title,
	boxed title style={%
			sharp corners,
			rounded corners=northwest,
			colback=tcbcolframe,
			boxrule=0pt,
		},
	underlay boxed title={%
			\path[fill=tcbcolframe] (title.south west)--(title.south east)
			to[out=0, in=180] ([xshift=5mm]title.east)--
			(title.center-|frame.east)
			[rounded corners=\kvtcb@arc] |-
			(frame.north) -| cycle;
		},
	#1
}{th}
\newtcbtheorem[use counter=commonbox]{strategie}{Stratégie }%
{
	enhanced,
	colback=white,
	colframe=mygr,
	attach boxed title to top left={yshift*=-\tcboxedtitleheight},
	fonttitle=\bfseries,
	title={#2},
	boxed title size=title,
	boxed title style={%
			sharp corners,
			rounded corners=northwest,
			colback=tcbcolframe,
			boxrule=0pt,
		},
	underlay boxed title={%
			\path[fill=tcbcolframe] (title.south west)--(title.south east)
			to[out=0, in=180] ([xshift=5mm]title.east)--
			(title.center-|frame.east)
			[rounded corners=\kvtcb@arc] |-
			(frame.north) -| cycle;
		},
	#1
}{th}
\newtcbtheorem[use counter=commonbox]{outil}{Outil }%
{
	enhanced,
	colback=white,
	colframe=mygr,
	attach boxed title to top left={yshift*=-\tcboxedtitleheight},
	fonttitle=\bfseries,
	title={#2},
	boxed title size=title,
	boxed title style={%
			sharp corners,
			rounded corners=northwest,
			colback=tcbcolframe,
			boxrule=0pt,
		},
	underlay boxed title={%
			\path[fill=tcbcolframe] (title.south west)--(title.south east)
			to[out=0, in=180] ([xshift=5mm]title.east)--
			(title.center-|frame.east)
			[rounded corners=\kvtcb@arc] |-
			(frame.north) -| cycle;
		},
	#1
}{th}
\newtcbtheorem[use counter=commonbox]{but}{Buts du chapitre }%
{
	enhanced,
	colback=white,
	colframe=mygr,
	attach boxed title to top left={yshift*=-\tcboxedtitleheight},
	fonttitle=\bfseries,
	title={#2},
	boxed title size=title,
	boxed title style={%
			sharp corners,
			rounded corners=northwest,
			colback=tcbcolframe,
			boxrule=0pt,
		},
	underlay boxed title={%
			\path[fill=tcbcolframe] (title.south west)--(title.south east)
			to[out=0, in=180] ([xshift=5mm]title.east)--
			(title.center-|frame.east)
			[rounded corners=\kvtcb@arc] |-
			(frame.north) -| cycle;
		},
	#1
}{th}
\newtcbtheorem[use counter=commonbox]{propriete}{Propriété }%
{
	enhanced,
	colback=white,
	colframe=mygr,
	attach boxed title to top left={yshift*=-\tcboxedtitleheight},
	fonttitle=\bfseries,
	title={#2},
	boxed title size=title,
	boxed title style={%
			sharp corners,
			rounded corners=northwest,
			colback=tcbcolframe,
			boxrule=0pt,
		},
	underlay boxed title={%
			\path[fill=tcbcolframe] (title.south west)--(title.south east)
			to[out=0, in=180] ([xshift=5mm]title.east)--
			(title.center-|frame.east)
			[rounded corners=\kvtcb@arc] |-
			(frame.north) -| cycle;
		},
	#1
}{th}
\newtcbtheorem[number within=commonbox]{definition}{Définition }%
{
	enhanced,
	colback=white,
	colframe=mygr,
	attach boxed title to top left={yshift*=-\tcboxedtitleheight},
	fonttitle=\bfseries,
	title={#2},
	boxed title size=title,
	boxed title style={%
			sharp corners,
			rounded corners=northwest,
			colback=tcbcolframe,
			boxrule=0pt,
		},
	underlay boxed title={%
			\path[fill=tcbcolframe] (title.south west)--(title.south east)
			to[out=0, in=180] ([xshift=5mm]title.east)--
			(title.center-|frame.east)
			[rounded corners=\kvtcb@arc] |-
			(frame.north) -| cycle;
		},
	#1
}{th}
\newtcbtheorem[number within=commonbox]{exemples}{Exemples }%
{
	enhanced,
	colback=white,
	colframe=mygr,
	attach boxed title to top left={yshift*=-\tcboxedtitleheight},
	fonttitle=\bfseries,
	title={#2},
	boxed title size=title,
	boxed title style={%
			sharp corners,
			rounded corners=northwest,
			colback=tcbcolframe,
			boxrule=0pt,
		},
	underlay boxed title={%
			\path[fill=tcbcolframe] (title.south west)--(title.south east)
			to[out=0, in=180] ([xshift=5mm]title.east)--
			(title.center-|frame.east)
			[rounded corners=\kvtcb@arc] |-
			(frame.north) -| cycle;
		},
	#1
}{th}
\newtcbtheorem[number within=commonbox]{exemple}{Exemple }%
{
	enhanced,
	colback=white,
	colframe=mygr,
	attach boxed title to top left={yshift*=-\tcboxedtitleheight},
	fonttitle=\bfseries,
	title={#2},
	boxed title size=title,
	boxed title style={%
			sharp corners,
			rounded corners=northwest,
			colback=tcbcolframe,
			boxrule=0pt,
		},
	underlay boxed title={%
			\path[fill=tcbcolframe] (title.south west)--(title.south east)
			to[out=0, in=180] ([xshift=5mm]title.east)--
			(title.center-|frame.east)
			[rounded corners=\kvtcb@arc] |-
			(frame.north) -| cycle;
		},
	#1
}{th}
\newtcbtheorem[number within=commonbox]{questions}{Questions guidantes }%
{
	enhanced,
	colback=white,
	colframe=mygr,
	attach boxed title to top left={yshift*=-\tcboxedtitleheight},
	fonttitle=\bfseries,
	title={#2},
	boxed title size=title,
	boxed title style={%
			sharp corners,
			rounded corners=northwest,
			colback=tcbcolframe,
			boxrule=0pt,
		},
	underlay boxed title={%
			\path[fill=tcbcolframe] (title.south west)--(title.south east)
			to[out=0, in=180] ([xshift=5mm]title.east)--
			(title.center-|frame.east)
			[rounded corners=\kvtcb@arc] |-
			(frame.north) -| cycle;
		},
	#1
}{th}
\makeatother

% corps
\newcommand{\R}{\mathbb{R}}
\newcommand{\Rnn}{\mathbb{R}^{2n}}
\newcommand{\Z}{\mathbb{Z}}
\newcommand{\N}{\mathbb{N}}
\newcommand{\Q}{\mathbb{Q}}

% domain
\newcommand{\D}{\mathcal{D}}
% for calligraphic C
\usepackage{calrsfs}
\newcommand{\C}{\mathcal{C}}

% date
\usepackage{advdate}

% ensembles tq. 
\newcommand{\xRtq}[1]{
	$\left\{ x \in \R \text{ tq. } #1 \right\}$
}

% vabs
\newcommand{\vabs}[1]{
	\left| #1 \right|
}

%pinfty minfty
\newcommand{\pinfty}{{+}\infty}
\newcommand{\minfty}{{-}\infty}

% plots
\usepackage{pgfplots}

%virgules
\usepackage{icomma}
\pgfplotsset{/pgf/number format/use comma}

%subfigures
\usepackage{subcaption}

%hyperlink footnote
\usepackage{hyperref}

%wider tabulars
\def\arraystretch{2}
\setlength\tabcolsep{15pt}

% tableaux var, signe
\usepackage{tkz-tab}

\SetDate[05/11/2025]

\begin{document}
\pagestyle{fancy}
\fancyhead[L]{Première}
\fancyhead[C]{\textbf{Suites géométriques}}
\fancyhead[R]{\today}


\exe{}{
	Jean de Florette élève une population de lapin de la race Romarin.
	Il étudie le nombre de lapins pendant plusieurs années avant de remarquer que, d'une année à l'autre, la population augmente systématiquement de $50\%$.
	Il note les premières valeurs obtenues dans le tableau en commençant à l'année $0$ pour simplifier les choses.
	\begin{center}
	\begin{tabular}{|c|c|c|c|c|c|}\hline
	Année & 0 & 1 & 2 & 3 & 4 \\ \hline
	Population & 256 & 384 & & & \\ \hline
	\end{tabular}
	\end{center}
	
	\begin{enumerate}
		\item Vérifier les premières valeurs du tableau ci-dessus et le compléter.
	\end{enumerate}
	On note $P(n)$ le nombre de lapins à l'année $n$.
	\begin{enumerate}[resume]
		\item Écrire l'expression algébrique de $P(n)$.
		\item Donner $P(20)$ en arrondissant à l'entier le plus proche.
		\item Trouver le plus petit entier naturel $N\in\N$ tel que
			\[ P(N) \geq 100~000.\]
	\end{enumerate}
}{exe:JeanFlorette}{
	\begin{center}
	\begin{tabular}{|c|c|c|c|c|c|}\hline
	Année & 0 & 1 & 2 & 3 & 4 \\ \hline
	Population & 256 & 384 & \color{red}576  & \color{red} 864 & \color{red} 1296 \\ \hline
	\end{tabular}
	\end{center}
	
	\begin{enumerate}[start=2]
		\item 
		La suite est géométrique de raison 1,5 et de terme initial $P(0) = 256$, donc
			\[ P(n) = 256 \times (1,5)^n. \]
		\item 
		À l'aide de la calculatrice, $P(20) = 256 \times (1,5)^{20} \approx 851~266$.
		\item 
		La suite $P(n)$ est croissante, donc le $N$ recherché doit être inférieur à 20 (et supérieur à 4, d'après le tableau).
		
		On teste donc $P(10) \approx 14~762$, qui implique que $N > 10$.
		
		Ensuite, $P(15) \approx 112~101$, qui implique que $N \leq 15$.
		
		Enfin, $P(14) \approx 74~734$, qui implique que $N > 14$, et qui conclut que $N=15$.
	\end{enumerate}
}




\exe{}{
	À l'âge de 17 ans une élève décide de placer 200€ en bourse qui lui rapportent 10\% d'intérêts chaque année.
	Chaque année, elle replace les intérêts gagnés.
	
	On souhaite étudier l'évolution de l'argent placé chaque année après ses 17 ans inclus.
	\begin{enumerate}
		\item Vérifier les premières valeurs du tableau suivantes et le compléter.
			\begin{center}
			\begin{tabular}{|c|c|c|c|c|c|c|}\hline
				Âge & 17 & 18 & 19 & 20 & 21 & 22 \\ \hline
				Argent placé (€) & 200 & 220 & 242 &  & & \\ \hline
				$n$ & 0&1&2&3&4&5 \\ \hline
			\end{tabular}
			\end{center}
	\end{enumerate}
	On appelle $A(n)$ la quantité d'argent placé à l'âge $17+n$, où $n\in \{0 ; 1 ; 2; \dots \}$.
	\begin{enumerate}[resume]
		\item Écrire l'expression algébrique de $A(n)$.	
		\item Combien d'argent aura l'élève à l'âge de $50$ ans ? 
		\item Calculer $A(50)$ et interpréter le résultat.
		\item À quel âge la somme d'argent dépassera-t-elle $100 \  000$€ ?		
	\end{enumerate}
 }{exe:interets}{
 
	\begin{center}
	\begin{tabular}{|c|c|c|c|c|c|c|}\hline
		Âge & 17 & 18 & 19 & 20 & 21 & 22 \\ \hline
		Argent placé (€) & 200 & 220 & 242 & \color{red} $266,2$  & \color{red} $292,82$ &\color{red} $322,102$   \\ \hline
		$n$ & 0&1&2&3&4&5 \\ \hline
	\end{tabular}
	\end{center}
			
	\begin{enumerate}
		\item Un gain de $10\%$ correspond à un coefficient multiplicateur de $1+\dfrac{10}{100} = 1,1$.
		\item 
		Pour obtenir $A(n+1)$, on multiplie $A(n)$ par $1,1$. On a donc la relation de récurrence suivante.
			\[ A(n+1) = 1,1 \times A(n), \]
		valable pour tout $n\in\N$ entier naturel.
		
		À l'âge de $50 = 17+n$ ans, on doit calculer $A(n)$ pour $n=50-17 = 34$, et donc $A(34)$.
		
		On utilise l'expression algébrique de $A(n)$ qui est
			\[ A(n) = 200 \times (1,1)^n, \]
		pour trouver $A(34) \approx 5109,5.$
		\item $A(50) = 200 \times (1,1)^{50} \approx 23~478,2$.
		C'est l'argent économisé à l'âge de $17+50 = 67$ ans.
		\item 
		On résoud à l'aide d'un tableau de valeurs ou par dichotomie.
		Comme $A(65) < 100~000 < A(66)$, le plus petit $n$ pour lequel $A(n)$ dépasse $100~000$ est $66$.
		Pour $n=66$, l'âge correspondant est $17+66 = 83$ ans.
	\end{enumerate}
 
}
 
\exe{}{
	Pour chacune des suites données algébriquement pour tout $n\in\N$, 
		\begin{itemize}[leftmargin=2cm]
			\item calculer les trois premiers termes ; et
			\item décider si elle est géométrique ou non.
		\end{itemize}
	\begin{multicols}{2}
	\begin{enumerate}
		\item $a(n) = 3^n$
		\item $f(n) = 3n + 2$
		\item $b(n) = \left(\dfrac25\right)^n$
		\item $c(n) = 5 \times 2^n$
		\item $g(n) = 3-n$
		\item $h(n) =  \dfrac3{n+1}$
	\end{enumerate}
	\end{multicols}

}{exe:geom-or-not}{
	\begin{enumerate}
		\item 
		$a$ est géométrique car elle respecte le théorème du cours avec $a(0) = 1$ et $q=3$, car $3^n = 3^n \times 1$.
		On peut également utiliser la définition d'une suite géométrique et que 
			\[ a(n+1) = 3^{n+1} = 3^{1} \times 3^{n} = 3 \times a(n). \]
		\item 
		Supposons que $f$ soit géométrique non nulle.
		Alors le ratio $\dfrac{f(n+1)}{f(n)} = q$, et est donc constant quelque soit $n$.
		
		En $n=0$, on calcule
			\[\dfrac{f(1)}{f(0)} = \dfrac52. \]
		En $n=1$, on calcule
			\[\dfrac{f(2)}{f(1)} = \dfrac85. \]
		Comme $\dfrac52 \neq \dfrac85$, la suite $f$ ne peut pas être géométrique (elle est arithmétique en fait).
		
		\item $b(n) = \left(\dfrac25\right)^n \times 1$, donc elle est géométrique.
		\item $c(n) = 5 \times 2^n$ est géométrique.
		\item On calcule deux ratios successifs.
			\begin{align*}
				\dfrac{g(1)}{g(0)} = \dfrac23 && \neq &&  \dfrac{g(2)}{g(1)} = \dfrac12,
			\end{align*}
		ce qui implique que $g$ ne peut pas être géométrique.
		\item On calcule deux ratios successifs.
			\begin{align*}
				\dfrac{h(1)}{h(0)} = \dfrac12 &&  \neq && \dfrac{h(2)}{h(1)} = \dfrac23,
			\end{align*}
		ce qui implique que $h$ ne peut pas être géométrique.
	\end{enumerate}

}

\newpage

\exe{, difficulty=1}{
	Lors de votre entretien d'embauche, une entreprise vous propose un salaire de départ de 35 000€ avec deux choix d'évolutions possibles pour ce salaire
	
	\begin{enumerate}[start=0,label={\bfseries Choix~\arabic*:},leftmargin=3cm]
		\item une augmentation annuelle de 2\% ;
		\item une augmentation annuelle de 7 00€.
	\end{enumerate}
	
	\begin{enumerate}
		\item Calculer le salaire (arrondi à l'euro près) à la 20ème année pour chacun des choix.
		\item Quel choix est, selon vous, le plus avantageux ? Expliquer.
		\item Selon les statistiques de l'OCDE, les français restent en moyenne 11 ans dans une même entreprise. En tenant compte de cette information, quel choix est probablement le plus avantageux ?
	\end{enumerate}
}{exe:exo9}{
	\begin{enumerate}
		\item
		Pour le premier choix : la suite est géométrique de raison $1,02$ et de terme initial 35 000.
		Ainsi, après 20 ans, le salaire est donné par
			\[ 35~000\times1,02^{20} \approx 52~008. \]
		Pour le second choix, la suite est arithmétique de raison 700 et de terme initial 35 000.
		Ainsi, après 20 ans, le salaire est donné par
			\[ 35~000 + 20 \times 700 = 49~000. \] 
		\item 
		Le premier choix semble donc plus avantageux après 20 ans d'ancienneté.
		\item
		On effectue les mêmes calculs sur 11 ans.
		Le premier choix donne
			\[ 35~000\times1,02^{11} \approx 43~518, \]
		et le deuxième choix
			\[ 35~000 + 11 \times 700 = 42~700. \] 
		Le premier choix est donc également plus avantageux.
	\end{enumerate}
}

\begin{multicols}{2}
\setlength\columnseprule{.1pt} 

\exe{}{
	Les suites suivantes données graphiquement peuvent-elles être géométriques ?
	
	\begin{center}
	\begin{tikzpicture}[>=stealth, scale=1]
		\begin{axis}[xmin = 0, xmax=4.2, xtick={ 0,1,2, 3, 4,5}, ymin=0, ymax=300, ytick={0, 30, ..., 300}, axis x line=middle, axis y line=middle, axis line style=->, ylabel={}, grid=both, extra x ticks = {0}]
			
			\addplot[black, thick, only marks, mark=star] coordinates {(0, 130) (1,140) (2,150) (3,160) (4,170)};
			
			\addplot[black, thick, only marks, mark=square] coordinates {(0,15) (1,30) (2,60) (3, 120) (4, 240)};
			
			\addplot[black, thick, only marks, mark=*] coordinates {(1,300) (2,200) (3,133.33) (4, 88.89)};
		\end{axis}
	
	\end{tikzpicture}
	\end{center}

}{exe:graph1}{
	On calcule les ratios successifs : s'ils ne sont pas constants, alors la suite n'est pas géométrique.
	Dans le cas contraire, on ne peut pas réellement conclure que la suite est géométrique, car nous n'avons qu'un échantillons restreint des images (et donc des ratios).
	
	Seule la suite $\star$ peut être écartée ici.
}

\exe{}{
	Donner le terme de rang $n \in \N$ des suites géométriques $\star$, $\bullet$, et $\square$ données graphiquement.

	\begin{center}
	\begin{tikzpicture}[>=stealth, scale=1]
		\begin{axis}[xmin = 0, xmax=4.2, xtick={ 0,1,2, 3, 4,5}, ymin=0, ymax=10000, ymode=log, log ticks with fixed point, axis x line=middle, axis y line=middle, axis line style=->, ylabel={}, grid=both, extra x ticks = {0}]
			
			\addplot[black, thick, only marks, mark=star] coordinates {(0, 1) (1,10) (2,100) (3,1000) (4,10000)};
			
			\addplot[black, thick, only marks, mark=square] coordinates {(0,10 000) (1,10 00) (2,100) (3,10) (4,1)};
			
			\addplot[black, thick, only marks, mark=*] coordinates {(0,3) (1,30) (2,300) (3, 3000)};
		\end{axis}
	
	\end{tikzpicture}
	\end{center}

}{exe:graph2}{
	Dans ce repère à échelle logarithmique, la première graduation des ordonnées est $1$.
	On obtient donc
		\begin{align*}
			\star(n) = 1 \times 10^n, && \square(n) = 10~000\times \left(\dfrac1{10}\right)^n && \bullet(n) = 3 \times 10^n
		\end{align*} 
}

\end{multicols}


\exe{}{
	Un ami vous propose de créer une entreprise dont le principe est le suivant :
	\begin{itemize}
		\item Vous recrutez des ``investisseurs'' en leurs proposant de verser 1 000€
		et en leur promettant qu'ils doubleront leur mise après une semaine. 
		\item Pour payer le premier investisseur après une semaine, vous recrutez deux nouveaux investisseurs.
		\item Vous poursuivez ainsi, en recrutant toujours deux nouveaux investisseurs pour en payer un.
	\end{itemize}
	On note $u(n)$ le nombre de nouveaux investisseurs recrutés en semaine numéro $n$. On suppose que l'entreprise démarre en semaine 0 avec un seul investisseur.
	\begin{enumerate}
		\item Calculer $u(0)$, $u(1)$ et $u(2)$.
		\item Quelle est la nature de la suite $u$ ? Justifier.
		\item Estimer le nombre de nouveaux investisseurs nécessaires après une année entière (52 semaines). 
		Le plan de votre ami semble-t-il possible ? Expliquer.
	\end{enumerate}
	\textit{Ce type de montage se nomme « système de Ponzi ».}
}{exe:5}{
	\begin{enumerate}
		\item 
		$u(0) = 1, u(1) = 2, u(2) = 4$.
		\item 
		$u$ est géométrique de raison 2 car pour passer d'un terme au suivant, on multiplie par 2.
		\item 
		$u(52) = 2^{52} \approx 4\times10^{15}$.
		Comme il existe moins de 9 milliards ($9\times10^9$) d'humains sur Terre, le plan n'a pas l'air très crédible.
	\end{enumerate}
}



%%%%%%%%%%%%

\newpage
\fancyhead[C]{\textbf{Solutions}}
\shipoutAnswer

\end{document}
