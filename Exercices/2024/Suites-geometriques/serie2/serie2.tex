% SOLUTION SWITCH
\newif\ifsolutions
				\solutionstrue
				\solutionsfalse
				
\documentclass[a4paper, 12pt]{extarticle}

\usepackage[utf8x]{inputenc}
%fonts
\usepackage{libertinus,libertinust1math}
\usepackage{amsmath,amsthm,amssymb,mathtools}

% SOLUTION SWITCH

\ifsolutions
	\newcommand{\exe}[2]{
		\begin{ex} #1  \end{ex}
		\begin{sol} #2 \end{sol}
	}
\else
	\newcommand{\exe}[2]{
		\begin{ex} #1  \end{ex}
	}
	
\fi


\usepackage[french]{babel}
\usepackage[
a4paper,
margin=2cm,
nomarginpar,% We don't want any margin paragraphs
]{geometry}

% HEADER, ARRAY, ENUM, MULTIOCL
\usepackage{fancyhdr}
\usepackage{array}
\usepackage{multicol, enumitem}
\newcolumntype{P}[1]{>{\centering\arraybackslash}p{#1}}
\usepackage{stackengine}
\newcommand\xrowht[2][0]{\addstackgap[.5\dimexpr#2\relax]{\vphantom{#1}}}

% theorems

\theoremstyle{theorem}
\newtheorem{thm}{Théorème}
\theoremstyle{plain}
\newtheorem*{sol}{Solution}
\theoremstyle{definition}
\newtheorem{ex}{Exercice}
\newtheorem{dfn}{Définition}
\newtheorem*{dfn*}{Définition}


%couleurs
\usepackage{tcolorbox}
\definecolor{myg}{RGB}{56, 140, 70}
\definecolor{myb}{RGB}{45, 111, 177}
\definecolor{myr}{RGB}{199, 68, 64}
\definecolor{mygr}{HTML}{2C3338}


\tcbuselibrary{theorems,skins,hooks}
\newcounter{commonbox}
\makeatletter
\newtcbtheorem[use counter=commonbox]{theorem}{Théorème }%
{
	enhanced,
	colback=white,
	colframe=mygr,
	attach boxed title to top left={yshift*=-\tcboxedtitleheight},
	fonttitle=\bfseries,
	title={#2},
	boxed title size=title,
	boxed title style={%
			sharp corners,
			rounded corners=northwest,
			colback=tcbcolframe,
			boxrule=0pt,
		},
	underlay boxed title={%
			\path[fill=tcbcolframe] (title.south west)--(title.south east)
			to[out=0, in=180] ([xshift=5mm]title.east)--
			(title.center-|frame.east)
			[rounded corners=\kvtcb@arc] |-
			(frame.north) -| cycle;
		},
	#1
}{th}
\newtcbtheorem[use counter=commonbox]{rappel}{Rappel }%
{
	enhanced,
	colback=white,
	colframe=mygr,
	attach boxed title to top left={yshift*=-\tcboxedtitleheight},
	fonttitle=\bfseries,
	title={#2},
	boxed title size=title,
	boxed title style={%
			sharp corners,
			rounded corners=northwest,
			colback=tcbcolframe,
			boxrule=0pt,
		},
	underlay boxed title={%
			\path[fill=tcbcolframe] (title.south west)--(title.south east)
			to[out=0, in=180] ([xshift=5mm]title.east)--
			(title.center-|frame.east)
			[rounded corners=\kvtcb@arc] |-
			(frame.north) -| cycle;
		},
	#1
}{th}
\newtcbtheorem[use counter=commonbox]{strategie}{Stratégie }%
{
	enhanced,
	colback=white,
	colframe=mygr,
	attach boxed title to top left={yshift*=-\tcboxedtitleheight},
	fonttitle=\bfseries,
	title={#2},
	boxed title size=title,
	boxed title style={%
			sharp corners,
			rounded corners=northwest,
			colback=tcbcolframe,
			boxrule=0pt,
		},
	underlay boxed title={%
			\path[fill=tcbcolframe] (title.south west)--(title.south east)
			to[out=0, in=180] ([xshift=5mm]title.east)--
			(title.center-|frame.east)
			[rounded corners=\kvtcb@arc] |-
			(frame.north) -| cycle;
		},
	#1
}{th}
\newtcbtheorem[use counter=commonbox]{outil}{Outil }%
{
	enhanced,
	colback=white,
	colframe=mygr,
	attach boxed title to top left={yshift*=-\tcboxedtitleheight},
	fonttitle=\bfseries,
	title={#2},
	boxed title size=title,
	boxed title style={%
			sharp corners,
			rounded corners=northwest,
			colback=tcbcolframe,
			boxrule=0pt,
		},
	underlay boxed title={%
			\path[fill=tcbcolframe] (title.south west)--(title.south east)
			to[out=0, in=180] ([xshift=5mm]title.east)--
			(title.center-|frame.east)
			[rounded corners=\kvtcb@arc] |-
			(frame.north) -| cycle;
		},
	#1
}{th}
\newtcbtheorem[use counter=commonbox]{but}{Buts du chapitre }%
{
	enhanced,
	colback=white,
	colframe=mygr,
	attach boxed title to top left={yshift*=-\tcboxedtitleheight},
	fonttitle=\bfseries,
	title={#2},
	boxed title size=title,
	boxed title style={%
			sharp corners,
			rounded corners=northwest,
			colback=tcbcolframe,
			boxrule=0pt,
		},
	underlay boxed title={%
			\path[fill=tcbcolframe] (title.south west)--(title.south east)
			to[out=0, in=180] ([xshift=5mm]title.east)--
			(title.center-|frame.east)
			[rounded corners=\kvtcb@arc] |-
			(frame.north) -| cycle;
		},
	#1
}{th}
\newtcbtheorem[use counter=commonbox]{propriete}{Propriété }%
{
	enhanced,
	colback=white,
	colframe=mygr,
	attach boxed title to top left={yshift*=-\tcboxedtitleheight},
	fonttitle=\bfseries,
	title={#2},
	boxed title size=title,
	boxed title style={%
			sharp corners,
			rounded corners=northwest,
			colback=tcbcolframe,
			boxrule=0pt,
		},
	underlay boxed title={%
			\path[fill=tcbcolframe] (title.south west)--(title.south east)
			to[out=0, in=180] ([xshift=5mm]title.east)--
			(title.center-|frame.east)
			[rounded corners=\kvtcb@arc] |-
			(frame.north) -| cycle;
		},
	#1
}{th}
\newtcbtheorem[number within=commonbox]{definition}{Définition }%
{
	enhanced,
	colback=white,
	colframe=mygr,
	attach boxed title to top left={yshift*=-\tcboxedtitleheight},
	fonttitle=\bfseries,
	title={#2},
	boxed title size=title,
	boxed title style={%
			sharp corners,
			rounded corners=northwest,
			colback=tcbcolframe,
			boxrule=0pt,
		},
	underlay boxed title={%
			\path[fill=tcbcolframe] (title.south west)--(title.south east)
			to[out=0, in=180] ([xshift=5mm]title.east)--
			(title.center-|frame.east)
			[rounded corners=\kvtcb@arc] |-
			(frame.north) -| cycle;
		},
	#1
}{th}
\newtcbtheorem[number within=commonbox]{exemples}{Exemples }%
{
	enhanced,
	colback=white,
	colframe=mygr,
	attach boxed title to top left={yshift*=-\tcboxedtitleheight},
	fonttitle=\bfseries,
	title={#2},
	boxed title size=title,
	boxed title style={%
			sharp corners,
			rounded corners=northwest,
			colback=tcbcolframe,
			boxrule=0pt,
		},
	underlay boxed title={%
			\path[fill=tcbcolframe] (title.south west)--(title.south east)
			to[out=0, in=180] ([xshift=5mm]title.east)--
			(title.center-|frame.east)
			[rounded corners=\kvtcb@arc] |-
			(frame.north) -| cycle;
		},
	#1
}{th}
\newtcbtheorem[number within=commonbox]{exemple}{Exemple }%
{
	enhanced,
	colback=white,
	colframe=mygr,
	attach boxed title to top left={yshift*=-\tcboxedtitleheight},
	fonttitle=\bfseries,
	title={#2},
	boxed title size=title,
	boxed title style={%
			sharp corners,
			rounded corners=northwest,
			colback=tcbcolframe,
			boxrule=0pt,
		},
	underlay boxed title={%
			\path[fill=tcbcolframe] (title.south west)--(title.south east)
			to[out=0, in=180] ([xshift=5mm]title.east)--
			(title.center-|frame.east)
			[rounded corners=\kvtcb@arc] |-
			(frame.north) -| cycle;
		},
	#1
}{th}
\newtcbtheorem[number within=commonbox]{questions}{Questions guidantes }%
{
	enhanced,
	colback=white,
	colframe=mygr,
	attach boxed title to top left={yshift*=-\tcboxedtitleheight},
	fonttitle=\bfseries,
	title={#2},
	boxed title size=title,
	boxed title style={%
			sharp corners,
			rounded corners=northwest,
			colback=tcbcolframe,
			boxrule=0pt,
		},
	underlay boxed title={%
			\path[fill=tcbcolframe] (title.south west)--(title.south east)
			to[out=0, in=180] ([xshift=5mm]title.east)--
			(title.center-|frame.east)
			[rounded corners=\kvtcb@arc] |-
			(frame.north) -| cycle;
		},
	#1
}{th}
\makeatother

% corps
\newcommand{\R}{\mathbb{R}}
\newcommand{\Rnn}{\mathbb{R}^{2n}}
\newcommand{\Z}{\mathbb{Z}}
\newcommand{\N}{\mathbb{N}}
\newcommand{\Q}{\mathbb{Q}}

% domain
\newcommand{\D}{\mathcal{D}}
% for calligraphic C
\usepackage{calrsfs}
\newcommand{\C}{\mathcal{C}}

% date
\usepackage{advdate}

% ensembles tq. 
\newcommand{\xRtq}[1]{
	$\left\{ x \in \R \text{ tq. } #1 \right\}$
}

% vabs
\newcommand{\vabs}[1]{
	\left| #1 \right|
}

%pinfty minfty
\newcommand{\pinfty}{{+}\infty}
\newcommand{\minfty}{{-}\infty}

% plots
\usepackage{pgfplots}

%virgules
\usepackage{icomma}
\pgfplotsset{/pgf/number format/use comma}

%subfigures
\usepackage{subcaption}

%hyperlink footnote
\usepackage{hyperref}

%wider tabulars
\def\arraystretch{2}
\setlength\tabcolsep{15pt}

% tableaux var, signe
\usepackage{tkz-tab}

\AdvanceDate[1]

\begin{document}
\pagestyle{fancy}
\fancyhead[L]{Première}
\fancyhead[C]{\textbf{Suites géométriques 2 \ifsolutions -- Solutions \fi}}
\fancyhead[R]{\today}

\begin{theorem*}{Propriétés des puissances}
	Soient $a, b, c \in \R$. On a les relations suivantes.
		\begin{gather*}
			a^{b+c} = a^{b} \times a^{c} \\
			\left(a^b\right)^c = a^{b \times c} \\
			a^{c} \times b^c = \left( a \times b \right)^c
		\end{gather*}
	En particulier, si $a \neq 0$ et pour tout $n \in \N$, on a
		\begin{align*}
			a^0 = 1, &
			&a^1 = a, &
			&a^{-n} = \dfrac1{a^n}. 
		\end{align*}
\end{theorem*}

\exe{
	Pour chacune des suites données algébriquement pour tout $n\in\N$, décider si elle est géométrique ou non.
	\begin{multicols}{2}
	\begin{enumerate}
		\item $a(n) = 3^n$
		\item $f(n) = 3n + 2$
		\item $b(n) = \left(\dfrac25\right)^n$
		\item $c(n) = 5 \times 2^n$
		\item $g(n) = 3-n$
		\item $h(n) =  \dfrac3{n+1}$
	\end{enumerate}
	\end{multicols}

}{
	\begin{enumerate}
		\item 
		$a$ est géométrique car elle respecte le théorème du cours avec $a(0) = 1$ et $q=3$, car $3^n = 3^n \times 1$.
		On peut également utiliser la définition d'une suite géométrique et que 
			\[ a(n+1) = 3^{n+1} = 3^{1} \times 3^{n} = 3 \times a(n). \]
		\item 
		Supposons que $f$ soit géométrique non nulle.
		Alors le ratio $\dfrac{f(n+1)}{f(n)} = q$, et est donc constant quelque soit $n$.
		
		En $n=0$, on calcule
			\[\dfrac{f(1)}{f(0)} = \dfrac52. \]
		En $n=1$, on calcule
			\[\dfrac{f(2)}{f(1)} = \dfrac85. \]
		Comme $\dfrac52 \neq \dfrac85$, la suite $f$ ne peut pas être géométrique (elle est arithmétique en fait).
		
		\item $b(n) = \left(\dfrac25\right)^n \times 1$, donc elle est géométrique.
		\item $c(n) = 5 \times 2^n$ est géométrique.
		\item On calcule deux ratios successifs.
			\begin{align*}
				\dfrac{g(1)}{g(0)} = \dfrac23, && \neq &&  \dfrac{g(2)}{g(1)} = \dfrac12,
			\end{align*}
		ce qui implique que $g$ ne peut pas être géométrique.
		\item On calcule deux ratios successifs.
			\begin{align*}
				\dfrac{h(1)}{h(0)} = \dfrac12, &&  \neq && \dfrac{h(2)}{h(1)} = \dfrac23,
			\end{align*}
		ce qui implique que $h$ ne peut pas être géométrique.
	\end{enumerate}

}

\exe{
	Pour chacune des suites géométriques données algébriquement pour tout $n\in\N$, donner sa raison et son terme initial.
	\begin{multicols}{2}
	\begin{enumerate}
		\item $u(n) = 2 \times 3^n$
		\item $v(n) = 7 \times \left(\dfrac12 \right)^n$
		\item $w(n) = (-6)^n$
		\item $\zeta(n) = - 6^n$
		\item $a(n) = 11 \times 5^{2n}$
		\item $b(n) = 3 \times 5^{2n+3}$
		\item $c(n) = 10^{-n}$
		\item $d(n) = \dfrac{4}{7^n}$
	\end{enumerate}
	\end{multicols}
}{
	Comme on sait que les suites sont géométriques, le terme initial est donné par $u(0)$ et la raison par $\dfrac{u(1)}{u(0)}$.
	Il suffit donc de savoir évaluer les suites en $0$ et $1$ pour conclure.
	L'identité $a^0 = 1$ du théorème sera utile.

	\begin{enumerate}
		\item 
			\begin{align*}
				u(0) = 2 && q = 3
			\end{align*}
		\item 
			\begin{align*}
				v(0) = 7 && q = \dfrac12
			\end{align*}
		\item 
			\begin{align*}
				w(0) = 1 && q = -6
			\end{align*}
		\item 
			\begin{align*}
				\zeta(0) = -1 && q = 6
			\end{align*}
		\item 
			\begin{align*}
				a(0) = 11 && q = 25
			\end{align*}
		\item 
			\begin{align*}
				b(0) = 375 && q = 25
			\end{align*}
		\item 
			\begin{align*}
				c(0) = 1 && q = \dfrac{1}{10}
			\end{align*}
		\item 
			\begin{align*}
				d(0) = 4 && q = \dfrac17
			\end{align*}
	\end{enumerate}
}


\exe{
	Trouver le plus petit entier naturel $n\in\N$ vérifiant les inéquations suivantes.
	\begin{multicols}{2}
	\begin{enumerate}
		\item $2 \times 3^n > 100~000$
		\item $7 \times \left(\dfrac32 \right)^n > 50~000$
		\item $7 \times \left(\dfrac32 \right)^n > 500~000$
		\item $3 \times \left( \dfrac43 \right)^n > 1~000$
		\item $3 \times \left( \dfrac43 \right)^n > 10~000$
		\item $3 \times \left( \dfrac43 \right)^n > 100~000$
	\end{enumerate}
	\end{multicols}
}{
	On obtient par dichotomie ou avec un tableau de valeurs les rangs suivants.
	
	\begin{multicols}{2}
		\begin{enumerate}
			\item $n=10$
			\item $n=22$
			\item $n=28$
			\item $n=21$
			\item $n=29$
			\item $n=37$
		\end{enumerate}
	\end{multicols}

}

\ifsolutions
\else
\newpage
\fi

\exe{
	Les suites suivantes données graphiquement peuvent-elles être géométriques ?
	
	\begin{center}
	\begin{tikzpicture}[>=stealth, scale=1.5]
		\begin{axis}[xmin = 0, xmax=4.2, xtick={ 0,1,2, 3, 4,5}, ymin=0, ymax=300, ytick={0, 30, ..., 300}, axis x line=middle, axis y line=middle, axis line style=->, ylabel={}, grid=both]
			
			\addplot[black, thick, only marks, mark=star] coordinates {(0, 130) (1,140) (2,150) (3,160) (4,170)};
			
			\addplot[black, thick, only marks, mark=square] coordinates {(0,15) (1,30) (2,60) (3, 120) (4, 240)};
			
			\addplot[black, thick, only marks, mark=*] coordinates {(1,300) (2,200) (3,133.33) (4, 88.89)};
		\end{axis}
	
	\end{tikzpicture}
	\end{center}

}{
	On calcule les ratios successifs : s'ils ne sont pas constants, alors la suite n'est pas géométrique.
	Dans le cas contraire, on ne peut pas réellement conclure que la suite est géométrique, car nous n'avons qu'un échantillons restreint des images (et donc des ratios).
	
	Seule la suite $\star$ peut être écartée ici.
}

\exe{
	Donner le terme de rang $n \in \N$ des suites géométriques $\star$, $\bullet$, et $\square$ données graphiquement.

	\begin{center}
	\begin{tikzpicture}[>=stealth, scale=1.5]
		\begin{axis}[xmin = 0, xmax=4.2, xtick={ 0,1,2, 3, 4,5}, ymin=0, ymax=10000, ymode=log, log ticks with fixed point, axis x line=middle, axis y line=middle, axis line style=->, ylabel={}, grid=both]
			
			\addplot[black, thick, only marks, mark=star] coordinates {(0, 1) (1,10) (2,100) (3,1000) (4,10000)};
			
			\addplot[black, thick, only marks, mark=square] coordinates {(0,10 000) (1,10 00) (2,100) (3,10) (4,1)};
			
			\addplot[black, thick, only marks, mark=*] coordinates {(0,3) (1,30) (2,300) (3, 3000)};
		\end{axis}
	
	\end{tikzpicture}
	\end{center}

}{
	Dans ce repère à échelle logarithmique, la première graduation des ordonnées est $1$.
	On obtient donc
		\begin{align*}
			\star(n) = 1 \times 10^n, && \square(n) = 10~000\times \left(\dfrac1{10}\right)^n && \bullet(n) = 3 \times 10^n
		\end{align*} 
}


\end{document}
