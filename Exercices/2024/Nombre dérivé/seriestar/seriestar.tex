% SOLUTION SWITCH
\newif\ifsolutions
				\solutionstrue
				\solutionsfalse
				
\documentclass[a4paper, 12pt]{extarticle}

\usepackage[utf8x]{inputenc}
%fonts
\usepackage{libertinus,libertinust1math}
\usepackage{amsmath,amsthm,amssymb,mathtools}

% SOLUTION SWITCH

\ifsolutions
	\newcommand{\exe}[2]{
		\begin{ex} #1  \end{ex}
		\begin{sol} #2 \end{sol}
	}
\else
	\newcommand{\exe}[2]{
		\begin{ex} #1  \end{ex}
	}
	
\fi


\usepackage[french]{babel}
\usepackage[
a4paper,
margin=2cm,
nomarginpar,% We don't want any margin paragraphs
]{geometry}

% HEADER, ARRAY, ENUM, MULTIOCL
\usepackage{fancyhdr}
\usepackage{array}
\usepackage{multicol, enumitem}
\newcolumntype{P}[1]{>{\centering\arraybackslash}p{#1}}
\usepackage{stackengine}
\newcommand\xrowht[2][0]{\addstackgap[.5\dimexpr#2\relax]{\vphantom{#1}}}

% theorems

\theoremstyle{theorem}
\newtheorem{thm}{Théorème}
\theoremstyle{plain}
\newtheorem*{sol}{Solution}
\theoremstyle{definition}
\newtheorem{ex}{Exercice}
\newtheorem{dfn}{Définition}
\newtheorem*{dfn*}{Définition}


%couleurs
\usepackage{tcolorbox}
\definecolor{myg}{RGB}{56, 140, 70}
\definecolor{myb}{RGB}{45, 111, 177}
\definecolor{myr}{RGB}{199, 68, 64}
\definecolor{mygr}{HTML}{2C3338}


\tcbuselibrary{theorems,skins,hooks}
\newcounter{commonbox}
\makeatletter
\newtcbtheorem[use counter=commonbox]{theorem}{Théorème }%
{
	enhanced,
	colback=white,
	colframe=mygr,
	attach boxed title to top left={yshift*=-\tcboxedtitleheight},
	fonttitle=\bfseries,
	title={#2},
	boxed title size=title,
	boxed title style={%
			sharp corners,
			rounded corners=northwest,
			colback=tcbcolframe,
			boxrule=0pt,
		},
	underlay boxed title={%
			\path[fill=tcbcolframe] (title.south west)--(title.south east)
			to[out=0, in=180] ([xshift=5mm]title.east)--
			(title.center-|frame.east)
			[rounded corners=\kvtcb@arc] |-
			(frame.north) -| cycle;
		},
	#1
}{th}
\newtcbtheorem[use counter=commonbox]{rappel}{Rappel }%
{
	enhanced,
	colback=white,
	colframe=mygr,
	attach boxed title to top left={yshift*=-\tcboxedtitleheight},
	fonttitle=\bfseries,
	title={#2},
	boxed title size=title,
	boxed title style={%
			sharp corners,
			rounded corners=northwest,
			colback=tcbcolframe,
			boxrule=0pt,
		},
	underlay boxed title={%
			\path[fill=tcbcolframe] (title.south west)--(title.south east)
			to[out=0, in=180] ([xshift=5mm]title.east)--
			(title.center-|frame.east)
			[rounded corners=\kvtcb@arc] |-
			(frame.north) -| cycle;
		},
	#1
}{th}
\newtcbtheorem[use counter=commonbox]{strategie}{Stratégie }%
{
	enhanced,
	colback=white,
	colframe=mygr,
	attach boxed title to top left={yshift*=-\tcboxedtitleheight},
	fonttitle=\bfseries,
	title={#2},
	boxed title size=title,
	boxed title style={%
			sharp corners,
			rounded corners=northwest,
			colback=tcbcolframe,
			boxrule=0pt,
		},
	underlay boxed title={%
			\path[fill=tcbcolframe] (title.south west)--(title.south east)
			to[out=0, in=180] ([xshift=5mm]title.east)--
			(title.center-|frame.east)
			[rounded corners=\kvtcb@arc] |-
			(frame.north) -| cycle;
		},
	#1
}{th}
\newtcbtheorem[use counter=commonbox]{outil}{Outil }%
{
	enhanced,
	colback=white,
	colframe=mygr,
	attach boxed title to top left={yshift*=-\tcboxedtitleheight},
	fonttitle=\bfseries,
	title={#2},
	boxed title size=title,
	boxed title style={%
			sharp corners,
			rounded corners=northwest,
			colback=tcbcolframe,
			boxrule=0pt,
		},
	underlay boxed title={%
			\path[fill=tcbcolframe] (title.south west)--(title.south east)
			to[out=0, in=180] ([xshift=5mm]title.east)--
			(title.center-|frame.east)
			[rounded corners=\kvtcb@arc] |-
			(frame.north) -| cycle;
		},
	#1
}{th}
\newtcbtheorem[use counter=commonbox]{but}{Buts du chapitre }%
{
	enhanced,
	colback=white,
	colframe=mygr,
	attach boxed title to top left={yshift*=-\tcboxedtitleheight},
	fonttitle=\bfseries,
	title={#2},
	boxed title size=title,
	boxed title style={%
			sharp corners,
			rounded corners=northwest,
			colback=tcbcolframe,
			boxrule=0pt,
		},
	underlay boxed title={%
			\path[fill=tcbcolframe] (title.south west)--(title.south east)
			to[out=0, in=180] ([xshift=5mm]title.east)--
			(title.center-|frame.east)
			[rounded corners=\kvtcb@arc] |-
			(frame.north) -| cycle;
		},
	#1
}{th}
\newtcbtheorem[use counter=commonbox]{propriete}{Propriété }%
{
	enhanced,
	colback=white,
	colframe=mygr,
	attach boxed title to top left={yshift*=-\tcboxedtitleheight},
	fonttitle=\bfseries,
	title={#2},
	boxed title size=title,
	boxed title style={%
			sharp corners,
			rounded corners=northwest,
			colback=tcbcolframe,
			boxrule=0pt,
		},
	underlay boxed title={%
			\path[fill=tcbcolframe] (title.south west)--(title.south east)
			to[out=0, in=180] ([xshift=5mm]title.east)--
			(title.center-|frame.east)
			[rounded corners=\kvtcb@arc] |-
			(frame.north) -| cycle;
		},
	#1
}{th}
\newtcbtheorem[number within=commonbox]{definition}{Définition }%
{
	enhanced,
	colback=white,
	colframe=mygr,
	attach boxed title to top left={yshift*=-\tcboxedtitleheight},
	fonttitle=\bfseries,
	title={#2},
	boxed title size=title,
	boxed title style={%
			sharp corners,
			rounded corners=northwest,
			colback=tcbcolframe,
			boxrule=0pt,
		},
	underlay boxed title={%
			\path[fill=tcbcolframe] (title.south west)--(title.south east)
			to[out=0, in=180] ([xshift=5mm]title.east)--
			(title.center-|frame.east)
			[rounded corners=\kvtcb@arc] |-
			(frame.north) -| cycle;
		},
	#1
}{th}
\newtcbtheorem[number within=commonbox]{exemples}{Exemples }%
{
	enhanced,
	colback=white,
	colframe=mygr,
	attach boxed title to top left={yshift*=-\tcboxedtitleheight},
	fonttitle=\bfseries,
	title={#2},
	boxed title size=title,
	boxed title style={%
			sharp corners,
			rounded corners=northwest,
			colback=tcbcolframe,
			boxrule=0pt,
		},
	underlay boxed title={%
			\path[fill=tcbcolframe] (title.south west)--(title.south east)
			to[out=0, in=180] ([xshift=5mm]title.east)--
			(title.center-|frame.east)
			[rounded corners=\kvtcb@arc] |-
			(frame.north) -| cycle;
		},
	#1
}{th}
\newtcbtheorem[number within=commonbox]{exemple}{Exemple }%
{
	enhanced,
	colback=white,
	colframe=mygr,
	attach boxed title to top left={yshift*=-\tcboxedtitleheight},
	fonttitle=\bfseries,
	title={#2},
	boxed title size=title,
	boxed title style={%
			sharp corners,
			rounded corners=northwest,
			colback=tcbcolframe,
			boxrule=0pt,
		},
	underlay boxed title={%
			\path[fill=tcbcolframe] (title.south west)--(title.south east)
			to[out=0, in=180] ([xshift=5mm]title.east)--
			(title.center-|frame.east)
			[rounded corners=\kvtcb@arc] |-
			(frame.north) -| cycle;
		},
	#1
}{th}
\newtcbtheorem[number within=commonbox]{questions}{Questions guidantes }%
{
	enhanced,
	colback=white,
	colframe=mygr,
	attach boxed title to top left={yshift*=-\tcboxedtitleheight},
	fonttitle=\bfseries,
	title={#2},
	boxed title size=title,
	boxed title style={%
			sharp corners,
			rounded corners=northwest,
			colback=tcbcolframe,
			boxrule=0pt,
		},
	underlay boxed title={%
			\path[fill=tcbcolframe] (title.south west)--(title.south east)
			to[out=0, in=180] ([xshift=5mm]title.east)--
			(title.center-|frame.east)
			[rounded corners=\kvtcb@arc] |-
			(frame.north) -| cycle;
		},
	#1
}{th}
\makeatother

% corps
\newcommand{\R}{\mathbb{R}}
\newcommand{\Rnn}{\mathbb{R}^{2n}}
\newcommand{\Z}{\mathbb{Z}}
\newcommand{\N}{\mathbb{N}}
\newcommand{\Q}{\mathbb{Q}}

% domain
\newcommand{\D}{\mathcal{D}}
% for calligraphic C
\usepackage{calrsfs}
\newcommand{\C}{\mathcal{C}}

% date
\usepackage{advdate}

% ensembles tq. 
\newcommand{\xRtq}[1]{
	$\left\{ x \in \R \text{ tq. } #1 \right\}$
}

% vabs
\newcommand{\vabs}[1]{
	\left| #1 \right|
}

%pinfty minfty
\newcommand{\pinfty}{{+}\infty}
\newcommand{\minfty}{{-}\infty}

% plots
\usepackage{pgfplots}

%virgules
\usepackage{icomma}
\pgfplotsset{/pgf/number format/use comma}

%subfigures
\usepackage{subcaption}

%hyperlink footnote
\usepackage{hyperref}

%wider tabulars
\def\arraystretch{2}
\setlength\tabcolsep{15pt}

% tableaux var, signe
\usepackage{tkz-tab}


\AdvanceDate[1]

\begin{document}
\pagestyle{fancy}
\fancyhead[L]{Première}
\fancyhead[C]{\textbf{Fonction dérivée : approfondissements \ifsolutions -- Solutions \fi}}
\fancyhead[R]{\today}


\exe{
	Spinoza, déterministe notoire, jette verticalement un objet de masse $m$ au temps $t=0$ et souhaite connaître sa hauteur $h(t)$ en tout temps $t>0$ à l'aide uniquement de ses conditions initiales : la hauteur initiale $h(0)$, la vitesse initiale $h'(0)$.
	
	La deuxième loi de Newton lui permet d'écrire, en négligeant toute force qui n'est pas la gravité, que l'accélération $h''$ est donnée par $h''(t) = -\dfrac{g}m$, où $m$ est la masse de l'objet et $g$ l'accélération de la pesanteur.
	
	\begin{enumerate}
		\item 
		Montrer, en dérivant deux fois $h$, qu'elle est de la forme
			\[ h(t) = -\dfrac{g}{2m} t^2 + b t + c, \]
		où $b$ et $c$ sont des constantes qu'on ne connaît pas encore.
	
		\item Montrer que $c$ est la hauteur de l'objet au temps initial $t=0$.
		\item Montrer que $b$ est la vitesse de l'objet au temps initial $t=0$.
		\item Écrire la hauteur de l'objet en fonction du temps avec les données $g=9,8 ; m=1 ; h(0) = 3 ; h'(0) = 2$. Quand est-ce que l'objet touche le sol, approximativement ? On cherche le $t$ positif tel que $h(t) = 0$.
	\end{enumerate}
}{}

\exe{
	Soit $f$ une fonction réelle et $c\in\R$ une constante quelconque.
	
	\begin{enumerate}
		\item Montrer que $f(x)$ et $g(x) = f(x)+c$ admettent les mêmes variations. Comment dessiner $\C_g$ en connaissant $\C_f$ ? Comparer par exemple les courbe représentatives de $x^2$ et $x^2 + 2$.
		\item Montrer que $f(x)$ et $c\cdot f(x)$ admettent
			\begin{enumerate}[label=\roman*)]
				\item les mêmes variations si $c > 0$ ; et
				\item des variations opposées si $c < 0$.
			\end{enumerate}
		 Comment dessiner $\C_g$ en connaissant $\C_f$ ? Comparer par exemple les courbe représentatives de $x^2, -x^2,$ et $3x^2$.
	\end{enumerate}
}{}

\exe{
	Donner une fonction $f$ telle que sa dérivée $f'$ soit
		\[ f'(x) =  x^2 - 4. \]
	En quels $x$ la fonction $f$ change-t-elle de variations ? Ce sont les antécédents des extrema locaux.
}{}

\exe{
	Donner une fonction $f$ telle que
		\[ f'(x) = (2-x)(x-1). \]
	En quels $x$ la fonction $f$ change-t-elle de variations ? Ce sont les antécédents des extrema locaux.
}{}

\exe{[$\star$]
	Soient $a_1 < a_2 < \dots < a_n$ des nombres réels ordonnés.
	On pose $f$ comme le produit des facteurs affines $x-a_i$ :
		\[ f(x) = (x-a_1) \cdot (x-a_2) \cdots (x-a_n). \]
	\begin{enumerate}
		\item Montrer que $f$ s'annule exactement en $x=a_1, a_2, \dots, a_n$.
		\item Montrer que $f(x)$ est positif pour $x<a_1$ si $n$ est pair, et négatif sinon.
		\item Montrer que le signe de $f(x)$ change $n$ fois en $x=a_1, a_2, \dots, a_n$.
	\end{enumerate}
}{}

\exe{
	Montrer que, pour $x, h \in \R$,
		\[ (x+h)^4 = x^4 + 4x^3 h +6x^2h^2 + 4xh^3 + h^4. \]
	En déduire que
		\[ \dfrac{(x+h)^4 - x^4}h \]
	tend vers $4x^3$ lorsque $h$ devient nul.
	Conclure que $(x^4)' = 4x^3$.
}{}

\exe{
	Montrer que, pour $x, h \in \R$ tels que $x\neq0$ et $h$ suffisamment petit pour que $x+h\neq0$,
		\[ \dfrac1{x+h} - \dfrac1x = \dfrac{-h}{x(x+h)}. \]
	Conclure que
		\[ \left(\dfrac1x\right)' = \dfrac{-1}{x^2}. \]
}{}

\exe{
	À l'aide du résultat de l'exercice précédent qui montre que
		\[ \left(\dfrac1x\right)' = \dfrac{-1}{x^2}, \]
	montrer que la fonction inverse $f(x) = \dfrac1x$ est strictement décroissante sur son domaine de définition $\R^*$, la droite réelle perforée en $0$.
}{}

\exe{[$\star$]
	Soit $n\in\N_{\geq2}$ un entier naturel supérieur ou égal à $2$. On souhaite montrer que
		\[ (x+h)^n = x^n + nhx^{n-1} + h^2 g(x, h), \]
	où $g$ est une fonction polynomiale en $x$ et $h$.
	
	\begin{enumerate}
		\item Montrer, en développant $(x+h)^2$, que cette égalité est vraie pour $n=2$.
		\item En utilisant que 
			\[ (x+h)^{n+1} = (x+h) \cdot (x+h)^{n} = x(x+h)^n + h(x+h)^n, \]
		montrer que si cette égalité est vraie pour un entier $n$, alors elle est aussi vraie pour $n+1$.
		 \item En déduire que l'égalité est vraie pour tout les entiers naturels $n\in\N_{\geq2}$.
	\end{enumerate}

}{}

\exe{
	Soit $n\in\N_{\geq2}$ un entier naturel supérieur ou égal à $2$. L'exercice précédent donne
		\[ (x+h)^n = x^n + nhx^{n-1} + h^2 g(x, h), \]
	où $g$ est une fonction polynomiale en $x$ et $h$.
	
	En déduire que $(x^n)' = nx^{n-1}$.
}{}

\end{document}