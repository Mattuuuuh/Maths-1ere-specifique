% SOLUTION SWITCH
\newif\ifsolutions
				\solutionstrue
				\solutionsfalse
				
\documentclass[a4paper, 12pt]{extarticle}
\usepackage[french]{babel}
\usepackage[
a4paper,
margin=2cm,
]{geometry}

\usepackage[utf8x]{inputenc}
%fonts
\usepackage{libertinus,libertinust1math}
\usepackage{amsmath,amsthm,amssymb,mathtools}

%virgules
\usepackage{icomma}

% HEADER, ARRAY, ENUM, MULTIOCL
\usepackage{fancyhdr}
\usepackage{array}
\usepackage{multicol, enumitem}
\newcolumntype{P}[1]{>{\centering\arraybackslash}p{#1}}
\usepackage{stackengine}
\newcommand\xrowht[2][0]{\addstackgap[.5\dimexpr#2\relax]{\vphantom{#1}}}

% theorems
\theoremstyle{definition}

\newtheorem{theorem}{Théorème}
\newtheorem{corollaire}[theorem]{Corollaire}
\newtheorem{lemme}[theorem]{Lemme}
\newtheorem{proposition}[theorem]{Proposition}
\newtheorem{exercice}[theorem]{Exercice}
\newtheorem{exemple}[theorem]{Exemple}
\newtheorem{definition}[theorem]{Définition}
\newtheorem*{question}{Question}
\newtheorem*{preuve}{Preuve}
\newtheorem*{remarque}{Remarque}
\newtheorem*{strategie}{Stratégie}
\newtheorem*{methode}{Méthode}
\newtheorem*{notation}{Notation}
\newtheorem*{nomenclature}{Nomenclature}
\newtheorem{axiome}[theorem]{Axiome}
\newtheorem*{heuristique}{Heuristique}

\newtheorem*{definition*}{Définition}
\newtheorem*{lemme*}{Lemme}
\newtheorem*{proposition*}{Proposition}
\newtheorem*{theorem*}{Théorème}
\newtheorem*{corollaire*}{Corollaire}

%%%%%%%%%%%%%%%%%%%%%%%%%%%%
% MDFRAMED SURROUND
%%%%%%%%%%%%%%%%%%%%%%%%%%%%

\usepackage[framemethod=pgf]{mdframed}
% def
\mdfdefinestyle{definition}{
	hidealllines=true,
	leftline=true,
	linecolor=BLUE_E,
	linewidth=2pt,
	innertopmargin=-4pt,
	innerrightmargin=0,
	nobreak=true,
}
\surroundwithmdframed[
	style=definition,
]{definition}
\surroundwithmdframed[
	style=definition,
]{definition*}

% thm
\mdfdefinestyle{theorem}{
	linecolor=MAROON_C,
	linewidth=2pt,
	roundcorner=4pt,
	innertopmargin=-4pt,
	nobreak=true,
}
\surroundwithmdframed[
	style=theorem,
]{theorem}
\surroundwithmdframed[
	style=theorem,
]{theorem*}

% prop
\mdfdefinestyle{proposition}{
	linecolor=GREEN_E,
	linewidth=2pt,
	innertopmargin=-4pt,
	nobreak=true,
}
\surroundwithmdframed[
	style=proposition,
]{proposition}
\surroundwithmdframed[
	style=proposition,
]{proposition*}

% lemme
\mdfdefinestyle{lemme}{
	linecolor=TEAL_E,
	linewidth=1pt,
	innertopmargin=-4pt,
	nobreak=true,
}
\surroundwithmdframed[
	style=lemme,
]{lemme}
\surroundwithmdframed[
	style=lemme,
]{lemme*}

% corollaire
\mdfdefinestyle{corollaire}{
	linecolor=YELLOW_E,
	linewidth=2pt,
	roundcorner=4pt,
	innertopmargin=-4pt,
	nobreak=true,
}
\surroundwithmdframed[
	style=corollaire,
]{corollaire}
\surroundwithmdframed[
	style=corollaire,
]{corollaire*}

% exercices
\usepackage[answerdelayed, lastexercise]{exercise}
\usepackage{ifthen}
\renewcommand{\ExerciseHeader}{
	\tikz[baseline=(R.base)]\node[draw,rectangle, thick, inner sep=2pt](R) {\textbf{\theExercise.}};\!
	\ifnum\ExerciseDifficulty=0
	\else
		(\theExerciseDifficulty)
	\fi
}
\renewcommand{\DifficultyMarker}{$\star$}
\renewcommand{\AnswerHeader}{
	\tikz[baseline=(R.base)]\node[draw,rectangle, thick, inner sep=2pt](R) {\textbf{\theExercise.}};\!
}
\newcommand{\exe}[4]{
	\begin{Exercise}[title=#1, label=#3]
		\if\relax\detokenize\expandafter{\ExerciseTitle}\relax
		%\marginpar{[Bonus]}
		\else
		\marginpar{\mbox{[\ExerciseTitle]}}
		\fi
		#2
	\end{Exercise}
	\begin{Answer}[ref=#3]
		#4
	\end{Answer}
}
\newcommand{\exemulticols}[5]{
	\begin{multicols}{2}
	\begin{Exercise}[title=#1, label=#4]
		\if\relax\detokenize\expandafter{\ExerciseTitle}\relax
		%\marginnote{[Bonus]}
		\else
		\marginnote{\mbox{[\ExerciseTitle]\qquad}}
		\fi
		#2
	\end{Exercise}
	\columnbreak
		#3
	\end{multicols}
	\begin{Answer}[ref=#4]
		#5
	\end{Answer}
}

% date
\usepackage{advdate}

% plots
\usepackage{pgfplots}
\tikzset{
	every axis/.style = {clip=false, axis lines=center, axis line style=<->, xlabel={}, ylabel={}, grid=both, grid style = {opacity=.5}, domain=-2:2}
}

%subfigures
\usepackage{subcaption}

%hyperlink footnote
\usepackage{hyperref}

% tableaux var, signe
\usepackage{tkz-tab}

%wider tabulars
\def\arraystretch{2}
\setlength\tabcolsep{15pt}
\usepackage{makecell} %pour \thead dans tabular ex3 (aligner verticalement le coeff de proportionnalité)

% for striked out implies sign (\centernot\implies)
\usepackage{centernot}

%%%%%%%%%%%%%%%%%%%%%%%%%%%%%%
% SELF MADE COLORS
%%%%%%%%%%%%%%%%%%%%%%%%%%%%%%

%!TEX encoding = UTF8
%!TEX root = 0-notes.tex

%%%%%%%%%%%%%%%%%%%%%%%%%%%%%%
% SELF MADE COLORS
%%%%%%%%%%%%%%%%%%%%%%%%%%%%%%


\definecolor{myg}{RGB}{56, 140, 70}
\definecolor{myb}{RGB}{45, 111, 177}
\definecolor{myr}{RGB}{199, 68, 64}
\definecolor{mytheorembg}{HTML}{F2F2F9}
\definecolor{mytheoremfr}{HTML}{00007B}
\definecolor{mylenmabg}{HTML}{FFFAF8}
\definecolor{mylenmafr}{HTML}{983b0f}
\definecolor{mypropbg}{HTML}{f2fbfc}
\definecolor{mypropfr}{HTML}{191971}
\definecolor{myexamplebg}{HTML}{F2FBF8}
\definecolor{myexamplefr}{HTML}{88D6D1}
\definecolor{myexampleti}{HTML}{2A7F7F}
\definecolor{mydefinitbg}{HTML}{E5E5FF}
\definecolor{mydefinitfr}{HTML}{3F3FA3}
\definecolor{notesgreen}{RGB}{0,162,0}
\definecolor{myp}{RGB}{197, 92, 212}
\definecolor{mygr}{HTML}{2C3338}
\definecolor{myred}{RGB}{127,0,0}
\definecolor{myyellow}{RGB}{169,121,69}
\definecolor{myexercisebg}{HTML}{F2FBF8}
\definecolor{myexercisefg}{HTML}{88D6D1}
\definecolor{doc}{RGB}{0,60,110}

% manim colors because they're beautiful
% https://docs.manim.community/en/stable/reference/manim.utils.color.manim_colors.html

\definecolor{BLACK}{HTML}{000000}\definecolor{BLUE}{HTML}{58C4DD}\definecolor{BLUE_A}{HTML}{C7E9F1}\definecolor{BLUE_B}{HTML}{9CDCEB}\definecolor{BLUE_C}{HTML}{58C4DD}\definecolor{BLUE_D}{HTML}{29ABCA}\definecolor{BLUE_E}{HTML}{236B8E}\definecolor{DARKER_GRAY}{HTML}{222222}\definecolor{DARKER_GREY}{HTML}{222222}\definecolor{DARK_BLUE}{HTML}{236B8E}\definecolor{DARK_BROWN}{HTML}{8B4513}\definecolor{DARK_GRAY}{HTML}{444444}\definecolor{DARK_GREY}{HTML}{444444}\definecolor{GOLD}{HTML}{F0AC5F}\definecolor{GOLD_A}{HTML}{F7C797}\definecolor{GOLD_B}{HTML}{F9B775}\definecolor{GOLD_C}{HTML}{F0AC5F}\definecolor{GOLD_D}{HTML}{E1A158}\definecolor{GOLD_E}{HTML}{C78D46}\definecolor{GRAY}{HTML}{888888}\definecolor{GRAY_A}{HTML}{DDDDDD}\definecolor{GRAY_B}{HTML}{BBBBBB}\definecolor{GRAY_BROWN}{HTML}{736357}\definecolor{GRAY_C}{HTML}{888888}\definecolor{GRAY_D}{HTML}{444444}\definecolor{GRAY_E}{HTML}{222222}\definecolor{GREEN}{HTML}{83C167}\definecolor{GREEN_A}{HTML}{C9E2AE}\definecolor{GREEN_B}{HTML}{A6CF8C}\definecolor{GREEN_C}{HTML}{83C167}\definecolor{GREEN_D}{HTML}{77B05D}\definecolor{GREEN_E}{HTML}{699C52}\definecolor{GREY}{HTML}{888888}\definecolor{GREY_A}{HTML}{DDDDDD}\definecolor{GREY_B}{HTML}{BBBBBB}\definecolor{GREY_BROWN}{HTML}{736357}\definecolor{GREY_C}{HTML}{888888}\definecolor{GREY_D}{HTML}{444444}\definecolor{GREY_E}{HTML}{222222}\definecolor{LIGHTER_GRAY}{HTML}{DDDDDD}\definecolor{LIGHTER_GREY}{HTML}{DDDDDD}\definecolor{LIGHT_BROWN}{HTML}{CD853F}\definecolor{LIGHT_GRAY}{HTML}{BBBBBB}\definecolor{LIGHT_GREY}{HTML}{BBBBBB}\definecolor{LIGHT_PINK}{HTML}{DC75CD}\definecolor{LOGO_BLACK}{HTML}{343434}\definecolor{LOGO_BLUE}{HTML}{525893}\definecolor{LOGO_GREEN}{HTML}{87C2A5}\definecolor{LOGO_RED}{HTML}{E07A5F}\definecolor{LOGO_WHITE}{HTML}{ECE7E2}\definecolor{MAROON}{HTML}{C55F73}\definecolor{MAROON_A}{HTML}{ECABC1}\definecolor{MAROON_B}{HTML}{EC92AB}\definecolor{MAROON_C}{HTML}{C55F73}\definecolor{MAROON_D}{HTML}{A24D61}\definecolor{MAROON_E}{HTML}{94424F}\definecolor{ORANGE}{HTML}{FF862F}\definecolor{PINK}{HTML}{D147BD}\definecolor{PURE_BLUE}{HTML}{0000FF}\definecolor{PURE_GREEN}{HTML}{00FF00}\definecolor{PURE_RED}{HTML}{FF0000}\definecolor{PURPLE}{HTML}{9A72AC}\definecolor{PURPLE_A}{HTML}{CAA3E8}\definecolor{PURPLE_B}{HTML}{B189C6}\definecolor{PURPLE_C}{HTML}{9A72AC}\definecolor{PURPLE_D}{HTML}{715582}\definecolor{PURPLE_E}{HTML}{644172}\definecolor{RED}{HTML}{FC6255}\definecolor{RED_A}{HTML}{F7A1A3}\definecolor{RED_B}{HTML}{FF8080}\definecolor{RED_C}{HTML}{FC6255}\definecolor{RED_D}{HTML}{E65A4C}\definecolor{RED_E}{HTML}{CF5044}\definecolor{TEAL}{HTML}{5CD0B3}\definecolor{TEAL_A}{HTML}{ACEAD7}\definecolor{TEAL_B}{HTML}{76DDC0}\definecolor{TEAL_C}{HTML}{5CD0B3}\definecolor{TEAL_D}{HTML}{55C1A7}\definecolor{TEAL_E}{HTML}{49A88F}\definecolor{WHITE}{HTML}{FFFFFF}\definecolor{YELLOW}{HTML}{FFFF00}\definecolor{YELLOW_A}{HTML}{FFF1B6}\definecolor{YELLOW_B}{HTML}{FFEA94}\definecolor{YELLOW_C}{HTML}{FFFF00}\definecolor{YELLOW_D}{HTML}{F4D345}\definecolor{YELLOW_E}{HTML}{E8C11C}

%%%%%%%%%%%%%%%%%%%%%%%%%%%%
% LETTERFONTS
%%%%%%%%%%%%%%%%%%%%%%%%%%%%

%!TEX encoding = UTF8
%!TEX root = 0-notes.tex

%fonts
\usepackage{libertinus,libertinust1math}
\usepackage[T1]{fontenc}

% for calligraphic C, D, P (important to import this after the font)
\usepackage{calrsfs}
\newcommand{\D}{\mathcal{D}}
\newcommand{\C}{\mathcal{C}}
\renewcommand{\P}{\mathcal{P}}

% Schwartz
\renewcommand{\S}{\mathcal{S}} % \S est le signe paragraphe normalement

% corps
\newcommand{\R}{\mathbb{R}}
\newcommand{\Rnn}{\mathbb{R}^{2n}}
\newcommand{\Z}{\mathbb{Z}}
\newcommand{\N}{\mathbb{N}}
\newcommand{\Q}{\mathbb{Q}}
\newcommand{\E}{\mathbb{E}}
\newcommand{\DD}{\mathbb{D}}

% order notations
\DeclareRobustCommand{\O}{%
  \text{\usefont{OMS}{cmsy}{m}{n}O}%
}

% japanese bracket
\newcommand{\japb}[1]{\langle #1 \rangle}

% arrows over partial derivatives
\newcommand{\lp}{\overleftarrow{\partial}}
\newcommand{\rp}{\overrightarrow{\partial}}

% quantization
\newcommand{\h}{\hbar}
\newcommand{\Opht}{\textrm{Op}_{\h}^{t}}
\newcommand{\Op}[2][\hbar]{\textrm{Op}_{#1}^{#2}}

% omega functions
\newcommand{\omegap}[2][\rho_0]{\omega(\partial_{#1},\partial_{#2})}
\newcommand{\omegar}[2][\rho_0]{\omega(#1,#2)}

% space before semicolon
\mathcode`\;="303B

% for \Lightning
\usepackage{marvosym}

% for \warning
\newcommand{\warning}{{\fontencoding{U}\fontfamily{futs}\selectfont\char 49\relax}}

% Q(\sqrt(d)) field
\newcommand{\Qsqrt}[1]{\Q\bigl(\mspace{-3mu}\sqrt{#1}\bigr)}


%%%%%%%%%%%%%%%%%%%%%%%%%%%%
% MACROS
%%%%%%%%%%%%%%%%%%%%%%%%%%%%

%!TEX encoding = UTF8
%!TEX root = 0-notes.tex

%%%%%%%%%%%%%%%%%%%%%%%%%%%%%%
% SELF MADE COMMANDS
%%%%%%%%%%%%%%%%%%%%%%%%%%%%%%


%%
% tcolor environments VS clean environments
%%

\ifclean

\newcommand{\thm}[3]{\begin{theorem}[#1]\label{#3}#2\end{theorem}}
\newcommand{\cor}[3]{\begin{corollaire}[#1]\label{#3}#2\end{corollaire}}
\newcommand{\lem}[3]{\begin{lemme}[#1]\label{#3}#2\end{lemme}}
\newcommand{\mprop}[3]{\begin{proposition}[#1]\label{#3}#2\end{proposition}}
\newcommand{\ex}[3]{\begin{exemple}[#1]\label{#3}#2\end{exemple}}
%\newcommand{\exe}[3]{\begin{exercice}[#1]\label{#3}#2\end{exercice}}
\newcommand{\dfn}[3]{\begin{definition}[#1]\label{#3}#2\end{definition}}
\newcommand{\qs}[2]{\begin{question}[#1]#2\end{question}}
\newcommand{\pf}[2]{\begin{preuve}[#1]#2\end{preuve}}
\newcommand{\nt}[1]{\begin{remarque}#1\end{remarque}}
\newcommand{\str}[1]{\begin{strategie}#1\end{strategie}}
\newcommand{\mth}[1]{\begin{methode}#1\end{methode}}
\newcommand{\ax}[3]{\begin{axiome}[#1]\label{#3}#2\end{axiome}}

\newcommand{\exe}[4]{
	\begin{Exercise}[title=#1, label=#3]
		\marginpar{\mbox{\scriptsize(solution p.\pageref{\ExerciseLabel-Answer})}}
		#2
	\end{Exercise}
	\begin{Answer}[ref=#3]
		#4
	\end{Answer}
}

\newcommand{\exemulticols}[5]{
	\begin{multicols}{2}
	\begin{Exercise}[title=#1, label=#4]
		\marginnote{\mbox{\scriptsize(solution p.\pageref{\ExerciseLabel-Answer})}}
		#2
	\end{Exercise}
		#3
	\end{multicols}
	\begin{Answer}[ref=#4]
		#5
	\end{Answer}
}

\else

\newcommand{\thm}[3]{\begin{Theorem}[label=#3]{#1}{}#2\end{Theorem}}
\newcommand{\cor}[3]{\begin{Corollary}[label=#3]{#1}{}#2\end{Corollary}}
\newcommand{\lem}[3]{\begin{Lemma}[label=#3]{#1}{}#2\end{Lemma}}
\newcommand{\mprop}[3]{\begin{Prop}[label=#3]{#1}{}#2\end{Prop}}
\newcommand{\ex}[3]{\begin{Example}[label=#3]{#1}{}#2\end{Example}}
%\newcommand{\exe}[3]{\begin{Exe}[label=#3]{#1}{}#2\end{Exe}}
\newcommand{\dfn}[3]{\begin{Definition}[colbacktitle=red!75!black, label=#3]{#1}{}#2\end{Definition}}
\newcommand{\qs}[2]{\begin{MyQuestion}{#1}{}#2\end{MyQuestion}}
\newcommand{\pf}[2]{\begin{myproof}[#1]#2\end{myproof}}
\newcommand{\nt}[1]{\begin{Note}#1\end{Note}}
\newcommand{\str}[1]{\begin{Strategy}#1\end{Strategy}}
\newcommand{\mth}[1]{\begin{Methode}#1\end{Methode}}
\newcommand{\axiome}[3]{\begin{Axiome}[label=#3]{#1}{}#2\end{Axiome}}

\newcommand{\exe}[4]{
	\begin{Exe}[label=#3]{}{}#2\end{Exe}
	\begin{Answer}[ref=#3]
		#4
	\end{Answer}
}

\fi

\newcommand{\notations}[1]{\begin{notation}#1 \end{notation}}
\newcommand{\nomen}[1]{\begin{nomenclature}#1 \end{nomenclature}}
\newcommand{\heur}[1]{\begin{heuristique}#1\end{heuristique}}

%%

% deliminators
\DeclarePairedDelimiter{\abs}{\lvert}{\rvert}
%\DeclarePairedDelimiter{\norm}{\lVert}{\rVert}

\DeclarePairedDelimiter{\ceil}{\lceil}{\rceil}
\DeclarePairedDelimiter{\floor}{\lfloor}{\rfloor}
\DeclarePairedDelimiter{\round}{\lfloor}{\rceil}

\newsavebox\diffdbox
\newcommand{\slantedromand}{{\mathpalette\makesl{d}}}
\newcommand{\makesl}[2]{%
\begingroup
\sbox{\diffdbox}{$\mathsurround=0pt#1\mathrm{#2}$}%
\pdfsave
\pdfsetmatrix{1 0 0.2 1}%
\rlap{\usebox{\diffdbox}}%
\pdfrestore
\hskip\wd\diffdbox
\endgroup
}
\newcommand{\dd}[1][]{\ensuremath{\mathop{}\!\ifstrempty{#1}{%
\slantedromand\@ifnextchar^{\hspace{0.2ex}}{\hspace{0.1ex}}}%
{\slantedromand\hspace{0.2ex}^{#1}}}}
\ProvideDocumentCommand\dv{o m g}{%
  \ensuremath{%
    \IfValueTF{#3}{%
      \IfNoValueTF{#1}{%
        \frac{\dd #2}{\dd #3}%
      }{%
        \frac{\dd^{#1} #2}{\dd #3^{#1}}%
      }%
    }{%
      \IfNoValueTF{#1}{%
        \frac{\dd}{\dd #2}%
      }{%
        \frac{\dd^{#1}}{\dd #2^{#1}}%
      }%
    }%
  }%
}
\providecommand*{\pdv}[3][]{\frac{\partial^{#1}#2}{\partial#3^{#1}}}
%  - others
\DeclareMathOperator{\Lap}{\mathcal{L}}
\DeclareMathOperator{\Var}{Var} % variance
\DeclareMathOperator{\Cov}{Cov} % covariance

% Since the amsthm package isn't loaded

% I prefer the slanted \leq
\let\oldleq\leq % save them in case they're every wanted
\let\oldgeq\geq
\renewcommand{\leq}{\leqslant}
\renewcommand{\geq}{\geqslant}

% tel que
\newcommand{\tqs}{\text{ tels que }}
\newcommand{\tq}{\text{ tq. }}
\newcommand{\et}{\text{ et }}
\newcommand{\ou}{\text{ ou }}
\newcommand{\pourtout}{\text{ pour tout }}
\newcommand{\sct}{\text{ sachant }}

% Lois
\newcommand{\Bern}{\text{Bern}}
\newcommand{\Binom}{\text{Binom}}

% ensemble avec bigl et bigr
\newcommand{\bigset}[1]{\bigl\{ #1 \bigr\}}
\newcommand{\Bigset}[1]{\Bigl\{ #1 \Bigr\}}
\newcommand{\bigpar}[1]{\bigl( #1 \bigr)}
\newcommand{\Bigpar}[1]{\Bigl( #1 \Bigr)}

% PLUS INFTY AND MINUS INFTY WITH NO SPACE
\newcommand{\pinfty}{{+}\infty}
\newcommand{\minfty}{{-}\infty}

% vecteur flèche
\renewcommand{\vec}[1]{\overrightarrow{#1}}

% vecteur pmatrix
\newcommand{\pvec}[2]{\begin{pmatrix} #1 \\ #2 \end{pmatrix}}

% vecteur norme
\newcommand{\norm}[1]{\left\Vert #1 \right\Vert}

% point plan
\newcommand{\point}[3]{
	#1\left(#2 ; #3 \right)
}

% \smash avant \underline pour coller la ligne au mot
\let\oldunderline\underline
\renewcommand{\underline}[1]{\oldunderline{\smash{#1}}}

% emph + index
\newcommand{\emphindex}[1]{\emph{#1}\index{#1}}

% tableau croisé
\newcommand{\tableaucroise}[4]{
\begin{tabular}{|c|c|c|c|}
	\cline{2-4}
	\multicolumn{1}{c|}{} & #1 \\ \hline
	#2 \\ \hline
	#3  \\ \hline
	#4  \\ \hline
\end{tabular}
}

% python minted
\newcommand{\python}[1]{
\inputminted[
		linenos,
		gobble=0,
		breaklines=true, % otherwise it breaks for no apparent reason?
		breakafter=,,
		fontsize=\small,
		numbersep=8pt,
		tabsize=4, % tab ident = 4 spaces
		fontfamily=courier, %important pour les signes <, >
]{python}{python/#1.py}
}



\AdvanceDate[3]

\begin{document}
\pagestyle{fancy}
\fancyhead[L]{Première}
\fancyhead[C]{\textbf{Chapitre 3 --- Dérivation \ifsolutions -- Solutions \fi}}
\fancyhead[R]{\today}


\begin{questions*}{}{}
	
	 \begin{enumerate}[label=\roman*)]
	 	\item Quelles sont les variations de $x^2 - 2x + 3$ sur $\R$ ?
	 	\item Quel est le maximum, le minimum de $x^3 - 6x + 9x - 5$ sur un intervalle donné ?
		\item Pour quel $x \in [0;2]$ la valeur $x \sqrt{4-x^2}$ est-elle maximale ?
	\end{enumerate}
\end{questions*}


\begin{but*}{}{}
	\begin{enumerate}
		\item Comprendre le sens de variations d'une fonction par analyse locale.
		\item En déduire les extrema (maximum, minimum) d'une fonction par analyse globale.
	\end{enumerate}
\end{but*}

\begin{figure}[h]
	\centering
	\begin{tabular}{|c|c|c|c|c|c|c|c|}\hline
		$x$ & $-3$ & $-2$ & $-1$ & $0$ &  1 & 2 & 3 \\ \hline
		$f(x) = -2x+1$ &&&&&&& \\ \hline
	\end{tabular}
	\caption{Tableau de valeurs de $f(x) = -2x+1$.}
	\label{fig:f(t)}
\end{figure}

\begin{figure}[h]
	\centering
	\begin{tikzpicture}[scale=.9]
	\begin{axis}[
	xmin = -3, xmax=3, ymin=-5, ymax=7, 
	grid = both,  xlabel={$x$}, ylabel={$f(x) = -2x+1$},
	xtick = {-3, ..., 3},
	ytick = {-5, ..., 7},
	]
	\end{axis}
	\end{tikzpicture}
	\caption{Graphe de $\C_f$ où $f(x) = -2x+1$.}
	\label{fig:Cf}
\end{figure}

\begin{rappel*}{fonctions affines}{}
	
	Une fonction $f$ est \emph{affine} si elle s'écrit
		\[ f(x) = \qquad\qquad\qquad, \]
	où %$a$ est le coefficient directeur.
	
	La courbe représentative de $f$ est une % droite de pente $a$.
\end{rappel*}

\begin{propriete}{}{}
	Soit $f$ une fonction affine et $a$ son coefficient directeur.
		\begin{enumerate}	
			\item si $a > 0$, alors $f$ est
			\item si $a < 0$, alors $f$ est
			\item si $a = 0$, alors $f$ est
		\end{enumerate}
\end{propriete}

\begin{outil*}{théorème du coefficient directeur}{}
	Soient $x, y\in\R$ deux nombres distincts, et $f$ une fonction affine de coefficient directeur $a$.
	Alors
	
\begin{multicols}{2}
		%\[ a = \dfrac{f(y) - f(x)}{y-x}. \]
		\[ a = \qquad\qquad\qquad \]
		\vfill
		\,
		
	\begin{tikzpicture}[scale=.8]
	\begin{axis}[xmin = -2, xmax=2, ymin=-5, ymax=3, grid = none,  xlabel={$x$}, ylabel={$f(x)$}]
		\addplot[domain=-2:2, samples=2, myb, thick] {2*x-1};
		
		\addplot[black, thick, mark=*, mark size = 1] (-1,-3) node[right=5pt] {$A(x; f(x))$};
		\addplot[black, thick, mark=*, mark size = 1] (1,1) node[above left] {$B(y; f(y))$};
	\end{axis}
	\end{tikzpicture}
\end{multicols}

\end{outil*}

\begin{exemple*}{étude de $\mathbf{f(x) = \dfrac12(x^3 + 2x^2 - x - 2)}$ autour de $\mathbf{x=-0,5}$}{}

\begin{center}
\begin{tikzpicture}[scale=.8]
\begin{axis}[xmin = -3, xmax=2, ymin=-5, ymax=6, grid = none,  xlabel={$x$}, ylabel={$f(x)$}]
	\addplot[domain=-3:2, samples=500, myb, thick] {.5*(x+2)*(x+1)*(x-1)};
	
	% zoom square A to B
	\newcommand\xA{-2}
	\newcommand\yA{-2}
	\newcommand\xB{1}
	\newcommand\yB{2}
	\draw[black, thick] (axis cs:\xA,\yA) -- (axis cs:\xA,\yB);
	\draw[black, thick] (axis cs:\xB,\yA) -- (axis cs:\xB, \yB);
	\draw[black, thick] (axis cs:\xA,\yB) -- (axis cs:\xB,\yB);
	\draw[black, thick] (axis cs:\xA,\yA) -- (axis cs:\xB,\yA);
\end{axis}
\end{tikzpicture}
\begin{tikzpicture}[scale=.8]
\begin{axis}[xmin = -2, xmax=1, ymin=-2, ymax=2, grid = none,  xlabel={$x$}, ylabel={$f(x)$}]
	\addplot[domain=-2:1, samples=500, myb, thick] {.5*(x+2)*(x+1)*(x-1)};
	
	% zoom square A to B
	\newcommand\xA{-1}
	\newcommand\yA{-1.5}
	\newcommand\xB{0}
	\newcommand\yB{0.5}
	\draw[black, thick] (axis cs:\xA,\yA) -- (axis cs:\xA,\yB);
	\draw[black, thick] (axis cs:\xB,\yA) -- (axis cs:\xB, \yB);
	\draw[black, thick] (axis cs:\xA,\yB) -- (axis cs:\xB,\yB);
	\draw[black, thick] (axis cs:\xA,\yA) -- (axis cs:\xB,\yA);
\end{axis}
\end{tikzpicture}

\begin{tikzpicture}[scale=.8]
\begin{axis}[xmin = -1, xmax=0, ymin=-1.5, ymax=.5, grid = none,  xlabel={$x$}, ylabel={$f(x)$}]
	\addplot[domain=-1:0, samples=500, myb, thick] {.5*(x+2)*(x+1)*(x-1)};
	
	% zoom square A to B
	\newcommand\xA{-.75}
	\newcommand\yA{-1}
	\newcommand\xB{-.25}
	\newcommand\yB{0}
	\draw[black, thick] (axis cs:\xA,\yA) -- (axis cs:\xA,\yB);
	\draw[black, thick] (axis cs:\xB,\yA) -- (axis cs:\xB, \yB);
	\draw[black, thick] (axis cs:\xA,\yB) -- (axis cs:\xB,\yB);
	\draw[black, thick] (axis cs:\xA,\yA) -- (axis cs:\xB,\yA);
\end{axis}
\end{tikzpicture}
\begin{tikzpicture}[scale=.8]
\begin{axis}[xmin = -.75, xmax=-.25, ymin=-1, ymax=0, grid = none,  xlabel={$x$}, ylabel={$f(x)$}]
	\addplot[domain=-1:0, samples=500, myb, thick] {.5*(x+2)*(x+1)*(x-1)};
	
	\addplot[black, thick, mark=*, mark size = 1] (-.5,-.5625) node[right=5pt] {$A(x; f(x))$};
	\addplot[black, thick, mark=*, mark size = 1] (-.6,-.448) ;
	\addplot[black] (-.5,-.388) node{$B(x-h; f(x-h))$};
	\addplot[black, thick, mark=*, mark size = 1] (-.4,-.672);
	\addplot[black] (-.4,-.75) node{$C(x+h; f(x+h))$};
\end{axis}
\end{tikzpicture}
\end{center}

La pente de la droite $(AC)$ est
	\[ a_h = \qquad\qquad\qquad\qquad\qquad\qquad. \]
La pente de la droite $(AB)$ est
	\[ a_h = \qquad\qquad\qquad\qquad\qquad\qquad. \]

\end{exemple*}


\begin{definition*}{nombre dérivé}{}
	On considère les coefficients directeurs
		\begin{align}
			a_h = \dfrac{f(x+h) - f(x)}h, \label{eq:ah}
		\end{align}
	où $h$ est de plus en plus petit.

	Si $f$ est suffisamment lisse, on admet que $a_h$ converge vers une valeur qu'on écrit
		\[ f'(x). \]
	C'est le \emph{nombre dérivé} de $f$ en $x$.
\end{definition*}

\begin{definition*}{tangente à un courbe}{}
	La droite approximant $\C_f$ localement autour de $x$ est la \emph{tangente} à $\C_f$ en $x$.
	
	Son coefficient directeur est $f'(x)$ et elle passe par $(x; f(x))$.
	
\end{definition*}

\begin{exemple*}{tangente à $\mathbf{f(x) = \dfrac12(x^3 + 2x^2 - x - 2)}$ en $\mathbf{x=-0,5}$}{}

On pose $x=-0,5$ et on prend $h$ de plus en plus petit dans l'équation \eqref{eq:ah} pour remplir le tableau figure \ref{fig:3}.
	%\[ a_h = \dfrac{f(-0,5+h)-f(-0,5)}{h}. \]
	\[ a_h = \hspace{5cm} \]
	
On trace la droite de pente \hspace{2cm} passant par $(-0,5 ; f(-0,5))$ ci-dessous.
\end{exemple*}

\begin{figure}[h!]
	\begin{subfigure}{0.18\textwidth}
	\begin{tabular}{|c|c|}\hline
		$h$ & $a_h$ \\ \hline
		1 & \hspace{1cm} \\ \hline
		0,1 & \\ \hline
		0,001 & \\ \hline
		-1 & \\ \hline
		-0,1 & \\ \hline
		-0,001 & \\ \hline
	\end{tabular}
	%\caption{Accroissements autour de $x=-0,5$.}
	%\label{fig:3a}
	\end{subfigure}
	\hspace{5cm}
	\begin{subfigure}{0.78\textwidth}
	\begin{tikzpicture}[scale=1]
	\begin{axis}[xmin = -3, xmax=2, ymin=-5, ymax=6, grid = none,  xlabel={$x$}, ylabel={$f(x)$}]
		\addplot[domain=-3:2, samples=500, myb, thick] {.5*(x+2)*(x+1)*(x-1)};
		
		% tangente
		\addplot[domain=-2:1, samples=2, myr, thick, <->] {-1.125*(x+.5)-.5625};
	\end{axis}
	\end{tikzpicture}
	%\caption{Tangente à $\C_f$ en $x=-0,5$.}
	%\label{fig:3b}
	\end{subfigure}
\caption{Calculs d'accroissements et tangente.}
\label{fig:3}
\end{figure}


%\begin{propriete}{}{}
%	$f'(x)$ est le coefficient directeur de la tangente à $f$ en $x$.
%\end{propriete}

\begin{propriete}{}{}
	Le signe de $f'(x)$ donne la variation de $f$ en $x$ et inversement :
		\begin{enumerate}
			\item si $f'(x) > 0$, alors 
			\item si $f$ est croissante autour de $x$, alors
			\item si $f'(x) < 0$, alors 
			\item si $f$ est décroissante autour de $x$, alors
		\end{enumerate}
\end{propriete}

\begin{exemple*}{variations de $\mathbf{f(x) = \dfrac12(x^3 + 2x^2 - x - 2)}$, signe de $\mathbf{f'(x)}$}{}
	\begin{center}
	\begin{tikzpicture}
		\tkzTabInit
		 %[lgt=3,espcl=1.5]
	       		{$x$ / 1 , Variation de $f(x)$ / 2, Signe de $f'(x)$ / 1}
	       		{-3,,,,2}
	\end{tikzpicture}
	\end{center}
\end{exemple*}

\begin{strategie*}{déterminer la variation de $\mathbf{f}$ autour de $\mathbf{x}$}{}
	On considère une courbe $\C_f$ lisse quelconque.
	Par zooms successifs, on remarque que $\C_f$ devient presque droite.
	L'erreur d'approximation devient de plus en plus petite en zoomant.
	
	\begin{enumerate}
		\item On suppose $h$ assez petit tel que $\C_f$ soit droite sur $[x-h ; x+h]$.
		\item On calcule le coefficient directeur de la droite à l'aide de l'équation \eqref{eq:ah}. Il tend vers $f'(x)$.
		\item On regarde le signe de $f'(x)$ qui donne la variation de $f$ autour de $x$.
	\end{enumerate}
\end{strategie*}

\begin{rappel*}{}{}
	Un extremum local (maximum ou minimum) de $f$ survient lorsque $f$ change de variations.
\end{rappel*}

\begin{propriete}{}{}
		\begin{center}
			Un extremum local de $f$ survient lorsque $f'$ change de \qquad\qquad\qquad\qquad
		\end{center}
		\begin{center}
			Si un extremum local de $f$ survient en $x$, on a nécessairement \qquad\qquad\qquad\qquad
		\end{center}
\end{propriete}
%\begin{strategie*}{déterminer les extrema de $f$}{}
%	\begin{enumerate}
%		\item On calcule $f'(x)$ comme précédemment.
%		\item On trouve les $x$ vérifiant $f'(x) = 0$.
%		\item On regarde le signe de $f'$ autour de $x$ : s'il change, c'est un extremum.
%	\end{enumerate}
%\end{strategie*}


\end{document}