\documentclass[12pt]{paper}
\usepackage[french]{babel}
\usepackage[
a4paper,
margin=2cm,
nomarginpar,% We don't want any margin paragraphs
]{geometry}
\usepackage{fancyhdr}
\usepackage{array}
\usepackage{amsmath,amsfonts,amsthm,amssymb,mathtools,multicol}
\newcolumntype{P}[1]{>{\centering\arraybackslash}p{#1}}

\usepackage{enumitem}

\usepackage{stackengine}
\newcommand\xrowht[2][0]{\addstackgap[.5\dimexpr#2\relax]{\vphantom{#1}}}

% theorems

\theoremstyle{plain}
\newtheorem{theorem}{Th\'eor\`eme}
\newtheorem{Sol}{Solution}
\newtheorem*{Sol*}{Solution}
\theoremstyle{definition}
\newtheorem{ex}{Exercice}
\newtheorem{definition}{Définition}


% corps
\newcommand{\C}{\mathbb{C}}
\newcommand{\R}{\mathbb{R}}
\newcommand{\Rnn}{\mathbb{R}^{2n}}
\newcommand{\Z}{\mathbb{Z}}
\newcommand{\N}{\mathbb{N}}
\newcommand{\Q}{\mathbb{Q}}

% domain
\newcommand{\D}{\mathbb{D}}


% date
\usepackage{advdate}
\AdvanceDate[1]

% plots
\usepackage{pgfplots, tikz}

% for calligraphic C
\usepackage{calrsfs}

% euro
\usepackage{lmodern,textcomp}

\begin{document}
\pagestyle{fancy}
\fancyhead[L]{Première G5}
\fancyhead[C]{\textbf{Évaluation -- Suites arithmétiques}}
\fancyhead[R]{\today}

\begin{definition}\label{def:1}
	Soit $u$ une suite. On dit que $u$ est \emph{arithmétique} dès que, pour tout $n\in\N$,
		\begin{align}\label{eq:def}
			u(n+1) - u(n) = r,
		\end{align}
	où $r\in\R$ est la \emph{raison} de la suite. La raison est fixe et ne dépend pas de $n$.
\end{definition}



\section*{Exercice d'application (5pts)}

\begin{ex}
	Le mathématicien français Abraham de Moivre, alors âgé de plus que 80 ans, étudia dans les années 1750 la durée de son sommeil ; il remarqua que celle-ci augmentait de façon inquiétante.
	Il prit donc note chaque jour de son heure de coucher et de réveil et compara avec une valeur initiale : la nuit $0$, il dormit $8$ heures.
	
	Au fil des jours, il compta que son temps de sommeil augmentait chaque nuit de $15$ minutes, soit $\frac14$ d'heure.
	En notant $v(n)$ la durée de son sommeil en heures lors de sa $n$-ième nuit, il put alors calculer la nuit de la mort : celle où il dormirait $24$ heures.
	\begin{enumerate}
		\item Donner le terme initial $v(0)$.
		\item Donner $v(4)$, la durée de son sommeil en heures lors de la $4$-ème nuit.
		\item Pourquoi la suite $v$ est-elle arithmétique ? Quelle est sa raison ?
		\item Pour tout $n\in\N$, donner $v(n)$ en fonction de $n$ sans justifier.
		\item Quelle nuit Abraham de Moivre mourut-t-il ?
	\end{enumerate}
\end{ex}

\section*{Exercices théoriques (12pts)}

\begin{ex}
	Compléter l'énoncé du théorème \ref{thm:1} vu en cours. Il n'est pas nécessaire de tout réécrire.
\end{ex}

\begin{theorem}[de Kylian]\label{thm:1}
	
	Une suite $u$ est arithmétique de raison $r\in\R$ et de terme initial $b\in\R$ si et seulement si le terme de rang $n\in\N$ de $u$ s'écrit
		\[ u(n) = \dots\dots\dots.\]
		
\end{theorem}

\begin{ex}
	Soit $u$ une suite arithmétique de raison $-7$ et telle que
		\[u(61) = 30. \]
	\begin{enumerate}
		\item Qu'est-ce que l'ensemble $\N$ ? Donner ses $4$ premiers éléments (dans l'ordre croissant).
		\item Écrire l'équation \eqref{eq:def} de la définition \ref{def:1} avec $r=-7$ et $n=61$.
		\item En déduire la valeur de $u(62)$.
		\item Écrire l'équation \eqref{eq:def} de la définition \ref{def:1} avec $r=-7$ et $n=60$.
		\item En déduire la valeur de $u(60)$.
	\end{enumerate}
\end{ex}

\begin{ex}
	Soit $\gamma$ une suite arithmétique donnée par, pour tout $n\in\N$,
		\[ \gamma(n) = 4 -8n . \]
	On définit les suites $A$ et $B$ suivantes, dépendantes de $\gamma$.
		\[ A(n) = \gamma\left(\dfrac12n+1\right) \qquad \text{ et } \qquad B(n) = \gamma(n^2 - 1), \qquad \text{ pour tout }n\in\N. \]
		
	\newpage
	Pour chacune des suites $A$ et $B$, 
		\begin{enumerate}[leftmargin=2cm, label=\roman*)]
			\item écrire le terme de rang $n$ de la suite \textbf{en substituant dans l'expression de} $\boldsymbol{\gamma}$ ; 
			\item  \textbf{justifier} si la suite est arithmétique ou non à l'aide de la définition \ref{def:1} ou en citant le théorème \ref{thm:1} ; et
			\item si la suite est arithmétique : donner sa \textbf{raison} et son \textbf{terme initial}.
		\end{enumerate}
\end{ex}

\begin{ex}
	Soient $S$ et $T$ deux suites arithmétiques données par
		\[ S(n) = \dfrac{1}{4}n + \dfrac{23}2, \qquad T(n) = 15 - \dfrac{1}{3}n, \qquad \text{ pour tout } n \in \N.\]
	\begin{enumerate}
		\item Donner les variations de $S$ et de $T$.
		\item Donner l'\emph{ensemble} des rangs $n\in\N$ pour lesquels la suite $S$ est inférieure ou égale à la suite $T$.
		\item À partir de quel rang $n\in\N$ la suite $T$ est-elle négative ou nulle ?
	\end{enumerate}
\end{ex}

\section*{Lecture graphique (3pts)}

\begin{ex}
	Écrire, en fonction de $n$, le terme de rang $n \in \N$ des suites arithmétiques $\bullet, \square,$ et $\star$ données graphiquement ci-dessous.

	\begin{center}
	\begin{tikzpicture}[>=stealth, scale=1.1]
		\begin{axis}[xmin = 0, xmax=4.9, xtick={ 0,1,2, 3, 4,5}, ymin=-2, ymax=5, ytick={-2, -1,0,1,2, 3,4}, axis x line=middle, axis y line=middle, axis line style=->, xlabel={$n$}, ylabel={}, grid=both]
			\addplot[black, thick, only marks, mark=*] coordinates {(0,1) (1,1.5) (2,2) (3,2.5) (4,3)};
			
			\addplot[black, thick, only marks, mark=star] coordinates {(2,4) (3,-1)};
			
			\addplot[black, thick, only marks, mark=square] coordinates {(0,4) (1,1) (2,-2)};
			
			%\addplot[black, thick, only marks, mark=triangle] coordinates {(0,1) (4,3)};
		\end{axis}
	
	\end{tikzpicture}
	\end{center}
\end{ex}




\section*{Approfondissement (bonus : 2pts)}

%\begin{ex}
%	Soit $v$ une suite arithmétique de raison $r$.
%	
%	Considérons la suite $S$ définie par, pour tout $n\in\N$,
%		\[ S(n) = v(0) + v(1) + v(2) + \dots + v(n-2) + v(n-1) + v(n). \]
%	La suite $S$ est-elle arithmétique ?
%\end{ex}

\begin{ex}
	\, \\
	
	
	\begin{multicols}{2}
	\begin{center}
	\begin{tikzpicture}[scale=0.5]
		\foreach \r in {0, ..., 10} {
			\draw[black,thick] (0,\r) -- (10,\r);
			\draw[black, thick] (\r,0) -- (\r, 10);
		}
		
		\draw[decoration={brace},decorate]
  			(0,10.2) -- node[above] {$n$} (10,10.2);
		\draw[decoration={brace, mirror},decorate]
  			(10.2,0) -- node[right] {$n$} (10.2,10);
  			
		\draw[->] (-0.5, 1.5) -- (1.5,-0.5) node[below] {$1$};
		\draw[->] (-0.5, 2.5) -- (2.5,-.5) node[below] {$2$};
		\draw[->] (-0.5, 3.5) -- (3.5,-.5) node[right=12pt, below] {$3$ $\cdots$};
	
	\end{tikzpicture}
	\end{center}
	
	
	Pour $n\in\N, n \geq 2$, on considère le carré $n \times n$ ci-contre, découpé en $n^2$ carrés unités.
	En comptant les carrés sur chaque diagonale, démontrer que
		\[ 1 + 2 + 3 + \dots + (n-2) + (n-1) = \dfrac{n(n-1)}2. \]
	\vspace{2cm}
		
	\end{multicols}
\end{ex}

\end{document}