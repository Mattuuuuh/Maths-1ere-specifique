\documentclass[12pt]{paper}
\usepackage[french]{babel}
\usepackage[
a4paper,
margin=2cm,
nomarginpar,% We don't want any margin paragraphs
]{geometry}
\usepackage{fancyhdr}
\usepackage{array}
\usepackage{amsmath,amsfonts,amsthm,amssymb,mathtools,}
\newcolumntype{P}[1]{>{\centering\arraybackslash}p{#1}}

\usepackage{enumitem}

\usepackage{stackengine}
\newcommand\xrowht[2][0]{\addstackgap[.5\dimexpr#2\relax]{\vphantom{#1}}}

% theorems

\theoremstyle{plain}
\newtheorem{theorem}{Th\'eor\`eme}
\newtheorem{Sol}{Solution}
\newtheorem*{Sol*}{Solution}
\theoremstyle{definition}
\newtheorem{ex}{Exercice}
\newtheorem{definition}{Définition}


% corps
\newcommand{\C}{\mathbb{C}}
\newcommand{\R}{\mathbb{R}}
\newcommand{\Rnn}{\mathbb{R}^{2n}}
\newcommand{\Z}{\mathbb{Z}}
\newcommand{\N}{\mathbb{N}}
\newcommand{\Q}{\mathbb{Q}}

% domain
\newcommand{\D}{\mathbb{D}}


% date
\usepackage{advdate}
\AdvanceDate[-1]

% plots
\usepackage{pgfplots}

% for calligraphic C
\usepackage{calrsfs}

% euro
\usepackage{lmodern,textcomp}

\begin{document}
\pagestyle{fancy}
\fancyhead[L]{Première G2}
\fancyhead[C]{\textbf{Évaluation -- Suites arithmétiques}}
\fancyhead[R]{\today}

\begin{definition}\label{def:1}
	Soit $u$ une suite. On dit que $u$ est \emph{arithmétique} dès que, pour tout $n\in\N$,
		\begin{align}\label{eq:def}
			u(n+1) - u(n) = r,
		\end{align}
	où $r\in\R$ est la \emph{raison} de la suite. La raison est fixe et ne dépend pas de $n$.
\end{definition}
	

\section*{Exercice d'application (5pts)}

\begin{ex}
	Un professeur hésite entre louer un appartement ou l'acheter.
	D'une part, le loyer mensuel de l'appartement est de $700$€, avec un coût initial (la caution) fixé à deux loyers, soit $1\ 400$€.
	D'autre part, à l'achat, cet appartement coûte $140 \ 000$€.
	
	Dans la première alternative, lorsque l'appartement est loué, on dénote par $u(n)$ le montant total payé en euros par le locataire après $n$ mois (et ceci pour tout $n\in\N$).
	Ainsi par exemple, après $1$ mois, un seul loyer a été payé en plus de la caution initiale. D'où $u(1) = 2 \ 100$.
	\begin{enumerate}
		\item Donner le terme initial $u(0)$.
		\item Donner $u(2)$, le montant payé par le locataire après $2$ mois.
		\item Justifier avec des mots le caractère arithmétique de la suite $u$. Quelle est sa raison ?
		\item Pour tout $n\in\N$, donner $u(n)$ en fonction de $n$ sans justifier.
		\item À partir de combien de mois le locataire aura-t-il payé plus que le coût de l'appartement à l'achat ?
	\end{enumerate}
\end{ex}

\section*{Exercices théoriques (12pts)}

\begin{ex}
	Compléter l'énoncé du théorème \ref{thm:1} vu en cours. Ne pas écrire sur le sujet.
\end{ex}

\begin{theorem}[Lili-Faustine]\label{thm:1}
	
	Une suite $u$ est arithmétique de raison $r\in\R$ et de terme initial $b\in\R$ si et seulement si le terme de rang $n\in\N$ de $u$ s'écrit
		\[ u(n) = \stackrel{?}{\underline{\hspace{2cm}}}.\]
		
\end{theorem}

\begin{ex}
	Soit $u$ une suite arithmétique de raison $-5$ et telle que
		\[u(10) = 21. \]
	\begin{enumerate}
		\item Donner les $4$ premiers éléments dans l'ordre croissant de l'ensemble $\N$ de la définition \ref{def:1}.
		\item Écrire l'équation \eqref{eq:def} de la définition \ref{def:1} avec $r=-5$ pour $n=10$.
		\item En déduire la valeur de $u(11)$.
		\item Écrire l'équation \eqref{eq:def} de la définition \ref{def:1} avec $r=-5$ pour $n=9$.
		\item En déduire la valeur de $u(9)$.
	\end{enumerate}
\end{ex}


\begin{ex}
	Soit $\phi$ une suite arithmétique donnée par, pour tout $n\in\N$,
		\[ \phi(n) = -3n + 10. \]
	Pour chacune des suites $a, b,$ et $c$ ci-après, 
		\begin{enumerate}[leftmargin=2cm, label=\roman*)]
			\item écrire le terme de rang $n$ de la suite \textbf{en fonction de n} ; 
			\item  \textbf{justifier} si la suite est arithmétique ou non à l'aide de la définition \ref{def:1} ou en citant le théorème \ref{thm:1} ; et
			\item si la suite est arithmétique : donner sa \textbf{raison} et son \textbf{terme initial}.
		\end{enumerate}
		\vspace{5pt}
		\begin{enumerate}[leftmargin=1cm]
			\item La suite $a$ définie par $a(n) = \phi\left(\dfrac13n+3\right)$ pour tout $n\in\N$.
			\item La suite $b$ définie par $b(n) = \phi(-2n+3)$ pour tout $n\in\N$.
			\item La suite $c$ définie par $c(n) = \phi(n)^2$ pour tout $n\in\N$.
		\end{enumerate}
\end{ex}

\begin{ex}
	Soient $A$ et $B$ deux suites arithmétiques données par
		\[ A(n) = \dfrac13n - 3, \qquad B(n) = 12 - \dfrac12n, \qquad \text{ pour tout } n \in \N.\]
	\begin{enumerate}
		\item Donner les variations de $A$ et de $B$.
		\item À partir de quel rang la suite $A$ est-elle plus grande ou égale à la suite $B$ ?
		\item Donner l'\emph{ensemble} des rangs $n\in\N$ pour lesquels $B(n)$ est  supérieure ou égale à $10$.
	\end{enumerate}
\end{ex}

\section*{Lecture graphique (3pts)}

\begin{ex}
	Écrire, en fonction de $n$, le terme de rang $n \in \N$ des suites arithmétiques $\bullet, \square,$ et $\star$ données graphiquement ci-dessous.

	\begin{center}
	\begin{tikzpicture}[>=stealth, scale=1.1]
		\begin{axis}[xmin = 0, xmax=4.9, xtick={ 0,1,2, 3, 4,5}, ymin=-2, ymax=5, ytick={-1,0,1,2, 3,4}, axis x line=middle, axis y line=middle, axis line style=->, xlabel={$n$}, ylabel={}, grid=both]
			\addplot[black, thick, only marks, mark=*] coordinates {(0,3) (1,2) (2,1) (3,0) (4,-1)};
			
			\addplot[black, thick, only marks, mark=star] coordinates {(2,-1) (3,4)};
			
			\addplot[black, thick, only marks, mark=square] coordinates {(0,2) (1,0) (2,-2)};
			
			%\addplot[black, thick, only marks, mark=triangle] coordinates {(0,1) (4,3)};
		\end{axis}
	
	\end{tikzpicture}
	\end{center}
\end{ex}

\section*{Approfondissement (bonus : 2pts)}

\begin{ex}
	Soit $v$ une suite arithmétique de raison $r$.
	
	Considérons la suite $S$ définie par, pour tout $n\in\N$,
		\[ S(n) = v(0) + v(1) + v(2) + \dots + v(n-2) + v(n-1) + v(n). \]
	La suite $S$ est-elle arithmétique ?
\end{ex}

%\begin{ex}
%	Soit $u$ une suite arithmétique de raison $r$ et de terme initial $b$.
%	En admettant que pour tout $n\in\N$ on ait 
%		\[ 0 + 1 + 2 + \dots + (n-1) + n = \dfrac{n(n+1)}2, \]
%	exprimer la somme
%		\[ u(0) + u(1) + u(2) + \dots + u(n-1) + u(n) \]
%	en fonction de $n, r,$ et $b$.
%\end{ex}

\end{document}