% SOLUTION SWITCH
\newif\ifsolutions
				\solutionstrue
				\solutionsfalse
				
\documentclass[a4paper, 12pt]{extarticle}

\usepackage[utf8x]{inputenc}
%fonts
\usepackage{libertinus,libertinust1math}
\usepackage{amsmath,amsthm,amssymb,mathtools}

% SOLUTION SWITCH

\ifsolutions
	\newcommand{\exe}[2]{
		\begin{ex} #1  \end{ex}
		\begin{sol} #2 \end{sol}
	}
\else
	\newcommand{\exe}[2]{
		\begin{ex} #1  \end{ex}
	}
	
\fi


\usepackage[french]{babel}
\usepackage[
a4paper,
margin=2cm,
nomarginpar,% We don't want any margin paragraphs
]{geometry}

% HEADER, ARRAY, ENUM, MULTIOCL
\usepackage{fancyhdr}
\usepackage{array}
\usepackage{multicol, enumitem}
\newcolumntype{P}[1]{>{\centering\arraybackslash}p{#1}}
\usepackage{stackengine}
\newcommand\xrowht[2][0]{\addstackgap[.5\dimexpr#2\relax]{\vphantom{#1}}}

% theorems

\theoremstyle{theorem}
\newtheorem{thm}{Théorème}
\theoremstyle{plain}
\newtheorem*{sol}{Solution}
\theoremstyle{definition}
\newtheorem{ex}{Exercice}
\newtheorem{dfn}{Définition}
\newtheorem*{dfn*}{Définition}


%couleurs
\usepackage{tcolorbox}
\definecolor{myg}{RGB}{56, 140, 70}
\definecolor{myb}{RGB}{45, 111, 177}
\definecolor{myr}{RGB}{199, 68, 64}
\definecolor{mygr}{HTML}{2C3338}


\tcbuselibrary{theorems,skins,hooks}
\newcounter{commonbox}
\makeatletter
\newtcbtheorem[use counter=commonbox]{theorem}{Théorème }%
{
	enhanced,
	colback=white,
	colframe=mygr,
	attach boxed title to top left={yshift*=-\tcboxedtitleheight},
	fonttitle=\bfseries,
	title={#2},
	boxed title size=title,
	boxed title style={%
			sharp corners,
			rounded corners=northwest,
			colback=tcbcolframe,
			boxrule=0pt,
		},
	underlay boxed title={%
			\path[fill=tcbcolframe] (title.south west)--(title.south east)
			to[out=0, in=180] ([xshift=5mm]title.east)--
			(title.center-|frame.east)
			[rounded corners=\kvtcb@arc] |-
			(frame.north) -| cycle;
		},
	#1
}{th}
\newtcbtheorem[use counter=commonbox]{rappel}{Rappel }%
{
	enhanced,
	colback=white,
	colframe=mygr,
	attach boxed title to top left={yshift*=-\tcboxedtitleheight},
	fonttitle=\bfseries,
	title={#2},
	boxed title size=title,
	boxed title style={%
			sharp corners,
			rounded corners=northwest,
			colback=tcbcolframe,
			boxrule=0pt,
		},
	underlay boxed title={%
			\path[fill=tcbcolframe] (title.south west)--(title.south east)
			to[out=0, in=180] ([xshift=5mm]title.east)--
			(title.center-|frame.east)
			[rounded corners=\kvtcb@arc] |-
			(frame.north) -| cycle;
		},
	#1
}{th}
\newtcbtheorem[use counter=commonbox]{strategie}{Stratégie }%
{
	enhanced,
	colback=white,
	colframe=mygr,
	attach boxed title to top left={yshift*=-\tcboxedtitleheight},
	fonttitle=\bfseries,
	title={#2},
	boxed title size=title,
	boxed title style={%
			sharp corners,
			rounded corners=northwest,
			colback=tcbcolframe,
			boxrule=0pt,
		},
	underlay boxed title={%
			\path[fill=tcbcolframe] (title.south west)--(title.south east)
			to[out=0, in=180] ([xshift=5mm]title.east)--
			(title.center-|frame.east)
			[rounded corners=\kvtcb@arc] |-
			(frame.north) -| cycle;
		},
	#1
}{th}
\newtcbtheorem[use counter=commonbox]{outil}{Outil }%
{
	enhanced,
	colback=white,
	colframe=mygr,
	attach boxed title to top left={yshift*=-\tcboxedtitleheight},
	fonttitle=\bfseries,
	title={#2},
	boxed title size=title,
	boxed title style={%
			sharp corners,
			rounded corners=northwest,
			colback=tcbcolframe,
			boxrule=0pt,
		},
	underlay boxed title={%
			\path[fill=tcbcolframe] (title.south west)--(title.south east)
			to[out=0, in=180] ([xshift=5mm]title.east)--
			(title.center-|frame.east)
			[rounded corners=\kvtcb@arc] |-
			(frame.north) -| cycle;
		},
	#1
}{th}
\newtcbtheorem[use counter=commonbox]{but}{Buts du chapitre }%
{
	enhanced,
	colback=white,
	colframe=mygr,
	attach boxed title to top left={yshift*=-\tcboxedtitleheight},
	fonttitle=\bfseries,
	title={#2},
	boxed title size=title,
	boxed title style={%
			sharp corners,
			rounded corners=northwest,
			colback=tcbcolframe,
			boxrule=0pt,
		},
	underlay boxed title={%
			\path[fill=tcbcolframe] (title.south west)--(title.south east)
			to[out=0, in=180] ([xshift=5mm]title.east)--
			(title.center-|frame.east)
			[rounded corners=\kvtcb@arc] |-
			(frame.north) -| cycle;
		},
	#1
}{th}
\newtcbtheorem[use counter=commonbox]{propriete}{Propriété }%
{
	enhanced,
	colback=white,
	colframe=mygr,
	attach boxed title to top left={yshift*=-\tcboxedtitleheight},
	fonttitle=\bfseries,
	title={#2},
	boxed title size=title,
	boxed title style={%
			sharp corners,
			rounded corners=northwest,
			colback=tcbcolframe,
			boxrule=0pt,
		},
	underlay boxed title={%
			\path[fill=tcbcolframe] (title.south west)--(title.south east)
			to[out=0, in=180] ([xshift=5mm]title.east)--
			(title.center-|frame.east)
			[rounded corners=\kvtcb@arc] |-
			(frame.north) -| cycle;
		},
	#1
}{th}
\newtcbtheorem[number within=commonbox]{definition}{Définition }%
{
	enhanced,
	colback=white,
	colframe=mygr,
	attach boxed title to top left={yshift*=-\tcboxedtitleheight},
	fonttitle=\bfseries,
	title={#2},
	boxed title size=title,
	boxed title style={%
			sharp corners,
			rounded corners=northwest,
			colback=tcbcolframe,
			boxrule=0pt,
		},
	underlay boxed title={%
			\path[fill=tcbcolframe] (title.south west)--(title.south east)
			to[out=0, in=180] ([xshift=5mm]title.east)--
			(title.center-|frame.east)
			[rounded corners=\kvtcb@arc] |-
			(frame.north) -| cycle;
		},
	#1
}{th}
\newtcbtheorem[number within=commonbox]{exemples}{Exemples }%
{
	enhanced,
	colback=white,
	colframe=mygr,
	attach boxed title to top left={yshift*=-\tcboxedtitleheight},
	fonttitle=\bfseries,
	title={#2},
	boxed title size=title,
	boxed title style={%
			sharp corners,
			rounded corners=northwest,
			colback=tcbcolframe,
			boxrule=0pt,
		},
	underlay boxed title={%
			\path[fill=tcbcolframe] (title.south west)--(title.south east)
			to[out=0, in=180] ([xshift=5mm]title.east)--
			(title.center-|frame.east)
			[rounded corners=\kvtcb@arc] |-
			(frame.north) -| cycle;
		},
	#1
}{th}
\newtcbtheorem[number within=commonbox]{exemple}{Exemple }%
{
	enhanced,
	colback=white,
	colframe=mygr,
	attach boxed title to top left={yshift*=-\tcboxedtitleheight},
	fonttitle=\bfseries,
	title={#2},
	boxed title size=title,
	boxed title style={%
			sharp corners,
			rounded corners=northwest,
			colback=tcbcolframe,
			boxrule=0pt,
		},
	underlay boxed title={%
			\path[fill=tcbcolframe] (title.south west)--(title.south east)
			to[out=0, in=180] ([xshift=5mm]title.east)--
			(title.center-|frame.east)
			[rounded corners=\kvtcb@arc] |-
			(frame.north) -| cycle;
		},
	#1
}{th}
\newtcbtheorem[number within=commonbox]{questions}{Questions guidantes }%
{
	enhanced,
	colback=white,
	colframe=mygr,
	attach boxed title to top left={yshift*=-\tcboxedtitleheight},
	fonttitle=\bfseries,
	title={#2},
	boxed title size=title,
	boxed title style={%
			sharp corners,
			rounded corners=northwest,
			colback=tcbcolframe,
			boxrule=0pt,
		},
	underlay boxed title={%
			\path[fill=tcbcolframe] (title.south west)--(title.south east)
			to[out=0, in=180] ([xshift=5mm]title.east)--
			(title.center-|frame.east)
			[rounded corners=\kvtcb@arc] |-
			(frame.north) -| cycle;
		},
	#1
}{th}
\makeatother

% corps
\newcommand{\R}{\mathbb{R}}
\newcommand{\Rnn}{\mathbb{R}^{2n}}
\newcommand{\Z}{\mathbb{Z}}
\newcommand{\N}{\mathbb{N}}
\newcommand{\Q}{\mathbb{Q}}

% domain
\newcommand{\D}{\mathcal{D}}
% for calligraphic C
\usepackage{calrsfs}
\newcommand{\C}{\mathcal{C}}

% date
\usepackage{advdate}

% ensembles tq. 
\newcommand{\xRtq}[1]{
	$\left\{ x \in \R \text{ tq. } #1 \right\}$
}

% vabs
\newcommand{\vabs}[1]{
	\left| #1 \right|
}

%pinfty minfty
\newcommand{\pinfty}{{+}\infty}
\newcommand{\minfty}{{-}\infty}

% plots
\usepackage{pgfplots}

%virgules
\usepackage{icomma}
\pgfplotsset{/pgf/number format/use comma}

%subfigures
\usepackage{subcaption}

%hyperlink footnote
\usepackage{hyperref}

%wider tabulars
\def\arraystretch{2}
\setlength\tabcolsep{15pt}

% tableaux var, signe
\usepackage{tkz-tab}


\AdvanceDate[1]

\begin{document}
\pagestyle{fancy}
\fancyhead[L]{Première}
\fancyhead[C]{\textbf{Fonctions exponentielles 2 \ifsolutions -- Solutions \fi}}
\fancyhead[R]{\today}

\exe{
	Montrer qu'on a environ $6^{8} \approx 10^6$ et $7^6 \approx 10^5$ : ce sont leur \emph{ordre de grandeur}.
	
	\begin{enumerate}
		\item En déduire l'ordre de grandeur de $6^{16}$ et le nombre de chiffres nécessaires pour l'écrire.
		\item En déduire l'ordre de grandeur de $7^{18}$ et le nombre de chiffres nécessaires pour l'écrire.
		\item En déduire l'ordre de grandeur de $6^{24} \times 7^{12}$ et le nombre de chiffres nécessaires pour l'écrire.
	\end{enumerate}
}{
	\begin{enumerate}
		\item En mettant au carré on obtient
			\[ 6^{16} = \left(6^8\right)^2 \approx \left(10^6\right)^2 = 10^{12}. \]
		Il faut donc $13$ chiffres pour écrire $6^{16}$.
		\item En mettant au cube on obtient
			\[ 7^{18} = \left(7^6\right)^3 \approx \left(10^5\right)^3 = 10^{15}. \]
		Il faut donc $16$ chiffres pour écrire $6^{16}$.
		\item On combine les méthodes pour obtenir
			\[ 6^{24} \times 7^{12} \approx \left(10^{6}\right)^3 \times \left(10^5 \right)^2 = 10^{18} \times 10^{10} = 10^{38}, \]
		et il faut $39$ chiffres pour écrire ce nombre.
	\end{enumerate}


}

\exe{
	On estime que, dans l'univers, on observe au moins
		\begin{itemize}
			\item $10^{11}$ galaxies ; que chacune contient
			\item $10^{11}$ étoiles ; dont la masse moyenne est de
			\item $10^{32}$ kilogrammes ; et que chaque gramme de matière contient
			\item $10^{24}$ atomes.
		\end{itemize}
	Estimer le nombre d'atomes dans l'univers observable à partir de ces données.
}{
	Des deux premières informations on obtient qu'il y a $10^{11} \times 10^{11} = 10^{22}$ étoiles dans l'univers observable.
	
	Ensuite, la masse totale est $10^{22} \times 10^{32} = 10^{54}$ kg.
	
	On multiplie par $10^{3}$ pour obtenir des grammes, et par $10^{24}$ pour compter les atomes, ce qui donne finalement $10^{54 + 3 + 24} = 10^{81}$ atomes dans l'univers observable.
	
	Il faut donc $82$ chiffres pour écrire ce nombre ! Et il nous suffit de $4$ symboles pour en parler.

}

\exe{
	Soit $S$ est une suite géométrique : $S(n) = S(0) \times q^n$ pour tout $n\in\N$.
	Donner l'expression algébrique de $S$ (c'est-à-dire trouver $S(0)$ et $q$) en sachant que
		\begin{align*}
			S(3) = \dfrac8{27}, && \text{ et } && S(4) = \dfrac{32}{81}.
		\end{align*}
}{
	On se rappelle de la définition des suites géométriques : pour passer d'une terme à l'autre, on multiplie par la raison $q$.
	Par conséquent, on obtient
		\[ \dfrac8{27} \times q = \dfrac{32}{81}. \]
	On résoud en multipliant par $\dfrac{27}{8}$, l'inverse de $\dfrac{8}{27}$ : 
		\begin{align*}
			\dfrac{27}{8} \times \dfrac8{27} \times q &= \dfrac{27}{8}\times\dfrac{32}{81} \\
			q &= \dfrac{32 \times 27}{8 \times 81} \\
			q &= \dfrac{32}8 \times \dfrac{27}{81} \\
			q &= 4 \times \dfrac13 = \dfrac43,
		\end{align*}
	où on a utilisé que $81 = 27 \times 3$ pour simplifier la fraction $\frac{27}{81}$

	On a trouvé $q$, et on peut déduire le terme initial en remontant en arrière :
		\begin{align*}
			S(3) = q \times S(2) && \iff && S(2) = \dfrac1q  \times S(3).
		\end{align*}
	D'où 	
		\begin{align*}
			S(2) &= \dfrac34 \times \dfrac8{27} =\dfrac29, \\
			S(1) &= \dfrac34 \times \dfrac23 = \dfrac16, \\
			S(0) &= \dfrac34 \times \dfrac12 = \dfrac14.
		\end{align*}
		
	Un autre façon de faire est d'écrire 
		\[ \dfrac{8}{27} = S(3) = S(0) \times q^3 = S(0) \times \left(\dfrac43\right)^3 = S(0) \times \dfrac{64}{27}, \]
	de quoi on déduit immédiatement
		\[ S(0) = \dfrac8{27}\times\dfrac{27}{64} = \dfrac8{64} = \dfrac14. \]
}

\exe{
	$g$ est une fonction exponentielle : $g(x) = g(0)\times q^x$ pour tout $x\in\R$.
	Donner l'expression algébrique de $g$ (c'est-à-dire trouver $g(0)$ et $q$) en sachant que
		\begin{align*}
			g(2,5) = 204,8 && \text{ et } && g(3) = 819,2.
		\end{align*}
}{
	La méthode de l'exercice précédent se généralise : au lieu de multiplier par $q = q^1$ pour passer de $S(3)$ à $S(4)$ (et donc d'avoir $\dfrac{S(4)}{S(3)} = q$), on multiplie par $q^{0,5}$ pour passer de $g(2,5)$ à $g(3)$.
	
	Ceci se démontre de la façon suivante :
		\begin{align*}
			g(2,5) &= g(0) \times q^{2,5} \\
			g(3) &=  g(0) \times q^{3},
		\end{align*}
	dont on déduit que
		\begin{align*}
			\dfrac{g(3)}{g(2,5)} &= \dfrac{g(0) \times q^{3}}{g(0) \times q^{2,5}} \\
									&= \dfrac{q^3}{q^{2,5}} \\
									&= q^3 \times q^{-2,5} \\
									&= q^{3-2,5} = q^{0,5}.
		\end{align*}
	On connaît le ratio, égal à $\dfrac{g(3)}{g(2,5)} = \dfrac{819,2}{204,8} = 4$.
	Pour obtenir $q$ à partir de $q^{0,5}$, on met à la puissance $\dfrac1{0,5} = 2$ :
		\[ q = q^1 = \left(q^{0,5}\right)^2 = 4^2 = 16. \]
	On conclut donc que $q=16$.

	Pour obtenir $g(0)$, on utilise une des deux équations déjà posées plus haut.
	Par exemple, $g(3) =  g(0) \times q^{3}$, qui implique
		\[ g(0) = \dfrac{g(3)}{q^3} = \dfrac{819,2}{16^3} = 0,2. \]
	
	Finalement, $g(x) = 0,2 \times 16^x$ pour tout $x\in\R$.
}

\exe{
	Soit $S$ est une suite géométrique.
	Donner l'expression algébrique de $S$ en sachant que
		\begin{align*}
			S(4) = 768, && \text{ et }  && S(7) = 49152.
		\end{align*}
}{
	Le même raisonnement qu'aux exercice précédent donne l'équation
		\[ S(7) = q^3 \times S(4), \]
	de laquelle on déduit
		\[ q^3 = \dfrac{49152}{768} = 64.\]
	Pour déduire $q$, on met à la puissance $\frac13$, qui donne
		\[ q = 64^{1/3} = 4. \]
	
	On cherche ensuite à connaître $S(0)$ :
		\[ S(0) = \dfrac{S(4)}{q^4} = \dfrac{768}{4^4} = 3. \]
	D'où $S(n) = 3 \times 4^n$ pour tout $n \in \N$.
}

\exe{
	$g$ est une fonction exponentielle.
	Donner l'expression algébrique de $g$ en sachant que
		\begin{align*}
			g(-3) = \dfrac1{49}, && \text{ et }  && g(3) = 2401.
		\end{align*}
}{
	On trouve d'abord $q$ en utilisant que
		\[ g(3) = g(-3) \times q^{6}, \]
	qui donne $q^6 = 2401\times49 = 117649$, et donc $q = 117649^{1/6} = 7$.
	
	On conclut à l'aide de
		\[ g(0) = \dfrac{g(3)}{q^3} = \dfrac{2401}{7^3} = 7. \]
	Il suit que $g(x) = 7 \times 7^x = 7^{x+1}$ pour tout $x\in\R$.
}



\end{document}