\documentclass[a4paper, 12pt]{extarticle}

\usepackage[utf8x]{inputenc}
%fonts
\usepackage{libertinus,libertinust1math}
\usepackage{amsmath,amsthm,amssymb,mathtools}

% SOLUTION SWITCH

\ifsolutions
	\newcommand{\exe}[2]{
		\begin{ex} #1  \end{ex}
		\begin{sol} #2 \end{sol}
	}
\else
	\newcommand{\exe}[2]{
		\begin{ex} #1  \end{ex}
	}
	
\fi


\usepackage[french]{babel}
\usepackage[
a4paper,
margin=2cm,
nomarginpar,% We don't want any margin paragraphs
]{geometry}

% HEADER, ARRAY, ENUM, MULTIOCL
\usepackage{fancyhdr}
\usepackage{array}
\usepackage{multicol, enumitem}
\newcolumntype{P}[1]{>{\centering\arraybackslash}p{#1}}
\usepackage{stackengine}
\newcommand\xrowht[2][0]{\addstackgap[.5\dimexpr#2\relax]{\vphantom{#1}}}

% theorems

\theoremstyle{theorem}
\newtheorem{thm}{Théorème}
\theoremstyle{plain}
\newtheorem*{sol}{Solution}
\theoremstyle{definition}
\newtheorem{ex}{Exercice}
\newtheorem{dfn}{Définition}
\newtheorem*{dfn*}{Définition}


%couleurs
\usepackage{tcolorbox}
\definecolor{myg}{RGB}{56, 140, 70}
\definecolor{myb}{RGB}{45, 111, 177}
\definecolor{myr}{RGB}{199, 68, 64}
\definecolor{mygr}{HTML}{2C3338}


\tcbuselibrary{theorems,skins,hooks}
\newcounter{commonbox}
\makeatletter
\newtcbtheorem[use counter=commonbox]{theorem}{Théorème }%
{
	enhanced,
	colback=white,
	colframe=mygr,
	attach boxed title to top left={yshift*=-\tcboxedtitleheight},
	fonttitle=\bfseries,
	title={#2},
	boxed title size=title,
	boxed title style={%
			sharp corners,
			rounded corners=northwest,
			colback=tcbcolframe,
			boxrule=0pt,
		},
	underlay boxed title={%
			\path[fill=tcbcolframe] (title.south west)--(title.south east)
			to[out=0, in=180] ([xshift=5mm]title.east)--
			(title.center-|frame.east)
			[rounded corners=\kvtcb@arc] |-
			(frame.north) -| cycle;
		},
	#1
}{th}
\newtcbtheorem[use counter=commonbox]{rappel}{Rappel }%
{
	enhanced,
	colback=white,
	colframe=mygr,
	attach boxed title to top left={yshift*=-\tcboxedtitleheight},
	fonttitle=\bfseries,
	title={#2},
	boxed title size=title,
	boxed title style={%
			sharp corners,
			rounded corners=northwest,
			colback=tcbcolframe,
			boxrule=0pt,
		},
	underlay boxed title={%
			\path[fill=tcbcolframe] (title.south west)--(title.south east)
			to[out=0, in=180] ([xshift=5mm]title.east)--
			(title.center-|frame.east)
			[rounded corners=\kvtcb@arc] |-
			(frame.north) -| cycle;
		},
	#1
}{th}
\newtcbtheorem[use counter=commonbox]{strategie}{Stratégie }%
{
	enhanced,
	colback=white,
	colframe=mygr,
	attach boxed title to top left={yshift*=-\tcboxedtitleheight},
	fonttitle=\bfseries,
	title={#2},
	boxed title size=title,
	boxed title style={%
			sharp corners,
			rounded corners=northwest,
			colback=tcbcolframe,
			boxrule=0pt,
		},
	underlay boxed title={%
			\path[fill=tcbcolframe] (title.south west)--(title.south east)
			to[out=0, in=180] ([xshift=5mm]title.east)--
			(title.center-|frame.east)
			[rounded corners=\kvtcb@arc] |-
			(frame.north) -| cycle;
		},
	#1
}{th}
\newtcbtheorem[use counter=commonbox]{outil}{Outil }%
{
	enhanced,
	colback=white,
	colframe=mygr,
	attach boxed title to top left={yshift*=-\tcboxedtitleheight},
	fonttitle=\bfseries,
	title={#2},
	boxed title size=title,
	boxed title style={%
			sharp corners,
			rounded corners=northwest,
			colback=tcbcolframe,
			boxrule=0pt,
		},
	underlay boxed title={%
			\path[fill=tcbcolframe] (title.south west)--(title.south east)
			to[out=0, in=180] ([xshift=5mm]title.east)--
			(title.center-|frame.east)
			[rounded corners=\kvtcb@arc] |-
			(frame.north) -| cycle;
		},
	#1
}{th}
\newtcbtheorem[use counter=commonbox]{but}{Buts du chapitre }%
{
	enhanced,
	colback=white,
	colframe=mygr,
	attach boxed title to top left={yshift*=-\tcboxedtitleheight},
	fonttitle=\bfseries,
	title={#2},
	boxed title size=title,
	boxed title style={%
			sharp corners,
			rounded corners=northwest,
			colback=tcbcolframe,
			boxrule=0pt,
		},
	underlay boxed title={%
			\path[fill=tcbcolframe] (title.south west)--(title.south east)
			to[out=0, in=180] ([xshift=5mm]title.east)--
			(title.center-|frame.east)
			[rounded corners=\kvtcb@arc] |-
			(frame.north) -| cycle;
		},
	#1
}{th}
\newtcbtheorem[use counter=commonbox]{propriete}{Propriété }%
{
	enhanced,
	colback=white,
	colframe=mygr,
	attach boxed title to top left={yshift*=-\tcboxedtitleheight},
	fonttitle=\bfseries,
	title={#2},
	boxed title size=title,
	boxed title style={%
			sharp corners,
			rounded corners=northwest,
			colback=tcbcolframe,
			boxrule=0pt,
		},
	underlay boxed title={%
			\path[fill=tcbcolframe] (title.south west)--(title.south east)
			to[out=0, in=180] ([xshift=5mm]title.east)--
			(title.center-|frame.east)
			[rounded corners=\kvtcb@arc] |-
			(frame.north) -| cycle;
		},
	#1
}{th}
\newtcbtheorem[number within=commonbox]{definition}{Définition }%
{
	enhanced,
	colback=white,
	colframe=mygr,
	attach boxed title to top left={yshift*=-\tcboxedtitleheight},
	fonttitle=\bfseries,
	title={#2},
	boxed title size=title,
	boxed title style={%
			sharp corners,
			rounded corners=northwest,
			colback=tcbcolframe,
			boxrule=0pt,
		},
	underlay boxed title={%
			\path[fill=tcbcolframe] (title.south west)--(title.south east)
			to[out=0, in=180] ([xshift=5mm]title.east)--
			(title.center-|frame.east)
			[rounded corners=\kvtcb@arc] |-
			(frame.north) -| cycle;
		},
	#1
}{th}
\newtcbtheorem[number within=commonbox]{exemples}{Exemples }%
{
	enhanced,
	colback=white,
	colframe=mygr,
	attach boxed title to top left={yshift*=-\tcboxedtitleheight},
	fonttitle=\bfseries,
	title={#2},
	boxed title size=title,
	boxed title style={%
			sharp corners,
			rounded corners=northwest,
			colback=tcbcolframe,
			boxrule=0pt,
		},
	underlay boxed title={%
			\path[fill=tcbcolframe] (title.south west)--(title.south east)
			to[out=0, in=180] ([xshift=5mm]title.east)--
			(title.center-|frame.east)
			[rounded corners=\kvtcb@arc] |-
			(frame.north) -| cycle;
		},
	#1
}{th}
\newtcbtheorem[number within=commonbox]{exemple}{Exemple }%
{
	enhanced,
	colback=white,
	colframe=mygr,
	attach boxed title to top left={yshift*=-\tcboxedtitleheight},
	fonttitle=\bfseries,
	title={#2},
	boxed title size=title,
	boxed title style={%
			sharp corners,
			rounded corners=northwest,
			colback=tcbcolframe,
			boxrule=0pt,
		},
	underlay boxed title={%
			\path[fill=tcbcolframe] (title.south west)--(title.south east)
			to[out=0, in=180] ([xshift=5mm]title.east)--
			(title.center-|frame.east)
			[rounded corners=\kvtcb@arc] |-
			(frame.north) -| cycle;
		},
	#1
}{th}
\newtcbtheorem[number within=commonbox]{questions}{Questions guidantes }%
{
	enhanced,
	colback=white,
	colframe=mygr,
	attach boxed title to top left={yshift*=-\tcboxedtitleheight},
	fonttitle=\bfseries,
	title={#2},
	boxed title size=title,
	boxed title style={%
			sharp corners,
			rounded corners=northwest,
			colback=tcbcolframe,
			boxrule=0pt,
		},
	underlay boxed title={%
			\path[fill=tcbcolframe] (title.south west)--(title.south east)
			to[out=0, in=180] ([xshift=5mm]title.east)--
			(title.center-|frame.east)
			[rounded corners=\kvtcb@arc] |-
			(frame.north) -| cycle;
		},
	#1
}{th}
\makeatother

% corps
\newcommand{\R}{\mathbb{R}}
\newcommand{\Rnn}{\mathbb{R}^{2n}}
\newcommand{\Z}{\mathbb{Z}}
\newcommand{\N}{\mathbb{N}}
\newcommand{\Q}{\mathbb{Q}}

% domain
\newcommand{\D}{\mathcal{D}}
% for calligraphic C
\usepackage{calrsfs}
\newcommand{\C}{\mathcal{C}}

% date
\usepackage{advdate}

% ensembles tq. 
\newcommand{\xRtq}[1]{
	$\left\{ x \in \R \text{ tq. } #1 \right\}$
}

% vabs
\newcommand{\vabs}[1]{
	\left| #1 \right|
}

%pinfty minfty
\newcommand{\pinfty}{{+}\infty}
\newcommand{\minfty}{{-}\infty}

% plots
\usepackage{pgfplots}

%virgules
\usepackage{icomma}
\pgfplotsset{/pgf/number format/use comma}

%subfigures
\usepackage{subcaption}

%hyperlink footnote
\usepackage{hyperref}

%wider tabulars
\def\arraystretch{2}
\setlength\tabcolsep{15pt}

% tableaux var, signe
\usepackage{tkz-tab}

\SetDate[18/02/2026]

\begin{document}
\pagestyle{fancy}
\fancyhead[L]{Première spécifique}
\fancyhead[C]{\textbf{Automatismes}}
\fancyhead[R]{\today}


\subsection*{Fractions}

	Soit $A = \dfrac12 - \dfrac12 \times \dfrac43$.
	Alors
	\begin{multicols}{4}
	\begin{enumerate}[label=\textbf{\alph*)}]
		\item
		$A = 0$
		\item
		$A = -\frac16$
		\item
		$A = \frac23$
		\item
		$A = -1$
	\end{enumerate}
	\end{multicols}
	
	Considérons les trois nombres $A = \frac15, B = \frac{19}{100}, C = 0,21$.
	Le classement par ordre croissant de ces trois nombres est 
	\begin{multicols}{4}
	\begin{enumerate}[label=\textbf{\alph*)}]
		\item
		$A < B < C$
		\item
		$A < C < B$
		\item
		$B < A < C$
		\item
		$C < B < A$
	\end{enumerate}
	\end{multicols}
	
	\framebox(450,100){} 
	
	Calculer et réduire les fractions.
	\begin{multicols}{3}
	\begin{enumerate}[itemsep=5pt]
		\item $1 - \dfrac12$
		\item $\dfrac{3}{4} + \dfrac{2}{5}$
		\item $\dfrac{7}{6} - \dfrac{5}{8}$
		\item $\dfrac{9}{10} + \dfrac{2}{3} - \dfrac{1}{4}$
		\item $\dfrac{-5}{7} + \dfrac{8}{9}$
		\item $\dfrac{\ \ -3 \ \ }{\dfrac54}$
		\item $\dfrac{63}{20} \times \dfrac{15}{14}$
		\item $\dfrac{-5}{7} \times \dfrac{8}{9}$
		\item $\dfrac{\ \ \dfrac{7}{6}\ \ }{\dfrac{5}{8}}$
	\end{enumerate}
	\end{multicols}
	
	Quatre croissants coûtent 6 euros. 
	Dix croissants coûtent
	\begin{multicols}{4}
	\begin{enumerate}[label=\textbf{\alph*)}]
		\item
		60 euros
		\item
		8 euros
		\item
		8,5 euros
		\item
		15 euros
	\end{enumerate}
	\end{multicols}


\subsection*{Équations linéaires}

	Le nombre $x$ vérifiant $6x + 8 = 16x - 2$ est
	\begin{multicols}{4}
	\begin{enumerate}[label=\textbf{\alph*)}]
		\item 
		$x = 1$
		\item 
		$x = \frac{10}{22}$
		\item 
		$x = -1$
		\item 
		$x = 0$
	\end{enumerate}
	\end{multicols}

	\framebox(450,100){} 

	Trouver le $x\in\R$ vérifiant chaque équation suivante.
	\begin{multicols}{2}
	\begin{enumerate}[itemsep=10pt]
		\item $3x + 5 = 2x + 7$
		\item $4x - 6 = 3x + 8$
		\item $5x + 10 = 2x + 4$
		\item $-2x + 3 = 4x - 1$
	\end{enumerate}
	\end{multicols}
	
	\newpage
	
	Le nombre $x$ vérifiant $12x + 8 = 32x - 2$ est
	\begin{multicols}{4}
	\begin{enumerate}[label=\textbf{\alph*)}]
		\item 
		$x = 1$
		\item 
		$x = \frac{5}{22}$
		\item 
		$x = -\frac12$
		\item 
		$x = 0$
	\end{enumerate}
	\end{multicols}

\subsection*{Évolutions}

%	Une diminution de 30\% correspond à
%	\begin{multicols}{2}
%	\begin{enumerate}[label=\textbf{\alph*)}]
%		\item
%		une multiplication par 0,3.
%		\item
%		une multiplication par 1,3.
%		\item
%		une multiplication par 0,7.
%		\item
%		une division par 1,3.
%	\end{enumerate}
%	\end{multicols}
	
	Le prix d'un objet passe de 50€ à 75€.
	Ceci correspond à
	\begin{multicols}{2}
	\begin{enumerate}[label=\textbf{\alph*)}]
		\item
		une augmentation de 25\%.
		\item
		une augmentation de 125\%.
		\item
		une augmentation de 150\%.
		\item
		une augmentation de 50\%.
	\end{enumerate}
	\end{multicols}
	
%	En 2019, les recettes fiscales recouvrées par les Finances publiques s'élèvent à 464 milliards d'euros.
%	En 2022, elles atteignent 544,4 milliards d'euros.
%	
%	Le calcul permettant d'obtenir le pourcentage d'augmentation annuel moyen est
%	\begin{multicols}{2}
%	\begin{enumerate}[label=\textbf{\alph*)}]
%		\item
%		$\left( \dfrac{544,4}{464} - 1\right)\times100$
%		\item
%		$\left(\left( \dfrac{464}{544,4}\right)^{1/3} - 1 \right)\times100$
%		\item
%		$\left(\left( \dfrac{544,4}{464}\right)^{3} - 1 \right)\times100$
%		\item
%		$\left(\left( \dfrac{544,4}{464} \right)^{1/3} - 1 \right)\times100$
%	\end{enumerate}
%	\end{multicols}
	
	\framebox(450,100){} 
	
	À l'issue d'une augmentation d 10\%, un article coûte 110 euros.
	Laquelle des quatre propositions suivantes est vraie ?
	%\begin{multicols}{1}
	\begin{enumerate}[label=\textbf{\alph*)}]
		\item
		Le prix de l'article avant augmentation était égal à 99 euros.
		\item
		Le prix de l'article avant augmentation était égal à 120 euros.
		\item
		Le prix a augmenté de 10 euros.
		\item
		Le prix a augmenté de 11 euros.
	\end{enumerate}
	%\end{multicols}
	
	Donner le coefficient multiplicateur associé à chaque évolution.
	\begin{multicols}{2}
	\begin{enumerate}
		\item
		une augmentation de 25\%
		\item
		une augmentation de 125\%
		\item
		une diminution de 20\%
		\item
		une diminution de 60\%
		\item
		cinq augmentations de 20\%
		\item
		dix diminutions de 10\%
	\end{enumerate}
	\end{multicols}

\subsection*{Suites}

	La suite de nombres $2 ; 4 ; 8 ; 16$ est en progression
	\begin{multicols}{2}
	\begin{enumerate}[label=\textbf{\alph*)}]
		\item
		 arithmétique uniquement.
		\item
		géométrique uniquement.
		\item
		arithmétique et géométrique.
		\item
		ni arithmétique ni géométrique.
	\end{enumerate}
	\end{multicols}

	\framebox(450,100){} 

	Soit $u$ une suite arithmétique de raison 6 et de terme initial $u(0) = 10$.
	\begin{multicols}{2}
	\begin{enumerate}
		\item
		Calculer $u(1)$.
		\item
		Calculer $u(2)$.
		\item
		Calculer $u(200)$.
		\item
		Calculer $u(725) - u(720)$.
	\end{enumerate}
	\end{multicols}

	Soit $w$ une suite géométrique de raison $1,5$ et de terme initial $w(0) = 16$.
	\begin{multicols}{2}
	\begin{enumerate}
		\item
		Calculer $w(1)$.
		\item
		Calculer $w(2)$.
		\item
		Calculer $w(200)$ à l'aide de la calculatrice.
		\item
		Calculer $\dfrac{w(725)}{w(723)}$.
	\end{enumerate}
	\end{multicols}

	Le volume d'un glacier diminue de 3\% chaque année.
	Si $V(n)$ désigne le volume du glacier pour l'année $n$, on a
	\begin{multicols}{2}
	\begin{enumerate}[label=\textbf{\alph*)}]
		\item
		$V(n+1) = V(n) - 0,03$
		\item
		$V(n+1) = 0,03\times V(n)$
		\item
		$V(n+1) = 0,97\times V(n)$
		\item
		$V(n+1) = V(n) - 0,97$
	\end{enumerate}
	\end{multicols}

\subsection*{Puissances}

	Soit $E = \dfrac{2^3}{2^8}\times2^{-1}$.
	Alors
	\begin{multicols}{4}
	\begin{enumerate}[label=\textbf{\alph*)}]
		\item
		$E = 2^{12}$
		\item
		$E = 2^{-6}$
		\item
		$E = 2^{10}$
		\item
		$E = 2^{-4}$
	\end{enumerate}
	\end{multicols}

	\framebox(450,100){} 
	
	Exprimer les nombre suivants sous la forme $10^x$ pour un entier $x\in\Z$.
	\begin{multicols}{4}
	\begin{enumerate}
		\item $10^{1,4} \times 10^{7,6}$
		\item $\left(10^{3,5}\right)^2$
		\item $\dfrac{10^{4,3}}{10^{12,3}}$
		\item $\dfrac{10^{11}}{10^{-3}}$
		\item $\dfrac{10^{0}}{10^{-12}}$
		\item $\dfrac{1}{10^{-6}}$
		\item $\dfrac{10^{32}}{10^{-16}}$
		\item $\left(\dfrac1{10^5}\right)^3$
	\end{enumerate}
	\end{multicols}
	
	
	Le nombre $5^2 + 10^{-100}$ est environ égal à
	\begin{multicols}{4}
	\begin{enumerate}[label=\textbf{\alph*)}]
		\item
		$-1~000$
		\item
		$10$
		\item
		$25$
		\item
		$1~000$
	\end{enumerate}
	\end{multicols}

\subsection*{Images}

	Considérons la fonction $f(x) = \dfrac{1}{2x+1}$.
	L'image de 2 par la fonction $f$ est égale à
	\begin{multicols}{4}
	\begin{enumerate}[label=\textbf{\alph*)}]
		\item
		0,2
		\item
		5
		\item
		0,5
		\item
		1,5
	\end{enumerate}
	\end{multicols}
	
	Considérons la fonction $f(x) = 2x^2 - 5x + 3$.
	Un antécédent de $0$ par la fonction $f$ est
	\begin{multicols}{4}
	\begin{enumerate}[label=\textbf{\alph*)}]
		\item
		1
		\item
		$-1$
		\item
		0
		\item
		2
	\end{enumerate}
	\end{multicols}
	
	\framebox(450,100){} 
	
%	L'ordonnée à l'origine de la fonction affine $f(x) = 3x - 4$ est égale à
%	\begin{multicols}{4}
%	\begin{enumerate}[label=\textbf{\alph*)}]
%		\item
%		$3$
%		\item
%		$-3$
%		\item
%		$4$
%		\item
%		$-4$
%	\end{enumerate}
%	\end{multicols}
	
\newpage

	Soit $f(x) = x^2 + \frac{1+x}{3+x} + 1$. Calculer les images suivantes. Une fraction est attendue.
	\begin{multicols}{4}
	\begin{enumerate}
		\item $f(0)$
		\item $f(1)$
		\item $f(-1)$
		\item $f(-2)$
	\end{enumerate}
	\end{multicols}


	Soit $f(x) = 7 - \frac12 (x-3)^2$. 
	L'image de 3 par la fonction $f$ est égale à
	\begin{multicols}{4}
	\begin{enumerate}[label=\textbf{\alph*)}]
		\item
		$7-\frac12$
		\item
		$7 - \frac12(9+9)$¨
		\item
		7
		\item
		0
	\end{enumerate}
	\end{multicols}

\subsection*{Courbes représentatives}
	
\begin{multicols}{2}
	Considérons la droite de la figure ci-contre.
	La seule équation pouvant lui correspondre est
	\begin{multicols}{2}
	\begin{enumerate}[label=\textbf{\alph*)}]
		\item $y=1-x$
		\item $y=-x-1$
		\item $y=1+x$
		\item $y=x-1$
	\end{enumerate}
	\end{multicols}
	\vfill\null
	\centering
	\begin{tikzpicture}[scale=.7]
	\begin{axis}[xmin = -2, xmax=2, ymin=-2, ymax=5, axis x line=middle, axis y line=middle, axis line style=<->, xlabel={}, ylabel={}, grid=none, grid style = {opacity=.5}, clip=true, ticks = none]
		\addplot[BLUE_E, very thick, domain =-2:4, samples=2] {1-x};% node[below right, pos=.6]{$\C_f$} ;
	\end{axis}
	\end{tikzpicture}
\end{multicols}

	
	Dans un repère du plan, on considère les points $A(1;100)$ et $B(4;106)$.
	On note $m$ le coefficient directeur de la droite $(AB)$.
	On peut affirmer que
	\begin{multicols}{4}
	\begin{enumerate}[label=\textbf{\alph*)}]
		\item $m=2$
		\item $m=0,5$
		\item $m=-2$
		\item $m=-0,5$
	\end{enumerate}
	\end{multicols}
	
%	Tracer la courbe représentative de chaque fonction pour $-2 \leq x \leq 2$.
%	\begin{multicols}{4}
%	\begin{enumerate}
%		\item $f(x) = $
%		\item $f(1)$
%		\item $f(-1)$
%		\item $f(-2)$
%	\end{enumerate}
%	\end{multicols}


	\framebox(450,100){} 


\begin{multicols}{2}
	Considérons la fonction exponentielle $f$ représentée ci-contre.
	La seule expression algébrique pouvant lui correspondre est
	\begin{multicols}{2}
	\begin{enumerate}[label=\textbf{\alph*)}]
		\item $f(x)=0,5 \times 2^x$
		\item $f(x)=0,5x + 3$
		\item $f(x)=2 \times 0,5^x$
		\item $f(x)=3x + 0,5$
	\end{enumerate}
	\end{multicols}
	\vfill\null
	\centering
	\begin{tikzpicture}[scale=1]
	\begin{axis}[xmin = -3, xmax=4, ymin=-3, ymax=7, axis x line=middle, axis y line=middle, axis line style=<->, xlabel={}, ylabel={}, grid=both, grid style = {opacity=.5}, clip=true, xtick distance=1, ytick distance=1]
		\addplot[RED_E, very thick, domain =-4:4, samples=50] {.5*2^x} node[right, pos=.7]{$\C_f$} ;
	\end{axis}
	\end{tikzpicture}
\end{multicols}
	

	\framebox(450,100){} 

%%%%%%%%%%%

\newpage
\fancyhead[C]{\textbf{Solutions}}
\fancyfoot[C]{\thepage}
\shipoutAnswer

\end{document}