\documentclass[a4paper, 12pt]{extarticle}

\usepackage[utf8x]{inputenc}
%fonts
\usepackage{libertinus,libertinust1math}
\usepackage{amsmath,amsthm,amssymb,mathtools}

% SOLUTION SWITCH

\ifsolutions
	\newcommand{\exe}[2]{
		\begin{ex} #1  \end{ex}
		\begin{sol} #2 \end{sol}
	}
\else
	\newcommand{\exe}[2]{
		\begin{ex} #1  \end{ex}
	}
	
\fi


\usepackage[french]{babel}
\usepackage[
a4paper,
margin=2cm,
nomarginpar,% We don't want any margin paragraphs
]{geometry}

% HEADER, ARRAY, ENUM, MULTIOCL
\usepackage{fancyhdr}
\usepackage{array}
\usepackage{multicol, enumitem}
\newcolumntype{P}[1]{>{\centering\arraybackslash}p{#1}}
\usepackage{stackengine}
\newcommand\xrowht[2][0]{\addstackgap[.5\dimexpr#2\relax]{\vphantom{#1}}}

% theorems

\theoremstyle{theorem}
\newtheorem{thm}{Théorème}
\theoremstyle{plain}
\newtheorem*{sol}{Solution}
\theoremstyle{definition}
\newtheorem{ex}{Exercice}
\newtheorem{dfn}{Définition}
\newtheorem*{dfn*}{Définition}


%couleurs
\usepackage{tcolorbox}
\definecolor{myg}{RGB}{56, 140, 70}
\definecolor{myb}{RGB}{45, 111, 177}
\definecolor{myr}{RGB}{199, 68, 64}
\definecolor{mygr}{HTML}{2C3338}


\tcbuselibrary{theorems,skins,hooks}
\newcounter{commonbox}
\makeatletter
\newtcbtheorem[use counter=commonbox]{theorem}{Théorème }%
{
	enhanced,
	colback=white,
	colframe=mygr,
	attach boxed title to top left={yshift*=-\tcboxedtitleheight},
	fonttitle=\bfseries,
	title={#2},
	boxed title size=title,
	boxed title style={%
			sharp corners,
			rounded corners=northwest,
			colback=tcbcolframe,
			boxrule=0pt,
		},
	underlay boxed title={%
			\path[fill=tcbcolframe] (title.south west)--(title.south east)
			to[out=0, in=180] ([xshift=5mm]title.east)--
			(title.center-|frame.east)
			[rounded corners=\kvtcb@arc] |-
			(frame.north) -| cycle;
		},
	#1
}{th}
\newtcbtheorem[use counter=commonbox]{rappel}{Rappel }%
{
	enhanced,
	colback=white,
	colframe=mygr,
	attach boxed title to top left={yshift*=-\tcboxedtitleheight},
	fonttitle=\bfseries,
	title={#2},
	boxed title size=title,
	boxed title style={%
			sharp corners,
			rounded corners=northwest,
			colback=tcbcolframe,
			boxrule=0pt,
		},
	underlay boxed title={%
			\path[fill=tcbcolframe] (title.south west)--(title.south east)
			to[out=0, in=180] ([xshift=5mm]title.east)--
			(title.center-|frame.east)
			[rounded corners=\kvtcb@arc] |-
			(frame.north) -| cycle;
		},
	#1
}{th}
\newtcbtheorem[use counter=commonbox]{strategie}{Stratégie }%
{
	enhanced,
	colback=white,
	colframe=mygr,
	attach boxed title to top left={yshift*=-\tcboxedtitleheight},
	fonttitle=\bfseries,
	title={#2},
	boxed title size=title,
	boxed title style={%
			sharp corners,
			rounded corners=northwest,
			colback=tcbcolframe,
			boxrule=0pt,
		},
	underlay boxed title={%
			\path[fill=tcbcolframe] (title.south west)--(title.south east)
			to[out=0, in=180] ([xshift=5mm]title.east)--
			(title.center-|frame.east)
			[rounded corners=\kvtcb@arc] |-
			(frame.north) -| cycle;
		},
	#1
}{th}
\newtcbtheorem[use counter=commonbox]{outil}{Outil }%
{
	enhanced,
	colback=white,
	colframe=mygr,
	attach boxed title to top left={yshift*=-\tcboxedtitleheight},
	fonttitle=\bfseries,
	title={#2},
	boxed title size=title,
	boxed title style={%
			sharp corners,
			rounded corners=northwest,
			colback=tcbcolframe,
			boxrule=0pt,
		},
	underlay boxed title={%
			\path[fill=tcbcolframe] (title.south west)--(title.south east)
			to[out=0, in=180] ([xshift=5mm]title.east)--
			(title.center-|frame.east)
			[rounded corners=\kvtcb@arc] |-
			(frame.north) -| cycle;
		},
	#1
}{th}
\newtcbtheorem[use counter=commonbox]{but}{Buts du chapitre }%
{
	enhanced,
	colback=white,
	colframe=mygr,
	attach boxed title to top left={yshift*=-\tcboxedtitleheight},
	fonttitle=\bfseries,
	title={#2},
	boxed title size=title,
	boxed title style={%
			sharp corners,
			rounded corners=northwest,
			colback=tcbcolframe,
			boxrule=0pt,
		},
	underlay boxed title={%
			\path[fill=tcbcolframe] (title.south west)--(title.south east)
			to[out=0, in=180] ([xshift=5mm]title.east)--
			(title.center-|frame.east)
			[rounded corners=\kvtcb@arc] |-
			(frame.north) -| cycle;
		},
	#1
}{th}
\newtcbtheorem[use counter=commonbox]{propriete}{Propriété }%
{
	enhanced,
	colback=white,
	colframe=mygr,
	attach boxed title to top left={yshift*=-\tcboxedtitleheight},
	fonttitle=\bfseries,
	title={#2},
	boxed title size=title,
	boxed title style={%
			sharp corners,
			rounded corners=northwest,
			colback=tcbcolframe,
			boxrule=0pt,
		},
	underlay boxed title={%
			\path[fill=tcbcolframe] (title.south west)--(title.south east)
			to[out=0, in=180] ([xshift=5mm]title.east)--
			(title.center-|frame.east)
			[rounded corners=\kvtcb@arc] |-
			(frame.north) -| cycle;
		},
	#1
}{th}
\newtcbtheorem[number within=commonbox]{definition}{Définition }%
{
	enhanced,
	colback=white,
	colframe=mygr,
	attach boxed title to top left={yshift*=-\tcboxedtitleheight},
	fonttitle=\bfseries,
	title={#2},
	boxed title size=title,
	boxed title style={%
			sharp corners,
			rounded corners=northwest,
			colback=tcbcolframe,
			boxrule=0pt,
		},
	underlay boxed title={%
			\path[fill=tcbcolframe] (title.south west)--(title.south east)
			to[out=0, in=180] ([xshift=5mm]title.east)--
			(title.center-|frame.east)
			[rounded corners=\kvtcb@arc] |-
			(frame.north) -| cycle;
		},
	#1
}{th}
\newtcbtheorem[number within=commonbox]{exemples}{Exemples }%
{
	enhanced,
	colback=white,
	colframe=mygr,
	attach boxed title to top left={yshift*=-\tcboxedtitleheight},
	fonttitle=\bfseries,
	title={#2},
	boxed title size=title,
	boxed title style={%
			sharp corners,
			rounded corners=northwest,
			colback=tcbcolframe,
			boxrule=0pt,
		},
	underlay boxed title={%
			\path[fill=tcbcolframe] (title.south west)--(title.south east)
			to[out=0, in=180] ([xshift=5mm]title.east)--
			(title.center-|frame.east)
			[rounded corners=\kvtcb@arc] |-
			(frame.north) -| cycle;
		},
	#1
}{th}
\newtcbtheorem[number within=commonbox]{exemple}{Exemple }%
{
	enhanced,
	colback=white,
	colframe=mygr,
	attach boxed title to top left={yshift*=-\tcboxedtitleheight},
	fonttitle=\bfseries,
	title={#2},
	boxed title size=title,
	boxed title style={%
			sharp corners,
			rounded corners=northwest,
			colback=tcbcolframe,
			boxrule=0pt,
		},
	underlay boxed title={%
			\path[fill=tcbcolframe] (title.south west)--(title.south east)
			to[out=0, in=180] ([xshift=5mm]title.east)--
			(title.center-|frame.east)
			[rounded corners=\kvtcb@arc] |-
			(frame.north) -| cycle;
		},
	#1
}{th}
\newtcbtheorem[number within=commonbox]{questions}{Questions guidantes }%
{
	enhanced,
	colback=white,
	colframe=mygr,
	attach boxed title to top left={yshift*=-\tcboxedtitleheight},
	fonttitle=\bfseries,
	title={#2},
	boxed title size=title,
	boxed title style={%
			sharp corners,
			rounded corners=northwest,
			colback=tcbcolframe,
			boxrule=0pt,
		},
	underlay boxed title={%
			\path[fill=tcbcolframe] (title.south west)--(title.south east)
			to[out=0, in=180] ([xshift=5mm]title.east)--
			(title.center-|frame.east)
			[rounded corners=\kvtcb@arc] |-
			(frame.north) -| cycle;
		},
	#1
}{th}
\makeatother

% corps
\newcommand{\R}{\mathbb{R}}
\newcommand{\Rnn}{\mathbb{R}^{2n}}
\newcommand{\Z}{\mathbb{Z}}
\newcommand{\N}{\mathbb{N}}
\newcommand{\Q}{\mathbb{Q}}

% domain
\newcommand{\D}{\mathcal{D}}
% for calligraphic C
\usepackage{calrsfs}
\newcommand{\C}{\mathcal{C}}

% date
\usepackage{advdate}

% ensembles tq. 
\newcommand{\xRtq}[1]{
	$\left\{ x \in \R \text{ tq. } #1 \right\}$
}

% vabs
\newcommand{\vabs}[1]{
	\left| #1 \right|
}

%pinfty minfty
\newcommand{\pinfty}{{+}\infty}
\newcommand{\minfty}{{-}\infty}

% plots
\usepackage{pgfplots}

%virgules
\usepackage{icomma}
\pgfplotsset{/pgf/number format/use comma}

%subfigures
\usepackage{subcaption}

%hyperlink footnote
\usepackage{hyperref}

%wider tabulars
\def\arraystretch{2}
\setlength\tabcolsep{15pt}

% tableaux var, signe
\usepackage{tkz-tab}

\SetDate[18/02/2026]

\begin{document}
\pagestyle{fancy}
\fancyhead[L]{Première spécifique}
\fancyhead[C]{\textbf{Automatismes et justifications}}
\fancyhead[R]{\today}


\subsection*{Fractions}

	\framebox(450,100){} 


\subsection*{Équations linéaires}

	\framebox(450,100){} 


\subsection*{Évolutions}

	\framebox(450,100){} 


\subsection*{Suites}

	\framebox(450,100){} 


\subsection*{Puissances}

	\framebox(450,100){} 


\subsection*{Fonctions}

	\framebox(450,100){} 


\begin{enumerate}[label=\textbf{\arabic*.}]
	\item 
	Le nombre $x$ vérifiant $6x + 8 = 16x - 2$ est
	\begin{multicols}{4}
	\begin{enumerate}[label=\textbf{\alph*)}]
		\item 
		$x = 1$
		\item 
		$x = \frac{10}{22}$
		\item 
		$x = -1$
		\item 
		$x = 0$
	\end{enumerate}
	\end{multicols}
	
	\item 
	La suite de nombre $2 ; 4 ; 8 ; 16$ est en progression
	\begin{multicols}{2}
	\begin{enumerate}[label=\textbf{\alph*)}]
		\item
		 arithmétique uniquement.
		\item
		géométrique uniquement.
		\item
		arithmétique et géométrique.
		\item
		ni arithmétique ni géométrique.
	\end{enumerate}
	\end{multicols}
	
	\item 
	Une diminution de 30\% correspond à
	\begin{multicols}{2}
	\begin{enumerate}[label=\textbf{\alph*)}]
		\item
		une multiplication par 0,3.
		\item
		une multiplication par 1,3.
		\item
		une multiplication par 0,7.
		\item
		une division par 1,3.
	\end{enumerate}
	\end{multicols}
	
	\item 
	Le prix d'un objet passe de 50€ à 75€.
	Ceci correspond à
	\begin{multicols}{2}
	\begin{enumerate}[label=\textbf{\alph*)}]
		\item
		une augmentation de 25\%.
		\item
		une augmentation de 125\%.
		\item
		une augmentation de 150\%.
		\item
		une augmentation de 50\%.
	\end{enumerate}
	\end{multicols}
	
	\item 
	Considérons la fonction $f(x) = \dfrac{1}{2x+1}$.
	L'image de 2 par la fonction $f$ est égale à
	\begin{multicols}{4}
	\begin{enumerate}[label=\textbf{\alph*)}]
		\item
		0,2
		\item
		5
		\item
		0,5
		\item
		1,5
	\end{enumerate}
	\end{multicols}
	
	\item 
	Soit $A = \dfrac12 - \dfrac12 \times \dfrac43$.
	Alors
	\begin{multicols}{4}
	\begin{enumerate}[label=\textbf{\alph*)}]
		\item
		$A = 0$
		\item
		$A = -\frac16$
		\item
		$A = \frac23$
		\item
		$A = -1$
	\end{enumerate}
	\end{multicols}
	
	\item 
	Soit $E = \dfrac{2^3}{2^8}\times2^{-1}$.
	Alors
	\begin{multicols}{4}
	\begin{enumerate}[label=\textbf{\alph*)}]
		\item
		$E = 2^{12}$
		\item
		$E = 2^{-6}$
		\item
		$E = 2^{10}$
		\item
		$E = 2^{-4}$
	\end{enumerate}
	\end{multicols}
	
	\item 
	Le nombre $5^2 + 10^{-100}$ est environ égal à
	\begin{multicols}{4}
	\begin{enumerate}[label=\textbf{\alph*)}]
		\item
		$-1~000$
		\item
		$10$
		\item
		$25$
		\item
		$1~000$
	\end{enumerate}
	\end{multicols}
	
	\item 
	L'ordonnée à l'origine de la fonction affine $f(x) = 3x - 4$ est égale à
	\begin{multicols}{4}
	\begin{enumerate}[label=\textbf{\alph*)}]
		\item
		$3$
		\item
		$-3$
		\item
		$4$
		\item
		$-4$
	\end{enumerate}
	\end{multicols}
\end{enumerate}

\begin{multicols}{2}
	\begin{enumerate}[label=\textbf{\arabic*.}, resume]
		\item 
		Considérons la droite de la figure ci-contre.
		La seule équation pouvant lui correspondre est
		\begin{multicols}{2}
		\begin{enumerate}[label=\textbf{\alph*)}]
			\item $y=1-x$
			\item $y=-x-1$
			\item $y=1+x$
			\item $y=x-1$
		\end{enumerate}
		\end{multicols}
	\end{enumerate}
	\vfill\null
	\centering
	\begin{tikzpicture}[scale=.7]
	\begin{axis}[xmin = -2, xmax=2, ymin=-2, ymax=5, axis x line=middle, axis y line=middle, axis line style=<->, xlabel={}, ylabel={}, grid=none, grid style = {opacity=.5}, clip=true, ticks = none]
		\addplot[BLUE_E, very thick, domain =-2:4, samples=2] {1-x};% node[below right, pos=.6]{$\C_f$} ;
	\end{axis}
	\end{tikzpicture}
\end{multicols}

\begin{enumerate}[label=\textbf{\arabic*.}]\setcounter{enumi}{10}
	
	\item 
	En 2019, les recettes fiscales recouvrées par les Finances publiques s'élèvent à 464 milliards d'euros.
	En 2022, elles atteignent 544,4 milliards d'euros.
	
	Le calcul permettant d'obtenir le pourcentage d'augmentation annuel moyen est
	\begin{multicols}{2}
	\begin{enumerate}[label=\textbf{\alph*)}]
		\item
		$\left( \dfrac{544,4}{464} - 1\right)\times100$
		\item
		$\left(\left( \dfrac{464}{544,4}\right)^{1/3} - 1 \right)\times100$
		\item
		$\left(\left( \dfrac{544,4}{464}\right)^{3} - 1 \right)\times100$
		\item
		$\left(\left( \dfrac{544,4}{464} \right)^{1/3} - 1 \right)\times100$
	\end{enumerate}
	\end{multicols}
		
\end{enumerate}


\begin{multicols}{2}
	\begin{enumerate}[label=\textbf{\arabic*.}]\setcounter{enumi}{11}
		\item 
		Considérons la fonction exponentielle $f$ représentée ci-contre.
		La seule expression algébrique pouvant lui correspondre est
		\begin{multicols}{2}
		\begin{enumerate}[label=\textbf{\alph*)}]
			\item $f(x)=0,5 \times 2^x$
			\item $f(x)=0,5x + 3$
			\item $f(x)=2 \times 0,5^x$
			\item $f(x)=3x + 0,5$
		\end{enumerate}
		\end{multicols}
	\end{enumerate}
	\vfill\null
	\centering
	\begin{tikzpicture}[scale=1]
	\begin{axis}[xmin = -3, xmax=4, ymin=-3, ymax=7, axis x line=middle, axis y line=middle, axis line style=<->, xlabel={}, ylabel={}, grid=both, grid style = {opacity=.5}, clip=true, xtick distance=1, ytick distance=1]
		\addplot[RED_E, very thick, domain =-4:4, samples=50] {.5*2^x} node[right, pos=.7]{$\C_f$} ;
	\end{axis}
	\end{tikzpicture}
\end{multicols}


%%%%%%%%%%%

\newpage
\fancyhead[C]{\textbf{Solutions}}
\fancyfoot[C]{\thepage}
\shipoutAnswer

\end{document}