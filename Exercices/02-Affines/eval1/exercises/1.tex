%!TEX root = ../eval1.tex

\begin{enumerate}[label=\textbf{\arabic*.}]
	\item 
	Le nombre $x$ vérifiant $3x + 4 = 8x - 1$ est
	\begin{multicols}{4}
	\begin{enumerate}[label=\textbf{\alph*)}]
		\item $x = $
	\end{enumerate}
	\end{multicols}
	
	\item 
	Le coefficient directeur de la fonction affine $f(x) = x + 2$ est
	\begin{multicols}{4}
	\begin{enumerate}[label=\textbf{\alph*)}]
		\item $0$
		\item $-\frac16$
		\item $\frac23$
		\item $-1$
	\end{enumerate}
	\end{multicols}
	
	\item 
	Considérons la droite ci-contre.
	La seule équation pouvant lui correspondre est
	\begin{multicols}{2}
	\begin{enumerate}[label=\textbf{\alph*)}]
		\item $y=2x+1$
		\item $y=-2x+1$
		\item $y=2x-1$
		\item $y=x^2+1$
	\end{enumerate}
	\end{multicols}
	
	\item 
	Le volume d'un pavé de 5cm de longueur, 4 cm de largeur, et 10 cm de hauteur est
	\begin{multicols}{4}
	\begin{enumerate}[label=\textbf{\alph*)}]
		\item
	\end{enumerate}
	\end{multicols}
	
	\item 
	Usain Bolt court à 10 mètres par seconde pendant $t$ secondes après avoir parcouru 30 mètres.
	Sa distance parcourue au total est
	\begin{multicols}{4}
	\begin{enumerate}[label=\textbf{\alph*)}]
		\item
	\end{enumerate}
	\end{multicols}
	
	\item 
	Considérons les deux droites ci-contre, courbes représentatives de fonctions $f$ et $g$.
	Le nombre $x$ vérifiant $f(x) = g(x)$ est donné environ égal à
	\begin{multicols}{4}
	\begin{enumerate}[label=\textbf{\alph*)}]
		\item
	\end{enumerate}
	\end{multicols}
\end{enumerate}

