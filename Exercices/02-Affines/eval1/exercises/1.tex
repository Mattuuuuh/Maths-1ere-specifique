%!TEX root = ../eval1.tex

\newcommand{\cm}{\text{ cm}}

\begin{enumerate}[label=\textbf{\arabic*.}]
	\item 
	Le nombre $x$ vérifiant $3x + 4 = 8x - 1$ est
	\begin{multicols}{4}
	\begin{enumerate}[label=\textbf{\alph*)}]
		\item $x = 1$
		\item $x = \frac5{11}$
		\item $x = -1$
		\item $x = 0$
	\end{enumerate}
	\end{multicols}
	
	\item 
	Le coefficient directeur de la fonction affine $f(x) = 2 - x$ est
	\begin{multicols}{4}
	\begin{enumerate}[label=\textbf{\alph*)}]
		\item $0$
		\item $1$
		\item $-1$
		\item $2$
	\end{enumerate}
	\end{multicols}
\end{enumerate}
\begin{multicols}{2}
	\begin{enumerate}[label=\textbf{\arabic*.}, resume]
		\item 
		Considérons la droite de la figure ci-contre.
		La seule équation pouvant lui correspondre est
		\begin{multicols}{2}
		\begin{enumerate}[label=\textbf{\alph*)}]
			\item $y=2x+1$
			\item $y=-2x+1$
			\item $y=2x-1$
			\item $y=x^2+1$
		\end{enumerate}
		\end{multicols}
	\end{enumerate}
	\vfill\null
	\centering
	\begin{tikzpicture}[scale=.7]
	\begin{axis}[xmin = -2, xmax=2, ymin=-2, ymax=5, axis x line=middle, axis y line=middle, axis line style=<->, xlabel={}, ylabel={}, grid=none, grid style = {opacity=.5}, clip=true, ticks = none]
		\addplot[BLUE_E, very thick, domain =-2:4, samples=2] {2*x+1};% node[below right, pos=.6]{$\C_f$} ;
	\end{axis}
	\end{tikzpicture}
\end{multicols}
\begin{enumerate}[label=\textbf{\arabic*.}]\setcounter{enumi}{3}
	\item 
	Le volume d'un pavé de 5cm de longueur, 4 cm de largeur, et $x$ cm de hauteur est égal à
	\begin{multicols}{4}
	\begin{enumerate}[label=\textbf{\alph*)}]
		\item $20x \cm^3$
		\item $25x \cm^3$
		\item $4x+5 \cm^3$
		\item $5x+4 \cm^3$
	\end{enumerate}
	\end{multicols}
	
	\item 
	Usain Bolt court à 10 mètres par seconde pendant $t$ secondes après avoir parcouru 30 mètres.
	Sa distance parcourue au total est égal à, en mètres, 
	\begin{multicols}{4}
	\begin{enumerate}[label=\textbf{\alph*)}]
		\item $10 + 30t$
		\item $30t$
		\item $40$
		\item $30 + 10t$
	\end{enumerate}
	\end{multicols}
	
\end{enumerate}
\begin{multicols}{2}
	\begin{enumerate}[label=\textbf{\arabic*.}]\setcounter{enumi}{5}
		\item 
		Considérons les deux droites ci-contre. %, courbes représentatives de fonctions $f$ et $g$.
		Le nombre $x$ vérifiant $f(x) = g(x)$ est donné environ égal à
		\begin{multicols}{2}
		\begin{enumerate}[label=\textbf{\alph*)}]
			\item $x=1$
			\item $x=2$
			\item $x=3$
			\item $x=-1$
		\end{enumerate}
		\end{multicols}
	\end{enumerate}
	\vfill\null
	\centering
	\begin{tikzpicture}[scale=1]
	\begin{axis}[xmin = -4, xmax=4, ymin=-3, ymax=7, axis x line=middle, axis y line=middle, axis line style=<->, xlabel={}, ylabel={}, grid=both, grid style = {opacity=.5}, clip=true, xtick distance=1]
		\addplot[GREEN_E, very thick, domain =-4:4, samples=2] {2  - x} node[below, pos=.9]{$\C_f$} ;
		\addplot[RED_E, very thick, domain =-4:4, samples=2] {3*x + 6} node[left, pos=.3]{$\C_g$} ;
	\end{axis}
	\end{tikzpicture}
\end{multicols}
