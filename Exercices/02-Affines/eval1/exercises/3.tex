%!TEX root = ../eval1.tex

\exe{7}{
	On étudie la température d'un plat sorti du four.
	
	Au début de l'expérience, le plat est à 215°C.
	Toutes les 30 secondes, on mesure sa température.
	Les premiers résultats sont notés dans le tableau ci-dessous.
	
	\begin{center}
	\begin{tabular}{|c|c|}\hline
		{Temps écoulé  (en minutes)} & {Température (°C)} \\ \hline
		0 & 215 \\\hline
		0,5 & 210 \\\hline
		1 & 205 \\\hline
		1,5 & 200 \\\hline
		2 & 195 \\ \hline
	\end{tabular}
	\end{center}
	
	Soit $n$ un entier naturel.
	On note $u(n)$ la température du plat après $n$ périodes de 30 secondes.
	Ainsi, $u(0) = 215 ; u(1) = 210 ; u(2) = 205 ; \dots$.
	\begin{enumerate}
		\item
		Justifier que les termes $u(0), u(1), u(2), u(3)$ sont en progression arithmétique.
		Quelle est la raison ?
		\item 
		En supposant que la température continue d'évoluer de façon arithmétique, donner $u(n)$ en fonction de $n$.
		\item 
		En combien de temps le plat atteindra-t-il une température de 25°C ? 
		Donner un résultat en minutes.
		\item
		Le modèle arithmétique de la température est-il réaliste ? 
	\end{enumerate}
}{exe:3}{
	\begin{enumerate}
		\item
		On calcule $u(1) - u(0) = -5, u(2) - u(1) = -5,$ et $u(3) - u(2) = -5$.
		Pour passer d'un terme à l'autre, on soustrait 5 : la progression est arithmétique de raison -5.
		\item 
		On a $u(n) = - 5n + 215$ d'après le cours.
		\item
		On pose $u(n) = 25$ et on résoud pour $n$.
		\begin{align*}
			u(n) &= 25 \\
			-5n + 215 &= 25 \\
			-5n &= -190 \\
			n &= \frac{-190}{-5} = \frac{190}5 \\
			n &= \frac{380}{10} = 38
		\end{align*}
		Après 38 périodes de 30 secondes, la température du plat est de 25°C.
		En minutes, cela fait $38 \times \frac12 = 19$ minutes.
		
		\item
		Le modèle n'est pas réaliste car la température continue de diminuer indéfiniment.
		D'après celui-ci, en 50 minutes, la température du plat sera $u(100) = -500 + 215 = -385$, ce qui n'est pas un température possible.
	\end{enumerate}
}