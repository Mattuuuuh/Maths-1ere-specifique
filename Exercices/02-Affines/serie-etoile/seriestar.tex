\documentclass[a4paper, 12pt]{extarticle}
\usepackage[french]{babel}
\usepackage[
a4paper,
margin=2cm,
]{geometry}

\usepackage[utf8x]{inputenc}
%fonts
\usepackage{libertinus,libertinust1math}
\usepackage{amsmath,amsthm,amssymb,mathtools}

%virgules
\usepackage{icomma}

% HEADER, ARRAY, ENUM, MULTIOCL
\usepackage{fancyhdr}
\usepackage{array}
\usepackage{multicol, enumitem}
\newcolumntype{P}[1]{>{\centering\arraybackslash}p{#1}}
\usepackage{stackengine}
\newcommand\xrowht[2][0]{\addstackgap[.5\dimexpr#2\relax]{\vphantom{#1}}}

% theorems
\theoremstyle{definition}

\newtheorem{theorem}{Théorème}
\newtheorem{corollaire}[theorem]{Corollaire}
\newtheorem{lemme}[theorem]{Lemme}
\newtheorem{proposition}[theorem]{Proposition}
\newtheorem{exercice}[theorem]{Exercice}
\newtheorem{exemple}[theorem]{Exemple}
\newtheorem{definition}[theorem]{Définition}
\newtheorem*{question}{Question}
\newtheorem*{preuve}{Preuve}
\newtheorem*{remarque}{Remarque}
\newtheorem*{strategie}{Stratégie}
\newtheorem*{methode}{Méthode}
\newtheorem*{notation}{Notation}
\newtheorem*{nomenclature}{Nomenclature}
\newtheorem{axiome}[theorem]{Axiome}
\newtheorem*{heuristique}{Heuristique}

\newtheorem*{definition*}{Définition}
\newtheorem*{lemme*}{Lemme}
\newtheorem*{proposition*}{Proposition}
\newtheorem*{theorem*}{Théorème}
\newtheorem*{corollaire*}{Corollaire}

%%%%%%%%%%%%%%%%%%%%%%%%%%%%
% MDFRAMED SURROUND
%%%%%%%%%%%%%%%%%%%%%%%%%%%%

\usepackage[framemethod=pgf]{mdframed}
% def
\mdfdefinestyle{definition}{
	hidealllines=true,
	leftline=true,
	linecolor=BLUE_E,
	linewidth=2pt,
	innertopmargin=-4pt,
	innerrightmargin=0,
	nobreak=true,
}
\surroundwithmdframed[
	style=definition,
]{definition}
\surroundwithmdframed[
	style=definition,
]{definition*}

% thm
\mdfdefinestyle{theorem}{
	linecolor=MAROON_C,
	linewidth=2pt,
	roundcorner=4pt,
	innertopmargin=-4pt,
	nobreak=true,
}
\surroundwithmdframed[
	style=theorem,
]{theorem}
\surroundwithmdframed[
	style=theorem,
]{theorem*}

% prop
\mdfdefinestyle{proposition}{
	linecolor=GREEN_E,
	linewidth=2pt,
	innertopmargin=-4pt,
	nobreak=true,
}
\surroundwithmdframed[
	style=proposition,
]{proposition}
\surroundwithmdframed[
	style=proposition,
]{proposition*}

% lemme
\mdfdefinestyle{lemme}{
	linecolor=TEAL_E,
	linewidth=1pt,
	innertopmargin=-4pt,
	nobreak=true,
}
\surroundwithmdframed[
	style=lemme,
]{lemme}
\surroundwithmdframed[
	style=lemme,
]{lemme*}

% corollaire
\mdfdefinestyle{corollaire}{
	linecolor=YELLOW_E,
	linewidth=2pt,
	roundcorner=4pt,
	innertopmargin=-4pt,
	nobreak=true,
}
\surroundwithmdframed[
	style=corollaire,
]{corollaire}
\surroundwithmdframed[
	style=corollaire,
]{corollaire*}

% exercices
\usepackage[answerdelayed, lastexercise]{exercise}
\usepackage{ifthen}
\renewcommand{\ExerciseHeader}{
	\tikz[baseline=(R.base)]\node[draw,rectangle, thick, inner sep=2pt](R) {\textbf{\theExercise.}};\!
	\ifnum\ExerciseDifficulty=0
	\else
		(\theExerciseDifficulty)
	\fi
}
\renewcommand{\DifficultyMarker}{$\star$}
\renewcommand{\AnswerHeader}{
	\tikz[baseline=(R.base)]\node[draw,rectangle, thick, inner sep=2pt](R) {\textbf{\theExercise.}};\!
}
\newcommand{\exe}[4]{
	\begin{Exercise}[title=#1, label=#3]
		\if\relax\detokenize\expandafter{\ExerciseTitle}\relax
		%\marginpar{[Bonus]}
		\else
		\marginpar{\mbox{[\ExerciseTitle]}}
		\fi
		#2
	\end{Exercise}
	\begin{Answer}[ref=#3]
		#4
	\end{Answer}
}
\newcommand{\exemulticols}[5]{
	\begin{multicols}{2}
	\begin{Exercise}[title=#1, label=#4]
		\if\relax\detokenize\expandafter{\ExerciseTitle}\relax
		%\marginnote{[Bonus]}
		\else
		\marginnote{\mbox{[\ExerciseTitle]\qquad}}
		\fi
		#2
	\end{Exercise}
	\columnbreak
		#3
	\end{multicols}
	\begin{Answer}[ref=#4]
		#5
	\end{Answer}
}

% date
\usepackage{advdate}

% plots
\usepackage{pgfplots}
\tikzset{
	every axis/.style = {clip=false, axis lines=center, axis line style=<->, xlabel={}, ylabel={}, grid=both, grid style = {opacity=.5}, domain=-2:2}
}

%subfigures
\usepackage{subcaption}

%hyperlink footnote
\usepackage{hyperref}

% tableaux var, signe
\usepackage{tkz-tab}

%wider tabulars
\def\arraystretch{2}
\setlength\tabcolsep{15pt}
\usepackage{makecell} %pour \thead dans tabular ex3 (aligner verticalement le coeff de proportionnalité)

% for striked out implies sign (\centernot\implies)
\usepackage{centernot}

%%%%%%%%%%%%%%%%%%%%%%%%%%%%%%
% SELF MADE COLORS
%%%%%%%%%%%%%%%%%%%%%%%%%%%%%%

%!TEX encoding = UTF8
%!TEX root = 0-notes.tex

%%%%%%%%%%%%%%%%%%%%%%%%%%%%%%
% SELF MADE COLORS
%%%%%%%%%%%%%%%%%%%%%%%%%%%%%%


\definecolor{myg}{RGB}{56, 140, 70}
\definecolor{myb}{RGB}{45, 111, 177}
\definecolor{myr}{RGB}{199, 68, 64}
\definecolor{mytheorembg}{HTML}{F2F2F9}
\definecolor{mytheoremfr}{HTML}{00007B}
\definecolor{mylenmabg}{HTML}{FFFAF8}
\definecolor{mylenmafr}{HTML}{983b0f}
\definecolor{mypropbg}{HTML}{f2fbfc}
\definecolor{mypropfr}{HTML}{191971}
\definecolor{myexamplebg}{HTML}{F2FBF8}
\definecolor{myexamplefr}{HTML}{88D6D1}
\definecolor{myexampleti}{HTML}{2A7F7F}
\definecolor{mydefinitbg}{HTML}{E5E5FF}
\definecolor{mydefinitfr}{HTML}{3F3FA3}
\definecolor{notesgreen}{RGB}{0,162,0}
\definecolor{myp}{RGB}{197, 92, 212}
\definecolor{mygr}{HTML}{2C3338}
\definecolor{myred}{RGB}{127,0,0}
\definecolor{myyellow}{RGB}{169,121,69}
\definecolor{myexercisebg}{HTML}{F2FBF8}
\definecolor{myexercisefg}{HTML}{88D6D1}
\definecolor{doc}{RGB}{0,60,110}

% manim colors because they're beautiful
% https://docs.manim.community/en/stable/reference/manim.utils.color.manim_colors.html

\definecolor{BLACK}{HTML}{000000}\definecolor{BLUE}{HTML}{58C4DD}\definecolor{BLUE_A}{HTML}{C7E9F1}\definecolor{BLUE_B}{HTML}{9CDCEB}\definecolor{BLUE_C}{HTML}{58C4DD}\definecolor{BLUE_D}{HTML}{29ABCA}\definecolor{BLUE_E}{HTML}{236B8E}\definecolor{DARKER_GRAY}{HTML}{222222}\definecolor{DARKER_GREY}{HTML}{222222}\definecolor{DARK_BLUE}{HTML}{236B8E}\definecolor{DARK_BROWN}{HTML}{8B4513}\definecolor{DARK_GRAY}{HTML}{444444}\definecolor{DARK_GREY}{HTML}{444444}\definecolor{GOLD}{HTML}{F0AC5F}\definecolor{GOLD_A}{HTML}{F7C797}\definecolor{GOLD_B}{HTML}{F9B775}\definecolor{GOLD_C}{HTML}{F0AC5F}\definecolor{GOLD_D}{HTML}{E1A158}\definecolor{GOLD_E}{HTML}{C78D46}\definecolor{GRAY}{HTML}{888888}\definecolor{GRAY_A}{HTML}{DDDDDD}\definecolor{GRAY_B}{HTML}{BBBBBB}\definecolor{GRAY_BROWN}{HTML}{736357}\definecolor{GRAY_C}{HTML}{888888}\definecolor{GRAY_D}{HTML}{444444}\definecolor{GRAY_E}{HTML}{222222}\definecolor{GREEN}{HTML}{83C167}\definecolor{GREEN_A}{HTML}{C9E2AE}\definecolor{GREEN_B}{HTML}{A6CF8C}\definecolor{GREEN_C}{HTML}{83C167}\definecolor{GREEN_D}{HTML}{77B05D}\definecolor{GREEN_E}{HTML}{699C52}\definecolor{GREY}{HTML}{888888}\definecolor{GREY_A}{HTML}{DDDDDD}\definecolor{GREY_B}{HTML}{BBBBBB}\definecolor{GREY_BROWN}{HTML}{736357}\definecolor{GREY_C}{HTML}{888888}\definecolor{GREY_D}{HTML}{444444}\definecolor{GREY_E}{HTML}{222222}\definecolor{LIGHTER_GRAY}{HTML}{DDDDDD}\definecolor{LIGHTER_GREY}{HTML}{DDDDDD}\definecolor{LIGHT_BROWN}{HTML}{CD853F}\definecolor{LIGHT_GRAY}{HTML}{BBBBBB}\definecolor{LIGHT_GREY}{HTML}{BBBBBB}\definecolor{LIGHT_PINK}{HTML}{DC75CD}\definecolor{LOGO_BLACK}{HTML}{343434}\definecolor{LOGO_BLUE}{HTML}{525893}\definecolor{LOGO_GREEN}{HTML}{87C2A5}\definecolor{LOGO_RED}{HTML}{E07A5F}\definecolor{LOGO_WHITE}{HTML}{ECE7E2}\definecolor{MAROON}{HTML}{C55F73}\definecolor{MAROON_A}{HTML}{ECABC1}\definecolor{MAROON_B}{HTML}{EC92AB}\definecolor{MAROON_C}{HTML}{C55F73}\definecolor{MAROON_D}{HTML}{A24D61}\definecolor{MAROON_E}{HTML}{94424F}\definecolor{ORANGE}{HTML}{FF862F}\definecolor{PINK}{HTML}{D147BD}\definecolor{PURE_BLUE}{HTML}{0000FF}\definecolor{PURE_GREEN}{HTML}{00FF00}\definecolor{PURE_RED}{HTML}{FF0000}\definecolor{PURPLE}{HTML}{9A72AC}\definecolor{PURPLE_A}{HTML}{CAA3E8}\definecolor{PURPLE_B}{HTML}{B189C6}\definecolor{PURPLE_C}{HTML}{9A72AC}\definecolor{PURPLE_D}{HTML}{715582}\definecolor{PURPLE_E}{HTML}{644172}\definecolor{RED}{HTML}{FC6255}\definecolor{RED_A}{HTML}{F7A1A3}\definecolor{RED_B}{HTML}{FF8080}\definecolor{RED_C}{HTML}{FC6255}\definecolor{RED_D}{HTML}{E65A4C}\definecolor{RED_E}{HTML}{CF5044}\definecolor{TEAL}{HTML}{5CD0B3}\definecolor{TEAL_A}{HTML}{ACEAD7}\definecolor{TEAL_B}{HTML}{76DDC0}\definecolor{TEAL_C}{HTML}{5CD0B3}\definecolor{TEAL_D}{HTML}{55C1A7}\definecolor{TEAL_E}{HTML}{49A88F}\definecolor{WHITE}{HTML}{FFFFFF}\definecolor{YELLOW}{HTML}{FFFF00}\definecolor{YELLOW_A}{HTML}{FFF1B6}\definecolor{YELLOW_B}{HTML}{FFEA94}\definecolor{YELLOW_C}{HTML}{FFFF00}\definecolor{YELLOW_D}{HTML}{F4D345}\definecolor{YELLOW_E}{HTML}{E8C11C}

%%%%%%%%%%%%%%%%%%%%%%%%%%%%
% LETTERFONTS
%%%%%%%%%%%%%%%%%%%%%%%%%%%%

%!TEX encoding = UTF8
%!TEX root = 0-notes.tex

%fonts
\usepackage{libertinus,libertinust1math}
\usepackage[T1]{fontenc}

% for calligraphic C, D, P (important to import this after the font)
\usepackage{calrsfs}
\newcommand{\D}{\mathcal{D}}
\newcommand{\C}{\mathcal{C}}
\renewcommand{\P}{\mathcal{P}}

% Schwartz
\renewcommand{\S}{\mathcal{S}} % \S est le signe paragraphe normalement

% corps
\newcommand{\R}{\mathbb{R}}
\newcommand{\Rnn}{\mathbb{R}^{2n}}
\newcommand{\Z}{\mathbb{Z}}
\newcommand{\N}{\mathbb{N}}
\newcommand{\Q}{\mathbb{Q}}
\newcommand{\E}{\mathbb{E}}
\newcommand{\DD}{\mathbb{D}}

% order notations
\DeclareRobustCommand{\O}{%
  \text{\usefont{OMS}{cmsy}{m}{n}O}%
}

% japanese bracket
\newcommand{\japb}[1]{\langle #1 \rangle}

% arrows over partial derivatives
\newcommand{\lp}{\overleftarrow{\partial}}
\newcommand{\rp}{\overrightarrow{\partial}}

% quantization
\newcommand{\h}{\hbar}
\newcommand{\Opht}{\textrm{Op}_{\h}^{t}}
\newcommand{\Op}[2][\hbar]{\textrm{Op}_{#1}^{#2}}

% omega functions
\newcommand{\omegap}[2][\rho_0]{\omega(\partial_{#1},\partial_{#2})}
\newcommand{\omegar}[2][\rho_0]{\omega(#1,#2)}

% space before semicolon
\mathcode`\;="303B

% for \Lightning
\usepackage{marvosym}

% for \warning
\newcommand{\warning}{{\fontencoding{U}\fontfamily{futs}\selectfont\char 49\relax}}

% Q(\sqrt(d)) field
\newcommand{\Qsqrt}[1]{\Q\bigl(\mspace{-3mu}\sqrt{#1}\bigr)}


%%%%%%%%%%%%%%%%%%%%%%%%%%%%
% MACROS
%%%%%%%%%%%%%%%%%%%%%%%%%%%%

%!TEX encoding = UTF8
%!TEX root = 0-notes.tex

%%%%%%%%%%%%%%%%%%%%%%%%%%%%%%
% SELF MADE COMMANDS
%%%%%%%%%%%%%%%%%%%%%%%%%%%%%%


%%
% tcolor environments VS clean environments
%%

\ifclean

\newcommand{\thm}[3]{\begin{theorem}[#1]\label{#3}#2\end{theorem}}
\newcommand{\cor}[3]{\begin{corollaire}[#1]\label{#3}#2\end{corollaire}}
\newcommand{\lem}[3]{\begin{lemme}[#1]\label{#3}#2\end{lemme}}
\newcommand{\mprop}[3]{\begin{proposition}[#1]\label{#3}#2\end{proposition}}
\newcommand{\ex}[3]{\begin{exemple}[#1]\label{#3}#2\end{exemple}}
%\newcommand{\exe}[3]{\begin{exercice}[#1]\label{#3}#2\end{exercice}}
\newcommand{\dfn}[3]{\begin{definition}[#1]\label{#3}#2\end{definition}}
\newcommand{\qs}[2]{\begin{question}[#1]#2\end{question}}
\newcommand{\pf}[2]{\begin{preuve}[#1]#2\end{preuve}}
\newcommand{\nt}[1]{\begin{remarque}#1\end{remarque}}
\newcommand{\str}[1]{\begin{strategie}#1\end{strategie}}
\newcommand{\mth}[1]{\begin{methode}#1\end{methode}}
\newcommand{\ax}[3]{\begin{axiome}[#1]\label{#3}#2\end{axiome}}

\newcommand{\exe}[4]{
	\begin{Exercise}[title=#1, label=#3]
		\marginpar{\mbox{\scriptsize(solution p.\pageref{\ExerciseLabel-Answer})}}
		#2
	\end{Exercise}
	\begin{Answer}[ref=#3]
		#4
	\end{Answer}
}

\newcommand{\exemulticols}[5]{
	\begin{multicols}{2}
	\begin{Exercise}[title=#1, label=#4]
		\marginnote{\mbox{\scriptsize(solution p.\pageref{\ExerciseLabel-Answer})}}
		#2
	\end{Exercise}
		#3
	\end{multicols}
	\begin{Answer}[ref=#4]
		#5
	\end{Answer}
}

\else

\newcommand{\thm}[3]{\begin{Theorem}[label=#3]{#1}{}#2\end{Theorem}}
\newcommand{\cor}[3]{\begin{Corollary}[label=#3]{#1}{}#2\end{Corollary}}
\newcommand{\lem}[3]{\begin{Lemma}[label=#3]{#1}{}#2\end{Lemma}}
\newcommand{\mprop}[3]{\begin{Prop}[label=#3]{#1}{}#2\end{Prop}}
\newcommand{\ex}[3]{\begin{Example}[label=#3]{#1}{}#2\end{Example}}
%\newcommand{\exe}[3]{\begin{Exe}[label=#3]{#1}{}#2\end{Exe}}
\newcommand{\dfn}[3]{\begin{Definition}[colbacktitle=red!75!black, label=#3]{#1}{}#2\end{Definition}}
\newcommand{\qs}[2]{\begin{MyQuestion}{#1}{}#2\end{MyQuestion}}
\newcommand{\pf}[2]{\begin{myproof}[#1]#2\end{myproof}}
\newcommand{\nt}[1]{\begin{Note}#1\end{Note}}
\newcommand{\str}[1]{\begin{Strategy}#1\end{Strategy}}
\newcommand{\mth}[1]{\begin{Methode}#1\end{Methode}}
\newcommand{\axiome}[3]{\begin{Axiome}[label=#3]{#1}{}#2\end{Axiome}}

\newcommand{\exe}[4]{
	\begin{Exe}[label=#3]{}{}#2\end{Exe}
	\begin{Answer}[ref=#3]
		#4
	\end{Answer}
}

\fi

\newcommand{\notations}[1]{\begin{notation}#1 \end{notation}}
\newcommand{\nomen}[1]{\begin{nomenclature}#1 \end{nomenclature}}
\newcommand{\heur}[1]{\begin{heuristique}#1\end{heuristique}}

%%

% deliminators
\DeclarePairedDelimiter{\abs}{\lvert}{\rvert}
%\DeclarePairedDelimiter{\norm}{\lVert}{\rVert}

\DeclarePairedDelimiter{\ceil}{\lceil}{\rceil}
\DeclarePairedDelimiter{\floor}{\lfloor}{\rfloor}
\DeclarePairedDelimiter{\round}{\lfloor}{\rceil}

\newsavebox\diffdbox
\newcommand{\slantedromand}{{\mathpalette\makesl{d}}}
\newcommand{\makesl}[2]{%
\begingroup
\sbox{\diffdbox}{$\mathsurround=0pt#1\mathrm{#2}$}%
\pdfsave
\pdfsetmatrix{1 0 0.2 1}%
\rlap{\usebox{\diffdbox}}%
\pdfrestore
\hskip\wd\diffdbox
\endgroup
}
\newcommand{\dd}[1][]{\ensuremath{\mathop{}\!\ifstrempty{#1}{%
\slantedromand\@ifnextchar^{\hspace{0.2ex}}{\hspace{0.1ex}}}%
{\slantedromand\hspace{0.2ex}^{#1}}}}
\ProvideDocumentCommand\dv{o m g}{%
  \ensuremath{%
    \IfValueTF{#3}{%
      \IfNoValueTF{#1}{%
        \frac{\dd #2}{\dd #3}%
      }{%
        \frac{\dd^{#1} #2}{\dd #3^{#1}}%
      }%
    }{%
      \IfNoValueTF{#1}{%
        \frac{\dd}{\dd #2}%
      }{%
        \frac{\dd^{#1}}{\dd #2^{#1}}%
      }%
    }%
  }%
}
\providecommand*{\pdv}[3][]{\frac{\partial^{#1}#2}{\partial#3^{#1}}}
%  - others
\DeclareMathOperator{\Lap}{\mathcal{L}}
\DeclareMathOperator{\Var}{Var} % variance
\DeclareMathOperator{\Cov}{Cov} % covariance

% Since the amsthm package isn't loaded

% I prefer the slanted \leq
\let\oldleq\leq % save them in case they're every wanted
\let\oldgeq\geq
\renewcommand{\leq}{\leqslant}
\renewcommand{\geq}{\geqslant}

% tel que
\newcommand{\tqs}{\text{ tels que }}
\newcommand{\tq}{\text{ tq. }}
\newcommand{\et}{\text{ et }}
\newcommand{\ou}{\text{ ou }}
\newcommand{\pourtout}{\text{ pour tout }}
\newcommand{\sct}{\text{ sachant }}

% Lois
\newcommand{\Bern}{\text{Bern}}
\newcommand{\Binom}{\text{Binom}}

% ensemble avec bigl et bigr
\newcommand{\bigset}[1]{\bigl\{ #1 \bigr\}}
\newcommand{\Bigset}[1]{\Bigl\{ #1 \Bigr\}}
\newcommand{\bigpar}[1]{\bigl( #1 \bigr)}
\newcommand{\Bigpar}[1]{\Bigl( #1 \Bigr)}

% PLUS INFTY AND MINUS INFTY WITH NO SPACE
\newcommand{\pinfty}{{+}\infty}
\newcommand{\minfty}{{-}\infty}

% vecteur flèche
\renewcommand{\vec}[1]{\overrightarrow{#1}}

% vecteur pmatrix
\newcommand{\pvec}[2]{\begin{pmatrix} #1 \\ #2 \end{pmatrix}}

% vecteur norme
\newcommand{\norm}[1]{\left\Vert #1 \right\Vert}

% point plan
\newcommand{\point}[3]{
	#1\left(#2 ; #3 \right)
}

% \smash avant \underline pour coller la ligne au mot
\let\oldunderline\underline
\renewcommand{\underline}[1]{\oldunderline{\smash{#1}}}

% emph + index
\newcommand{\emphindex}[1]{\emph{#1}\index{#1}}

% tableau croisé
\newcommand{\tableaucroise}[4]{
\begin{tabular}{|c|c|c|c|}
	\cline{2-4}
	\multicolumn{1}{c|}{} & #1 \\ \hline
	#2 \\ \hline
	#3  \\ \hline
	#4  \\ \hline
\end{tabular}
}

% python minted
\newcommand{\python}[1]{
\inputminted[
		linenos,
		gobble=0,
		breaklines=true, % otherwise it breaks for no apparent reason?
		breakafter=,,
		fontsize=\small,
		numbersep=8pt,
		tabsize=4, % tab ident = 4 spaces
		fontfamily=courier, %important pour les signes <, >
]{python}{python/#1.py}
}


\AdvanceDate[1]

\begin{document}
\pagestyle{fancy}
\fancyhead[L]{Première spécifique}
\fancyhead[C]{\textbf{Fonctions affines — approfondissements}}
\fancyhead[R]{\today}


\exe{, difficulty=1}{
    Soit $f(x) = ax + b$ une fonction affine sur $\R$ de paramètres $a, b\in\R$ et $P(x_P;y_P)$ un point du plan.

    \begin{enumerate}
        \item 
        Montrer que la fonction affine $g$ donnée par
            \[ g(x) = a(x-x_P) + y_P, \]
        pour tout $x\in\R$ vérifie que $\C_f$ est parallèle à $\C_g$ et que $P \in \C_g$.
        \item
        Montrer que si $P\in\C_f$, alors $f=g$ ($f$ et $g$ admettent le même coefficient directeur et la même ordonnée à l'origine).
    \end{enumerate}

}{exe:1}{
    \begin{enumerate}
        \item 
        Sont coefficient directeur est $a$, le même que celui de $f$.
        Sa courbe est donc parallèle à celle de $f$.
        
        De plus, $g(x_P) = a(x_P - x_P) + y_P = y_P$, donc $P$ appartient à la courbe de $g$.
        \item
        Les coefficients directeurs sont déjà égaux.
        L'ordonnée à l'origine de $g$ est $y_P - ax_P$.
        Or, comme $P$ appartient à la courbe de $f$, on a $f(x_P) = ax_P + b = y_P$.
        Il suit que $y_P - ax_P = b$, ce qui conclut.
    \end{enumerate}
}

\exe{, difficulty=1}{
    Soient $f(x) = 3x^2 + 17x - 11$ et $g(x) = 2x^2 + 17x - 10$ pour tout $x \in \R$ deux fonctions quadratiques.

    Déterminer entièrement $\C_f \cap \C_g$.
}{exe:2}{
	On pose $f(x) = g(x)$, qu'on simplifie en $x^2 = 1$.
	On trouve donc deux solution, $x=1$ et $x=-1$.
	
	Comme $f(1) = 9$ et $f(-1) = -25$, les deux points d'intersections sont donc $( 1 ; 9)$ et $(-1 ; -25)$.
}

\exe{, difficulty=2}{
    Soient $f(x) = x$ et $g(x) = x^3 - 3x^2 + 4x - 1$ deux fonctions polynomiales sur $\R$.

    \begin{enumerate}
        \item 
        Montrer que $(x-1)^3 = x^3 - 3x^2 + 3x - 1$ pour tout $x\in\R$.
        \item
        Déterminer entièrement $\C_f \cap \C_g$.
        \item
        Créer une fonction polynomiale $h$ de degré $4$ telle que $\C_f \cap \C_h = \bigset{ (1;1) }.$
    \end{enumerate}
}{exe:3}{
    \begin{enumerate}
        \item 
        On utilise que $(x-1)^3 = (x-1)(x-1)(x-1) = (x^2 - 2x + 1)(x-1)$ et on développe tranquillement.
        \item
        Après avoir posé $f(x) = g(x)$, on obtient $(x-1)^3 = 0$, d'après la première question.
        C'est un produit nul (trois fois le même terme), donc $x-1=0$ et $x=1$.
        Le point d'intersection est donc $(1 ; 1)$.
        \item
        On choisit $h(x) = (x-1)^4 + x$
        En développant, on trouve $h(x) = x^4 - 4x^3 + 6x^2 - 3x + 1$.
    \end{enumerate}
}

\exe{, difficulty=1}{
    Soient $f(x) = -2x^2 + 7x + 2$ et $g(x) = -3x^2 + 2x +16$ pour tout $x \in \R$ deux fonctions quadratiques.
    
    \begin{enumerate}
        \item 
        Déterminer l'autre solution de $f(x)-g(x)=0$, sachant que $x=2$ en est une et donc que $f(x)-g(x) = (x-2)h(x)$, où $h$ est affine.
        \item
        Déterminer entièrement $\C_f \cap \C_g$.
        \item
        Créer une fonction polynomiale $F$ de degré $2$ telle que $\C_f \cap \C_F = \bigset{ (2;8), (-1; -7) }.$
    \end{enumerate}
}{exe:4}{
    \begin{enumerate}
        \item 
        $f(x) - g(x) = x^2 + 5x - 14 = (x-2)h(x)$
        Comme $h$ est affine, $h(x) = ax+b$, et $x^2 + 5x - 14 = (x-2)(ax+b) = ax^2 + (b-2a)x - 2b$.
        
        En identifiant les coefficients devant les $x^2$, on trouve $1=a$.
        Les coefficients constants donnent $-14 = -2b$, et $b=7$.
        Donc $f(x)-g(x) = (x-2)(x+7)$.
        \item
        On cherche les $x$ vérifiant $(x-2)(x+7)=0$. 
        C'est un produit nul, et donc un des deux termes est nul.
        Il suit que $x=2$ et $x=-7$ sont les deux solutions.
        
        Ainsi, $(2 ; 8)$ et $(-7 ; -145)$ sont les deux points d'intersection.
        \item
        On vérifiera bien sûr que $f(-1)=-7$ avant de tenter quoi que ce soit...
        
        Par analogie au raisonnement précédent, on souhaite désormais créer $F$ de telle sorte que $f(x) - F(x) = (x-2)(x+1)$.
        Ainsi $F(x) = f(x) - (x-2)(x+1) = -3x^2 +8x +4$.
    \end{enumerate}
}

\exe{, difficulty=2}{
    Soit $f(x) = ax+b$ et $g(x) = a'x + b'$ pour tout $x \in \R$ deux fonctions affines de paramètres $a, a', b, b' \in\R$.

    Montrer que si $a\neq a'$, alors
        \[ \C_f \cap \C_g = \left\{ \left(\ \dfrac{b'-b}{a-a'} ; \dfrac{ab' - ba'}{a-a'} \right) \right\}. \]
    Comparer avec les réultats obtenus dans la série d'exercices précédente.
}{exe:5}{
	Soit $P(x_P ; y_P) \in \C_f \cap \C_g$.
	Alors $f(x_P) = y_P = g(x_P)$.
	Il suit que 
		\[ f(x_P) = g(x_P) \iff ax_P + b = a' x_P + b' \iff x_P = \dfrac{b'-b}{a-a'}. \]
	
	Ensuite, 
		\[ y_P = f(x_P) = a\dfrac{b'-b}{a-a'} + b = \dfrac{ab' -ab + ba -ba'}{a-a'} = \dfrac{ab' -ba'}{a-a'}. \]
}

\exe{, difficulty=1}{
	Soit $f(x) = ax+b$ une fonction affine sur $\R$ à paramètre $a, b\in\R$.
	
	Montrer que si $f(r)=0$ pour un certain $r\in\R$ on a alors, pour tout $x\in\R$,
		\[ f(x) = a(x-r). \]
}{exe:6}{
	L'exercice \ref{exe:1} conclut, grâce au point $P(r ; 0)$.
}

\exe{, difficulty=2}{
	Considérons une fonction quadratique 
		\[ f(x) = ax^2 + bx + c, \]
	où $a, b, c\in\R$ sont trois paramètres réels.
	On appelle cette forme, la \emph{forme développée} de $f$.
	
	Supposons de surcroît qu'on connaisse deux zéros distincts de $f$, c'est-à-dire qu'on connaisse $\alpha, \beta\in\R$ tels que $\alpha\neq\beta$ et
		\[ f(\alpha) = f(\beta) = 0. \]
	\begin{enumerate}
		\item Montrer que la fonction $g$ donnée par
			\[ g(x) = f(x) - a (x-\alpha)(x-\beta) \qquad \text{ pour tout } x\in\R \]
		est affine.
		\item Montrer que $g(\alpha) = g(\beta) = 0$.
		\item En déduire que $g$ est constamment nulle et donc que
			\[ f(x) = a (x-\alpha)(x-\beta)  \qquad \text{ pour tout } x\in\R.  \]
	\end{enumerate}
	
	\textbf{Conclusion} : une fonction quadratique $f$ se factorise comme produit de facteurs linéaires $x-r$ où $r$ est une \emph{racine} de $f$ (c'est-à-dire que $f(r) = 0$).
	C'est la \emph{forme factorisée} du polynôme.
	
	Ce résultat est généralisé par le \emph{théorème fondamental de l'algèbre} énoncé ci-dessous.
}{exe:7}{
	\begin{enumerate}
		\item 
			En développant l'expression de $g$, le terme quadratique (en $x^2$) s'annule.
			Ne subsiste qu'une expression de la forme $ax+b$, expression d'une fonction affine.
		\item 
			$g(\alpha) = f(\alpha) - a(\alpha-\alpha)(\alpha-\beta) = 0$, car $f(\alpha)=0$, et $\alpha-\alpha=0$.
			Idem pour $\beta$.
		\item 
			$g$ est affine et passe par $(\alpha ;0)$, et $(\beta ; 0)$, deux points distincts.
			$g$ est donc la fonction nulle, $g(x) = 0$.
			Or comme $g(x) = f(x) - a (x-\alpha)(x-\beta) =0$, on déduit que $f(x) = a (x-\alpha)(x-\beta)$.
	\end{enumerate}
}


\exe{, difficulty=1}{
	Considérons, pour tout $x\in\R$, la fonction quadratique suivante.
		\[ f(x) = x^2 -9x + 20 \]
	\begin{enumerate}
		\item Montrer que si $f(x) = (x-\alpha) (x-\beta)$ pour certains nombres réels $\alpha, \beta\in\R$, alors on a forcément $f(\alpha) = f(\beta) = 0$.
	\end{enumerate}
	Supposons désormais que $f(x) = (x-\alpha) (x-\beta)$.
	\begin{enumerate}[resume]
		\item Développer l'expression de droite, identifier les coeffients en $1, x, x^2$, et déduire que
			\[\begin{cases} \alpha + \beta = 9, \\ \alpha \beta = 20. \end{cases} \]
		\item Trouver $\alpha$ et $\beta$, les racines de $f$.
	\end{enumerate}
}{exe:8}{
	\begin{enumerate}
		\item Évaluer en $x=\alpha$ ou $x=\beta$ annule un des termes du produit, et donc le produit entier.
		C'est pour cela que la forme factorisée donne directement les racines d'un polynôme.
		\item 
			\[ f(x) = x^2 - 9x + 20 = (x-\alpha)(x-\beta) = x^2 - (\alpha + \beta)x + \alpha\beta. \]
		En $x^2$, on trouve $1=1$, en $x$, on trouve $\alpha+\beta=9$, et en $1$, on trouve $\alpha\beta=20$.
		\item 
		On cherche deux nombres donc le produit vaut 20 et la somme 9.
		$(\alpha, \beta) = (5 ; 4)$ fonctionne. À noter qu'on pourrait échanger les valeurs de $\alpha$ et $\beta$, par symétrie du problème.
			\[ x^2 - 9x + 20 = (x-5)(x-4). \]
	\end{enumerate}
}

\begin{thm}[théorème fondamental de l'algèbre]
	Si $f$ polynomiale admet une racine $r$ (c'est-à-dire que $f(r) = 0$), alors $f(x) = (x-r)g(x)$, avec $g$ polynomiale.
\end{thm}

\exe{, difficulty=1}{
	Soit $f(x) = x^3 - 3x^2 - 10x + 24$ une fonction cubique.
	
	\begin{enumerate}
		\item Montrer que $f(2) = 0$.
		\item Par division polynomiale, trouver $g$ quadratique telle que $f(x) = (x-2)g(x)$.
		\item Montrer que $g(4) = 0$, et factoriser complètement $f$ comme produit de facteurs linéaires.
	\end{enumerate}
}{exe:9}{
	\begin{enumerate}
		\item Trivial.
		\item La division polynômiale fonctionne ainsi : à chaque étape, le diviseur ($x-2$ ici) annule la plus grande puissance de $x$ du divisé.
		Ainsi, $f(x) - x^2(x-2) = -x^2 - 10x + 24$ annule le terme en $x^3$.
		On continue : 
			\[ f(x) - x^2(x-2) + x(x-2) = -12x + 24, \]
		annule le terme en $x^2$.
		Puis, 
			\[ f(x) - x^2(x-2) + x(x-2) + 12(x-2) = 0, \]
		qui permet de conclure que
			\[ f(x) = (x^2 - x - 12)(x-2) = g(x)(x-2). \]
		\item 
		On diviser $x^2 - x - 12$ par $x-4$ pour trouver
			\[ x^2 - x - 12 = (x-4)(x+3). \]
		On conclut que $x^3 - 3x^2 - 10x + 24 = (x-2)(x-4)(x+3)$.
	\end{enumerate}
}

\exe{, difficulty=3}{
	Soit $f, g$ deux polynômes.
	Montrer qu'il existe deux polynômes, $r$, et $q$, tels que
		\[ f = gq + r, \]
	avec le degré de $r$ strictement inférieur à celui de $q$.
	
	En déduire le théorème fondamental de l'algèbre.
}{exe:10}{
	La division polynomiale comme décrite à l'exercice \ref{exe:10} permet de se convaincre qu'on peut enlever un multiple de $g$ à $f$ pour réduire son degré, jusqu'à obtenir un reste $r$ de degré strictement inférieur.
	
	Le théorème fondamental de l'algèbre se démontre donc ainsi : si $f$ est polynomiale et admet une racine $r$, alors la division $f(x) = q(x) (x-r) + r(x)$ implique que $r(x) = c$ est une constante (car son degré est strictement inférieur à celui de $x-r$.
	Or, en évaluant en $x=r$, on trouve que $0 = 0 + c$, et donc que $c=0$.
	D'où $f(x) = q(x) (x-r)$, comme voulu.
}

% prochaine série
%\exe{, difficulty=2}{
%	Supposons que le polynôme du second degré $p(x) = ax^2 + bx + c$ s'écrive de la forme $p(x) = a'(x-\alpha)^2 - \beta$.
%	On appelle cette forme, la \emph{forme canonique} de $f$.
%	
%	Montrer, par identification des coefficients en $x^2, x, 1$, que $a'=a$, puis que $\alpha = \dfrac{-b}{2a}$, et enfin que $\beta = \dfrac{b^2 - 4ac}{4a}$.
%}{exe:déterminant-2nddeg}{
%	Sol11.
%}

%%%%%%%%%%%%

\newpage
\fancyhead[C]{\textbf{Solutions}}
\shipoutAnswer

\end{document}
