\documentclass[a4paper, 12pt]{extarticle}
\usepackage[french]{babel}
\usepackage[
a4paper,
margin=2cm,
]{geometry}

\usepackage[utf8x]{inputenc}
%fonts
\usepackage{libertinus,libertinust1math}
\usepackage{amsmath,amsthm,amssymb,mathtools}

%virgules
\usepackage{icomma}

% HEADER, ARRAY, ENUM, MULTIOCL
\usepackage{fancyhdr}
\usepackage{array}
\usepackage{multicol, enumitem}
\newcolumntype{P}[1]{>{\centering\arraybackslash}p{#1}}
\usepackage{stackengine}
\newcommand\xrowht[2][0]{\addstackgap[.5\dimexpr#2\relax]{\vphantom{#1}}}

% theorems
\theoremstyle{definition}

\newtheorem{theorem}{Théorème}
\newtheorem{corollaire}[theorem]{Corollaire}
\newtheorem{lemme}[theorem]{Lemme}
\newtheorem{proposition}[theorem]{Proposition}
\newtheorem{exercice}[theorem]{Exercice}
\newtheorem{exemple}[theorem]{Exemple}
\newtheorem{definition}[theorem]{Définition}
\newtheorem*{question}{Question}
\newtheorem*{preuve}{Preuve}
\newtheorem*{remarque}{Remarque}
\newtheorem*{strategie}{Stratégie}
\newtheorem*{methode}{Méthode}
\newtheorem*{notation}{Notation}
\newtheorem*{nomenclature}{Nomenclature}
\newtheorem{axiome}[theorem]{Axiome}
\newtheorem*{heuristique}{Heuristique}

\newtheorem*{definition*}{Définition}
\newtheorem*{lemme*}{Lemme}
\newtheorem*{proposition*}{Proposition}
\newtheorem*{theorem*}{Théorème}
\newtheorem*{corollaire*}{Corollaire}

%%%%%%%%%%%%%%%%%%%%%%%%%%%%
% MDFRAMED SURROUND
%%%%%%%%%%%%%%%%%%%%%%%%%%%%

\usepackage[framemethod=pgf]{mdframed}
% def
\mdfdefinestyle{definition}{
	hidealllines=true,
	leftline=true,
	linecolor=BLUE_E,
	linewidth=2pt,
	innertopmargin=-4pt,
	innerrightmargin=0,
	nobreak=true,
}
\surroundwithmdframed[
	style=definition,
]{definition}
\surroundwithmdframed[
	style=definition,
]{definition*}

% thm
\mdfdefinestyle{theorem}{
	linecolor=MAROON_C,
	linewidth=2pt,
	roundcorner=4pt,
	innertopmargin=-4pt,
	nobreak=true,
}
\surroundwithmdframed[
	style=theorem,
]{theorem}
\surroundwithmdframed[
	style=theorem,
]{theorem*}

% prop
\mdfdefinestyle{proposition}{
	linecolor=GREEN_E,
	linewidth=2pt,
	innertopmargin=-4pt,
	nobreak=true,
}
\surroundwithmdframed[
	style=proposition,
]{proposition}
\surroundwithmdframed[
	style=proposition,
]{proposition*}

% lemme
\mdfdefinestyle{lemme}{
	linecolor=TEAL_E,
	linewidth=1pt,
	innertopmargin=-4pt,
	nobreak=true,
}
\surroundwithmdframed[
	style=lemme,
]{lemme}
\surroundwithmdframed[
	style=lemme,
]{lemme*}

% corollaire
\mdfdefinestyle{corollaire}{
	linecolor=YELLOW_E,
	linewidth=2pt,
	roundcorner=4pt,
	innertopmargin=-4pt,
	nobreak=true,
}
\surroundwithmdframed[
	style=corollaire,
]{corollaire}
\surroundwithmdframed[
	style=corollaire,
]{corollaire*}

% exercices
\usepackage[answerdelayed, lastexercise]{exercise}
\usepackage{ifthen}
\renewcommand{\ExerciseHeader}{
	\tikz[baseline=(R.base)]\node[draw,rectangle, thick, inner sep=2pt](R) {\textbf{\theExercise.}};\!
	\ifnum\ExerciseDifficulty=0
	\else
		(\theExerciseDifficulty)
	\fi
}
\renewcommand{\DifficultyMarker}{$\star$}
\renewcommand{\AnswerHeader}{
	\tikz[baseline=(R.base)]\node[draw,rectangle, thick, inner sep=2pt](R) {\textbf{\theExercise.}};\!
}
\newcommand{\exe}[4]{
	\begin{Exercise}[title=#1, label=#3]
		\if\relax\detokenize\expandafter{\ExerciseTitle}\relax
		%\marginpar{[Bonus]}
		\else
		\marginpar{\mbox{[\ExerciseTitle]}}
		\fi
		#2
	\end{Exercise}
	\begin{Answer}[ref=#3]
		#4
	\end{Answer}
}
\newcommand{\exemulticols}[5]{
	\begin{multicols}{2}
	\begin{Exercise}[title=#1, label=#4]
		\if\relax\detokenize\expandafter{\ExerciseTitle}\relax
		%\marginnote{[Bonus]}
		\else
		\marginnote{\mbox{[\ExerciseTitle]\qquad}}
		\fi
		#2
	\end{Exercise}
	\columnbreak
		#3
	\end{multicols}
	\begin{Answer}[ref=#4]
		#5
	\end{Answer}
}

% date
\usepackage{advdate}

% plots
\usepackage{pgfplots}
\tikzset{
	every axis/.style = {clip=false, axis lines=center, axis line style=<->, xlabel={}, ylabel={}, grid=both, grid style = {opacity=.5}, domain=-2:2}
}

%subfigures
\usepackage{subcaption}

%hyperlink footnote
\usepackage{hyperref}

% tableaux var, signe
\usepackage{tkz-tab}

%wider tabulars
\def\arraystretch{2}
\setlength\tabcolsep{15pt}
\usepackage{makecell} %pour \thead dans tabular ex3 (aligner verticalement le coeff de proportionnalité)

% for striked out implies sign (\centernot\implies)
\usepackage{centernot}

%%%%%%%%%%%%%%%%%%%%%%%%%%%%%%
% SELF MADE COLORS
%%%%%%%%%%%%%%%%%%%%%%%%%%%%%%

%!TEX encoding = UTF8
%!TEX root = 0-notes.tex

%%%%%%%%%%%%%%%%%%%%%%%%%%%%%%
% SELF MADE COLORS
%%%%%%%%%%%%%%%%%%%%%%%%%%%%%%


\definecolor{myg}{RGB}{56, 140, 70}
\definecolor{myb}{RGB}{45, 111, 177}
\definecolor{myr}{RGB}{199, 68, 64}
\definecolor{mytheorembg}{HTML}{F2F2F9}
\definecolor{mytheoremfr}{HTML}{00007B}
\definecolor{mylenmabg}{HTML}{FFFAF8}
\definecolor{mylenmafr}{HTML}{983b0f}
\definecolor{mypropbg}{HTML}{f2fbfc}
\definecolor{mypropfr}{HTML}{191971}
\definecolor{myexamplebg}{HTML}{F2FBF8}
\definecolor{myexamplefr}{HTML}{88D6D1}
\definecolor{myexampleti}{HTML}{2A7F7F}
\definecolor{mydefinitbg}{HTML}{E5E5FF}
\definecolor{mydefinitfr}{HTML}{3F3FA3}
\definecolor{notesgreen}{RGB}{0,162,0}
\definecolor{myp}{RGB}{197, 92, 212}
\definecolor{mygr}{HTML}{2C3338}
\definecolor{myred}{RGB}{127,0,0}
\definecolor{myyellow}{RGB}{169,121,69}
\definecolor{myexercisebg}{HTML}{F2FBF8}
\definecolor{myexercisefg}{HTML}{88D6D1}
\definecolor{doc}{RGB}{0,60,110}

% manim colors because they're beautiful
% https://docs.manim.community/en/stable/reference/manim.utils.color.manim_colors.html

\definecolor{BLACK}{HTML}{000000}\definecolor{BLUE}{HTML}{58C4DD}\definecolor{BLUE_A}{HTML}{C7E9F1}\definecolor{BLUE_B}{HTML}{9CDCEB}\definecolor{BLUE_C}{HTML}{58C4DD}\definecolor{BLUE_D}{HTML}{29ABCA}\definecolor{BLUE_E}{HTML}{236B8E}\definecolor{DARKER_GRAY}{HTML}{222222}\definecolor{DARKER_GREY}{HTML}{222222}\definecolor{DARK_BLUE}{HTML}{236B8E}\definecolor{DARK_BROWN}{HTML}{8B4513}\definecolor{DARK_GRAY}{HTML}{444444}\definecolor{DARK_GREY}{HTML}{444444}\definecolor{GOLD}{HTML}{F0AC5F}\definecolor{GOLD_A}{HTML}{F7C797}\definecolor{GOLD_B}{HTML}{F9B775}\definecolor{GOLD_C}{HTML}{F0AC5F}\definecolor{GOLD_D}{HTML}{E1A158}\definecolor{GOLD_E}{HTML}{C78D46}\definecolor{GRAY}{HTML}{888888}\definecolor{GRAY_A}{HTML}{DDDDDD}\definecolor{GRAY_B}{HTML}{BBBBBB}\definecolor{GRAY_BROWN}{HTML}{736357}\definecolor{GRAY_C}{HTML}{888888}\definecolor{GRAY_D}{HTML}{444444}\definecolor{GRAY_E}{HTML}{222222}\definecolor{GREEN}{HTML}{83C167}\definecolor{GREEN_A}{HTML}{C9E2AE}\definecolor{GREEN_B}{HTML}{A6CF8C}\definecolor{GREEN_C}{HTML}{83C167}\definecolor{GREEN_D}{HTML}{77B05D}\definecolor{GREEN_E}{HTML}{699C52}\definecolor{GREY}{HTML}{888888}\definecolor{GREY_A}{HTML}{DDDDDD}\definecolor{GREY_B}{HTML}{BBBBBB}\definecolor{GREY_BROWN}{HTML}{736357}\definecolor{GREY_C}{HTML}{888888}\definecolor{GREY_D}{HTML}{444444}\definecolor{GREY_E}{HTML}{222222}\definecolor{LIGHTER_GRAY}{HTML}{DDDDDD}\definecolor{LIGHTER_GREY}{HTML}{DDDDDD}\definecolor{LIGHT_BROWN}{HTML}{CD853F}\definecolor{LIGHT_GRAY}{HTML}{BBBBBB}\definecolor{LIGHT_GREY}{HTML}{BBBBBB}\definecolor{LIGHT_PINK}{HTML}{DC75CD}\definecolor{LOGO_BLACK}{HTML}{343434}\definecolor{LOGO_BLUE}{HTML}{525893}\definecolor{LOGO_GREEN}{HTML}{87C2A5}\definecolor{LOGO_RED}{HTML}{E07A5F}\definecolor{LOGO_WHITE}{HTML}{ECE7E2}\definecolor{MAROON}{HTML}{C55F73}\definecolor{MAROON_A}{HTML}{ECABC1}\definecolor{MAROON_B}{HTML}{EC92AB}\definecolor{MAROON_C}{HTML}{C55F73}\definecolor{MAROON_D}{HTML}{A24D61}\definecolor{MAROON_E}{HTML}{94424F}\definecolor{ORANGE}{HTML}{FF862F}\definecolor{PINK}{HTML}{D147BD}\definecolor{PURE_BLUE}{HTML}{0000FF}\definecolor{PURE_GREEN}{HTML}{00FF00}\definecolor{PURE_RED}{HTML}{FF0000}\definecolor{PURPLE}{HTML}{9A72AC}\definecolor{PURPLE_A}{HTML}{CAA3E8}\definecolor{PURPLE_B}{HTML}{B189C6}\definecolor{PURPLE_C}{HTML}{9A72AC}\definecolor{PURPLE_D}{HTML}{715582}\definecolor{PURPLE_E}{HTML}{644172}\definecolor{RED}{HTML}{FC6255}\definecolor{RED_A}{HTML}{F7A1A3}\definecolor{RED_B}{HTML}{FF8080}\definecolor{RED_C}{HTML}{FC6255}\definecolor{RED_D}{HTML}{E65A4C}\definecolor{RED_E}{HTML}{CF5044}\definecolor{TEAL}{HTML}{5CD0B3}\definecolor{TEAL_A}{HTML}{ACEAD7}\definecolor{TEAL_B}{HTML}{76DDC0}\definecolor{TEAL_C}{HTML}{5CD0B3}\definecolor{TEAL_D}{HTML}{55C1A7}\definecolor{TEAL_E}{HTML}{49A88F}\definecolor{WHITE}{HTML}{FFFFFF}\definecolor{YELLOW}{HTML}{FFFF00}\definecolor{YELLOW_A}{HTML}{FFF1B6}\definecolor{YELLOW_B}{HTML}{FFEA94}\definecolor{YELLOW_C}{HTML}{FFFF00}\definecolor{YELLOW_D}{HTML}{F4D345}\definecolor{YELLOW_E}{HTML}{E8C11C}

%%%%%%%%%%%%%%%%%%%%%%%%%%%%
% LETTERFONTS
%%%%%%%%%%%%%%%%%%%%%%%%%%%%

%!TEX encoding = UTF8
%!TEX root = 0-notes.tex

%fonts
\usepackage{libertinus,libertinust1math}
\usepackage[T1]{fontenc}

% for calligraphic C, D, P (important to import this after the font)
\usepackage{calrsfs}
\newcommand{\D}{\mathcal{D}}
\newcommand{\C}{\mathcal{C}}
\renewcommand{\P}{\mathcal{P}}

% Schwartz
\renewcommand{\S}{\mathcal{S}} % \S est le signe paragraphe normalement

% corps
\newcommand{\R}{\mathbb{R}}
\newcommand{\Rnn}{\mathbb{R}^{2n}}
\newcommand{\Z}{\mathbb{Z}}
\newcommand{\N}{\mathbb{N}}
\newcommand{\Q}{\mathbb{Q}}
\newcommand{\E}{\mathbb{E}}
\newcommand{\DD}{\mathbb{D}}

% order notations
\DeclareRobustCommand{\O}{%
  \text{\usefont{OMS}{cmsy}{m}{n}O}%
}

% japanese bracket
\newcommand{\japb}[1]{\langle #1 \rangle}

% arrows over partial derivatives
\newcommand{\lp}{\overleftarrow{\partial}}
\newcommand{\rp}{\overrightarrow{\partial}}

% quantization
\newcommand{\h}{\hbar}
\newcommand{\Opht}{\textrm{Op}_{\h}^{t}}
\newcommand{\Op}[2][\hbar]{\textrm{Op}_{#1}^{#2}}

% omega functions
\newcommand{\omegap}[2][\rho_0]{\omega(\partial_{#1},\partial_{#2})}
\newcommand{\omegar}[2][\rho_0]{\omega(#1,#2)}

% space before semicolon
\mathcode`\;="303B

% for \Lightning
\usepackage{marvosym}

% for \warning
\newcommand{\warning}{{\fontencoding{U}\fontfamily{futs}\selectfont\char 49\relax}}

% Q(\sqrt(d)) field
\newcommand{\Qsqrt}[1]{\Q\bigl(\mspace{-3mu}\sqrt{#1}\bigr)}


%%%%%%%%%%%%%%%%%%%%%%%%%%%%
% MACROS
%%%%%%%%%%%%%%%%%%%%%%%%%%%%

%!TEX encoding = UTF8
%!TEX root = 0-notes.tex

%%%%%%%%%%%%%%%%%%%%%%%%%%%%%%
% SELF MADE COMMANDS
%%%%%%%%%%%%%%%%%%%%%%%%%%%%%%


%%
% tcolor environments VS clean environments
%%

\ifclean

\newcommand{\thm}[3]{\begin{theorem}[#1]\label{#3}#2\end{theorem}}
\newcommand{\cor}[3]{\begin{corollaire}[#1]\label{#3}#2\end{corollaire}}
\newcommand{\lem}[3]{\begin{lemme}[#1]\label{#3}#2\end{lemme}}
\newcommand{\mprop}[3]{\begin{proposition}[#1]\label{#3}#2\end{proposition}}
\newcommand{\ex}[3]{\begin{exemple}[#1]\label{#3}#2\end{exemple}}
%\newcommand{\exe}[3]{\begin{exercice}[#1]\label{#3}#2\end{exercice}}
\newcommand{\dfn}[3]{\begin{definition}[#1]\label{#3}#2\end{definition}}
\newcommand{\qs}[2]{\begin{question}[#1]#2\end{question}}
\newcommand{\pf}[2]{\begin{preuve}[#1]#2\end{preuve}}
\newcommand{\nt}[1]{\begin{remarque}#1\end{remarque}}
\newcommand{\str}[1]{\begin{strategie}#1\end{strategie}}
\newcommand{\mth}[1]{\begin{methode}#1\end{methode}}
\newcommand{\ax}[3]{\begin{axiome}[#1]\label{#3}#2\end{axiome}}

\newcommand{\exe}[4]{
	\begin{Exercise}[title=#1, label=#3]
		\marginpar{\mbox{\scriptsize(solution p.\pageref{\ExerciseLabel-Answer})}}
		#2
	\end{Exercise}
	\begin{Answer}[ref=#3]
		#4
	\end{Answer}
}

\newcommand{\exemulticols}[5]{
	\begin{multicols}{2}
	\begin{Exercise}[title=#1, label=#4]
		\marginnote{\mbox{\scriptsize(solution p.\pageref{\ExerciseLabel-Answer})}}
		#2
	\end{Exercise}
		#3
	\end{multicols}
	\begin{Answer}[ref=#4]
		#5
	\end{Answer}
}

\else

\newcommand{\thm}[3]{\begin{Theorem}[label=#3]{#1}{}#2\end{Theorem}}
\newcommand{\cor}[3]{\begin{Corollary}[label=#3]{#1}{}#2\end{Corollary}}
\newcommand{\lem}[3]{\begin{Lemma}[label=#3]{#1}{}#2\end{Lemma}}
\newcommand{\mprop}[3]{\begin{Prop}[label=#3]{#1}{}#2\end{Prop}}
\newcommand{\ex}[3]{\begin{Example}[label=#3]{#1}{}#2\end{Example}}
%\newcommand{\exe}[3]{\begin{Exe}[label=#3]{#1}{}#2\end{Exe}}
\newcommand{\dfn}[3]{\begin{Definition}[colbacktitle=red!75!black, label=#3]{#1}{}#2\end{Definition}}
\newcommand{\qs}[2]{\begin{MyQuestion}{#1}{}#2\end{MyQuestion}}
\newcommand{\pf}[2]{\begin{myproof}[#1]#2\end{myproof}}
\newcommand{\nt}[1]{\begin{Note}#1\end{Note}}
\newcommand{\str}[1]{\begin{Strategy}#1\end{Strategy}}
\newcommand{\mth}[1]{\begin{Methode}#1\end{Methode}}
\newcommand{\axiome}[3]{\begin{Axiome}[label=#3]{#1}{}#2\end{Axiome}}

\newcommand{\exe}[4]{
	\begin{Exe}[label=#3]{}{}#2\end{Exe}
	\begin{Answer}[ref=#3]
		#4
	\end{Answer}
}

\fi

\newcommand{\notations}[1]{\begin{notation}#1 \end{notation}}
\newcommand{\nomen}[1]{\begin{nomenclature}#1 \end{nomenclature}}
\newcommand{\heur}[1]{\begin{heuristique}#1\end{heuristique}}

%%

% deliminators
\DeclarePairedDelimiter{\abs}{\lvert}{\rvert}
%\DeclarePairedDelimiter{\norm}{\lVert}{\rVert}

\DeclarePairedDelimiter{\ceil}{\lceil}{\rceil}
\DeclarePairedDelimiter{\floor}{\lfloor}{\rfloor}
\DeclarePairedDelimiter{\round}{\lfloor}{\rceil}

\newsavebox\diffdbox
\newcommand{\slantedromand}{{\mathpalette\makesl{d}}}
\newcommand{\makesl}[2]{%
\begingroup
\sbox{\diffdbox}{$\mathsurround=0pt#1\mathrm{#2}$}%
\pdfsave
\pdfsetmatrix{1 0 0.2 1}%
\rlap{\usebox{\diffdbox}}%
\pdfrestore
\hskip\wd\diffdbox
\endgroup
}
\newcommand{\dd}[1][]{\ensuremath{\mathop{}\!\ifstrempty{#1}{%
\slantedromand\@ifnextchar^{\hspace{0.2ex}}{\hspace{0.1ex}}}%
{\slantedromand\hspace{0.2ex}^{#1}}}}
\ProvideDocumentCommand\dv{o m g}{%
  \ensuremath{%
    \IfValueTF{#3}{%
      \IfNoValueTF{#1}{%
        \frac{\dd #2}{\dd #3}%
      }{%
        \frac{\dd^{#1} #2}{\dd #3^{#1}}%
      }%
    }{%
      \IfNoValueTF{#1}{%
        \frac{\dd}{\dd #2}%
      }{%
        \frac{\dd^{#1}}{\dd #2^{#1}}%
      }%
    }%
  }%
}
\providecommand*{\pdv}[3][]{\frac{\partial^{#1}#2}{\partial#3^{#1}}}
%  - others
\DeclareMathOperator{\Lap}{\mathcal{L}}
\DeclareMathOperator{\Var}{Var} % variance
\DeclareMathOperator{\Cov}{Cov} % covariance

% Since the amsthm package isn't loaded

% I prefer the slanted \leq
\let\oldleq\leq % save them in case they're every wanted
\let\oldgeq\geq
\renewcommand{\leq}{\leqslant}
\renewcommand{\geq}{\geqslant}

% tel que
\newcommand{\tqs}{\text{ tels que }}
\newcommand{\tq}{\text{ tq. }}
\newcommand{\et}{\text{ et }}
\newcommand{\ou}{\text{ ou }}
\newcommand{\pourtout}{\text{ pour tout }}
\newcommand{\sct}{\text{ sachant }}

% Lois
\newcommand{\Bern}{\text{Bern}}
\newcommand{\Binom}{\text{Binom}}

% ensemble avec bigl et bigr
\newcommand{\bigset}[1]{\bigl\{ #1 \bigr\}}
\newcommand{\Bigset}[1]{\Bigl\{ #1 \Bigr\}}
\newcommand{\bigpar}[1]{\bigl( #1 \bigr)}
\newcommand{\Bigpar}[1]{\Bigl( #1 \Bigr)}

% PLUS INFTY AND MINUS INFTY WITH NO SPACE
\newcommand{\pinfty}{{+}\infty}
\newcommand{\minfty}{{-}\infty}

% vecteur flèche
\renewcommand{\vec}[1]{\overrightarrow{#1}}

% vecteur pmatrix
\newcommand{\pvec}[2]{\begin{pmatrix} #1 \\ #2 \end{pmatrix}}

% vecteur norme
\newcommand{\norm}[1]{\left\Vert #1 \right\Vert}

% point plan
\newcommand{\point}[3]{
	#1\left(#2 ; #3 \right)
}

% \smash avant \underline pour coller la ligne au mot
\let\oldunderline\underline
\renewcommand{\underline}[1]{\oldunderline{\smash{#1}}}

% emph + index
\newcommand{\emphindex}[1]{\emph{#1}\index{#1}}

% tableau croisé
\newcommand{\tableaucroise}[4]{
\begin{tabular}{|c|c|c|c|}
	\cline{2-4}
	\multicolumn{1}{c|}{} & #1 \\ \hline
	#2 \\ \hline
	#3  \\ \hline
	#4  \\ \hline
\end{tabular}
}

% python minted
\newcommand{\python}[1]{
\inputminted[
		linenos,
		gobble=0,
		breaklines=true, % otherwise it breaks for no apparent reason?
		breakafter=,,
		fontsize=\small,
		numbersep=8pt,
		tabsize=4, % tab ident = 4 spaces
		fontfamily=courier, %important pour les signes <, >
]{python}{python/#1.py}
}


\AdvanceDate[1]

\begin{document}
\pagestyle{fancy}
\fancyhead[L]{Première spécifique}
\fancyhead[C]{\textbf{Fonctions affines}}
\fancyhead[R]{\today}

\exe{}{
	Pour chaque fonction affine sur $\R$ suivante, déterminer son coefficient directeur $a$ et son ordonnée à l'origine $b$.
	\begin{multicols}{4}
	\begin{enumerate}
		\item $f(x) = 2x + 1$
		\item $f(x) = 1 + 2x$
		\item $f(x) = - x$
		\item $f(x) = -42$
		\item $f(x) = 10x + 2$
		\item $f(x) = 2 + 10x$
		\item $f(x) = 1 - x$
		\item $f(x) = 0$
	\end{enumerate}
	\end{multicols}
}{exe:parametres-affines}{
	\begin{multicols}{4}
	\begin{enumerate}
		\item $a=2, b=1$
		\item $a=2, b=1$
		\item $a=-1, b=0$
		\item $a=0, b=-42$
		\item $a=10, b=2$
		\item $a=10, b=2$
		\item $a=-1, b=1$
		\item $a=0, b=0$
	\end{enumerate}
	\end{multicols}
}

\exe{}{
	Un vase droit, de base carrée de $5$cm de côté, a une hauteur de $20$cm.
	On y dépose une couche initial de sable de $2$cm
	
	\begin{enumerate}
		\item Quel est le volume du vase ?
		\item Quel est le volume de la couche initiale de sable ?
	\end{enumerate}
	
	On note $x$ la hauteur supplémentaire de sable que l'on rajoute.
	\begin{enumerate}[resume]
		\item À quel intervalle appartient $x$ ?
		\item Exprimer, en fonction de $x$, le volume total $V(x)$ de sable dans le vase.
		\item Calculer $V(0), V(3),$ et $V(18)$.
		\item Quelle hauteur de sable a été rajoutée si le volume total est de $335 \text{cm}^3$ ?
		\item Quelle hauteur minimale de sable a été rajoutée si le volume total dépasse $440 \text{cm}^3$ ?
	\end{enumerate}
}{exe:volume-affine}{
	\begin{enumerate}
		\item On fait base $\times$ hauteur, d'où $5^2 \cdot 20 = 500 cm^3$.
		\item Idem, la hauteur maintenant étant de $2$cm : $5^2 \cdot 2 = 50cm^3$.
		\item La valeur minimale de $x$ est $0$ (hauteur non nulle), et sa valeur maximale est $18$, car il y a déjà une couche de $2$cm présente.
		
		$x$ peut donc prendre toutes les valeurs entre $0$ et $18$ incluses, d'où $x \in [0;18]$.
		\item La base reste d'aire $5^2 = 25cm^2$, et la hauteur du sable est de $2+x$cm, car il y a déjà une couche de $2$cm présente.
		
		D'où $V(x) = 25\cdot(2+x) = 25x + 50$.
		$V$ est affine.
		\item $V(0) = 50$ et $V(18) = 500$, ce qui est cohérent avec les valeurs trouvées aux deux premières questions de l'exercice.
		
			$V(3) = 25\cdot3 + 50 = 125$.
		\item On résoud pour $x$ l'équation $V(x) = 335$.
			\begin{align*}
				V(x) &= 335 \\
				25x + 50 &= 335 \\
				25x &= 285 \\
				x &= \dfrac{285}{25} = \dfrac{570}{50} = \dfrac{1140}{100} = 11,4.
			\end{align*}
		\item On résoud pour $x$ l'inéquation $V(x) \geq 440$.
			\begin{align*}
				V(x) &\geq 440 \\
				25x + 50 &\geq 440 \\
				25x &\geq 390 \\
				x &\geq 15,6
			\end{align*}
			La hauteur minimale ajoutée est donc $15,6$cm, qui est bien dans l'intervalle de définition de $x$.
	\end{enumerate}

}

\exe{}{
	\begin{multicols}{2}
		\begin{center}
		\begin{tikzpicture}[scale=0.8]
		% real line
		\draw[black, thick] (0,0) -- (7,0);
		\draw[black,thick] (7,0) -- (7,7);
		\draw[black,thick] (7,7) -- (2,7);
		\draw[black,thick] (2,7)--(2,2);
		\draw[black, thick](2,2)--(0,2);
		\draw[black,thick](0,2)--(0,0);
		
		\draw[black,thick, dotted](2,0)--(2,2);
		
		\draw[<->, thick] (2,-.2) -- (7,-.2) node [midway, below] {$5$} ;
		\draw[<->, thick] (0,-.2) -- (2,-.2) node [midway, below] {$x$} ;
		\draw[<->, thick] (-.2,0) -- (-.2,2) node [midway, left] {$x$} ;
		\draw[<->, thick] (7.2,0) -- (7.2,7) node [midway, right] {$12$} ;
	\end{tikzpicture}
	\end{center}
	
	Considérons la figure ci-contre. La longueur du côté du carré de gauche doit rester inférieure à la longueur du rectangle de droite.
	\begin{enumerate}
		\item À quel intervalle appartient $x$ ?
		\item Exprimer le périmètre de la figure en fonction de $x$. Est-ce une fonction affine ? Si oui, donner son coefficient directeur de son ordonnée à l'origine.
		\item Donner l'ensemble des valeurs de $x$ pour lesquelles le périmètre est supérieur ou égal à $50$.
	\end{enumerate}
	
	\end{multicols}
}{exe:perimetre-affine}{
	
	\begin{enumerate}
		\item En considérant les valeurs minimales et maximales de $x$, on trouve $x \in [0;12]$.
		\item Le périmètre est la longueur du pourtour, soit $12 + 5 + 5 + 12 + 2x = 34 + 2x$.
		
		La fonction $f(x) = 2x+34$ est affine, de coefficient directeur $2$ et d'ordonnée à l'origine $34$.
		\item On résoud pour $x$ l'inéquation $f(x) \geq 50$.
			\begin{align*}
				f(x) &\geq 50 \\
				34+2x &\geq 50 \\
				2x &\geq 16 \\
				x &\geq 8
			\end{align*}
		L'ensemble des valeurs est donc donné par $[8;12]$, car on a toujours $x \leq 12$ d'après la question 1.
	\end{enumerate}

}

\exe{}{
    Achille dispute une course avec une tortue. On suppose que les deux participants avancent à vitesse constante. Posons $v = 60$ mètres par minute la vitesse d'Achille.
    La tortue, elle, avance 10 fois moins vite qu’Achille, soit à $6$ mètres par minute. 
    Achille décide donc de lui laisser généreusement 10 minutes d’avance.

    
	\begin{enumerate}
        \item Calculer la distance parcourue par la tortue après les 10 minutes laissées gracieusement par Achille.
        \item \`A la minute $10+t$, on appelle $A(t)$ et $T(t)$ les distances parcourues respectivement par Achille et la tortue, c'est-à-dire après les $10$ premières minutes accordées par Achille.
        Montrer que
            \begin{align*}
                A(t) = 60 t, && \text{ et } && T(t) = 60 + 6t.
            \end{align*}
        \item Combien de temps Achille met-il pour rattraper la tortue ?
        \item Quelle est la distance parcourue par les deux participants au moment où ils se croisent ?
        \item($\star$) Supposons qu'on ne connaisse pas la vitesse exacte d'Achille mais qu'on sache tout de même que la tortue avance 10 fois moins vite que celui-ci.
        Montrer qu'Achille rattrape la tortue en $t = \frac{10}{9}$ minutes, indépendamment de la vitesse d'Achille.
	\end{enumerate}
	
}{exe:achille}{
	\begin{enumerate}
		\item (6 mètres par minute) $\times$ (10 minutes) = 60 mètres
        	\item Achille court pendant $t$ minutes, et il parcourt donc $60t$ mètres.
        	La tortue court pendant $t$ minutes et parcourt $6t$ mètres, auxquels ont ajoute les 60 mètres d'avance.
        	\item On pose l'équation, on substitue les expressions, et on résoud pour $t$.
        		\begin{align*}
        			A(t) &= T(t) \\
        			60t &= 60 + 6t \\
        			54t &= 60 \\
        			t &= \dfrac{60}{54} = \dfrac{10}9
        		\end{align*}
		Achille met donc $\frac{10}9 \approx 1,1111$ minutes à ratrapper la tortue.
        	\item On calcule soit $A(\frac{10}9)$, soit $T(\frac{10}9)$, les deux ayant la même valeur car $t$ a été trouvé vérifiant $A(t) = T(t)$.
        		\[ A\left(\dfrac{10}9 \right) = 60 \times \dfrac{10}9 = \dfrac{600}{9} = \dfrac{200}3. \]
		Achillet et la tortue se croisent donc après avoir parcouru chacun $\frac{200}3 \approx 66,666$ mètres.
        	\item($\star$) 
        	Posons $v$ la vitesse d'Achille, inconnue a priori.
        	Dans ce cas, on a $A(t) = vt$, et $T(t)= 10\frac{v}{10} + \frac{v}{10}t$.
        	On pose l'équation et on résoud pour $t$.
        		\begin{align*}
        			A(t) &= T(t) \\
        			vt &= 10\dfrac{v}{10} + \dfrac{v}{10}t \\
        			vt &= v\Bigl( 1 + \dfrac1{10}t\Bigr) \\
        			t &= 1 + \dfrac1{10}t \\
        			\dfrac9{10} t &= 1 \\
        			t &= \dfrac{10}9
        		\end{align*}
        		
	\end{enumerate}
}


\exe{}{
	Déterminer les paramètres des fonction affines $f, g, h$ dont les courbes sont représentées ci-dessous.

	\begin{center}
		\begin{tikzpicture}[>=stealth, scale=1.5, x=1cm, y=4cm]
		\begin{axis}[xmin = -10, xmax=10, ymin=-10, ymax=10, axis x line=middle, axis y line=middle, axis line style=<->, xlabel={}, ylabel={}, xtick = {-10, -8, ..., 8, 10}, ytick = {-10, -8, ..., 8, 10}, grid=both]
		
			\addplot[RED_E, thick, domain =-9:9, samples=2] {-x}  node[above=6pt] {$(\mathcal{C}_f)$};
			\addplot[RED_E, thick, dotted, domain =-10:-9, samples=2] {-x} ;
			\addplot[RED_E, thick, dotted, domain =9:10, samples=2] {-x};
		
		
			\addplot[GREEN_E, thick, domain =-9:9, samples=2] {x/2+1}  node[below=6pt] {$(\mathcal{C}_g)$};
			\addplot[GREEN_E, thick, dotted, domain =-10:-9, samples=2] {x/2+1} ;
			\addplot[GREEN_E, thick, dotted, domain =9:10, samples=2] {x/2+1};
		
		
			\addplot[black, thick, domain =-9:9, samples=2] {7}  node[above=10pt, left] {$(\mathcal{C}_h)$};
			\addplot[black, thick, dotted, domain =-10:-9, samples=2] {7} ;
			\addplot[black, thick, dotted, domain =9:10, samples=2] {7};
		
			
		\end{axis}
	\end{tikzpicture}
	\end{center}
}{exe:lecture-graphique-affine}{
	Pour la fonction $f(x) = ax+b (x\in\R)$, on peut d'abord trouver $b$ comme l'ordonnée à l'origine (l'origine du point lui appartenant et d'abscisse nulle).
	D'où $b=0$ car $(0;0) \in \C_f$.
	
	Pour $a$, on peut utiliser la formule $\dfrac{y_B - y_A}{x_B - x_A}$, avec $A$ et $B$ deux points de la droite.
	
	Pour $A(-4;4)$ et $B(2;-2)$, on trouve,	
	\begin{align*}
		a = \dfrac{4 - (-2)}{-4 - 2} = -1.
	\end{align*}
	
	\hrule\vspace{10pt}
	
	Pour la fonction $g(x) = ax+b (x\in\R)$, on peut faire idem en choisissant de bons points pour faciliter les calculs.
	Par exemple, $A(0;1)$ et $B(-2;0)$ sont pratiques car il y a beaucoup de zéros qui annulent les expressions.
	\begin{align*}
		a = \dfrac{1-0}{0-(-2)} = \dfrac12, && b = 1.
	\end{align*}
	
	\hrule\vspace{10pt}
	
	Pour la fonction $h(x) = ax+b (x\in\R)$, la fonction est constante, donc $a=0$.
	N'importe quel point lui appartenant a pour ordonnée $b=7$, ce qui conclut.

}

L'exercice suivant est tiré du premier sujet zéro de l'examen de 1ère spécifique.

\exe{}{
	Sur un axe gradué en mètres, on organise une course entre une tortue et un escargot.
	\begin{itemize}
		\item
		La tortue part du point d'abscisse $x=0$, et se déplace vers la droite à une vitesse de 2 mètres par minute.
	
		\item
		L'escargot part du point d'abscisse $x=12$, et se déplace vers la droite à une vitesse de 50 centimètres par minute.
	
		\item
		Les deux concurrents partent en même temps.
	\end{itemize}
	
	À quel endroit la tortue rattrapera-t-elle l'escargot ?
	
	\begin{center}
	\includegraphics[scale=.6]{zero1.png}
	\end{center}
	
	\emph{Toute trace de recherche, même infructueuse, sera prise en compte.}
}{exe:zero1-affine}{
	Il s'agit ici de calculer les distances parcourues respectivement par la tortue et par l'escargot à un temps $t$ variable (donné en minutes).
	
	Pour l'escargot, celui-ci part de 12 mètres, et avance à $0,5$ mètres par minute. 
	Sa distance parcourue au temps $t$ est donc de $12 + 0,5 t$.
	
	Pour la tortue, on obtient similairement $2t$.
	
	On pose l'équation et on résoud.
		\begin{align*}
			2t &= 12 + 0,5t \\
			1,5 t &= 12 \\
			t &= \dfrac{12}{1,5} = \dfrac{24}{3} = 8
		\end{align*}
	Au temps $t=8$ minutes, la tortue et l'escargot auront parcourus la même distance.
	Cette distance est égale à $2\times8 = 16$, ou $12 + 0,5\times8 = 16$ mètres.
	
	Les deux participants se croiseront donc en $x=16$.
}

%%%%%%%%%%%%

\newpage
\fancyhead[C]{\textbf{Solutions}}
\shipoutAnswer

\end{document}
