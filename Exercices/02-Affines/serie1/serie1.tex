\documentclass[a4paper, 12pt]{extarticle}

\usepackage[utf8x]{inputenc}
%fonts
\usepackage{libertinus,libertinust1math}
\usepackage{amsmath,amsthm,amssymb,mathtools}

% SOLUTION SWITCH

\ifsolutions
	\newcommand{\exe}[2]{
		\begin{ex} #1  \end{ex}
		\begin{sol} #2 \end{sol}
	}
\else
	\newcommand{\exe}[2]{
		\begin{ex} #1  \end{ex}
	}
	
\fi


\usepackage[french]{babel}
\usepackage[
a4paper,
margin=2cm,
nomarginpar,% We don't want any margin paragraphs
]{geometry}

% HEADER, ARRAY, ENUM, MULTIOCL
\usepackage{fancyhdr}
\usepackage{array}
\usepackage{multicol, enumitem}
\newcolumntype{P}[1]{>{\centering\arraybackslash}p{#1}}
\usepackage{stackengine}
\newcommand\xrowht[2][0]{\addstackgap[.5\dimexpr#2\relax]{\vphantom{#1}}}

% theorems

\theoremstyle{theorem}
\newtheorem{thm}{Théorème}
\theoremstyle{plain}
\newtheorem*{sol}{Solution}
\theoremstyle{definition}
\newtheorem{ex}{Exercice}
\newtheorem{dfn}{Définition}
\newtheorem*{dfn*}{Définition}


%couleurs
\usepackage{tcolorbox}
\definecolor{myg}{RGB}{56, 140, 70}
\definecolor{myb}{RGB}{45, 111, 177}
\definecolor{myr}{RGB}{199, 68, 64}
\definecolor{mygr}{HTML}{2C3338}


\tcbuselibrary{theorems,skins,hooks}
\newcounter{commonbox}
\makeatletter
\newtcbtheorem[use counter=commonbox]{theorem}{Théorème }%
{
	enhanced,
	colback=white,
	colframe=mygr,
	attach boxed title to top left={yshift*=-\tcboxedtitleheight},
	fonttitle=\bfseries,
	title={#2},
	boxed title size=title,
	boxed title style={%
			sharp corners,
			rounded corners=northwest,
			colback=tcbcolframe,
			boxrule=0pt,
		},
	underlay boxed title={%
			\path[fill=tcbcolframe] (title.south west)--(title.south east)
			to[out=0, in=180] ([xshift=5mm]title.east)--
			(title.center-|frame.east)
			[rounded corners=\kvtcb@arc] |-
			(frame.north) -| cycle;
		},
	#1
}{th}
\newtcbtheorem[use counter=commonbox]{rappel}{Rappel }%
{
	enhanced,
	colback=white,
	colframe=mygr,
	attach boxed title to top left={yshift*=-\tcboxedtitleheight},
	fonttitle=\bfseries,
	title={#2},
	boxed title size=title,
	boxed title style={%
			sharp corners,
			rounded corners=northwest,
			colback=tcbcolframe,
			boxrule=0pt,
		},
	underlay boxed title={%
			\path[fill=tcbcolframe] (title.south west)--(title.south east)
			to[out=0, in=180] ([xshift=5mm]title.east)--
			(title.center-|frame.east)
			[rounded corners=\kvtcb@arc] |-
			(frame.north) -| cycle;
		},
	#1
}{th}
\newtcbtheorem[use counter=commonbox]{strategie}{Stratégie }%
{
	enhanced,
	colback=white,
	colframe=mygr,
	attach boxed title to top left={yshift*=-\tcboxedtitleheight},
	fonttitle=\bfseries,
	title={#2},
	boxed title size=title,
	boxed title style={%
			sharp corners,
			rounded corners=northwest,
			colback=tcbcolframe,
			boxrule=0pt,
		},
	underlay boxed title={%
			\path[fill=tcbcolframe] (title.south west)--(title.south east)
			to[out=0, in=180] ([xshift=5mm]title.east)--
			(title.center-|frame.east)
			[rounded corners=\kvtcb@arc] |-
			(frame.north) -| cycle;
		},
	#1
}{th}
\newtcbtheorem[use counter=commonbox]{outil}{Outil }%
{
	enhanced,
	colback=white,
	colframe=mygr,
	attach boxed title to top left={yshift*=-\tcboxedtitleheight},
	fonttitle=\bfseries,
	title={#2},
	boxed title size=title,
	boxed title style={%
			sharp corners,
			rounded corners=northwest,
			colback=tcbcolframe,
			boxrule=0pt,
		},
	underlay boxed title={%
			\path[fill=tcbcolframe] (title.south west)--(title.south east)
			to[out=0, in=180] ([xshift=5mm]title.east)--
			(title.center-|frame.east)
			[rounded corners=\kvtcb@arc] |-
			(frame.north) -| cycle;
		},
	#1
}{th}
\newtcbtheorem[use counter=commonbox]{but}{Buts du chapitre }%
{
	enhanced,
	colback=white,
	colframe=mygr,
	attach boxed title to top left={yshift*=-\tcboxedtitleheight},
	fonttitle=\bfseries,
	title={#2},
	boxed title size=title,
	boxed title style={%
			sharp corners,
			rounded corners=northwest,
			colback=tcbcolframe,
			boxrule=0pt,
		},
	underlay boxed title={%
			\path[fill=tcbcolframe] (title.south west)--(title.south east)
			to[out=0, in=180] ([xshift=5mm]title.east)--
			(title.center-|frame.east)
			[rounded corners=\kvtcb@arc] |-
			(frame.north) -| cycle;
		},
	#1
}{th}
\newtcbtheorem[use counter=commonbox]{propriete}{Propriété }%
{
	enhanced,
	colback=white,
	colframe=mygr,
	attach boxed title to top left={yshift*=-\tcboxedtitleheight},
	fonttitle=\bfseries,
	title={#2},
	boxed title size=title,
	boxed title style={%
			sharp corners,
			rounded corners=northwest,
			colback=tcbcolframe,
			boxrule=0pt,
		},
	underlay boxed title={%
			\path[fill=tcbcolframe] (title.south west)--(title.south east)
			to[out=0, in=180] ([xshift=5mm]title.east)--
			(title.center-|frame.east)
			[rounded corners=\kvtcb@arc] |-
			(frame.north) -| cycle;
		},
	#1
}{th}
\newtcbtheorem[number within=commonbox]{definition}{Définition }%
{
	enhanced,
	colback=white,
	colframe=mygr,
	attach boxed title to top left={yshift*=-\tcboxedtitleheight},
	fonttitle=\bfseries,
	title={#2},
	boxed title size=title,
	boxed title style={%
			sharp corners,
			rounded corners=northwest,
			colback=tcbcolframe,
			boxrule=0pt,
		},
	underlay boxed title={%
			\path[fill=tcbcolframe] (title.south west)--(title.south east)
			to[out=0, in=180] ([xshift=5mm]title.east)--
			(title.center-|frame.east)
			[rounded corners=\kvtcb@arc] |-
			(frame.north) -| cycle;
		},
	#1
}{th}
\newtcbtheorem[number within=commonbox]{exemples}{Exemples }%
{
	enhanced,
	colback=white,
	colframe=mygr,
	attach boxed title to top left={yshift*=-\tcboxedtitleheight},
	fonttitle=\bfseries,
	title={#2},
	boxed title size=title,
	boxed title style={%
			sharp corners,
			rounded corners=northwest,
			colback=tcbcolframe,
			boxrule=0pt,
		},
	underlay boxed title={%
			\path[fill=tcbcolframe] (title.south west)--(title.south east)
			to[out=0, in=180] ([xshift=5mm]title.east)--
			(title.center-|frame.east)
			[rounded corners=\kvtcb@arc] |-
			(frame.north) -| cycle;
		},
	#1
}{th}
\newtcbtheorem[number within=commonbox]{exemple}{Exemple }%
{
	enhanced,
	colback=white,
	colframe=mygr,
	attach boxed title to top left={yshift*=-\tcboxedtitleheight},
	fonttitle=\bfseries,
	title={#2},
	boxed title size=title,
	boxed title style={%
			sharp corners,
			rounded corners=northwest,
			colback=tcbcolframe,
			boxrule=0pt,
		},
	underlay boxed title={%
			\path[fill=tcbcolframe] (title.south west)--(title.south east)
			to[out=0, in=180] ([xshift=5mm]title.east)--
			(title.center-|frame.east)
			[rounded corners=\kvtcb@arc] |-
			(frame.north) -| cycle;
		},
	#1
}{th}
\newtcbtheorem[number within=commonbox]{questions}{Questions guidantes }%
{
	enhanced,
	colback=white,
	colframe=mygr,
	attach boxed title to top left={yshift*=-\tcboxedtitleheight},
	fonttitle=\bfseries,
	title={#2},
	boxed title size=title,
	boxed title style={%
			sharp corners,
			rounded corners=northwest,
			colback=tcbcolframe,
			boxrule=0pt,
		},
	underlay boxed title={%
			\path[fill=tcbcolframe] (title.south west)--(title.south east)
			to[out=0, in=180] ([xshift=5mm]title.east)--
			(title.center-|frame.east)
			[rounded corners=\kvtcb@arc] |-
			(frame.north) -| cycle;
		},
	#1
}{th}
\makeatother

% corps
\newcommand{\R}{\mathbb{R}}
\newcommand{\Rnn}{\mathbb{R}^{2n}}
\newcommand{\Z}{\mathbb{Z}}
\newcommand{\N}{\mathbb{N}}
\newcommand{\Q}{\mathbb{Q}}

% domain
\newcommand{\D}{\mathcal{D}}
% for calligraphic C
\usepackage{calrsfs}
\newcommand{\C}{\mathcal{C}}

% date
\usepackage{advdate}

% ensembles tq. 
\newcommand{\xRtq}[1]{
	$\left\{ x \in \R \text{ tq. } #1 \right\}$
}

% vabs
\newcommand{\vabs}[1]{
	\left| #1 \right|
}

%pinfty minfty
\newcommand{\pinfty}{{+}\infty}
\newcommand{\minfty}{{-}\infty}

% plots
\usepackage{pgfplots}

%virgules
\usepackage{icomma}
\pgfplotsset{/pgf/number format/use comma}

%subfigures
\usepackage{subcaption}

%hyperlink footnote
\usepackage{hyperref}

%wider tabulars
\def\arraystretch{2}
\setlength\tabcolsep{15pt}

% tableaux var, signe
\usepackage{tkz-tab}

\AdvanceDate[1]

\begin{document}
\pagestyle{fancy}
\fancyhead[L]{Première spécifique}
\fancyhead[C]{\textbf{Fonctions affines}}
\fancyhead[R]{\today}

\exe{}{
	Pour chaque fonction affine sur $\R$ suivante, déterminer son coefficient directeur $a$ et son ordonnée à l'origine $b$.
	\begin{multicols}{4}
	\begin{enumerate}
		\item $f(x) = 2x + 1$
		\item $f(x) = 1 + 2x$
		\item $f(x) = - x$
		\item $f(x) = -42$
		\item $f(x) = 10x + 2$
		\item $f(x) = 2 + 10x$
		\item $f(x) = 1 - x$
		\item $f(x) = 0$
	\end{enumerate}
	\end{multicols}
}{exe:parametres-affines}{
	TODO
}

\exe{}{
	Un vase droit, de base carrée de $5$cm de côté, a une hauteur de $20$cm.
	On y dépose une couche initial de sable de $2$cm
	
	\begin{enumerate}
		\item Quel est le volume du vase ?
		\item Quel est le volume de la couche initiale de sable ?
	\end{enumerate}
	
	On note $x$ la hauteur supplémentaire de sable que l'on rajoute.
	\begin{enumerate}[resume]
		\item À quel intervalle appartient $x$ ?
		\item Exprimer, en fonction de $x$, le volume total $V(x)$ de sable dans le vase.
		\item Calculer $V(0), V(3),$ et $V(18)$.
		\item Quelle hauteur de sable a été rajoutée si le volume total est de $335 \text{cm}^3$ ?
		\item Quelle hauteur minimale de sable a été rajoutée si le volume total dépasse $440 \text{cm}^3$ ?
	\end{enumerate}
}{exe:volume-affine}{
	\begin{enumerate}
		\item On fait base $\times$ hauteur, d'où $5^2 \cdot 20 = 500 cm^3$.
		\item Idem, la hauteur maintenant étant de $2$cm : $5^2 \cdot 2 = 50cm^3$.
		\item La valeur minimale de $x$ est $0$ (hauteur non nulle), et sa valeur maximale est $18$, car il y a déjà une couche de $2$cm présente.
		
		$x$ peut donc prendre toutes les valeurs entre $0$ et $18$ incluses, d'où $x \in [0;18]$.
		\item La base reste d'aire $5^2 = 25cm^2$, et la hauteur du sable est de $2+x$cm, car il y a déjà une couche de $2$cm présente.
		
		D'où $V(x) = 25\cdot(2+x) = 25x + 50$.
		$V$ est affine.
		\item $V(0) = 50$ et $V(18) = 500$, ce qui est cohérent avec les valeurs trouvées aux deux premières questions de l'exercice.
		
			$V(3) = 25\cdot3 + 50 = 125$.
		\item On résoud pour $x$ l'équation $V(x) = 335$.
			\begin{align*}
				V(x) &= 335 \\
				25x + 50 &= 335 \\
				25x &= 285 \\
				x &= \dfrac{285}{25} = \dfrac{570}{50} = \dfrac{1140}{100} = 11,4.
			\end{align*}
		\item On résoud pour $x$ l'inéquation $V(x) \geq 440$.
			\begin{align*}
				V(x) &\geq 440 \\
				25x + 50 &\geq 440 \\
				25x &\geq 390 \\
				x &\geq 15,6
			\end{align*}
			La hauteur minimale ajoutée est donc $15,6$cm, qui est bien dans l'intervalle de définition de $x$.
	\end{enumerate}

}

\exe{}{
	\begin{multicols}{2}
		\begin{center}
		\begin{tikzpicture}[scale=0.8]
		% real line
		\draw[black, thick] (0,0) -- (7,0);
		\draw[black,thick] (7,0) -- (7,7);
		\draw[black,thick] (7,7) -- (2,7);
		\draw[black,thick] (2,7)--(2,2);
		\draw[black, thick](2,2)--(0,2);
		\draw[black,thick](0,2)--(0,0);
		
		\draw[black,thick, dotted](2,0)--(2,2);
		
		\draw[<->, thick] (2,-.2) -- (7,-.2) node [midway, below] {$5$} ;
		\draw[<->, thick] (0,-.2) -- (2,-.2) node [midway, below] {$x$} ;
		\draw[<->, thick] (-.2,0) -- (-.2,2) node [midway, left] {$x$} ;
		\draw[<->, thick] (7.2,0) -- (7.2,7) node [midway, right] {$12$} ;
	\end{tikzpicture}
	\end{center}
	
	Considérons la figure ci-contre. La longueur du côté du carré de gauche doit rester inférieure à la longueur du rectangle de droite.
	\begin{enumerate}
		\item À quel intervalle appartient $x$ ?
		\item Exprimer le périmètre de la figure en fonction de $x$. Est-ce une fonction affine ? Si oui, donner son coefficient directeur de son ordonnée à l'origine.
		\item Donner l'ensemble des valeurs de $x$ pour lesquelles le périmètre est supérieur ou égal à $50$.
	\end{enumerate}
	
	\end{multicols}
}{exe:perimetre-affine}{
	
	\begin{enumerate}
		\item En considérant les valeurs minimales et maximales de $x$, on trouve $x \in [0;12]$.
		\item Le périmètre est la longueur du pourtour, soit $12 + 5 + 5 + 12 + 2x = 34 + 2x$.
		
		La fonction $f(x) = 2x+34$ est affine, de coefficient directeur $2$ et d'ordonnée à l'origine $34$.
		\item On résoud pour $x$ l'inéquation $f(x) \geq 50$.
			\begin{align*}
				f(x) &\geq 50 \\
				34+2x &\geq 50 \\
				2x &\geq 16 \\
				x &\geq 8
			\end{align*}
		L'ensemble des valeurs est donc donné par $[8;12]$, car on a toujours $x \leq 12$ d'après la question 1.
	\end{enumerate}

}

\exe{}{
    Achille dispute une course avec une tortue. On suppose que les deux participants avancent à vitesse constante. Posons $v = 60$ mètres par minute la vitesse d'Achille.
    La tortue, elle, avance 10 fois moins vite qu’Achille, soit à $6$ mètres par minute. 
    Achille décide donc de lui laisser généreusement 10 minutes d’avance.

    
	\begin{enumerate}
        \item Calculer la distance parcourue par la tortue après les 10 minutes laissées gracieusement par Achille.
        \item \`A la minute $10+t$, on appelle $A(t)$ et $T(t)$ les distances parcourues respectivement par Achille et la tortue, c'est-à-dire après les $10$ premières minutes accordées par Achille.
        Montrer que
            \begin{align*}
                A(t) = 60 t, && \text{ et } && T(t) = 60 + 6t.
            \end{align*}
        \item Combien de temps Achille met-il pour rattraper la tortue ?
        \item Quelle est la distance parcourue par les deux participants au moment où ils se croisent ?
        \item($\star$) Supposons qu'on ne connaisse pas la vitesse exacte d'Achille mais qu'on sache tout de même que la tortue avance 10 fois moins vite que celui-ci.
        Montrer qu'Achille rattrape la tortue en $t = \frac{10}{9}$ minutes, indépendamment de la vitesse d'Achille.
	\end{enumerate}
	
}{}{}


\exe{}{
	Déterminer les paramètres des fonction affines $f, g, h$ dont les courbes sont représentées ci-dessous.

	\begin{center}
		\begin{tikzpicture}[>=stealth, scale=1.5, x=1cm, y=4cm]
		\begin{axis}[xmin = -10, xmax=10, ymin=-10, ymax=10, axis x line=middle, axis y line=middle, axis line style=<->, xlabel={}, ylabel={}, xtick = {-10, -8, ..., 8, 10}, ytick = {-10, -8, ..., 8, 10}, grid=both]
		
			\addplot[RED_E, thick, domain =-9:9, samples=2] {-x}  node[above=6pt] {$(\mathcal{C}_f)$};
			\addplot[RED_E, thick, dotted, domain =-10:-9, samples=2] {-x} ;
			\addplot[RED_E, thick, dotted, domain =9:10, samples=2] {-x};
		
		
			\addplot[GREEN_E, thick, domain =-9:9, samples=2] {x/2+1}  node[below=6pt] {$(\mathcal{C}_g)$};
			\addplot[GREEN_E, thick, dotted, domain =-10:-9, samples=2] {x/2+1} ;
			\addplot[GREEN_E, thick, dotted, domain =9:10, samples=2] {x/2+1};
		
		
			\addplot[black, thick, domain =-9:9, samples=2] {7}  node[above=10pt, left] {$(\mathcal{C}_h)$};
			\addplot[black, thick, dotted, domain =-10:-9, samples=2] {7} ;
			\addplot[black, thick, dotted, domain =9:10, samples=2] {7};
		
			
		\end{axis}
	\end{tikzpicture}
	\end{center}
}{exe:lecture-graphique-affine}{
	Pour la fonction $f(x) = ax+b (x\in\R)$, on peut par exemple choisir deux points lui appartenant et interpoler linéairement à partir de ces deux points.
	Par unicité de la droite passant par deux points, l'interpolation trouvée doit nécessairement être $f$.
	Prenons $A(-4;4)$ et $B(2;-2)$ et calculons les paramètres de la droite $(AB) = \C_f$.
	
	\begin{align*}
		a = \dfrac{4 - (-2)}{-4 - 2} = -1, && b = \dfrac{(-4)\cdot(-2) - 4\cdot2}{-4-2} = 0.
	\end{align*}
	
	\hrule\vspace{10pt}
	
	Pour la fonction $g(x) = ax+b (x\in\R)$, on peut faire idem en choisissant de bons points pour faciliter les calculs.
	Par exemple, $A(0;1)$ et $B(-2;0)$ sont pratiques car il y a beaucoup de zéros qui annulent les expressions.
	En fait, le point $A$ donne directement l'ordonnée à l'origine $b=1$, mais on peut calculer à l'aide du théorème du cours nonobstant.
	\begin{align*}
		a = \dfrac{1-0}{0-(-2)} = \dfrac12, && b = \dfrac{0 \cdot0 - 1 \cdot (-2)}{0-(-2)} = 1.
	\end{align*}
	
	\hrule\vspace{10pt}
	
	Pour la fonction $h(x) = ax+b (x\in\R)$, on procède par parallélisme et appartenance.
	La fonction est constante, donc $a=0$.
	N'importe quel point lui appartenant a pour ordonnée $b=7$, ce qui conclut.

}

L'exercice suivant est tiré du premier sujet zéro de l'examen de 1ère spécifique.

\exe{}{
	Sur un axe gradué en mètres, on organise une course entre une tortue et un escargot.
	\begin{itemize}
		\item
		La tortue part du point d'abscisse $x=0$, et se déplace vers la droite à une vitesse de 2 mètres par minute.
	
		\item
		L'escargot part du point d'abscisse $x=12$, et se déplace vers la droite à une vitesse de 50 centimètres par minute.
	
		\item
		Les deux concurrents partent en même temps.
	\end{itemize}
	
	À quel endroit la tortue rattrapera-t-elle l'escargot ?
	
	\begin{center}
	\includegraphics[scale=.6]{zero1.png}
	\end{center}
	
	\emph{Toute trace de recherche, même infructueuse, sera prise en compte.}
}{exe:zero1-affine}{}

%%%%%%%%%%%%

\newpage
\fancyhead[C]{\textbf{Solutions}}
\shipoutAnswer

\end{document}
