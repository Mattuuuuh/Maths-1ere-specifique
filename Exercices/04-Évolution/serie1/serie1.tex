\documentclass[a4paper, 12pt]{extarticle}

\usepackage[utf8x]{inputenc}
%fonts
\usepackage{libertinus,libertinust1math}
\usepackage{amsmath,amsthm,amssymb,mathtools}

% SOLUTION SWITCH

\ifsolutions
	\newcommand{\exe}[2]{
		\begin{ex} #1  \end{ex}
		\begin{sol} #2 \end{sol}
	}
\else
	\newcommand{\exe}[2]{
		\begin{ex} #1  \end{ex}
	}
	
\fi


\usepackage[french]{babel}
\usepackage[
a4paper,
margin=2cm,
nomarginpar,% We don't want any margin paragraphs
]{geometry}

% HEADER, ARRAY, ENUM, MULTIOCL
\usepackage{fancyhdr}
\usepackage{array}
\usepackage{multicol, enumitem}
\newcolumntype{P}[1]{>{\centering\arraybackslash}p{#1}}
\usepackage{stackengine}
\newcommand\xrowht[2][0]{\addstackgap[.5\dimexpr#2\relax]{\vphantom{#1}}}

% theorems

\theoremstyle{theorem}
\newtheorem{thm}{Théorème}
\theoremstyle{plain}
\newtheorem*{sol}{Solution}
\theoremstyle{definition}
\newtheorem{ex}{Exercice}
\newtheorem{dfn}{Définition}
\newtheorem*{dfn*}{Définition}


%couleurs
\usepackage{tcolorbox}
\definecolor{myg}{RGB}{56, 140, 70}
\definecolor{myb}{RGB}{45, 111, 177}
\definecolor{myr}{RGB}{199, 68, 64}
\definecolor{mygr}{HTML}{2C3338}


\tcbuselibrary{theorems,skins,hooks}
\newcounter{commonbox}
\makeatletter
\newtcbtheorem[use counter=commonbox]{theorem}{Théorème }%
{
	enhanced,
	colback=white,
	colframe=mygr,
	attach boxed title to top left={yshift*=-\tcboxedtitleheight},
	fonttitle=\bfseries,
	title={#2},
	boxed title size=title,
	boxed title style={%
			sharp corners,
			rounded corners=northwest,
			colback=tcbcolframe,
			boxrule=0pt,
		},
	underlay boxed title={%
			\path[fill=tcbcolframe] (title.south west)--(title.south east)
			to[out=0, in=180] ([xshift=5mm]title.east)--
			(title.center-|frame.east)
			[rounded corners=\kvtcb@arc] |-
			(frame.north) -| cycle;
		},
	#1
}{th}
\newtcbtheorem[use counter=commonbox]{rappel}{Rappel }%
{
	enhanced,
	colback=white,
	colframe=mygr,
	attach boxed title to top left={yshift*=-\tcboxedtitleheight},
	fonttitle=\bfseries,
	title={#2},
	boxed title size=title,
	boxed title style={%
			sharp corners,
			rounded corners=northwest,
			colback=tcbcolframe,
			boxrule=0pt,
		},
	underlay boxed title={%
			\path[fill=tcbcolframe] (title.south west)--(title.south east)
			to[out=0, in=180] ([xshift=5mm]title.east)--
			(title.center-|frame.east)
			[rounded corners=\kvtcb@arc] |-
			(frame.north) -| cycle;
		},
	#1
}{th}
\newtcbtheorem[use counter=commonbox]{strategie}{Stratégie }%
{
	enhanced,
	colback=white,
	colframe=mygr,
	attach boxed title to top left={yshift*=-\tcboxedtitleheight},
	fonttitle=\bfseries,
	title={#2},
	boxed title size=title,
	boxed title style={%
			sharp corners,
			rounded corners=northwest,
			colback=tcbcolframe,
			boxrule=0pt,
		},
	underlay boxed title={%
			\path[fill=tcbcolframe] (title.south west)--(title.south east)
			to[out=0, in=180] ([xshift=5mm]title.east)--
			(title.center-|frame.east)
			[rounded corners=\kvtcb@arc] |-
			(frame.north) -| cycle;
		},
	#1
}{th}
\newtcbtheorem[use counter=commonbox]{outil}{Outil }%
{
	enhanced,
	colback=white,
	colframe=mygr,
	attach boxed title to top left={yshift*=-\tcboxedtitleheight},
	fonttitle=\bfseries,
	title={#2},
	boxed title size=title,
	boxed title style={%
			sharp corners,
			rounded corners=northwest,
			colback=tcbcolframe,
			boxrule=0pt,
		},
	underlay boxed title={%
			\path[fill=tcbcolframe] (title.south west)--(title.south east)
			to[out=0, in=180] ([xshift=5mm]title.east)--
			(title.center-|frame.east)
			[rounded corners=\kvtcb@arc] |-
			(frame.north) -| cycle;
		},
	#1
}{th}
\newtcbtheorem[use counter=commonbox]{but}{Buts du chapitre }%
{
	enhanced,
	colback=white,
	colframe=mygr,
	attach boxed title to top left={yshift*=-\tcboxedtitleheight},
	fonttitle=\bfseries,
	title={#2},
	boxed title size=title,
	boxed title style={%
			sharp corners,
			rounded corners=northwest,
			colback=tcbcolframe,
			boxrule=0pt,
		},
	underlay boxed title={%
			\path[fill=tcbcolframe] (title.south west)--(title.south east)
			to[out=0, in=180] ([xshift=5mm]title.east)--
			(title.center-|frame.east)
			[rounded corners=\kvtcb@arc] |-
			(frame.north) -| cycle;
		},
	#1
}{th}
\newtcbtheorem[use counter=commonbox]{propriete}{Propriété }%
{
	enhanced,
	colback=white,
	colframe=mygr,
	attach boxed title to top left={yshift*=-\tcboxedtitleheight},
	fonttitle=\bfseries,
	title={#2},
	boxed title size=title,
	boxed title style={%
			sharp corners,
			rounded corners=northwest,
			colback=tcbcolframe,
			boxrule=0pt,
		},
	underlay boxed title={%
			\path[fill=tcbcolframe] (title.south west)--(title.south east)
			to[out=0, in=180] ([xshift=5mm]title.east)--
			(title.center-|frame.east)
			[rounded corners=\kvtcb@arc] |-
			(frame.north) -| cycle;
		},
	#1
}{th}
\newtcbtheorem[number within=commonbox]{definition}{Définition }%
{
	enhanced,
	colback=white,
	colframe=mygr,
	attach boxed title to top left={yshift*=-\tcboxedtitleheight},
	fonttitle=\bfseries,
	title={#2},
	boxed title size=title,
	boxed title style={%
			sharp corners,
			rounded corners=northwest,
			colback=tcbcolframe,
			boxrule=0pt,
		},
	underlay boxed title={%
			\path[fill=tcbcolframe] (title.south west)--(title.south east)
			to[out=0, in=180] ([xshift=5mm]title.east)--
			(title.center-|frame.east)
			[rounded corners=\kvtcb@arc] |-
			(frame.north) -| cycle;
		},
	#1
}{th}
\newtcbtheorem[number within=commonbox]{exemples}{Exemples }%
{
	enhanced,
	colback=white,
	colframe=mygr,
	attach boxed title to top left={yshift*=-\tcboxedtitleheight},
	fonttitle=\bfseries,
	title={#2},
	boxed title size=title,
	boxed title style={%
			sharp corners,
			rounded corners=northwest,
			colback=tcbcolframe,
			boxrule=0pt,
		},
	underlay boxed title={%
			\path[fill=tcbcolframe] (title.south west)--(title.south east)
			to[out=0, in=180] ([xshift=5mm]title.east)--
			(title.center-|frame.east)
			[rounded corners=\kvtcb@arc] |-
			(frame.north) -| cycle;
		},
	#1
}{th}
\newtcbtheorem[number within=commonbox]{exemple}{Exemple }%
{
	enhanced,
	colback=white,
	colframe=mygr,
	attach boxed title to top left={yshift*=-\tcboxedtitleheight},
	fonttitle=\bfseries,
	title={#2},
	boxed title size=title,
	boxed title style={%
			sharp corners,
			rounded corners=northwest,
			colback=tcbcolframe,
			boxrule=0pt,
		},
	underlay boxed title={%
			\path[fill=tcbcolframe] (title.south west)--(title.south east)
			to[out=0, in=180] ([xshift=5mm]title.east)--
			(title.center-|frame.east)
			[rounded corners=\kvtcb@arc] |-
			(frame.north) -| cycle;
		},
	#1
}{th}
\newtcbtheorem[number within=commonbox]{questions}{Questions guidantes }%
{
	enhanced,
	colback=white,
	colframe=mygr,
	attach boxed title to top left={yshift*=-\tcboxedtitleheight},
	fonttitle=\bfseries,
	title={#2},
	boxed title size=title,
	boxed title style={%
			sharp corners,
			rounded corners=northwest,
			colback=tcbcolframe,
			boxrule=0pt,
		},
	underlay boxed title={%
			\path[fill=tcbcolframe] (title.south west)--(title.south east)
			to[out=0, in=180] ([xshift=5mm]title.east)--
			(title.center-|frame.east)
			[rounded corners=\kvtcb@arc] |-
			(frame.north) -| cycle;
		},
	#1
}{th}
\makeatother

% corps
\newcommand{\R}{\mathbb{R}}
\newcommand{\Rnn}{\mathbb{R}^{2n}}
\newcommand{\Z}{\mathbb{Z}}
\newcommand{\N}{\mathbb{N}}
\newcommand{\Q}{\mathbb{Q}}

% domain
\newcommand{\D}{\mathcal{D}}
% for calligraphic C
\usepackage{calrsfs}
\newcommand{\C}{\mathcal{C}}

% date
\usepackage{advdate}

% ensembles tq. 
\newcommand{\xRtq}[1]{
	$\left\{ x \in \R \text{ tq. } #1 \right\}$
}

% vabs
\newcommand{\vabs}[1]{
	\left| #1 \right|
}

%pinfty minfty
\newcommand{\pinfty}{{+}\infty}
\newcommand{\minfty}{{-}\infty}

% plots
\usepackage{pgfplots}

%virgules
\usepackage{icomma}
\pgfplotsset{/pgf/number format/use comma}

%subfigures
\usepackage{subcaption}

%hyperlink footnote
\usepackage{hyperref}

%wider tabulars
\def\arraystretch{2}
\setlength\tabcolsep{15pt}

% tableaux var, signe
\usepackage{tkz-tab}

\SetDate[08/10/2025]

\begin{document}
\pagestyle{fancy}
\fancyhead[L]{Première spécifique}
\fancyhead[C]{\textbf{Évolution}}
\fancyhead[R]{\today}

\exe{}{
	Calculer sans calculatrice les valeurs suivantes.
	\begin{multicols}{2}
	\begin{enumerate}
		\item $75\%$ de $60$
		\item $60\%$ de $75$
		\item $72\%$ de $25$
		\item $68\%$ de $20$
		\item $125\%$ de $40$
		\item $40\%$ de $125$
	\end{enumerate}
	\end{multicols}
}{exe:1}{
	\begin{multicols}{2}
	\begin{enumerate}
		\item $\dfrac34 \cdot 60 = 3 \cdot \dfrac{60}4 = 3 \cdot 15 = 45$
		\item $45$
		\item $\dfrac14 \cdot 72 = 18$
		\item $\dfrac15 \cdot 68 = \dfrac{136}{10} = 13,6$
		\item $40 + \dfrac14 \cdot 40 = 50$
		\item $50$
	\end{enumerate}
	\end{multicols}
}

\exe{}{
	Approximer sans calculatrice les valeurs suivantes.
	\begin{multicols}{2}
	\begin{enumerate}
		\item $33\%$ de $150$
		\item $166\%$ de $180$
		\item $11\%$ de $90$
		\item $89\%$ de $81$
		\item $16,6\%$ de $18$
		\item $83,4\%$ de $36$
	\end{enumerate}
	\end{multicols}
}{exe:2}{
	\begin{multicols}{2}
	\begin{enumerate}
		\item $\approx \dfrac13 \cdot 150 = 50$
		\item $\approx 180 + \dfrac23 \cdot 180 = 180 + 120 = 300$
		\item $\approx \dfrac19 \cdot 90 = 10$
		\item $\approx 81 - \dfrac19 \cdot 81 = 81 - 9 = 72$
		\item $\approx \dfrac16 \cdot 18 = 3$
		\item $\approx 36 - \dfrac16 \cdot 36 = 36 - 6 = 30$
	\end{enumerate}
	\end{multicols}
}

\exe{}{
On estime la biomasse totale des fourmis sur Terre à $12$ millions de tonnes.
Ceci serait égal à $20\%$ de la biomasse humaine.

Estimer la biomasse totale des humains sur Terre en tonnes.
}{exe:3}{
	On a la relation
		\[ \dfrac{\text{biomasse des fourmis}}{\text{biomasse humaine}} = 0,2. \]
	D'où
		\[ \text{biomasse humaine} = \dfrac{12 \times 10^6}{0,2} \text{T} = 60 \times 10^6 \text{T}.\] 

}


% à changer pour éviter la calculatrice ?
\exe{}{
	Considérons deux tailleurs, l'un à $250$€ et l'autre à $360$€.
	\begin{enumerate}
		\item Quelle augmentation de prix faut-il appliquer au premier tailleur pour qu'il ait le prix du second ?
		\item Quel rabais faut-il appliquer au second tailleur pour qu'il ait le prix du premier ?
	\end{enumerate}
}{exe:4}{
	\begin{enumerate}
		\item On calcule l'évolution $\dfrac{360}{250}= 1,44 = 144\%$. Ainsi, le deuxième tailleur vaut $144\%$ du prix du premier : une augmentation de $44\%$ est nécessaire.
		\item On calcule l'évolution $\dfrac{250}{360}\approx 0,7 = 70\%$. Le premier tailleur vaut environ $70\%$ du prix du second : une diminution de $30\%$ est nécessaire.
	\end{enumerate}
}

\exe{}{
À quelle évolution correspond une augmentation de $20\%$ suivie d'une diminution de $20\%$ ?
}{exe:5}{
	Augmenter une quantité $N$ de $20\%$ correspond à la multiplier par $1,2$.
	Une diminution, elle, multiplie par $0,8$.
	
	La quantité finale est donné par 
		\[ 0,8 \cdot (1,2 \cdot N) = (0,8 \cdot 1,2) \cdot N = 0,96 \cdot N, \]
	qui correspond à une diminution de $4\%$.
}

\exe{}{
	Si on augmente le prix d'un objet de $150\%$, quel rabais faut-il appliquer pour retrouver le prix initial de l'objet ?
}{exe:6}{
	Notons $P$ le prix initial de l'objet.
	Le prix augmenté vaut donc $1,5 \cdot P$.
	Pour retrouver $P$, il faut multiplier le prix augmenté par l'inverse de $1,5$, soit $1,5^{-1} = \dfrac23 \approx 0,666 = 66,6\%$.
	Ceci correspond à une diminution de $33,4\%$.
}

\exe{}{
	Un marchand décide de changer le prix de sa marchandise de 1 000€ à 999€, prix psychologique.
	Il compare le nombre de ventes avant et après le changement de prix.
	\begin{enumerate}
		\item De quel pourcentage les ventes doivent-elles augmenter pour que le chiffre d'affaire reste inchangé ?
		\item De quel pourcentage les ventes doivent-elles augmenter pour que le chiffre d'affaire augmente de 10\% ?
	\end{enumerate}
}{exe:prix-psychologique}{
	TODO
}

\exe{}{
	Une jeune femme dépose $10$€ à la banque. Celle-ci lui promet un taux d'intérêt à l'année de $3\%$.
	Ainsi, après la première année, il y aura $1,03 \times 10 = 10,3$€ sur son compte.
	La deuxième année, il y aura $1,03 \times 10,35 = 10,609$€, etc...
	
	\begin{enumerate}
		\item Combien d'argent aura-t-elle après $5$ ans ?
		\item Combien d'argent aura-t-elle après $50$ ans ?
		\item Combien d'argent y aura-t-il sur son compte après $1000$ ans ?
	\end{enumerate}
	
	Aide aux calculs
}{exe:evol4}{
	\begin{enumerate}
		\item On multiplie $5$ fois par $1,03$, ce qui donne $1,03^5 \times 10\approx 11,59$€.
		\item On multiplie $50$ fois par $1,03$, ce qui donne $1,03^{50} \times 10\approx 43,84$€.
		\item On multiplie $1000$ fois par $1,03$, ce qui donne $1,03^{1000} \times 10\approx 6,87 \times 10^{13}$€, c'est-à-dire environ $68$ billions d'euros ($1$ billion = $1000$ milliards).
	\end{enumerate}
}


% évol moyenne. Chiant ? mais dans le BO
%\exe{}{
%	Le nombre de visiteurs du Musée du Louvre au cours des années $2015$ à $2018$ sont donnés ci-dessous.
%	\begin{center}
%	\begin{tikzpicture}[scale=.8]
%		% nodes
%		\draw (0,0) ellipse (2cm and .5cm) node {$8,6$ millions};
%		
%		\draw (5,0) ellipse (2cm and .5cm) node {$7,3$ millions};
%		
%		\draw (10,0) ellipse (2cm and .5cm) node {$8,1$ millions};
%		
%		\draw (15,0) ellipse (2cm and .5cm) node {$10,2$ millions};
%		
%		% vertices
%		\draw[->, thick, myg] (1cm,.6cm) arc (105:75:7);
%		\draw[->, thick, myg] (6cm,.6cm) arc (105:75:7);
%		\draw[->, thick, myg] (11cm,.6cm) arc (105:75:7);
%		
%		\draw[->, thick, myr] (1cm,-.5cm) arc (-105:-75:25);
%	\end{tikzpicture}
%	\end{center}
%
%	\begin{enumerate}
%		\item Compléter le schéma en ajoutant les coefficients multiplicateurs.
%		\item Calculer le coefficient multiplicateur moyen et en déduire le taux d'évolution moyen.
%	\end{enumerate}
%
%}{exe:7}{
%	TODO
%}
%
%\exe{}{
%	Le nombre de visiteurs du Musée du Louvre semble augmenter de $6\%$ chaque année.
%	À l'année $0$, on compte $8,6$ millions de visiteurs.
%
%	\begin{enumerate}
%		\item Écrire $V(n)$, le nombre de visiteurs du musée après $n$ années, où $n\in\N$ est un entier naturel.
%		\item Écrire $V(x)$, le nombre de visiteurs du musée après $x$ années, où $x\geq0$ est un nombre réel positif ou nul.
%		\item À partir de quand le nombre de visiteurs dépassera $13,3$ millions ?
%		\item À partir de quand le nombre de visiteurs dépassera $14,78$ millions ?
%	\end{enumerate}
%
%}{exe:8}{
%	TODO
%}

%%%%%%%%%%%%

\newpage
\fancyhead[C]{\textbf{Solutions}}
\shipoutAnswer

\end{document}
