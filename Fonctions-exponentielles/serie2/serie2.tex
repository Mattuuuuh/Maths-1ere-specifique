% SOLUTION SWITCH
\newif\ifsolutions
				\solutionstrue
				\solutionsfalse
				
\documentclass[a4paper, 12pt]{extarticle}

\usepackage[utf8x]{inputenc}
%fonts
\usepackage{libertinus,libertinust1math}
\usepackage{amsmath,amsthm,amssymb,mathtools}

% SOLUTION SWITCH

\ifsolutions
	\newcommand{\exe}[2]{
		\begin{ex} #1  \end{ex}
		\begin{sol} #2 \end{sol}
	}
\else
	\newcommand{\exe}[2]{
		\begin{ex} #1  \end{ex}
	}
	
\fi


\usepackage[french]{babel}
\usepackage[
a4paper,
margin=2cm,
nomarginpar,% We don't want any margin paragraphs
]{geometry}

% HEADER, ARRAY, ENUM, MULTIOCL
\usepackage{fancyhdr}
\usepackage{array}
\usepackage{multicol, enumitem}
\newcolumntype{P}[1]{>{\centering\arraybackslash}p{#1}}
\usepackage{stackengine}
\newcommand\xrowht[2][0]{\addstackgap[.5\dimexpr#2\relax]{\vphantom{#1}}}

% theorems

\theoremstyle{theorem}
\newtheorem{thm}{Théorème}
\theoremstyle{plain}
\newtheorem*{sol}{Solution}
\theoremstyle{definition}
\newtheorem{ex}{Exercice}
\newtheorem{dfn}{Définition}
\newtheorem*{dfn*}{Définition}


%couleurs
\usepackage{tcolorbox}
\definecolor{myg}{RGB}{56, 140, 70}
\definecolor{myb}{RGB}{45, 111, 177}
\definecolor{myr}{RGB}{199, 68, 64}
\definecolor{mygr}{HTML}{2C3338}


\tcbuselibrary{theorems,skins,hooks}
\newcounter{commonbox}
\makeatletter
\newtcbtheorem[use counter=commonbox]{theorem}{Théorème }%
{
	enhanced,
	colback=white,
	colframe=mygr,
	attach boxed title to top left={yshift*=-\tcboxedtitleheight},
	fonttitle=\bfseries,
	title={#2},
	boxed title size=title,
	boxed title style={%
			sharp corners,
			rounded corners=northwest,
			colback=tcbcolframe,
			boxrule=0pt,
		},
	underlay boxed title={%
			\path[fill=tcbcolframe] (title.south west)--(title.south east)
			to[out=0, in=180] ([xshift=5mm]title.east)--
			(title.center-|frame.east)
			[rounded corners=\kvtcb@arc] |-
			(frame.north) -| cycle;
		},
	#1
}{th}
\newtcbtheorem[use counter=commonbox]{rappel}{Rappel }%
{
	enhanced,
	colback=white,
	colframe=mygr,
	attach boxed title to top left={yshift*=-\tcboxedtitleheight},
	fonttitle=\bfseries,
	title={#2},
	boxed title size=title,
	boxed title style={%
			sharp corners,
			rounded corners=northwest,
			colback=tcbcolframe,
			boxrule=0pt,
		},
	underlay boxed title={%
			\path[fill=tcbcolframe] (title.south west)--(title.south east)
			to[out=0, in=180] ([xshift=5mm]title.east)--
			(title.center-|frame.east)
			[rounded corners=\kvtcb@arc] |-
			(frame.north) -| cycle;
		},
	#1
}{th}
\newtcbtheorem[use counter=commonbox]{strategie}{Stratégie }%
{
	enhanced,
	colback=white,
	colframe=mygr,
	attach boxed title to top left={yshift*=-\tcboxedtitleheight},
	fonttitle=\bfseries,
	title={#2},
	boxed title size=title,
	boxed title style={%
			sharp corners,
			rounded corners=northwest,
			colback=tcbcolframe,
			boxrule=0pt,
		},
	underlay boxed title={%
			\path[fill=tcbcolframe] (title.south west)--(title.south east)
			to[out=0, in=180] ([xshift=5mm]title.east)--
			(title.center-|frame.east)
			[rounded corners=\kvtcb@arc] |-
			(frame.north) -| cycle;
		},
	#1
}{th}
\newtcbtheorem[use counter=commonbox]{outil}{Outil }%
{
	enhanced,
	colback=white,
	colframe=mygr,
	attach boxed title to top left={yshift*=-\tcboxedtitleheight},
	fonttitle=\bfseries,
	title={#2},
	boxed title size=title,
	boxed title style={%
			sharp corners,
			rounded corners=northwest,
			colback=tcbcolframe,
			boxrule=0pt,
		},
	underlay boxed title={%
			\path[fill=tcbcolframe] (title.south west)--(title.south east)
			to[out=0, in=180] ([xshift=5mm]title.east)--
			(title.center-|frame.east)
			[rounded corners=\kvtcb@arc] |-
			(frame.north) -| cycle;
		},
	#1
}{th}
\newtcbtheorem[use counter=commonbox]{but}{Buts du chapitre }%
{
	enhanced,
	colback=white,
	colframe=mygr,
	attach boxed title to top left={yshift*=-\tcboxedtitleheight},
	fonttitle=\bfseries,
	title={#2},
	boxed title size=title,
	boxed title style={%
			sharp corners,
			rounded corners=northwest,
			colback=tcbcolframe,
			boxrule=0pt,
		},
	underlay boxed title={%
			\path[fill=tcbcolframe] (title.south west)--(title.south east)
			to[out=0, in=180] ([xshift=5mm]title.east)--
			(title.center-|frame.east)
			[rounded corners=\kvtcb@arc] |-
			(frame.north) -| cycle;
		},
	#1
}{th}
\newtcbtheorem[use counter=commonbox]{propriete}{Propriété }%
{
	enhanced,
	colback=white,
	colframe=mygr,
	attach boxed title to top left={yshift*=-\tcboxedtitleheight},
	fonttitle=\bfseries,
	title={#2},
	boxed title size=title,
	boxed title style={%
			sharp corners,
			rounded corners=northwest,
			colback=tcbcolframe,
			boxrule=0pt,
		},
	underlay boxed title={%
			\path[fill=tcbcolframe] (title.south west)--(title.south east)
			to[out=0, in=180] ([xshift=5mm]title.east)--
			(title.center-|frame.east)
			[rounded corners=\kvtcb@arc] |-
			(frame.north) -| cycle;
		},
	#1
}{th}
\newtcbtheorem[number within=commonbox]{definition}{Définition }%
{
	enhanced,
	colback=white,
	colframe=mygr,
	attach boxed title to top left={yshift*=-\tcboxedtitleheight},
	fonttitle=\bfseries,
	title={#2},
	boxed title size=title,
	boxed title style={%
			sharp corners,
			rounded corners=northwest,
			colback=tcbcolframe,
			boxrule=0pt,
		},
	underlay boxed title={%
			\path[fill=tcbcolframe] (title.south west)--(title.south east)
			to[out=0, in=180] ([xshift=5mm]title.east)--
			(title.center-|frame.east)
			[rounded corners=\kvtcb@arc] |-
			(frame.north) -| cycle;
		},
	#1
}{th}
\newtcbtheorem[number within=commonbox]{exemples}{Exemples }%
{
	enhanced,
	colback=white,
	colframe=mygr,
	attach boxed title to top left={yshift*=-\tcboxedtitleheight},
	fonttitle=\bfseries,
	title={#2},
	boxed title size=title,
	boxed title style={%
			sharp corners,
			rounded corners=northwest,
			colback=tcbcolframe,
			boxrule=0pt,
		},
	underlay boxed title={%
			\path[fill=tcbcolframe] (title.south west)--(title.south east)
			to[out=0, in=180] ([xshift=5mm]title.east)--
			(title.center-|frame.east)
			[rounded corners=\kvtcb@arc] |-
			(frame.north) -| cycle;
		},
	#1
}{th}
\newtcbtheorem[number within=commonbox]{exemple}{Exemple }%
{
	enhanced,
	colback=white,
	colframe=mygr,
	attach boxed title to top left={yshift*=-\tcboxedtitleheight},
	fonttitle=\bfseries,
	title={#2},
	boxed title size=title,
	boxed title style={%
			sharp corners,
			rounded corners=northwest,
			colback=tcbcolframe,
			boxrule=0pt,
		},
	underlay boxed title={%
			\path[fill=tcbcolframe] (title.south west)--(title.south east)
			to[out=0, in=180] ([xshift=5mm]title.east)--
			(title.center-|frame.east)
			[rounded corners=\kvtcb@arc] |-
			(frame.north) -| cycle;
		},
	#1
}{th}
\newtcbtheorem[number within=commonbox]{questions}{Questions guidantes }%
{
	enhanced,
	colback=white,
	colframe=mygr,
	attach boxed title to top left={yshift*=-\tcboxedtitleheight},
	fonttitle=\bfseries,
	title={#2},
	boxed title size=title,
	boxed title style={%
			sharp corners,
			rounded corners=northwest,
			colback=tcbcolframe,
			boxrule=0pt,
		},
	underlay boxed title={%
			\path[fill=tcbcolframe] (title.south west)--(title.south east)
			to[out=0, in=180] ([xshift=5mm]title.east)--
			(title.center-|frame.east)
			[rounded corners=\kvtcb@arc] |-
			(frame.north) -| cycle;
		},
	#1
}{th}
\makeatother

% corps
\newcommand{\R}{\mathbb{R}}
\newcommand{\Rnn}{\mathbb{R}^{2n}}
\newcommand{\Z}{\mathbb{Z}}
\newcommand{\N}{\mathbb{N}}
\newcommand{\Q}{\mathbb{Q}}

% domain
\newcommand{\D}{\mathcal{D}}
% for calligraphic C
\usepackage{calrsfs}
\newcommand{\C}{\mathcal{C}}

% date
\usepackage{advdate}

% ensembles tq. 
\newcommand{\xRtq}[1]{
	$\left\{ x \in \R \text{ tq. } #1 \right\}$
}

% vabs
\newcommand{\vabs}[1]{
	\left| #1 \right|
}

%pinfty minfty
\newcommand{\pinfty}{{+}\infty}
\newcommand{\minfty}{{-}\infty}

% plots
\usepackage{pgfplots}

%virgules
\usepackage{icomma}
\pgfplotsset{/pgf/number format/use comma}

%subfigures
\usepackage{subcaption}

%hyperlink footnote
\usepackage{hyperref}

%wider tabulars
\def\arraystretch{2}
\setlength\tabcolsep{15pt}

% tableaux var, signe
\usepackage{tkz-tab}


\AdvanceDate[2]

\begin{document}
\pagestyle{fancy}
\fancyhead[L]{Première}
\fancyhead[C]{\textbf{Fonctions exponentielles \ifsolutions -- Solutions \fi}}
\fancyhead[R]{\today}

Dans la suite, $q\in\R^*_+$ est un nombre réel strictement positif qu'on appelle \emph{base}.

\begin{definition*}{Puissance sur $\N^*$}{}
	Soit $m \in \N^*$ un entier naturel non nul.
	Alors \og $q$ puissance $m$ \fg~est égal à
		\[ q^{m} = \underbrace{q \times q \times \cdots \times q}_{\text{$m$ fois}}. \]
	%En particulier, $q^1 = q$.
\end{definition*}

\begin{definition*}{Extension de la puissance à $\Z$}{}
	Pour $m\in\N^*$, on a 
		\[ q^{m+1} = \underbrace{q \times q \times \cdots \times q}_{\text{$m+1$ fois}} = q \times q^{m} \]
	On étend cette relation à $m \in \Z$ pour obtenir les valeurs suivantes.
		\begin{center}
		\begin{tabular}{|c|c|c|c|c|c|c|}\hline
			$m$ & $-2$ & $-1$ & $0$ & $1$ & $2$ & $3$ \\ \hline
			$q^m$ & & & & $q$ & & \\ \hline
		\end{tabular}
		\end{center}
	D'où
		\begin{align*}
			q^{0} = \qquad, && q^{-1} = \qquad, && \text{ et } && q^{-m} = \qquad.
		\end{align*}
\end{definition*}


\begin{definition*}{Extension de la puissance à $\Q$}{}
	Pour $m, n \in \N^*$, on a 
		\[ \left( q^m \right)^n = \underbrace{q^m \times q^m \times \cdots \times q^m}_{\text{$n$ fois}} = q^{m \times n} \]
	On étend cette relation à $\Q$ pour obtenir
		\[ \left( q^{1/m} \right)^{m} = q. \]
	Le nombre $q^{1/m}$ est donc l'unique nombre réel $x\in\R^*_+$ strictement positif tel que
		\[ x^m = q. \]		
	On étend alors la puissance à tout nombre rationnel $\frac{n}{m} \in \Q$ par
		\[ q^{n/m} = q^{n \times 1/m} = \left( q^{1/m} \right)^n. \]
\end{definition*}

\begin{example*}{}{}
	Le nombre $q^{1/2}$ est défini par la quantité $x$ vérifiant $x^2 = q$, d'où
		\[ q^{1/2} = \qquad. \]
%		\begin{align*}
%			7^3 = 343 && \iff &&  343^{1/3} = 7 && \iff && 343^{2/3} = 7^2 = 49
%		\end{align*}
	\vspace{1.4625cm}
\end{example*}

\newpage

\begin{definition*}{Fonction exponentielle de base $q$}{}
	On dit qu'une fonction $g$ est exponentielle de base $q\in\R_+^*$ si elle est de la forme 
		\[ g(x) = g(0) \times q^{x}. \]
\end{definition*}

\exe{
	Calculer sans calculatrice.
	
	\begin{multicols}{2}
	\begin{enumerate}
		\item $2^3$
		\item $8^{1/3}$
		\item $3^4$
		\item $81^{1/4}$
		\item $8^{2/3}$
		\item $2^{-1}$
		\item $8^{-2/3}$
		\item $27^{-2/3}$
	\end{enumerate}
	\end{multicols}
}{}

\exe{
	Montrer qu'on a environ $2^{10} \approx 10^3$. 
	On dira que $2^{10}$ a pour \emph{ordre de grandeur} $10^3$ car c'est la puissance de $10$ la plus proche.
	
	\begin{enumerate}
		\item En déduire approximativement l'ordre de grandeur de $2^{20}$ et le nombre de chiffres nécessaires pour écrire le nombre.
		\item En déduire approximativement l'ordre de grandeur de $2^{35}$ et le nombre de chiffres nécessaires pour écrire le nombre.
	\end{enumerate}
}{}

\exe{
	Approximer à l'aide de la calculatrice en arrondissant à $10^{-2}$ près.
	
	\begin{multicols}{2}
	\begin{enumerate}
		\item $70^{1/3}$
		\item $1000^{1/10}$
		\item $\dfrac{1}{30^{-1/2}}$
		\item $2^{\pi}$
	\end{enumerate}
	\end{multicols}
}{}

\exe{
	Soit $S$ une suite géométrique de terme initial $S(0) = 4$ et de raison $q$ encore inconnue.
	On rappelle que $S$ s'écrit $S(n) = 4 \times q^{n}$ pour tout $n\in\N$.
	
	%\begin{multicols}{2}
	\begin{enumerate}
		\item Calculer $q$ si $S(5) = 390,625$.
		\item Calculer $q$ si $S(10) = 4096$.
		\item Calculer $q$ si $S(7) = 700$. Arrondir à $10^{-2}$ près.
		\item Calculer $q$ si $S(13) = 13~521$. Arrondir à $10^{-2}$ près.
	\end{enumerate}
	%end{multicols}

}{}

\exe{
	$g$ est une fonction exponentielle de base $q$ encore inconnue et telle que $g(0) = 350$.
	
	\begin{enumerate}
		%\item Montrer que $g(x)/g(0) = q^x$ pour tout $x\in\R$.
		\item Si $g(3,4) = 45875200$, que vaut $q$ ?
		\item Si $g\left(-\dfrac43\right) = 21,875$, que vaut $q$ ?
		\item Si $g(-4,1) = 15$, que vaut $q$ ? Arrondir à $10^{-4}$ près.
		\item Si $g(\pi) = \pi$, que vaut $q$ ? Arrondir à $10^{-4}$ près.
	\end{enumerate}
}

\end{document}