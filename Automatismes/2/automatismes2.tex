\documentclass[14pt]{beamer}
\usepackage[french]{babel}


\usetheme{CambridgeUS}
\usecolortheme{rose}
\beamertemplatenavigationsymbolsempty

\usepackage{array}
\usepackage{amsmath,amsfonts,amsthm,amssymb,mathtools}
\newcolumntype{P}[1]{>{\centering\arraybackslash}p{#1}}


\usepackage{stackengine}
\newcommand\xrowht[2][0]{\addstackgap[.5\dimexpr#2\relax]{\vphantom{#1}}}


% corps
\newcommand{\C}{\mathbb{C}}
\newcommand{\R}{\mathbb{R}}
\newcommand{\Rnn}{\mathbb{R}^{2n}}
\newcommand{\Z}{\mathbb{Z}}
\newcommand{\N}{\mathbb{N}}
\newcommand{\Q}{\mathbb{Q}}

% domain
\newcommand{\D}{\mathbb{D}}


% date
\usepackage{advdate}
\AdvanceDate[1]

\begin{document}

\section{Automatismes n°1}

\subsection{Raison}

\begin{frame}
	
	Soit $u$ la suite donnée, pour tout $n\in\N$, par
		\[ u(n) = 4n+1. \]
	Calculer $u(n+1) - u(n)$.

\end{frame}

\subsection{Récurrence}

\begin{frame}
	Soit $u$ une suite arithmétique de raison $4$ et de terme de rang $8$ donné par
		\[ u(8) = -13. \]
	Calculer $u(10)$.

\end{frame}

\subsection{Substitution}

\begin{frame}

	Soit $u$ la suite donnée, pour tout $n\in\N$, par
		\[ u(n) = 4n+1, \]
	et $v$ la suite définie par, pour tout $n\in\N$, 
		\[ v(n) = u(2n-3). \]
	Écrire $v(n)$ en fonction de $n$.	
\end{frame}

%\begin{frame}
%
%	Soit $u$ la suite donnée, pour tout $n\in\N$, par
%		\[ u(n) = 4n+1, \]
%	et $w$ la suite définie par, pour tout $n\in\N$, 
%		\[ w(n) = u(-n^2+1). \]
%	Donner le terme de rang $n$ de $v$ en fonction de $n$.	
%
%\end{frame}

\subsection{Inégalités}

\begin{frame}

	Soient $u$ et $v$ les suites données, pour tout $n\in\N$, par
		\begin{align*}
			u(n) &= 4n+1, & v(n) &= -6n+\dfrac34
		\end{align*}
	Donner l'\emph{ensemble} des rangs $n$ vérifiants
		\[ u(n) \leq v(n). \]
\end{frame}




\end{document}