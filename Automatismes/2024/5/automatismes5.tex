\documentclass[14pt]{beamer}
\usepackage[french]{babel}


\usetheme{CambridgeUS}
\usecolortheme{rose}
\beamertemplatenavigationsymbolsempty

\usepackage{array}
\usepackage{amsmath,amsfonts,amsthm,calrsfs,mathtools}
\newcolumntype{P}[1]{>{\centering\arraybackslash}p{#1}}


\usepackage{stackengine}
\newcommand\xrowht[2][0]{\addstackgap[.5\dimexpr#2\relax]{\vphantom{#1}}}


% corps
\newcommand{\C}{\mathbb{C}}
\newcommand{\R}{\mathbb{R}}
\newcommand{\Rnn}{\mathbb{R}^{2n}}
\newcommand{\Z}{\mathbb{Z}}
\newcommand{\N}{\mathbb{N}}
\newcommand{\Q}{\mathbb{Q}}

% domain
\newcommand{\D}{\mathbb{D}}


% date
\usepackage{advdate}
\AdvanceDate[0]

\begin{document}

\section{Automatismes n°5}

\begin{frame}

\centering \huge
Automatismes

\end{frame}

\subsection{Appartenances}

\begin{frame}{1}

	Soit $f$ une fonction affine donnée, pour tout $x\in\R$, par
		\[ f(x) = - 5. \]
	Est-ce que $(0;-1) \in \mathcal{C}_f$ ?
\end{frame}


\begin{frame}{2}
	Soit $f$ une fonction affine donnée, pour tout $x\in\R$, par
		\[ f(x) =  -5. \]
	Est-ce que $(-1;-5) \in \mathcal{C}_f$ ?
\end{frame}

\subsection{Paramètres}

\begin{frame}{3}
    Identifier le coefficient directeur $a$ et l'ordonnée à l'origine $b$ de la fonction affine
		\[ g(x) = -5, \qquad x \in \R. \]
    \begin{align*}
        a = && && b=
    \end{align*}
\end{frame}

\subsection{Variations}

\begin{frame}{4}

	Soit $h$ une fonction affine donnée, pour tout $x\in\R$, par
		\[ h(x) = -5. \]
	Donner les variations de $g$.

\end{frame}




\end{document}