\documentclass[14pt]{beamer}
\usepackage[french]{babel}


\usetheme{CambridgeUS}
\usecolortheme{rose}
\beamertemplatenavigationsymbolsempty

\usepackage{array}
\usepackage{amsmath,amsfonts,amsthm,calrsfs,mathtools}
\usepackage{multicol}
\usepackage{eurosym}
\newcolumntype{P}[1]{>{\centering\arraybackslash}p{#1}}


\usepackage{stackengine}
\newcommand\xrowht[2][0]{\addstackgap[.5\dimexpr#2\relax]{\vphantom{#1}}}


% corps
\usepackage{calrsfs}
\newcommand{\C}{\mathcal{C}}
\newcommand{\R}{\mathbb{R}}
\newcommand{\Rnn}{\mathbb{R}^{2n}}
\newcommand{\Z}{\mathbb{Z}}
\newcommand{\N}{\mathbb{N}}
\newcommand{\Q}{\mathbb{Q}}

% domain
\newcommand{\D}{\mathbb{D}}


% date
\usepackage{advdate}
\AdvanceDate[0]


\usepackage{pgfplots, subcaption}
\definecolor{myg}{RGB}{56, 140, 70}
\definecolor{myb}{RGB}{45, 111, 177}
\definecolor{myr}{RGB}{199, 68, 64}

\begin{document}

\section{Automatismes}

\begin{frame}

\centering \huge
Automatismes

\large
Calculatrice interdite

\end{frame}

\subsection{Manipulation des puissances}

\begin{frame}{1}
	\centering
    Calculer et donner $2^4$.
\end{frame}

\begin{frame}{2}
	\centering
    Donner $5^{-1}$ sous forme de fraction.
\end{frame}

\begin{frame}{3}
	\centering
    Donner la valeur de $16^{1/2}$ à l'aide des égalités ci-dessous.

    \vfill
    \begin{align*}
        2^4 &= 16 \\
        16^2 &= 256 \\
        16\times \dfrac12 &= 8 \\
        4^2 &= 16 \\
        2^3 &= 8
    \end{align*}
\end{frame}

\begin{frame}{4}
	\centering
    Calculer et la valeur de $343^{2/3}$ à l'aide du tableau ci-dessous.

    \vfill
    \begin{tabular}{|c|c|c|c|c|c|} \hline
        Puissance de $7$ & $7^0$ & $7^1$ & $7^2$ & $7^3$ & $7^4$ \\ \hline
        Valeur & $1$ & $7$ & $49$ & $343$ & $2401$ \\ \hline
    \end{tabular}
\end{frame}



\end{document}