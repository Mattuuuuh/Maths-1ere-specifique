\documentclass[14pt]{beamer}
\usepackage[french]{babel}


\usetheme{CambridgeUS}
\usecolortheme{rose}
\beamertemplatenavigationsymbolsempty

\usepackage{array}
\usepackage{amsmath,amsfonts,amsthm,calrsfs,mathtools}
\newcolumntype{P}[1]{>{\centering\arraybackslash}p{#1}}


\usepackage{stackengine}
\newcommand\xrowht[2][0]{\addstackgap[.5\dimexpr#2\relax]{\vphantom{#1}}}


% corps
\usepackage{calrsfs}
\newcommand{\C}{\mathcal{C}}
\newcommand{\R}{\mathbb{R}}
\newcommand{\Rnn}{\mathbb{R}^{2n}}
\newcommand{\Z}{\mathbb{Z}}
\newcommand{\N}{\mathbb{N}}
\newcommand{\Q}{\mathbb{Q}}

% domain
\newcommand{\D}{\mathbb{D}}


% date
\usepackage{advdate}
\AdvanceDate[0]


\usepackage{pgfplots, subcaption}
\definecolor{myg}{RGB}{56, 140, 70}
\definecolor{myb}{RGB}{45, 111, 177}
\definecolor{myr}{RGB}{199, 68, 64}

\begin{document}

\section{Automatismes n°7}

\begin{frame}

\centering \huge
Automatismes

\end{frame}

\subsection{Suites géométriques}

\begin{frame}{1}
    Soit $P$ une suite géométrique de raison $\dfrac43$ et de terme initial
        \[ P(0) = 3. \]
    Donner $P(1)$.
\end{frame}

\begin{frame}{2}
    On considère une suite géométrique dont les $5$ premiers termes sont donnés ci-dessous.
    \begin{center}
    \begin{tabular}{|c|c|c|c|c|c|}\hline
        Étape & 0 & 1 & 2 & 3 & 4 \\ \hline
        Temps (min) & 1 & 0,1 & 0,01 & 0,001 & $10^{-4}$ \\ \hline
    \end{tabular}
    \end{center}
    Donner la raison $q\in\R$ de la suite.
\end{frame}

\begin{frame}{3}
    Soit $P$ une suite géométrique de raison $\dfrac32$ et de terme initial
        \[ P(0) = 8. \]
    Donner $P(3)$.
\end{frame}

\begin{frame}{4}
    Soit $P$ une suite géométrique de raison $10$ et de terme initial
        \[ P(0) = 3. \]
    Donner $P(100)$ en écriture scientifique.
\end{frame}


\end{document}