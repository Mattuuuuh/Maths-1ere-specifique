\documentclass[14pt]{beamer}
\usepackage[french]{babel}


\usetheme{CambridgeUS}
\usecolortheme{rose}
\beamertemplatenavigationsymbolsempty

\usepackage{array}
\usepackage{amsmath,amsfonts,amsthm,calrsfs,mathtools}
\usepackage{multicol}
\newcolumntype{P}[1]{>{\centering\arraybackslash}p{#1}}


\usepackage{stackengine}
\newcommand\xrowht[2][0]{\addstackgap[.5\dimexpr#2\relax]{\vphantom{#1}}}


% corps
\usepackage{calrsfs}
\newcommand{\C}{\mathcal{C}}
\newcommand{\R}{\mathbb{R}}
\newcommand{\Rnn}{\mathbb{R}^{2n}}
\newcommand{\Z}{\mathbb{Z}}
\newcommand{\N}{\mathbb{N}}
\newcommand{\Q}{\mathbb{Q}}

% domain
\newcommand{\D}{\mathbb{D}}


% date
\usepackage{advdate}
\AdvanceDate[0]


\usepackage{pgfplots, subcaption}
\definecolor{myg}{RGB}{56, 140, 70}
\definecolor{myb}{RGB}{45, 111, 177}
\definecolor{myr}{RGB}{199, 68, 64}

\begin{document}

\section{Automatismes}

\begin{frame}

\centering \huge
Automatismes

\end{frame}

\subsection{Suites géométriques}

\begin{frame}{1}
	Soit $u$, suite géométrique donnée algébriquement pour tout $n\in\N$ par 
		\[ u(n) = 3 \times 2^n. \]
    Donner le terme initial $u(0)$ ainsi que la raison $q$ de la suite.
\end{frame}

\begin{frame}{2}
	Soit $u$ donnée algébriquement pour tout $n\in\N$ par 
		\[ u(n) = 3 \times 2^n. \]
    Calculer $u(3)$.
\end{frame}

\begin{frame}{3}
	Soit $u$ donnée algébriquement pour tout $n\in\N$ par 
		\[ u(n) = 3 \times 2^n. \]
    Donner le plus petit entier naturel $N\in\N$ vérifiant
        \[ u(N) \geq 30. \]
\end{frame}

\begin{frame}{4}

    Soit $u(n)$ la suite dont les $9$ premiers termes sont donnés graphiquement ci-dessous.
    Donner $u(8)$.
    
	\begin{center}
	\begin{tikzpicture}[>=stealth, scale=.9]
		\begin{axis}[xmin = 0, xmax=8.2, xtick={ 0, ..., 8}, ymin=0, ymax=300, ytick={0, 32, ..., 300}, axis x line=middle, axis y line=middle, axis line style=->, ylabel={}, grid=both]
			\addplot[black, thick, only marks, mark=square] coordinates {(0,1) (1,2) (2,4) (3, 8) (4, 16) (5, 32) (6, 64) (7, 128) (8, 256)};
		\end{axis}
	
	\end{tikzpicture}
	\end{center}

\end{frame}



\end{document}