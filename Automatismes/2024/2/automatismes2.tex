\documentclass[14pt]{beamer}
\usepackage[french]{babel}


\usetheme{CambridgeUS}
\usecolortheme{rose}
\beamertemplatenavigationsymbolsempty

\usepackage{array}
\usepackage{amsmath,amsfonts,amsthm,amssymb,mathtools}
\newcolumntype{P}[1]{>{\centering\arraybackslash}p{#1}}


\usepackage{stackengine}
\newcommand\xrowht[2][0]{\addstackgap[.5\dimexpr#2\relax]{\vphantom{#1}}}


% corps
\newcommand{\C}{\mathbb{C}}
\newcommand{\R}{\mathbb{R}}
\newcommand{\Rnn}{\mathbb{R}^{2n}}
\newcommand{\Z}{\mathbb{Z}}
\newcommand{\N}{\mathbb{N}}
\newcommand{\Q}{\mathbb{Q}}

% domain
\newcommand{\D}{\mathbb{D}}


% date
\usepackage{advdate}
\AdvanceDate[1]

\begin{document}

\section{Automatismes n°2}

\begin{frame}

\centering \huge
Automatismes

\end{frame}

\subsection{Fonctions}

\begin{frame}{1}

	Soit $u$ une fonction donnée, pour tout $n\in\N$, par
		\[ u(n) = 4n+1. \]
	Calculer $u(0)$.
\end{frame}


\begin{frame}{2}
	Soit $u$ une fonction donnée, pour tout $n\in\N$, par
		\[ u(n) = 10-2n. \]
	Calculer $u(10)$.
\end{frame}

\subsection{Équations linéaires}

\begin{frame}{3}

	Soient $u$ et $v$ les suites données, pour tout $n\in\N$, par
		\begin{align*}
			u(n) &= 4n+1, & v(n) &= -n+11
		\end{align*}
	À quel rang les suites $u$ et $v$ sont-elles égales ?

\end{frame}

\subsection{Substitution}

\begin{frame}{4}

	Soit $u$ la suite donnée, pour tout $n\in\N$, par
		\[ u(n) = 4n+14, \]
	et $v$ la suite définie par, pour tout $k\in\N$, 
		\[ v(k) = u(2k-3). \]
	Écrire $v(k)$ en substituant dans l'expression de $u$.

\end{frame}




\end{document}