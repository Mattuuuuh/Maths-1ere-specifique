% SOLUTION SWITCH
\newif\ifsolutions
				\solutionstrue
				\solutionsfalse
				
\documentclass[a4paper, 12pt]{extarticle}

\usepackage[utf8x]{inputenc}
%fonts
\usepackage{libertinus,libertinust1math}
\usepackage{amsmath,amsthm,amssymb,mathtools}

% SOLUTION SWITCH

\ifsolutions
	\newcommand{\exe}[2]{
		\begin{ex} #1  \end{ex}
		\begin{sol} #2 \end{sol}
	}
\else
	\newcommand{\exe}[2]{
		\begin{ex} #1  \end{ex}
	}
	
\fi


\usepackage[french]{babel}
\usepackage[
a4paper,
margin=2cm,
nomarginpar,% We don't want any margin paragraphs
]{geometry}

% HEADER, ARRAY, ENUM, MULTIOCL
\usepackage{fancyhdr}
\usepackage{array}
\usepackage{multicol, enumitem}
\newcolumntype{P}[1]{>{\centering\arraybackslash}p{#1}}
\usepackage{stackengine}
\newcommand\xrowht[2][0]{\addstackgap[.5\dimexpr#2\relax]{\vphantom{#1}}}

% theorems

\theoremstyle{theorem}
\newtheorem{thm}{Théorème}
\theoremstyle{plain}
\newtheorem*{sol}{Solution}
\theoremstyle{definition}
\newtheorem{ex}{Exercice}
\newtheorem{dfn}{Définition}
\newtheorem*{dfn*}{Définition}


%couleurs
\usepackage{tcolorbox}
\definecolor{myg}{RGB}{56, 140, 70}
\definecolor{myb}{RGB}{45, 111, 177}
\definecolor{myr}{RGB}{199, 68, 64}
\definecolor{mygr}{HTML}{2C3338}


\tcbuselibrary{theorems,skins,hooks}
\newcounter{commonbox}
\makeatletter
\newtcbtheorem[use counter=commonbox]{theorem}{Théorème }%
{
	enhanced,
	colback=white,
	colframe=mygr,
	attach boxed title to top left={yshift*=-\tcboxedtitleheight},
	fonttitle=\bfseries,
	title={#2},
	boxed title size=title,
	boxed title style={%
			sharp corners,
			rounded corners=northwest,
			colback=tcbcolframe,
			boxrule=0pt,
		},
	underlay boxed title={%
			\path[fill=tcbcolframe] (title.south west)--(title.south east)
			to[out=0, in=180] ([xshift=5mm]title.east)--
			(title.center-|frame.east)
			[rounded corners=\kvtcb@arc] |-
			(frame.north) -| cycle;
		},
	#1
}{th}
\newtcbtheorem[use counter=commonbox]{rappel}{Rappel }%
{
	enhanced,
	colback=white,
	colframe=mygr,
	attach boxed title to top left={yshift*=-\tcboxedtitleheight},
	fonttitle=\bfseries,
	title={#2},
	boxed title size=title,
	boxed title style={%
			sharp corners,
			rounded corners=northwest,
			colback=tcbcolframe,
			boxrule=0pt,
		},
	underlay boxed title={%
			\path[fill=tcbcolframe] (title.south west)--(title.south east)
			to[out=0, in=180] ([xshift=5mm]title.east)--
			(title.center-|frame.east)
			[rounded corners=\kvtcb@arc] |-
			(frame.north) -| cycle;
		},
	#1
}{th}
\newtcbtheorem[use counter=commonbox]{strategie}{Stratégie }%
{
	enhanced,
	colback=white,
	colframe=mygr,
	attach boxed title to top left={yshift*=-\tcboxedtitleheight},
	fonttitle=\bfseries,
	title={#2},
	boxed title size=title,
	boxed title style={%
			sharp corners,
			rounded corners=northwest,
			colback=tcbcolframe,
			boxrule=0pt,
		},
	underlay boxed title={%
			\path[fill=tcbcolframe] (title.south west)--(title.south east)
			to[out=0, in=180] ([xshift=5mm]title.east)--
			(title.center-|frame.east)
			[rounded corners=\kvtcb@arc] |-
			(frame.north) -| cycle;
		},
	#1
}{th}
\newtcbtheorem[use counter=commonbox]{outil}{Outil }%
{
	enhanced,
	colback=white,
	colframe=mygr,
	attach boxed title to top left={yshift*=-\tcboxedtitleheight},
	fonttitle=\bfseries,
	title={#2},
	boxed title size=title,
	boxed title style={%
			sharp corners,
			rounded corners=northwest,
			colback=tcbcolframe,
			boxrule=0pt,
		},
	underlay boxed title={%
			\path[fill=tcbcolframe] (title.south west)--(title.south east)
			to[out=0, in=180] ([xshift=5mm]title.east)--
			(title.center-|frame.east)
			[rounded corners=\kvtcb@arc] |-
			(frame.north) -| cycle;
		},
	#1
}{th}
\newtcbtheorem[use counter=commonbox]{but}{Buts du chapitre }%
{
	enhanced,
	colback=white,
	colframe=mygr,
	attach boxed title to top left={yshift*=-\tcboxedtitleheight},
	fonttitle=\bfseries,
	title={#2},
	boxed title size=title,
	boxed title style={%
			sharp corners,
			rounded corners=northwest,
			colback=tcbcolframe,
			boxrule=0pt,
		},
	underlay boxed title={%
			\path[fill=tcbcolframe] (title.south west)--(title.south east)
			to[out=0, in=180] ([xshift=5mm]title.east)--
			(title.center-|frame.east)
			[rounded corners=\kvtcb@arc] |-
			(frame.north) -| cycle;
		},
	#1
}{th}
\newtcbtheorem[use counter=commonbox]{propriete}{Propriété }%
{
	enhanced,
	colback=white,
	colframe=mygr,
	attach boxed title to top left={yshift*=-\tcboxedtitleheight},
	fonttitle=\bfseries,
	title={#2},
	boxed title size=title,
	boxed title style={%
			sharp corners,
			rounded corners=northwest,
			colback=tcbcolframe,
			boxrule=0pt,
		},
	underlay boxed title={%
			\path[fill=tcbcolframe] (title.south west)--(title.south east)
			to[out=0, in=180] ([xshift=5mm]title.east)--
			(title.center-|frame.east)
			[rounded corners=\kvtcb@arc] |-
			(frame.north) -| cycle;
		},
	#1
}{th}
\newtcbtheorem[number within=commonbox]{definition}{Définition }%
{
	enhanced,
	colback=white,
	colframe=mygr,
	attach boxed title to top left={yshift*=-\tcboxedtitleheight},
	fonttitle=\bfseries,
	title={#2},
	boxed title size=title,
	boxed title style={%
			sharp corners,
			rounded corners=northwest,
			colback=tcbcolframe,
			boxrule=0pt,
		},
	underlay boxed title={%
			\path[fill=tcbcolframe] (title.south west)--(title.south east)
			to[out=0, in=180] ([xshift=5mm]title.east)--
			(title.center-|frame.east)
			[rounded corners=\kvtcb@arc] |-
			(frame.north) -| cycle;
		},
	#1
}{th}
\newtcbtheorem[number within=commonbox]{exemples}{Exemples }%
{
	enhanced,
	colback=white,
	colframe=mygr,
	attach boxed title to top left={yshift*=-\tcboxedtitleheight},
	fonttitle=\bfseries,
	title={#2},
	boxed title size=title,
	boxed title style={%
			sharp corners,
			rounded corners=northwest,
			colback=tcbcolframe,
			boxrule=0pt,
		},
	underlay boxed title={%
			\path[fill=tcbcolframe] (title.south west)--(title.south east)
			to[out=0, in=180] ([xshift=5mm]title.east)--
			(title.center-|frame.east)
			[rounded corners=\kvtcb@arc] |-
			(frame.north) -| cycle;
		},
	#1
}{th}
\newtcbtheorem[number within=commonbox]{exemple}{Exemple }%
{
	enhanced,
	colback=white,
	colframe=mygr,
	attach boxed title to top left={yshift*=-\tcboxedtitleheight},
	fonttitle=\bfseries,
	title={#2},
	boxed title size=title,
	boxed title style={%
			sharp corners,
			rounded corners=northwest,
			colback=tcbcolframe,
			boxrule=0pt,
		},
	underlay boxed title={%
			\path[fill=tcbcolframe] (title.south west)--(title.south east)
			to[out=0, in=180] ([xshift=5mm]title.east)--
			(title.center-|frame.east)
			[rounded corners=\kvtcb@arc] |-
			(frame.north) -| cycle;
		},
	#1
}{th}
\newtcbtheorem[number within=commonbox]{questions}{Questions guidantes }%
{
	enhanced,
	colback=white,
	colframe=mygr,
	attach boxed title to top left={yshift*=-\tcboxedtitleheight},
	fonttitle=\bfseries,
	title={#2},
	boxed title size=title,
	boxed title style={%
			sharp corners,
			rounded corners=northwest,
			colback=tcbcolframe,
			boxrule=0pt,
		},
	underlay boxed title={%
			\path[fill=tcbcolframe] (title.south west)--(title.south east)
			to[out=0, in=180] ([xshift=5mm]title.east)--
			(title.center-|frame.east)
			[rounded corners=\kvtcb@arc] |-
			(frame.north) -| cycle;
		},
	#1
}{th}
\makeatother

% corps
\newcommand{\R}{\mathbb{R}}
\newcommand{\Rnn}{\mathbb{R}^{2n}}
\newcommand{\Z}{\mathbb{Z}}
\newcommand{\N}{\mathbb{N}}
\newcommand{\Q}{\mathbb{Q}}

% domain
\newcommand{\D}{\mathcal{D}}
% for calligraphic C
\usepackage{calrsfs}
\newcommand{\C}{\mathcal{C}}

% date
\usepackage{advdate}

% ensembles tq. 
\newcommand{\xRtq}[1]{
	$\left\{ x \in \R \text{ tq. } #1 \right\}$
}

% vabs
\newcommand{\vabs}[1]{
	\left| #1 \right|
}

%pinfty minfty
\newcommand{\pinfty}{{+}\infty}
\newcommand{\minfty}{{-}\infty}

% plots
\usepackage{pgfplots}

%virgules
\usepackage{icomma}
\pgfplotsset{/pgf/number format/use comma}

%subfigures
\usepackage{subcaption}

%hyperlink footnote
\usepackage{hyperref}

%wider tabulars
\def\arraystretch{2}
\setlength\tabcolsep{15pt}

% tableaux var, signe
\usepackage{tkz-tab}


\AdvanceDate[1]

\begin{document}
\pagestyle{fancy}
\fancyhead[L]{Première}
\fancyhead[C]{\textbf{Évaluation blanche : Suites géométriques \ifsolutions -- Solutions \fi}}
\fancyhead[R]{\today}

\begin{definition*}{Suite géométrique} \label{def:1}
	Soit $u$ une suite. On dit que $u$ est \emph{géométrique} de raison $q\in\R$ dès que, pour tout $n\in\N$,
		\begin{align}\label{eq:def}
			u(n+1) = q \times u(n).
		\end{align}
\end{definition*}

\begin{theorem}[label=thm:1]{d'Eden}
	Considérons une suite $u$ géométrique de raison $q\in\R$ et de terme initial $u(0)$.
    Alors le terme de rang $n\in\N$ de $u$ s'écrit
		\[ u(n) = \dots\dots\dots\dots\dots.\]
		
\end{theorem}

\exe{[2pts]
    Compléter le théorème \ref{thm:1} vu en cours.
}

\exe{[4pts]
    Soit $v$ une suite géométrique de raison $2$ et telle que
        \[ v(10) = 4. \]
    \begin{multicols}{2}
    \begin{enumerate}
        \item Calculer $v(11)$.
        \item Calculer $v(12)$.
        \item Calculer $v(10)$.
        \item Calculer $v(9)$.
    \end{enumerate}
    \end{multicols}
}{}

\exe{[6pts]
	Pour chacune des suites géométriques données algébriquement pour tout $n\in\N$, donner sa raison et son terme initial.
	\begin{multicols}{2}
	\begin{enumerate}
		\item $u(n) = 2 \times 3^n$
		\item $v(n) = 7 \times \left(\dfrac12 \right)^n$
		\item $a(n) = 11 \times 5^{2n}$
		\item $b(n) = 3 \times 5^{2n+3}$
		\item $c(n) = 10^{-n}$
		\item $d(n) = \dfrac{4}{7^n}$
	\end{enumerate}
	\end{multicols}
}{}

\exe{[4pts]
    Un étudiant souhaite étudier la vitesse de son algorithme de recherche par dichotomie.
    Il se demande, étant donné une suite géométrique $v$ positive et non constante et un seuil $M>0$ : en combien d'étapes puis-je trouver le plus petit entier naturel $N\in\N$ tel que
        \[ v(N) \geq M ?\]
    Il commence par évaluer $v(0)$ et $v(768)$ et déduit que le $N$ recherché appartient forcément à l'intervalle $[0 ; 768]$.
    A chaque étape, une évaluation de $v$ lui permet de diviser en deux l'intervalle dans lequel $N$ se trouve et de continuer avec une des deux moitiés.
    \begin{center}
    \begin{tabular}{|c|c|c|c|c|c|}\hline
        Etape & 0 & 1 & 2 & 3 & 4 \\ \hline
        Taille de l'intervalle & 768 & 384 & & & \\ \hline
    \end{tabular}
    \end{center}

    \begin{enumerate}
        \item Vérifier les premières valeurs du tableau ci-dessus et le compléter.
        \item En notant par $I(n)$ la taille de l'intervalle à l'étape $n$, donner l'expression algébrique de $I(n)$.
        \item Donner le plus petit entier naturel $k\in\N$ tel que
            \[ I(k) < 1.\]
        \item Décrire avec des mots ce que $k$ représente.
        \item Combien d'évaluations de $v$ l'étudiant a-t-il eu besoin de faire pour trouver le $N$ recherché ?
    \end{enumerate}
}{}

\newpage

\exe{[4pts]
	Déterminer le terme inital et la raison des suites géométriques donnés graphiquement ci-dessous.

	\begin{center}
	\begin{tikzpicture}[>=stealth, scale=1.5]
		\begin{axis}[xmin = 0, xmax=4.2, xtick={ 0,1,2, 3, 4,5}, ymin=0, ymax=10000, ymode=log, log ticks with fixed point, axis x line=middle, axis y line=middle, axis line style=->, ylabel={}, grid=both]
			
			\addplot[black, thick, only marks, mark=star] coordinates {(0, 1) (1,10) (2,100) (3,1000) (4,10000)};
			
			\addplot[black, thick, only marks, mark=square] coordinates {(0,10 000) (1,10 00) (2,100) (3,10) (4,1)};
			
			\addplot[black, thick, only marks, mark=*] coordinates {(0,3) (1,30) (2,300) (3, 3000)};
		\end{axis}
	
	\end{tikzpicture}
	\end{center}
}{}



\subsection*{Bonus (2pts)}

\exe{
    Considérons la suite $v$ définie par la relation de récurrence suivante, pour tout $n\in\N$.
    \[
    \begin{cases}
        v(0) = 0, \\
        v(n+1) = 3v(n) + 2.
    \end{cases}
    \]
    \begin{enumerate}
        \item Montrer que la suite intermédiaire $w$ définie pour tout $n\in\N$ par
            \[ w(n) = v(n) + 1,\]
        est géométrique.
        \item En déduire que, pour tout $n\in\N$,
            \[ v(n) = 3^n - 1.\]
    \end{enumerate}
}

\end{document}