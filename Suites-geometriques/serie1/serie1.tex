% SOLUTION SWITCH
\newif\ifsolutions
				\solutionstrue
				%\solutionsfalse
				
\documentclass[a4paper, 12pt]{extarticle}

\usepackage[utf8x]{inputenc}
%fonts
\usepackage{libertinus,libertinust1math}
\usepackage{amsmath,amsthm,amssymb,mathtools}

% SOLUTION SWITCH

\ifsolutions
	\newcommand{\exe}[2]{
		\begin{ex} #1  \end{ex}
		\begin{sol} #2 \end{sol}
	}
\else
	\newcommand{\exe}[2]{
		\begin{ex} #1  \end{ex}
	}
	
\fi


\usepackage[french]{babel}
\usepackage[
a4paper,
margin=2cm,
nomarginpar,% We don't want any margin paragraphs
]{geometry}

% HEADER, ARRAY, ENUM, MULTIOCL
\usepackage{fancyhdr}
\usepackage{array}
\usepackage{multicol, enumitem}
\newcolumntype{P}[1]{>{\centering\arraybackslash}p{#1}}
\usepackage{stackengine}
\newcommand\xrowht[2][0]{\addstackgap[.5\dimexpr#2\relax]{\vphantom{#1}}}

% theorems

\theoremstyle{theorem}
\newtheorem{thm}{Théorème}
\theoremstyle{plain}
\newtheorem*{sol}{Solution}
\theoremstyle{definition}
\newtheorem{ex}{Exercice}
\newtheorem{dfn}{Définition}
\newtheorem*{dfn*}{Définition}


%couleurs
\usepackage{tcolorbox}
\definecolor{myg}{RGB}{56, 140, 70}
\definecolor{myb}{RGB}{45, 111, 177}
\definecolor{myr}{RGB}{199, 68, 64}
\definecolor{mygr}{HTML}{2C3338}


\tcbuselibrary{theorems,skins,hooks}
\newcounter{commonbox}
\makeatletter
\newtcbtheorem[use counter=commonbox]{theorem}{Théorème }%
{
	enhanced,
	colback=white,
	colframe=mygr,
	attach boxed title to top left={yshift*=-\tcboxedtitleheight},
	fonttitle=\bfseries,
	title={#2},
	boxed title size=title,
	boxed title style={%
			sharp corners,
			rounded corners=northwest,
			colback=tcbcolframe,
			boxrule=0pt,
		},
	underlay boxed title={%
			\path[fill=tcbcolframe] (title.south west)--(title.south east)
			to[out=0, in=180] ([xshift=5mm]title.east)--
			(title.center-|frame.east)
			[rounded corners=\kvtcb@arc] |-
			(frame.north) -| cycle;
		},
	#1
}{th}
\newtcbtheorem[use counter=commonbox]{rappel}{Rappel }%
{
	enhanced,
	colback=white,
	colframe=mygr,
	attach boxed title to top left={yshift*=-\tcboxedtitleheight},
	fonttitle=\bfseries,
	title={#2},
	boxed title size=title,
	boxed title style={%
			sharp corners,
			rounded corners=northwest,
			colback=tcbcolframe,
			boxrule=0pt,
		},
	underlay boxed title={%
			\path[fill=tcbcolframe] (title.south west)--(title.south east)
			to[out=0, in=180] ([xshift=5mm]title.east)--
			(title.center-|frame.east)
			[rounded corners=\kvtcb@arc] |-
			(frame.north) -| cycle;
		},
	#1
}{th}
\newtcbtheorem[use counter=commonbox]{strategie}{Stratégie }%
{
	enhanced,
	colback=white,
	colframe=mygr,
	attach boxed title to top left={yshift*=-\tcboxedtitleheight},
	fonttitle=\bfseries,
	title={#2},
	boxed title size=title,
	boxed title style={%
			sharp corners,
			rounded corners=northwest,
			colback=tcbcolframe,
			boxrule=0pt,
		},
	underlay boxed title={%
			\path[fill=tcbcolframe] (title.south west)--(title.south east)
			to[out=0, in=180] ([xshift=5mm]title.east)--
			(title.center-|frame.east)
			[rounded corners=\kvtcb@arc] |-
			(frame.north) -| cycle;
		},
	#1
}{th}
\newtcbtheorem[use counter=commonbox]{outil}{Outil }%
{
	enhanced,
	colback=white,
	colframe=mygr,
	attach boxed title to top left={yshift*=-\tcboxedtitleheight},
	fonttitle=\bfseries,
	title={#2},
	boxed title size=title,
	boxed title style={%
			sharp corners,
			rounded corners=northwest,
			colback=tcbcolframe,
			boxrule=0pt,
		},
	underlay boxed title={%
			\path[fill=tcbcolframe] (title.south west)--(title.south east)
			to[out=0, in=180] ([xshift=5mm]title.east)--
			(title.center-|frame.east)
			[rounded corners=\kvtcb@arc] |-
			(frame.north) -| cycle;
		},
	#1
}{th}
\newtcbtheorem[use counter=commonbox]{but}{Buts du chapitre }%
{
	enhanced,
	colback=white,
	colframe=mygr,
	attach boxed title to top left={yshift*=-\tcboxedtitleheight},
	fonttitle=\bfseries,
	title={#2},
	boxed title size=title,
	boxed title style={%
			sharp corners,
			rounded corners=northwest,
			colback=tcbcolframe,
			boxrule=0pt,
		},
	underlay boxed title={%
			\path[fill=tcbcolframe] (title.south west)--(title.south east)
			to[out=0, in=180] ([xshift=5mm]title.east)--
			(title.center-|frame.east)
			[rounded corners=\kvtcb@arc] |-
			(frame.north) -| cycle;
		},
	#1
}{th}
\newtcbtheorem[use counter=commonbox]{propriete}{Propriété }%
{
	enhanced,
	colback=white,
	colframe=mygr,
	attach boxed title to top left={yshift*=-\tcboxedtitleheight},
	fonttitle=\bfseries,
	title={#2},
	boxed title size=title,
	boxed title style={%
			sharp corners,
			rounded corners=northwest,
			colback=tcbcolframe,
			boxrule=0pt,
		},
	underlay boxed title={%
			\path[fill=tcbcolframe] (title.south west)--(title.south east)
			to[out=0, in=180] ([xshift=5mm]title.east)--
			(title.center-|frame.east)
			[rounded corners=\kvtcb@arc] |-
			(frame.north) -| cycle;
		},
	#1
}{th}
\newtcbtheorem[number within=commonbox]{definition}{Définition }%
{
	enhanced,
	colback=white,
	colframe=mygr,
	attach boxed title to top left={yshift*=-\tcboxedtitleheight},
	fonttitle=\bfseries,
	title={#2},
	boxed title size=title,
	boxed title style={%
			sharp corners,
			rounded corners=northwest,
			colback=tcbcolframe,
			boxrule=0pt,
		},
	underlay boxed title={%
			\path[fill=tcbcolframe] (title.south west)--(title.south east)
			to[out=0, in=180] ([xshift=5mm]title.east)--
			(title.center-|frame.east)
			[rounded corners=\kvtcb@arc] |-
			(frame.north) -| cycle;
		},
	#1
}{th}
\newtcbtheorem[number within=commonbox]{exemples}{Exemples }%
{
	enhanced,
	colback=white,
	colframe=mygr,
	attach boxed title to top left={yshift*=-\tcboxedtitleheight},
	fonttitle=\bfseries,
	title={#2},
	boxed title size=title,
	boxed title style={%
			sharp corners,
			rounded corners=northwest,
			colback=tcbcolframe,
			boxrule=0pt,
		},
	underlay boxed title={%
			\path[fill=tcbcolframe] (title.south west)--(title.south east)
			to[out=0, in=180] ([xshift=5mm]title.east)--
			(title.center-|frame.east)
			[rounded corners=\kvtcb@arc] |-
			(frame.north) -| cycle;
		},
	#1
}{th}
\newtcbtheorem[number within=commonbox]{exemple}{Exemple }%
{
	enhanced,
	colback=white,
	colframe=mygr,
	attach boxed title to top left={yshift*=-\tcboxedtitleheight},
	fonttitle=\bfseries,
	title={#2},
	boxed title size=title,
	boxed title style={%
			sharp corners,
			rounded corners=northwest,
			colback=tcbcolframe,
			boxrule=0pt,
		},
	underlay boxed title={%
			\path[fill=tcbcolframe] (title.south west)--(title.south east)
			to[out=0, in=180] ([xshift=5mm]title.east)--
			(title.center-|frame.east)
			[rounded corners=\kvtcb@arc] |-
			(frame.north) -| cycle;
		},
	#1
}{th}
\newtcbtheorem[number within=commonbox]{questions}{Questions guidantes }%
{
	enhanced,
	colback=white,
	colframe=mygr,
	attach boxed title to top left={yshift*=-\tcboxedtitleheight},
	fonttitle=\bfseries,
	title={#2},
	boxed title size=title,
	boxed title style={%
			sharp corners,
			rounded corners=northwest,
			colback=tcbcolframe,
			boxrule=0pt,
		},
	underlay boxed title={%
			\path[fill=tcbcolframe] (title.south west)--(title.south east)
			to[out=0, in=180] ([xshift=5mm]title.east)--
			(title.center-|frame.east)
			[rounded corners=\kvtcb@arc] |-
			(frame.north) -| cycle;
		},
	#1
}{th}
\makeatother

% corps
\newcommand{\R}{\mathbb{R}}
\newcommand{\Rnn}{\mathbb{R}^{2n}}
\newcommand{\Z}{\mathbb{Z}}
\newcommand{\N}{\mathbb{N}}
\newcommand{\Q}{\mathbb{Q}}

% domain
\newcommand{\D}{\mathcal{D}}
% for calligraphic C
\usepackage{calrsfs}
\newcommand{\C}{\mathcal{C}}

% date
\usepackage{advdate}

% ensembles tq. 
\newcommand{\xRtq}[1]{
	$\left\{ x \in \R \text{ tq. } #1 \right\}$
}

% vabs
\newcommand{\vabs}[1]{
	\left| #1 \right|
}

%pinfty minfty
\newcommand{\pinfty}{{+}\infty}
\newcommand{\minfty}{{-}\infty}

% plots
\usepackage{pgfplots}

%virgules
\usepackage{icomma}
\pgfplotsset{/pgf/number format/use comma}

%subfigures
\usepackage{subcaption}

%hyperlink footnote
\usepackage{hyperref}

%wider tabulars
\def\arraystretch{2}
\setlength\tabcolsep{15pt}

% tableaux var, signe
\usepackage{tkz-tab}


\AdvanceDate[-7]

\begin{document}
\pagestyle{fancy}
\fancyhead[L]{Première}
\fancyhead[C]{\textbf{Suites géométriques 1 \ifsolutions -- Solutions \fi}}
\fancyhead[R]{\today}

\exe{[Paradoxe de la frontière]

	\begin{multicols}{2}
	\begin{center}
	\begin{tikzpicture}[scale=4]
		\node (A) at (0,0) {$\bullet$};
		\node (B) at (1,0) {$\bullet$};
		\node (C) at (.5,0.866) {$\bullet$};
		
		\draw[black, thick] (A) -- (B) node[midway, below] {$1$};
		\draw[black, thick] (A) -- (C) node[midway, left] {$1$};
		\draw[black, thick] (C) -- (B) node[midway, right] {$1$};
	\end{tikzpicture}
	
	Étape $0$.
	\end{center}
	
	
	\begin{center}
	\begin{tikzpicture}[scale=4]
		\node (A) at (0,0) {$\bullet$};
		\node (B) at (1,0) {$\bullet$};
		\node (C) at (.5,0.866) {$\bullet$};
		
		\node (D) at (.33,0) {$\bullet$};
		\node (E) at (.66,0) {$\bullet$};
		\node (F) at (.5,-0.288) {$\bullet$};
		
		\node (G) at (.16,.2887) {$\bullet$};
		\node (H) at (.33,.5773) {$\bullet$};
		\node (I) at (0,0.5773) {$\bullet$};
		
		\node (J) at (.833,.2887) {$\bullet$};
		\node (K) at (.66,.5773) {$\bullet$};
		\node (L) at (1,0.5773) {$\bullet$};
		
		
		\draw[black, thick] (A) -- (D) -- (F) -- (E) -- (B) node[midway, below] {$\frac13$};
		\draw[black, thick] (A) -- (G) -- (I) -- (H) -- (C);
		\draw[black, thick] (B) -- (J) -- (L) -- (K) -- (C);
		
	\end{tikzpicture}
	
	Étape $1$.
	\end{center}
	\end{multicols}

	On considère une construction qui commence par un triangle équilatéral de côté $1$.
	À chaque étape, on divise chaque segment de la figure précédente par trois, et on remplace le tiers du milieu par un triangle équilatéral qui pointe vers l'extérieur.
	On continue ainsi indéfiniment et on se pose les questions suivantes.
	
	\begin{enumerate}
		\item Calculer le périmètre de la figure à l'étape $0$ puis à l'étape $1$.
		\item Comment le nombre de segment évolue-t-il d'une étape à l'autre ?
		\item Comment la longueur de chaque segment évolue-t-elle d'une étape à l'autre ?
		\item En déduire l'évolution du périmètre d'une étape à l'autre :
		en notant $P(n)$ le périmètre de la figure à l'étape $n=0, 1, 2, 3,\dots$, montrer que
			\[ P(n+1) = \dfrac43 P(n). \]
		\item Donner $P(2)$ et $P(3)$.
		\item Le périmètre peut-il être aussi grand qu'on le souhaite ? Par exemple, existe-t-il un rang $n$ pour lequel $P(n) \geq 1 \ 000$ ?
		\item Donner approximativement $P(100)$.
	\end{enumerate}
}{
	\begin{enumerate}
		\item Le périmètre est la longueur du contour.
		À l'étape $0$, le périmètre est donc égal à $1+1+1 = 3$. 
		À l'étape $1$, on compte $12$ segments de longueur $\frac13$.
		Le périmètre est donc égal à $\frac13 \times 12 = 4$.
		\item Chaque segment se scinde pour engendrer $4$ segments à l'étape d'après.
		Le nombre de segments quadruple donc d'une étape à l'autre. C'est bien le cas de l'étape $0$ à l'étape $1$ d'ailleurs.
		\item La longueur d'un segment est divisée par trois, c'est-à-dire multipliée par $\frac13$.
		\item Le périmètre étant égal au produit de la longueur d'un segment par le nombre de segments, il est multiplié par $4$ puis divisé par $3$ en passant d'une étape à l'autre d'après les questions précédentes.
		Bien sûr $4\times\dfrac13 = \dfrac43$, et donc
			\[ P(n+1) = \dfrac43 P(n), \]
		pour chaque étape $n\in\N$, entier naturel.
		\item On utilise la relation de récurrence de la question précédente pour trouver
			\[ P(2) = \dfrac43 P(1) = \dfrac43 \times 4 = \dfrac{16}3. \]
		Idem pour $P(3)$, 
			\[ P(3) = \dfrac43 P(2) = \dfrac43 \times \dfrac{16}3 = \dfrac{64}9. \]
		\item Pour passer d'une étape à l'autre, on multiplie par une quantité fixe strictement plus grande que $1$.
		Le périmètre croît donc exponentiellement et devient aussi grand que souhaité.
		On peut multiplier par $\dfrac43$ plusieurs fois pour se rendre compte qu'il existe forcément un rang $n$ pour lequel $P(n) \geq 1 \ 000$.
		\item On démontre facilement que $P(n) = 3 \times \left(\dfrac43\right)^n$, par définition de la puissance $n$.
		En évaluant en $n=100$, on trouve
			\[ P(100) = 3 \times \left(\dfrac43\right)^{100} \approx 9,35 \times 10^{12}. \]
	\end{enumerate}

}

\exe{[Paradoxe d'Achille et de la tortue]
	Achille dispute une course avec une tortue. On suppose que les deux participants avancent à vitesse constante et que la tortue avance $10$ fois moins vite qu'Achille.
	Celui-ci décide donc de lui laisser généreusement $10$ minutes d'avance.

	En analysant la situation, Zénon décide de diviser la course en plusieurs étapes.
	À chaque étape, Achille court jusqu'au point d'où a démarré la tortue à la dernière étape.
	Il déduit que, comme la tortue avance pendant qu'Achille court, il ne pourra jamais la dépasser.

	\begin{enumerate}
		\item Vérifier les premières valeurs du tableau suivantes et le compléter.
			\begin{center}
			\begin{tabular}{|c|c|c|c|c|c|}\hline
				Étape & 0 & 1 & 2 & 3 & 4 \\ \hline
				Temps qu'Achille met pour finir l'étape (min) & 1 & 0,1 & \ifsolutions \color{red} 0,01 \fi & \ifsolutions \color{red} 0,001 \fi & \ifsolutions \color{red} $10^{-4}$ \fi \\ \hline
			\end{tabular}
			\end{center}
		\item En déduire l'évolution du temps que prend Achille d'une étape à l'autre :
		en notant $T(n)$ ce temps à l'étape $n=0, 1, 2, 3,\dots$, montrer que
			\[ T(n+1) = \dfrac1{10} T(n). \]
		\item Donner exactement $T(100)$ en écriture scientifique.
		\item On considère la somme des temps de chaque étape pour comprendre quand Achille atteindra la tortue.
			\[ S(n) = T(0) + T(1) + \dots + T(n). \]
		Vérifier les premières valeurs du tableau suivantes et le compléter.
			\begin{center}
			\begin{tabular}{|c|c|c|c|c|c|}\hline
				$n$ & 0 & 1 & 2 & 3 & 4 \\ \hline
				$S(n)$ & 1 & 1,1 & \ifsolutions \color{red} $1,11$ \fi  & \ifsolutions \color{red} $1,111$ \fi  & \ifsolutions \color{red} $1,1111$ \fi   \\ \hline
			\end{tabular}
			\end{center}
		\item Est-ce que $S(n)$ peut être aussi grand qu'on le souhaite ? Par exemple, existe-t-il un rang $n$ pour lequel $S(n) \geq 2$ ?
		\item En combien de temps Achille arrive-t-il à dépasser la tortue ? Donner une valeur exacte sous forme de fraction.
	\end{enumerate}
}{

	\begin{enumerate}
		\item Cf. tableau de l'énoncé.
		\item D'une étape à l'autre, le temps pris par Achille pour rattraper la tortue est divisé par $10$ car celui-ci va $10$ fois plus vite.
		\item $T(100) = 10^{-100}$.
		\item On complète en utilisant que $S(2) = T(0) + T(1) + T(2)$, puis $S(3) = T(0) + T(1) + T(2) + T(3)$, et idem pour $S(4)$.
		\item 
		$S(n)$ ne dépasse jamais $1,2$ et donc ne peut pas être aussi grand que souhaité.
		\item 
		On cherche à étudier la limite de $S(n)$ quand $n$ tend vers l'infini.
		On souhaite donc déterminer $x = 1,1111\dots$ avec une infinité de $1$.
		En multipliant par $3$, on remarque que
			\[ 3x = 3,333\dots = \dfrac{10}3. \]
		On en déduit que $x= \dfrac{10}9$, qu'on vérifiera avec une calculatrice par exemple.
	\end{enumerate}

}

\exe{[Intérêts sur intérêts]\label{ex:2}

	À l'âge de $17$ ans une élève décide de placer $200$€ en bourse qui lui rapportent $10\%$ d'intérêts chaque année.
	Chaque année, elle replace les intérêts gagnés.
	
	On souhaite étudier l'évolution de l'argent placé chaque année après ses $17$ ans inclus.
	\begin{enumerate}
		\item Vérifier les premières valeurs du tableau suivantes et le compléter.
			\begin{center}
			\begin{tabular}{|c|c|c|c|c|c|c|}\hline
				Âge & 17 & 18 & 19 & 20 & 21 & 22 \\ \hline
				Argent placé (€) & 200 & 220 & 242 & \ifsolutions \color{red} $266,2$ \fi  & \ifsolutions \color{red} $292,82$ \fi  & \ifsolutions \color{red} $322,102$ \fi   \\ \hline
			\end{tabular}
			\end{center}
		\item On appelle $A(n)$ la quantité d'argent placé à l'âge $17+n$, où $n\in \{0 ; 1 ; 2; \dots \}$.
		Décrire comment obtenir $A(n+1)$ en connaissant $A(n)$
		.%, c'est-à-dire comment obtenir la quantité d'argent placé en l'an $2017+(n+1)$ en connaissant la quantité en $2017+n$.
		
		\item Combien d'argent aura l'élève à l'âge de $50$ ans ? 
		\item Calculer $A(50)$ et interpréter le résultat.
		\item À quel âge la somme d'argent dépassera-t-elle $100 \  000$€ ?		
	\end{enumerate}
 }{
 
	\begin{enumerate}
		\item Un gain de $10\%$ correspond à un coefficient multiplicateur de $1+\dfrac{10}{100} = 1,1$.
		\item 
		Pour obtenir $A(n+1)$, on multiplie $A(n)$ par $1,1$. On a donc la relation de récurrence suivante.
			\[ A(n+1) = 1,1 \times A(n), \]
		valable pour tout $n\in\N$ entier naturel.
		\item 
		À l'âge de $50 = 17+n$ ans, on doit calculer $A(n)$ pour $n=50-17 = 34$, et donc $A(34)$.
		
		On utilise l'expression algébrique de $A(n)$ qui est
			\[ A(n) = 200 \times (1,1)^n, \]
		pour trouver $A(34) \approx 5109,5.$
		\item $A(50) = 200 \times (1,1)^{50} \approx 23~478,2$.
		C'est l'argent économisé à l'âge de $17+50 = 67$ ans.
		\item 
		On résoud à l'aide d'un tableau de valeurs ou par dichotomie.
		Comme $A(65) < 100~000 < A(66)$, le plus petit $n$ pour lequel $A(n)$ dépasse $100~000$ est $66$.
		Pour $n=66$, l'âge correspondant est $17+66 = 83$ ans.
	\end{enumerate}
 
 }
 
 
 \exe{[Intérêts sur intérêts 2]
 
 	On reprend l'exercice \ref{ex:2} en prenant en plus en compte l'inflation, qu'on suppose constante à $2\%$ par an.
 	Cela signifie que les prix augmentent en moyenne de $2\%$ par an.
 	
 	Afin de rendre comparables les sommes d'argent dans le temps, on souhaite fixer les prix : au lieu de voir les prix comme augmentant, on voit l'euro comme se dépréciant.
 
	\begin{enumerate}
		\item Calculer l'évolution réciproque de $+2\%$. 
		C'est la diminution qu'on appliquera à l'euro chaque année.
		\item Vérifier les premières valeurs du tableau suivantes et le compléter.
			\begin{center}
			\begin{tabular}{|c|c|c|c|c|c|c|}\hline
				Année & 17 & 18 & 19 & 20 & 21 & 22 \\ \hline
				Argent placé (€, prix fixes) & 200 & 215,69 & 232,60 & \ifsolutions \color{red} $250,84$ \fi & \ifsolutions \color{red} $270,5$ \fi & \ifsolutions \color{red} $291,7$ \fi  \\ \hline
			\end{tabular}
			\end{center}
		\item On appelle $B(n)$ la quantité d'argent placé à prix fixes en l'an $2017+n$, où $n\in \{0 ; 1 ; 2; \dots \}$.
		Décrire comment obtenir $B(n+1)$ en connaissant $B(n)$
		.%, c'est-à-dire comment obtenir la quantité d'argent placé en l'an $2017+(n+1)$ en connaissant la quantité en $2017+n$.
		
		\item Calculer $B(70)$ et interpréter le résultat.
	\end{enumerate}
 }{
 
 	\begin{enumerate}
		\item 
		Le coefficient multiplicateur de l'augmentation de $2\%$ est $\times1,02$.
		La réciproque est donnée par la multiplication par $\dfrac1{1,02} \approx 0,98 = 1 - 0,02$.
		C'est donc une diminution de $2\%$ à appliquer après les intérêts chaque année.
		\item 
			Pour passer d'un terme à l'autre, on multiplie par $\dfrac{1,1}{1,02}$.
		\item 
			\[ B(n+1) = \dfrac{1,1}{1,02} \times B(n). \]
		
		\item 
			L'expression algébrique de $B(n)$ est 
				\[ B(n) = 200 \times \left(\dfrac{1,1}{1,02}\right)^n, \]
			qui donne $B(70) = 39~491,7$.
			C'est l'argent à prix fixes accumulé après $17+70 = 87$ ans.
	\end{enumerate}
 
 
 }

\end{document}