% SOLUTION SWITCH
\newif\ifsolutions
				\solutionstrue
				\solutionsfalse
				
\documentclass[a4paper, 12pt]{extarticle}

\usepackage[utf8x]{inputenc}
%fonts
\usepackage{libertinus,libertinust1math}
\usepackage{amsmath,amsthm,amssymb,mathtools}

% SOLUTION SWITCH

\ifsolutions
	\newcommand{\exe}[2]{
		\begin{ex} #1  \end{ex}
		\begin{sol} #2 \end{sol}
	}
\else
	\newcommand{\exe}[2]{
		\begin{ex} #1  \end{ex}
	}
	
\fi


\usepackage[french]{babel}
\usepackage[
a4paper,
margin=2cm,
nomarginpar,% We don't want any margin paragraphs
]{geometry}

% HEADER, ARRAY, ENUM, MULTIOCL
\usepackage{fancyhdr}
\usepackage{array}
\usepackage{multicol, enumitem}
\newcolumntype{P}[1]{>{\centering\arraybackslash}p{#1}}
\usepackage{stackengine}
\newcommand\xrowht[2][0]{\addstackgap[.5\dimexpr#2\relax]{\vphantom{#1}}}

% theorems

\theoremstyle{theorem}
\newtheorem{thm}{Théorème}
\theoremstyle{plain}
\newtheorem*{sol}{Solution}
\theoremstyle{definition}
\newtheorem{ex}{Exercice}
\newtheorem{dfn}{Définition}
\newtheorem*{dfn*}{Définition}


%couleurs
\usepackage{tcolorbox}
\definecolor{myg}{RGB}{56, 140, 70}
\definecolor{myb}{RGB}{45, 111, 177}
\definecolor{myr}{RGB}{199, 68, 64}
\definecolor{mygr}{HTML}{2C3338}


\tcbuselibrary{theorems,skins,hooks}
\newcounter{commonbox}
\makeatletter
\newtcbtheorem[use counter=commonbox]{theorem}{Théorème }%
{
	enhanced,
	colback=white,
	colframe=mygr,
	attach boxed title to top left={yshift*=-\tcboxedtitleheight},
	fonttitle=\bfseries,
	title={#2},
	boxed title size=title,
	boxed title style={%
			sharp corners,
			rounded corners=northwest,
			colback=tcbcolframe,
			boxrule=0pt,
		},
	underlay boxed title={%
			\path[fill=tcbcolframe] (title.south west)--(title.south east)
			to[out=0, in=180] ([xshift=5mm]title.east)--
			(title.center-|frame.east)
			[rounded corners=\kvtcb@arc] |-
			(frame.north) -| cycle;
		},
	#1
}{th}
\newtcbtheorem[use counter=commonbox]{rappel}{Rappel }%
{
	enhanced,
	colback=white,
	colframe=mygr,
	attach boxed title to top left={yshift*=-\tcboxedtitleheight},
	fonttitle=\bfseries,
	title={#2},
	boxed title size=title,
	boxed title style={%
			sharp corners,
			rounded corners=northwest,
			colback=tcbcolframe,
			boxrule=0pt,
		},
	underlay boxed title={%
			\path[fill=tcbcolframe] (title.south west)--(title.south east)
			to[out=0, in=180] ([xshift=5mm]title.east)--
			(title.center-|frame.east)
			[rounded corners=\kvtcb@arc] |-
			(frame.north) -| cycle;
		},
	#1
}{th}
\newtcbtheorem[use counter=commonbox]{strategie}{Stratégie }%
{
	enhanced,
	colback=white,
	colframe=mygr,
	attach boxed title to top left={yshift*=-\tcboxedtitleheight},
	fonttitle=\bfseries,
	title={#2},
	boxed title size=title,
	boxed title style={%
			sharp corners,
			rounded corners=northwest,
			colback=tcbcolframe,
			boxrule=0pt,
		},
	underlay boxed title={%
			\path[fill=tcbcolframe] (title.south west)--(title.south east)
			to[out=0, in=180] ([xshift=5mm]title.east)--
			(title.center-|frame.east)
			[rounded corners=\kvtcb@arc] |-
			(frame.north) -| cycle;
		},
	#1
}{th}
\newtcbtheorem[use counter=commonbox]{outil}{Outil }%
{
	enhanced,
	colback=white,
	colframe=mygr,
	attach boxed title to top left={yshift*=-\tcboxedtitleheight},
	fonttitle=\bfseries,
	title={#2},
	boxed title size=title,
	boxed title style={%
			sharp corners,
			rounded corners=northwest,
			colback=tcbcolframe,
			boxrule=0pt,
		},
	underlay boxed title={%
			\path[fill=tcbcolframe] (title.south west)--(title.south east)
			to[out=0, in=180] ([xshift=5mm]title.east)--
			(title.center-|frame.east)
			[rounded corners=\kvtcb@arc] |-
			(frame.north) -| cycle;
		},
	#1
}{th}
\newtcbtheorem[use counter=commonbox]{but}{Buts du chapitre }%
{
	enhanced,
	colback=white,
	colframe=mygr,
	attach boxed title to top left={yshift*=-\tcboxedtitleheight},
	fonttitle=\bfseries,
	title={#2},
	boxed title size=title,
	boxed title style={%
			sharp corners,
			rounded corners=northwest,
			colback=tcbcolframe,
			boxrule=0pt,
		},
	underlay boxed title={%
			\path[fill=tcbcolframe] (title.south west)--(title.south east)
			to[out=0, in=180] ([xshift=5mm]title.east)--
			(title.center-|frame.east)
			[rounded corners=\kvtcb@arc] |-
			(frame.north) -| cycle;
		},
	#1
}{th}
\newtcbtheorem[use counter=commonbox]{propriete}{Propriété }%
{
	enhanced,
	colback=white,
	colframe=mygr,
	attach boxed title to top left={yshift*=-\tcboxedtitleheight},
	fonttitle=\bfseries,
	title={#2},
	boxed title size=title,
	boxed title style={%
			sharp corners,
			rounded corners=northwest,
			colback=tcbcolframe,
			boxrule=0pt,
		},
	underlay boxed title={%
			\path[fill=tcbcolframe] (title.south west)--(title.south east)
			to[out=0, in=180] ([xshift=5mm]title.east)--
			(title.center-|frame.east)
			[rounded corners=\kvtcb@arc] |-
			(frame.north) -| cycle;
		},
	#1
}{th}
\newtcbtheorem[number within=commonbox]{definition}{Définition }%
{
	enhanced,
	colback=white,
	colframe=mygr,
	attach boxed title to top left={yshift*=-\tcboxedtitleheight},
	fonttitle=\bfseries,
	title={#2},
	boxed title size=title,
	boxed title style={%
			sharp corners,
			rounded corners=northwest,
			colback=tcbcolframe,
			boxrule=0pt,
		},
	underlay boxed title={%
			\path[fill=tcbcolframe] (title.south west)--(title.south east)
			to[out=0, in=180] ([xshift=5mm]title.east)--
			(title.center-|frame.east)
			[rounded corners=\kvtcb@arc] |-
			(frame.north) -| cycle;
		},
	#1
}{th}
\newtcbtheorem[number within=commonbox]{exemples}{Exemples }%
{
	enhanced,
	colback=white,
	colframe=mygr,
	attach boxed title to top left={yshift*=-\tcboxedtitleheight},
	fonttitle=\bfseries,
	title={#2},
	boxed title size=title,
	boxed title style={%
			sharp corners,
			rounded corners=northwest,
			colback=tcbcolframe,
			boxrule=0pt,
		},
	underlay boxed title={%
			\path[fill=tcbcolframe] (title.south west)--(title.south east)
			to[out=0, in=180] ([xshift=5mm]title.east)--
			(title.center-|frame.east)
			[rounded corners=\kvtcb@arc] |-
			(frame.north) -| cycle;
		},
	#1
}{th}
\newtcbtheorem[number within=commonbox]{exemple}{Exemple }%
{
	enhanced,
	colback=white,
	colframe=mygr,
	attach boxed title to top left={yshift*=-\tcboxedtitleheight},
	fonttitle=\bfseries,
	title={#2},
	boxed title size=title,
	boxed title style={%
			sharp corners,
			rounded corners=northwest,
			colback=tcbcolframe,
			boxrule=0pt,
		},
	underlay boxed title={%
			\path[fill=tcbcolframe] (title.south west)--(title.south east)
			to[out=0, in=180] ([xshift=5mm]title.east)--
			(title.center-|frame.east)
			[rounded corners=\kvtcb@arc] |-
			(frame.north) -| cycle;
		},
	#1
}{th}
\newtcbtheorem[number within=commonbox]{questions}{Questions guidantes }%
{
	enhanced,
	colback=white,
	colframe=mygr,
	attach boxed title to top left={yshift*=-\tcboxedtitleheight},
	fonttitle=\bfseries,
	title={#2},
	boxed title size=title,
	boxed title style={%
			sharp corners,
			rounded corners=northwest,
			colback=tcbcolframe,
			boxrule=0pt,
		},
	underlay boxed title={%
			\path[fill=tcbcolframe] (title.south west)--(title.south east)
			to[out=0, in=180] ([xshift=5mm]title.east)--
			(title.center-|frame.east)
			[rounded corners=\kvtcb@arc] |-
			(frame.north) -| cycle;
		},
	#1
}{th}
\makeatother

% corps
\newcommand{\R}{\mathbb{R}}
\newcommand{\Rnn}{\mathbb{R}^{2n}}
\newcommand{\Z}{\mathbb{Z}}
\newcommand{\N}{\mathbb{N}}
\newcommand{\Q}{\mathbb{Q}}

% domain
\newcommand{\D}{\mathcal{D}}
% for calligraphic C
\usepackage{calrsfs}
\newcommand{\C}{\mathcal{C}}

% date
\usepackage{advdate}

% ensembles tq. 
\newcommand{\xRtq}[1]{
	$\left\{ x \in \R \text{ tq. } #1 \right\}$
}

% vabs
\newcommand{\vabs}[1]{
	\left| #1 \right|
}

%pinfty minfty
\newcommand{\pinfty}{{+}\infty}
\newcommand{\minfty}{{-}\infty}

% plots
\usepackage{pgfplots}

%virgules
\usepackage{icomma}
\pgfplotsset{/pgf/number format/use comma}

%subfigures
\usepackage{subcaption}

%hyperlink footnote
\usepackage{hyperref}

%wider tabulars
\def\arraystretch{2}
\setlength\tabcolsep{15pt}

% tableaux var, signe
\usepackage{tkz-tab}

\AdvanceDate[1]

\begin{document}
\pagestyle{fancy}
\fancyhead[L]{Première}
\fancyhead[C]{\textbf{Chapitre 4 --- Phénomènes aléatoires \ifsolutions -- Solutions \fi}}
\fancyhead[R]{\today}

\begin{questions*}{}{}
	\begin{enumerate}[label=\roman*)]
		\item La probabilité d'un événement change-t-elle lorsqu'une nouvelle information est disponible ? Est-ce quantifiable ?
		\item En connaissant la probabilité qu'une personne vaccinée soit infectée, peut-on connaître la probabilité qu'une personne infectée soit vaccinée ?
		\item Quand est-ce que deux événements sont indépendants ? Que peut-on dire dans ce cas ?
	\end{enumerate}
\end{questions*}


\begin{but*}{}{}
	\begin{enumerate}
		\item Construire un arbre de probabilité associé à une expérience aléatoire.
		\item Calculer des fréquences, des probabilités conditionnelles.
		\item Compléter et interpréter un tableau croisé.
	\end{enumerate}
\end{but*}

\section{Événements conditionnels}

\begin{definition*}{événement conditionnel}{}
	Soient $A, B$ deux événements. L'événement « $A$ sachant $B$ », aussi noté $A|B$
	correspond à l'événement associé à $A$ dans le nouvel univers où on sait que $B$ se réalise.
\end{definition*}

\subsection{Arbres de probabilité}

\begin{exemple*}{}{}
	On considère un arbre modélisant une expérience aléatoire à deux épreuves.
	A et B sont les deux issues de la première épreuve. Chaque feuille correspond à une issue finale.
	\begin{center}
	\begin{tikzpicture}
		% depth 1
		\foreach \i in {-3, 3}
		\draw[-, thick, black] (0,0) node {$\bullet$} -- (\i,-2);
		
		\draw (0,0) node[above] {racine};
		\draw (-3,-2) node[above left] {A};
		\draw (3,-2) node[above right] {B};
		% depth 2
		\foreach \i in {-3, 3} \foreach \j in {-1, 1}
			\draw[-, thick, black] (\i,-2) node {$\bullet$} -- (\i+\j,-4) node {$\bullet$};
			
		\draw (-4,-4) node[below] {issue 1};
		\draw (2,-4) node[below] {issue 3};
		\draw (-2,-4) node[below] {issue 2};
		\draw (4,-4) node[below] {issue 4};
		
		% sols
		\draw (1.5,-.5) node {$0,6$};
		\draw (-1.5,-.5) node {$0,4$};
		
		\draw (4,-3) node {$0,3$};
		\draw (2,-3) node {$0,7$};
		
		\draw (-2,-3) node {$\frac14$};
		\draw (-4,-3) node {$\frac34$};
		
		
		\draw (5,-3) node[right] {\thead{probabilités \\ conditionnelles \\ sachant B}};
		\draw (-7.5,-3) node[right] {\thead{probabilités \\ conditionnelles \\ sachant A}};
		\draw[->, thick] (5,-3) -- (4.5,-3);
		\draw[->, thick] (-5,-3) -- (-4.5,-3);
	\end{tikzpicture}
	\end{center}
\end{exemple*}


\begin{proprietes*}{}{}
	\begin{enumerate}[label=$\bullet$]
		\item Les probabilités de l'arbre sont conditionnées par le chemin déjà parcouru.
		\item Pour chaque nœud, la somme des probabilités des sous-branches vaut $1$.
		\item Chaque feuille est une issue (et vice versa). Pour connaître la probabilité d'une issue, il faut multiplier les probabilités le long du chemin racine-feuille.
		\item Un événement est un ensemble d'issues. Pour connaître la probabilité d'un événement, il faut sommer la probabilité de chaque feuille lui correspondant.
	\end{enumerate}
\end{proprietes*}



\subsection{Probabilité conditionnelle}

\begin{definition*}{probabilité conditionnelle}{}
		\[ P(B \sct A) = \dfrac{P(A \cap B)}{P(A)} \]
\end{definition*}



\begin{theorem*}{de Bayes}{}
	Soient $A, B$ deux événements. Alors
		\[ P(A \sct B) = \dfrac{P(A)}{P(B)} \times P(B \sct A). \]
\end{theorem*}


\begin{definition*}{événements indépendants}{}
	Les événements $A$ et $B$ sont \emph{indépendants} si
		\begin{align*}
			P(B \sct A) = P(B) && \iff && \ifsolutions P(B \sct A) = P(B) \else \hspace{3cm} \fi && \iff && \ifsolutions P(A\cap B) = P(A) \times P(B) \else \hspace{3cm} \fi 
		\end{align*}
	On dira alors que $A$ et $B$ sont \emph{décorrélés}, ou de \emph{corrélation nulle}.
	
\end{definition*}

\begin{remarque*}{corrélation}{}
	\begin{enumerate}[label=$\bullet$]
		\item Si $P(A \sct B) > P(A)$, savoir que $B$ se réalise augmente la probabilité que $A$ se réalise. Les événements sont \emph{positivement} corrélés.
		\item Si $P(A \sct B) < P(A)$, savoir que $B$ se réalise diminue la probabilité que $A$ se réalise. Les événements sont \emph{négativement} corrélés.
	\end{enumerate}
\end{remarque*}

\section{Les fréquences comme probabilités : tableaux croisés}

\begin{exemple*}{}{}
	On interroge $500$ personnes pour savoir si elles sont allées chez le dentiste cette année. 
	
	\begin{center}
	\tableaucroise{Dentiste & Pas dentiste & Total}{Enfant & 160 & 40 & 200}{Adulte & 240 & 10 & 300}{Total & 400 & 100 & 500}
	\end{center}
	On choisit une personne uniformément au hasard.
	Les événements $D$ : « la personne choisie est allée chez le dentiste » et $E$ : « la personne choisie est un enfant » sont-ils indépendants ?
	\begin{align*}
		P(D) = &&  P(D \sct E) = \\ P(E) = &&  P(E \sct D) = 
	\end{align*}
\end{exemple*}

\begin{proprietes*}{}{}
	\begin{enumerate}[label=$\bullet$]
		\item On voit la fréquence d'un événement comme sa probabilité.
		\item Pour lire la fréquence conditionnelle, on se restreint à une colonne ou une ligne.
	\end{enumerate}
\end{proprietes*}


\end{document}