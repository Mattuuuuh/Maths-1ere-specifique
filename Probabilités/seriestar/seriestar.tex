% SOLUTION SWITCH
\newif\ifsolutions
				\solutionstrue
				\solutionsfalse
				
\documentclass[a4paper, 12pt]{extarticle}

\usepackage[utf8x]{inputenc}
%fonts
\usepackage{libertinus,libertinust1math}
\usepackage{amsmath,amsthm,amssymb,mathtools}

% SOLUTION SWITCH

\ifsolutions
	\newcommand{\exe}[2]{
		\begin{ex} #1  \end{ex}
		\begin{sol} #2 \end{sol}
	}
\else
	\newcommand{\exe}[2]{
		\begin{ex} #1  \end{ex}
	}
	
\fi


\usepackage[french]{babel}
\usepackage[
a4paper,
margin=2cm,
nomarginpar,% We don't want any margin paragraphs
]{geometry}

% HEADER, ARRAY, ENUM, MULTIOCL
\usepackage{fancyhdr}
\usepackage{array}
\usepackage{multicol, enumitem}
\newcolumntype{P}[1]{>{\centering\arraybackslash}p{#1}}
\usepackage{stackengine}
\newcommand\xrowht[2][0]{\addstackgap[.5\dimexpr#2\relax]{\vphantom{#1}}}

% theorems

\theoremstyle{theorem}
\newtheorem{thm}{Théorème}
\theoremstyle{plain}
\newtheorem*{sol}{Solution}
\theoremstyle{definition}
\newtheorem{ex}{Exercice}
\newtheorem{dfn}{Définition}
\newtheorem*{dfn*}{Définition}


%couleurs
\usepackage{tcolorbox}
\definecolor{myg}{RGB}{56, 140, 70}
\definecolor{myb}{RGB}{45, 111, 177}
\definecolor{myr}{RGB}{199, 68, 64}
\definecolor{mygr}{HTML}{2C3338}


\tcbuselibrary{theorems,skins,hooks}
\newcounter{commonbox}
\makeatletter
\newtcbtheorem[use counter=commonbox]{theorem}{Théorème }%
{
	enhanced,
	colback=white,
	colframe=mygr,
	attach boxed title to top left={yshift*=-\tcboxedtitleheight},
	fonttitle=\bfseries,
	title={#2},
	boxed title size=title,
	boxed title style={%
			sharp corners,
			rounded corners=northwest,
			colback=tcbcolframe,
			boxrule=0pt,
		},
	underlay boxed title={%
			\path[fill=tcbcolframe] (title.south west)--(title.south east)
			to[out=0, in=180] ([xshift=5mm]title.east)--
			(title.center-|frame.east)
			[rounded corners=\kvtcb@arc] |-
			(frame.north) -| cycle;
		},
	#1
}{th}
\newtcbtheorem[use counter=commonbox]{rappel}{Rappel }%
{
	enhanced,
	colback=white,
	colframe=mygr,
	attach boxed title to top left={yshift*=-\tcboxedtitleheight},
	fonttitle=\bfseries,
	title={#2},
	boxed title size=title,
	boxed title style={%
			sharp corners,
			rounded corners=northwest,
			colback=tcbcolframe,
			boxrule=0pt,
		},
	underlay boxed title={%
			\path[fill=tcbcolframe] (title.south west)--(title.south east)
			to[out=0, in=180] ([xshift=5mm]title.east)--
			(title.center-|frame.east)
			[rounded corners=\kvtcb@arc] |-
			(frame.north) -| cycle;
		},
	#1
}{th}
\newtcbtheorem[use counter=commonbox]{strategie}{Stratégie }%
{
	enhanced,
	colback=white,
	colframe=mygr,
	attach boxed title to top left={yshift*=-\tcboxedtitleheight},
	fonttitle=\bfseries,
	title={#2},
	boxed title size=title,
	boxed title style={%
			sharp corners,
			rounded corners=northwest,
			colback=tcbcolframe,
			boxrule=0pt,
		},
	underlay boxed title={%
			\path[fill=tcbcolframe] (title.south west)--(title.south east)
			to[out=0, in=180] ([xshift=5mm]title.east)--
			(title.center-|frame.east)
			[rounded corners=\kvtcb@arc] |-
			(frame.north) -| cycle;
		},
	#1
}{th}
\newtcbtheorem[use counter=commonbox]{outil}{Outil }%
{
	enhanced,
	colback=white,
	colframe=mygr,
	attach boxed title to top left={yshift*=-\tcboxedtitleheight},
	fonttitle=\bfseries,
	title={#2},
	boxed title size=title,
	boxed title style={%
			sharp corners,
			rounded corners=northwest,
			colback=tcbcolframe,
			boxrule=0pt,
		},
	underlay boxed title={%
			\path[fill=tcbcolframe] (title.south west)--(title.south east)
			to[out=0, in=180] ([xshift=5mm]title.east)--
			(title.center-|frame.east)
			[rounded corners=\kvtcb@arc] |-
			(frame.north) -| cycle;
		},
	#1
}{th}
\newtcbtheorem[use counter=commonbox]{but}{Buts du chapitre }%
{
	enhanced,
	colback=white,
	colframe=mygr,
	attach boxed title to top left={yshift*=-\tcboxedtitleheight},
	fonttitle=\bfseries,
	title={#2},
	boxed title size=title,
	boxed title style={%
			sharp corners,
			rounded corners=northwest,
			colback=tcbcolframe,
			boxrule=0pt,
		},
	underlay boxed title={%
			\path[fill=tcbcolframe] (title.south west)--(title.south east)
			to[out=0, in=180] ([xshift=5mm]title.east)--
			(title.center-|frame.east)
			[rounded corners=\kvtcb@arc] |-
			(frame.north) -| cycle;
		},
	#1
}{th}
\newtcbtheorem[use counter=commonbox]{propriete}{Propriété }%
{
	enhanced,
	colback=white,
	colframe=mygr,
	attach boxed title to top left={yshift*=-\tcboxedtitleheight},
	fonttitle=\bfseries,
	title={#2},
	boxed title size=title,
	boxed title style={%
			sharp corners,
			rounded corners=northwest,
			colback=tcbcolframe,
			boxrule=0pt,
		},
	underlay boxed title={%
			\path[fill=tcbcolframe] (title.south west)--(title.south east)
			to[out=0, in=180] ([xshift=5mm]title.east)--
			(title.center-|frame.east)
			[rounded corners=\kvtcb@arc] |-
			(frame.north) -| cycle;
		},
	#1
}{th}
\newtcbtheorem[number within=commonbox]{definition}{Définition }%
{
	enhanced,
	colback=white,
	colframe=mygr,
	attach boxed title to top left={yshift*=-\tcboxedtitleheight},
	fonttitle=\bfseries,
	title={#2},
	boxed title size=title,
	boxed title style={%
			sharp corners,
			rounded corners=northwest,
			colback=tcbcolframe,
			boxrule=0pt,
		},
	underlay boxed title={%
			\path[fill=tcbcolframe] (title.south west)--(title.south east)
			to[out=0, in=180] ([xshift=5mm]title.east)--
			(title.center-|frame.east)
			[rounded corners=\kvtcb@arc] |-
			(frame.north) -| cycle;
		},
	#1
}{th}
\newtcbtheorem[number within=commonbox]{exemples}{Exemples }%
{
	enhanced,
	colback=white,
	colframe=mygr,
	attach boxed title to top left={yshift*=-\tcboxedtitleheight},
	fonttitle=\bfseries,
	title={#2},
	boxed title size=title,
	boxed title style={%
			sharp corners,
			rounded corners=northwest,
			colback=tcbcolframe,
			boxrule=0pt,
		},
	underlay boxed title={%
			\path[fill=tcbcolframe] (title.south west)--(title.south east)
			to[out=0, in=180] ([xshift=5mm]title.east)--
			(title.center-|frame.east)
			[rounded corners=\kvtcb@arc] |-
			(frame.north) -| cycle;
		},
	#1
}{th}
\newtcbtheorem[number within=commonbox]{exemple}{Exemple }%
{
	enhanced,
	colback=white,
	colframe=mygr,
	attach boxed title to top left={yshift*=-\tcboxedtitleheight},
	fonttitle=\bfseries,
	title={#2},
	boxed title size=title,
	boxed title style={%
			sharp corners,
			rounded corners=northwest,
			colback=tcbcolframe,
			boxrule=0pt,
		},
	underlay boxed title={%
			\path[fill=tcbcolframe] (title.south west)--(title.south east)
			to[out=0, in=180] ([xshift=5mm]title.east)--
			(title.center-|frame.east)
			[rounded corners=\kvtcb@arc] |-
			(frame.north) -| cycle;
		},
	#1
}{th}
\newtcbtheorem[number within=commonbox]{questions}{Questions guidantes }%
{
	enhanced,
	colback=white,
	colframe=mygr,
	attach boxed title to top left={yshift*=-\tcboxedtitleheight},
	fonttitle=\bfseries,
	title={#2},
	boxed title size=title,
	boxed title style={%
			sharp corners,
			rounded corners=northwest,
			colback=tcbcolframe,
			boxrule=0pt,
		},
	underlay boxed title={%
			\path[fill=tcbcolframe] (title.south west)--(title.south east)
			to[out=0, in=180] ([xshift=5mm]title.east)--
			(title.center-|frame.east)
			[rounded corners=\kvtcb@arc] |-
			(frame.north) -| cycle;
		},
	#1
}{th}
\makeatother

% corps
\newcommand{\R}{\mathbb{R}}
\newcommand{\Rnn}{\mathbb{R}^{2n}}
\newcommand{\Z}{\mathbb{Z}}
\newcommand{\N}{\mathbb{N}}
\newcommand{\Q}{\mathbb{Q}}

% domain
\newcommand{\D}{\mathcal{D}}
% for calligraphic C
\usepackage{calrsfs}
\newcommand{\C}{\mathcal{C}}

% date
\usepackage{advdate}

% ensembles tq. 
\newcommand{\xRtq}[1]{
	$\left\{ x \in \R \text{ tq. } #1 \right\}$
}

% vabs
\newcommand{\vabs}[1]{
	\left| #1 \right|
}

%pinfty minfty
\newcommand{\pinfty}{{+}\infty}
\newcommand{\minfty}{{-}\infty}

% plots
\usepackage{pgfplots}

%virgules
\usepackage{icomma}
\pgfplotsset{/pgf/number format/use comma}

%subfigures
\usepackage{subcaption}

%hyperlink footnote
\usepackage{hyperref}

%wider tabulars
\def\arraystretch{2}
\setlength\tabcolsep{15pt}

% tableaux var, signe
\usepackage{tkz-tab}


\AdvanceDate[1]

\begin{document}
\pagestyle{fancy}
\fancyhead[L]{Première}
\fancyhead[C]{\textbf{Phénomènes aléatoires : approfondissements \ifsolutions -- Solutions \fi}}
\fancyhead[R]{\today}

\exe{
	L'univers associé à une expérience aléatoire est $\{ a, b, c\}$.
	La loi de probabilité $P$ vérifie $P(a) = t^2$, $P(b) = -t$, et $P(c) = \frac14$, pour un réel $t \in \R$.
	
	Développer le carré $\left(t-\frac12\right)^2$ et déterminer $t$.
}{
	On développe le carré à l'aide de l'identité remarquable
		\[ (a-b)^2 = a^2 + b^2 - 2ab, \]
	où, ici, on a $a=t$ et $b=\frac12.$
		\begin{align*}
			\left(t-\dfrac12\right)^2 &= t^2 + \left(\dfrac12\right)^2 - 2 \cdot t \cdot \dfrac12 \\
									&= t^2 + \dfrac14 - t
		\end{align*}
	On cherche désormais le $t\in\R$ pour lequel $P$ est une loi de probabilité. 
	Un loi vérifie les deux propriétés suivantes :
		\begin{itemize}
			\item $P(\omega) \in [0;1]$ pour chaque issue $\omega \in \Omega$ ; et
			\item $P(\Omega) = 1$.
		\end{itemize}
	La deuxième identité donne donc
		\begin{align*}
			P(a) + P(b) + P(c) = 1 && \iff && t^2 - t + \dfrac14 = 1.
		\end{align*}
	Le carré développé nous permet d'écrire
		\[ \left(t-\dfrac12\right)^2 = 1, \]
	et donc
		\[ \left|t-\dfrac12\right| = \sqrt{1} = 1, \]
	en utilisant le fait que $\sqrt{x^2} = |x|.$
	L'expression à l'intérieur de la valeur absolue est donc soit $+1$, soit $-1$, et on a donc deux alternatives :
		\begin{align*}
			t-\dfrac12 = 1 && \text{ ou } && t - \dfrac12 = -1 \\
			t = \dfrac32 && \text{ ou } && t = -\dfrac12.
		\end{align*}
	Pour s'entraîner à ce genre de résolution, voir la feuille d'exercices Fonctions 3.
	
	Comme les probabilités sont des nombres entre $0$ et $1$, on peut écarter la première solution car $P(a) = t^2$ serait strictement supérieur à $1$, et $P(b) = -t$, serait strictement négatif.
	Il ne reste donc que $t = -\frac12$, qui donne le tableau de probabilités suivant.
	\begin{center}
	\begin{tabular}{|c|c|c|c|} \hline
		Issue & $a$ & $b$ & $c$ \\ \hline
		Probabilité & $\frac14$ & $\frac12$ & $\frac14$ \\ \hline
	\end{tabular}
	\end{center}
}

\exe{
	On lance $3$ fois de suite une pièce de monnaie bien équilibrée.
	On note par $P$ (pile) ou $F$ (face) le résultat de chaque lancer.
	Donner $\Omega$, l'univers de l'expérience, et $|\Omega|$ son cardinal.
	
	Calculer la probabilité des événements suivants.
		\begin{enumerate}
			\item Obtenir $3$ fois face.
			\item Le deuxième lancer donne pile.
			\item Le troisième lancer est différent du premier.
			\item On obtient au moins une fois pile.
		\end{enumerate}
}{
	Comme les lancers sont distingués, il y a $8$ issues possibles.
		\[ \Omega = \{ FFF ; FFP ; FPF ; FPP ; PFF ; PFP ; PPF ; PPP \} \]
	Le cardinal de l'univers est $|\Omega| = 8$.
	On aurait pû aussi noter les issues avec des parenthèses, p.ex. $(F;P;F)$, mais pas avec des accolades $\{ \cdot \}$.
	\begin{enumerate}
		\item
		Les probabilités se multiplient, on a donc $P(FFF) = \dfrac12 \times \dfrac12 \times \dfrac12 = \left( \dfrac12 \right)^3 = \dfrac18$.
		En fait, nous sommes en situation d'équiprobabilité, et $|\Omega| = 8$.
		\item 
		Les lancers sont indépendants (le résultat des précédents n'influe en rien celui des prochains), donc la probabilité que le deuxième donne pile est $\frac12$.
		On aurait également pu sommer la probabilité des événements concernés :
			\[ P(FPF) + P(PPF) +  P(FPP) + P(PPP) = \dfrac48 = \dfrac12. \]
		\item
		Il y a quatre issues qui correspondent à cet événement. 
			\[ P(PFF) + P(PPF) + P(FFP) + P(FPP) = \dfrac48 = \dfrac12. \]
		On aurait pû tout aussi bien supprimer le deuxième lancer, car il n'a aucune influence sur les autres --- cela donne le même résultat.
		\item 
		Lorsqu'on étudie un événement de la forme \og au moins [\dots] \fg, il est toujours utile de passer par le complémentaire.
		L'événement complémentaire est \og on obtient trois fois face \fg, dont la probabilité est $P(FFF) = \dfrac18.$
		La probabilité recherché est donc $1-\dfrac18 = \dfrac78$.
		
		On aurait également pû énumérer les issues de l'événement et sommer leur probabilité. 
		Seul l'événement $FFF$ n'apparaît alors pas dans cette somme qui vaut $\dfrac78$.
	\end{enumerate}
}

\exe{
	On lance deux D$6$ équilibrés, dés à $6$ faces l'un après l'autre. Les deux dés sont distinguables car de couleurs différentes.
	\begin{enumerate}
		\item Donner l'univers $\Omega$ et son cardinal $|\Omega|$. Est-ce une situation d'équiprobabilité ?
		\item Quelle est la probabilité d'obtenir un double $6$ ?
		\item Quelle est la probabilité qu'après $10$ tels lancers, on obtienne au moins une fois un double $6$ ?
	\end{enumerate}
}{
	\begin{enumerate}
		\item Donner l'univers $\Omega$ et son cardinal $|\Omega|$. Est-ce une situation d'équiprobabilité ?
		L'univers est formé par tous les couples $(a ;b)$ de résultats.
		On utilise des parenthèses ici car on distingue le premier du deuxième lancer.
			\[ \Omega = \left\{ (a ; b) \text{ où } a, b \in \{ 1 ; 2 ; 3 ; 4 ;5 ; 6 \} \right\}, \]
		de cardinal $|\Omega| = 6 \times 6 = 36$.
		
		La situation est bien d'équiprobabilité car il y a $36$ issues et chacune admet comme probabilité $\dfrac16 \times \dfrac16 = \dfrac1{36}$, car les dés sont bien équilibrés.
		
		\item 
		La probabilité de l'issue $(6;6)$ est $\dfrac1{36}$ par équiprobabilité.
		
		\item Quelle est la probabilité qu'après $10$ tels lancers, on obtienne au moins une fois un double $6$ ?
		Lorsqu'on étudie un événement de la forme \og au moins [\dots] \fg, il est toujours utile de passer par le complémentaire.
		L'événement complémentaire est \og obtenir aucun double $6$ \fg, dont la probabilité est
			\[ \left( \dfrac{35}{36} \right)^{10} \approx 0,75. \]
		En effet, on peut construire un arbre réduit à deux événements  : \og double 6 \fg (probabilité $\frac1{36}$) et \og pas double 6 \fg (probabilité $\frac{35}{36}$), de profondeur $10$.
		La feuille qui correspond à \og obtenir aucun double $6$ \fg est obtenue en obtenant \og pas double 6 \fg dix fois dans l'arbre.
		
		La probabilité de l'événement \og on obtient aucun double $6$ \fg est donc
			\[ \left( \dfrac{35}{36} \right)^{10} \approx 0,75. \]
		On conclut en faisant $1-0,75 = 0,25 = \frac14$, probabilité approximative qu'au moins un des $10$ lancers donne un double $6$.
	\end{enumerate}
}

\exe{
	Une expérience aléatoire à deux épreuves d'univers $\{A ; B ; C ; D\}$ admet un arbre de probabilité binaire comme ci-dessous, où $t\in\R$ est un paramètre réel encore inconnu.

	\begin{center}
	\begin{tikzpicture}
		% depth 1
		\foreach \i in {-3, 3}
		\draw[-, thick, black] (0,0) node {$\bullet$} -- (\i,-2);
		% depth 2
		\foreach \i in {-3, 3} \foreach \j in {-1, 1}
			\draw[-, thick, black] (\i,-2) node {$\bullet$} -- (\i+\j,-4) node {$\bullet$};
		
		% probas
		\draw (-1.5,-1) node[above left] {$-t$};
		\draw (1.5,-1) node[above right] {$1+t$};
		
		\draw (-3.5,-3) node[above left] {$-t$};
		\draw (-2.5,-3) node[above right] {$1+t$};
		
		\draw (2.5,-3) node[above left] {$-t$};
		\draw (3.5,-3) node[above right] {$1+t$};
		
		% issues
		\draw (-4,-4) node[below] {A};
		\draw (2,-4) node[below] {B};
		\draw (-2,-4) node[below] {C};
		\draw (4,-4) node[below] {D};
	\end{tikzpicture}
	\end{center}
	
	\begin{enumerate}
		\item Montrer qu'on a forcément 
			\[ -1 \leq t \leq 0. \]
		\item Déterminer le paramètre $t$ tel que
			\[ P(D) = \frac19. \]
	\end{enumerate}
}{
	

	\begin{enumerate}
		\item Un probabilité est nécessairement entre $0$ et $1$, donc
			\[ 0 \leq -t \leq 1, \]
		et multiplier par un nombre négatif inverse l'ordre des inégalités :
			\[ -1 \leq t \leq 0. \]
		\item On a la suite d'égalités suivante.
			\begin{align*}
				P(D) &= \dfrac19 \\
				(1+t)^2 &= \dfrac19 \\
				| 1 + t | &= \sqrt{\dfrac19} = \dfrac{\sqrt{1}}{\sqrt{9}} = \dfrac13,
			\end{align*}
		où à la dernière ligne on a utilisé que $\sqrt{x^2} = |x|$ et que $\sqrt{\dfrac{a}{b}} = \dfrac{\sqrt{a}}{\sqrt{b}}$ pour $b$ non nul.
		Les propriétés des racines sont décrites sur un feuille dédiée (semaine 9 décembre sur Éléa).
		
		On continue avec
			\begin{align*}
				1+t = \dfrac13 && \text{ou} && 1+t = -\dfrac13 \\
				t = -\dfrac23 && \text{ou} && t = -\dfrac43
			\end{align*}
		Pour revoir la résolution des équations du type $E^2 = a$, revoir la feuille Fonctions 3 (semaine du 25 novembre sur Éléa).
		
		La première question implique qu'on a nécessairement $t=-\dfrac23$.
		On s'assurera que les probabilités de l'arbre ont bien un sens (ce sont des nombres entre $0$ et $1$).
	\end{enumerate}

}

\end{document}