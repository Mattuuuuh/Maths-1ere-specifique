\documentclass[12pt]{paper}
\usepackage[french]{babel}
\usepackage[
a4paper,
margin=2cm,
nomarginpar,% We don't want any margin paragraphs
]{geometry}
\usepackage{fancyhdr}
\usepackage{array}
\usepackage{amsmath,amsfonts,amsthm,amssymb,mathtools,enumitem}
\newcolumntype{P}[1]{>{\centering\arraybackslash}p{#1}}


\usepackage{stackengine}
\newcommand\xrowht[2][0]{\addstackgap[.5\dimexpr#2\relax]{\vphantom{#1}}}

% theorems

\theoremstyle{plain}
\newtheorem{theorem}{Th\'eor\`eme}
\newtheorem*{sol}{Solution}
\theoremstyle{definition}
\newtheorem{ex}{Exercice}


% corps
\newcommand{\C}{\mathcal{C}}
\newcommand{\R}{\mathbb{R}}
\newcommand{\Rnn}{\mathbb{R}^{2n}}
\newcommand{\Z}{\mathbb{Z}}
\newcommand{\N}{\mathbb{N}}
\newcommand{\Q}{\mathbb{Q}}

% domain
\newcommand{\D}{\mathbb{D}}


% date
\usepackage{advdate}
\AdvanceDate[0]

% plots
\usepackage{tikz}
\usepackage{multicol}
\usepackage{pgfplots}

% for calligraphic C
\usepackage{calrsfs}

% euro
\usepackage{lmodern,textcomp}


% SOLUTION SWITCH
\newif\ifsolutions
				\solutionstrue
				\solutionsfalse

\ifsolutions
	\newcommand{\exe}[2]{
		\begin{ex} #1  \end{ex}
		\begin{sol} #2 \end{sol}
	}
\else
	\newcommand{\exe}[2]{
		\begin{ex} #1  \end{ex}
	}
	
\fi

\begin{document}
\pagestyle{fancy}
\fancyhead[L]{Première}
\fancyhead[C]{\textbf{Fonctions affines 4 \ifsolutions -- Solutions \fi}}
\fancyhead[R]{\today}

\exe{[Interpolation]
	Pour chacune des paires de points $A,B \in \R^2$ suivantes, calculer les paramètres de la fonction affine $f$ dont la courbe passe par ces deux points.
	
	\begin{multicols}{2}
	\begin{enumerate}
		\item $A(x_A;y_A), B(x_B; y_B)$.
		    \item $A(1; 3), B(4; 5)$
		
		    \item $A(0; -2), B(3; 4)$
		
		    \item $A(-3; -1), B(-1; 2)$
		
		    \item $A(2; 7), B(5; 7)$
		
		    \item $A(-4; 6), B(2; -3)$

	\end{enumerate}
	\end{multicols}
}
{
	On utilise la convention $a, b \in \R$ pour désigner respectivement le coefficient directeur et l'ordonnée à l'origine de la fonction affine $f$.
	$f(x) = ax+b$ pour tout $x\in\R$.

	\begin{enumerate}
		\item 
			D'après le cours,
			\begin{align*}
				a = \dfrac{y_A - y_B}{x_A - x_B}, && b = \dfrac{x_A y_B - x_B y_A}{x_A - x_B}.
			\end{align*}
		 \item 
		    \begin{align*}
		        a &= \frac{3-5}{1-4} = \frac{2}{3}, & b &= \frac{1 \cdot 5 - 4 \cdot 3}{1 - 4} = -\dfrac73.
		    \end{align*}
		
		    \item 
		    \begin{align*}
		        a &= \frac{-2 - 4}{0-3} = 2, & b &= \frac{0 \cdot 4 - 3 \cdot (-2)}{0 - 3} = -2.
		    \end{align*}
		    
		    \item
		    \begin{align*}
		        a &= \frac{-1 - 2}{-3 - (-1)} = \frac{3}{2}, & b &= \frac{-3 \cdot 2 - (-1) \cdot (-1)}{-3 - (-1)} = \dfrac72.
		    \end{align*}
		
		    \item
		    \begin{align*}
		        a &= \frac{7 - 7}{2 - 5} = 0, & b &= \frac{2 \cdot 7 - 5 \cdot 7}{2 - 5} = 7.
		    \end{align*}
		
		    \item
		    \begin{align*}
		        a &= \frac{6-(-3)}{-4 - 2} = -\frac{3}{2}, & b &= \frac{-4 \cdot (-3) - 2 \cdot 6}{-4 - 2} = 0.
		    \end{align*}

	\end{enumerate}


}

\exe{[Intersection]
	Pour chacune des paires de fonctions affines $f,g$, calculer l'intersection des droites $\C_f \cap \C_g$.
	
	\begin{multicols}{2}
	\begin{enumerate}
		\item $f(x) = 3-2x, g(x) = -x+1$ $(x\in\R)$.
		\item $f(x) = -7x -2, g(x) = 4-2x$ $(x\in\R)$.
		\item $f(x) = 3+7x, g(x) = 7x+3$ $(x\in\R)$.
		\item $f(x) = x, g(x) = 3$ $(x\in\R)$.
		\item $f(x) = -123, g(x) = 2x+1$ $(x\in\R)$.
		\item $f(x) = 284, g(x) = -\dfrac13$ $(x\in\R)$.
	\end{enumerate}
	\end{multicols}
}{

	\begin{enumerate}
		\item 
			On cherche un point $P(x_P, y_P) \in \R^2$ vérifiant les deux équations
				\begin{align*}
					y_P = f(x_P) && \text{ et } && y_P = g(x_P)
				\end{align*}
			Comme bien sûr $y_P = y_P$, on peut résoudre l'équation
				\[ f(x_P) = g(x_P) \]
			pour trouver $x_P$.
			Ici, on pose calmement
				\begin{align*}
					f(x_P) &= g(x_P) \\
					3 - 2x_P &= -x_P + 1 \\
					x_P &= 2,
				\end{align*}
			ce qui nous fournit $x_P = 2$.
			Reste plus qu'à utiliser que $y_P = f(x_P) = g(x_P)$ pour calculer $y_P$.
			
			D'une part, 
				\[ f(x_P) = f(2) = 3-2\cdot2 = -1. \]
			D'autre part, on aurait pû calculer
				\[ g(x_P) = g(2) = -2 + 1 = -1. \]
			Rien de surprenant à ce qu'on trouve la même valeur $y_P = -1$, car $x_P$ a été trouvé vérifiant $f(x_P) = g(x_P)$.
			
			En conclusion, $\C_f \cap \C_g = \{ (2;-1) \}$.
			
		\item 
			Similairement,
			\begin{align*}
				f(x_P) &= g(x_P) \\
				-7x_P -2 & = 4-2x_P \\
				5x_P &= -6 \\
				x_P &= -\dfrac65,
			\end{align*}
			et $y_P = g\left(-\dfrac65\right) = 4-2\left(-\dfrac65\right) = 4+\dfrac{12}5 = \dfrac{32}5$.
			
			Donc $\C_f \cap \C_g = \left\{ \left(-\dfrac65;\dfrac{32}5 \right) \right\}$.
		
		
		\item 
			On remarque que $f=g$ et donc que les droites associées aux fonctions sont confondues.			
			Donc $\C_f \cap \C_g = \C_f = \C_g$.
			
		\item
			On pose
				\begin{align*}
				f(x_P) &= g(x_P) \\
				x_P & = 3,
			\end{align*}
			ce qui donne ensuite $y_P = f(3) = 3 = g(3)$.
			
			Donc $\C_f \cap \C_g = \left\{ \left( 3; 3 \right) \right\}$.
			
			
		\item
			On pose
				\begin{align*}
				f(x_P) &= g(x_P) \\
				-123 & = 2x_P+1 \\
				x_P = -62,
			\end{align*}
			ce qui donne ensuite $y_P = f(-62) = -123$.
			
			Donc $\C_f \cap \C_g = \left\{ \left( -62; -123 \right) \right\}$.
			
			
		\item
			On pose
				\begin{align*}
				f(x_P) &= g(x_P) \\
				284 & = -\dfrac13,
			\end{align*}
			équation évidemment fausse.
			Il n'existe donc aucun $P \in \C_f \cap \C_g$. 
			En d'autres termes, $\C_f \cap \C_g = \emptyset$, l'ensemble vide.
			
			En fait, les deux fonctions sont constantes et donc en particulier parallèles (les coefficients directeurs sont tous les deux égaux à $0$).
			Or comme les ordonnées à l'origine des fonctions ne sont pas les mêmes, le théorème du cours nous donne directement le résultat.
	\end{enumerate}

}

\exe{[Parallélisme]
	Pour chacun des couples de point $P$ et fonction affine $f$, trouver la fonction affine $g$ telle que $\C_f$ et $\C_g$ soient parallèles et que $P \in \C_g$.
	
	
	\begin{multicols}{2}
	\begin{enumerate}
		\item $P=(-1;-2)$ et $f(x) = \dfrac13 - 8x$ $(x\in\R)$.
		\item $P=(-3;-4)$ et $f(x) = \dfrac13 x - 7^{8^9}$ $(x\in\R)$.
		\item $P=(-5;-6)$ et $f(x) = -\dfrac{8}{11}x$ $(x\in\R)$.
		\item $P=(7;8)$ et $f(x) = -7^{6^{5}}$ $(x\in\R)$.
	\end{enumerate}
	\end{multicols}
}{
	On pose $a,b\in\R$ les paramètres de la fonction affine $g(x)=ax+b$, $x\in\R$.
	\begin{enumerate}
		\item 
			L'information de parallélisme donne immédiatement $a=-8$.
			D'où $g(x) = -8x+b$ pour tout $x\in\R$.
			
			Reste à trouver $b$ avec l'information d'appartenance.
			Celle-ci est équivalente à la contrainte
				\begin{align*}
					-2 &= g(-1) \\
					-2 &= -8\cdot(-1) + b \\
					b &= -10
				\end{align*}
			D'où $g(x) = -8x -10$ ($x\in\R$) est la fonction affine recherchée.
		
		
		\item 
			L'information de parallélisme donne immédiatement $a=\dfrac13$.
			D'où $g(x) = \dfrac13x+b$ pour tout $x\in\R$.
			
			Reste à trouver $b$ avec l'information d'appartenance.
			Celle-ci est équivalente à la contrainte
				\begin{align*}
					-4 &= g(-3) \\
					-4 &= -1 + b \\
					b &= -3
				\end{align*}
			D'où $g(x) = \dfrac13x-3$ ($x\in\R$) est la fonction affine recherchée.
			
		\item 
			L'information de parallélisme donne immédiatement $a=-\dfrac8{11}$.
			D'où $g(x)=-\dfrac8{11}x + b$ pour tout $x\in\R$.
			
			Reste à trouver $b$ avec l'information d'appartenance.
			Celle-ci est équivalente à la contrainte
				\begin{align*}
					-6 &= g(-5) \\
					-6 &=  \dfrac{40}{11} + b \\
					b &= -\dfrac{106}{11}
				\end{align*}
			D'où $g(x) = -\dfrac8{11}x -\dfrac{106}{11}$ ($x\in\R$) est la fonction affine recherchée.
			
		\item 
			L'information de parallélisme donne immédiatement $a=0$.
			D'où $g(x) = b$ pour tout $x\in\R$.
			
			Reste à trouver $b$ avec l'information d'appartenance.
			Celle-ci est équivalente à la contrainte
				\begin{align*}
					8 &= g(7) \\
					b &= 8
				\end{align*}
			D'où $g(x) = 8$ ($x\in\R$) est la fonction affine recherchée.
	\end{enumerate}

}


\exe{[Lecture graphique]

	Déterminer les paramètres des fonction affines $f,g,h$ dont les courbes sont représentées ci-dessous.

	\begin{center}
		\begin{tikzpicture}[>=stealth, scale=1.5]
		\begin{axis}[xmin = -10, xmax=10, ymin=-10, ymax=10, axis x line=middle, axis y line=middle, axis line style=<->, xlabel={}, ylabel={}, xtick = {-10, -8, ..., 8, 10}, ytick = {-10, -8, ..., 8, 10}, grid=both]
		
			\addplot[red, thick, domain =-9:9, samples=2] {-x/3 + 1}  node[below=6pt] {$(\mathcal{C}_f)$};
			\addplot[red, thick, dotted, domain =-10:-9, samples=2] {-x/3 + 1} ;
			\addplot[red, thick, dotted, domain =9:10, samples=2] {-x/3 + 1};
		
		
			\addplot[green, thick, domain =-5:7, samples=2] {3*x/2-2}  node[midway, below=18pt, right=-8pt] {$(\mathcal{C}_g)$};
			\addplot[green, thick, dotted, domain =-6:-5, samples=2] {3*x/2-2} ;
			\addplot[green, thick, dotted, domain =7:8, samples=2] {3*x/2-2};
		
		
			\addplot[black, thick, domain =-9:9, samples=2] {-x/6 + 5}  node[above=10pt, left] {$(\mathcal{C}_h)$};
			\addplot[black, thick, dotted, domain =-10:-9, samples=2] {-x/6 + 5} ;
			\addplot[black, thick, dotted, domain =9:10, samples=2] {-x/6 + 5};
		
			
		\end{axis}
	\end{tikzpicture}
	\end{center}
}{
	Les fonctions sont données, pour tout $x\in\R$, par
	\begin{align*}
			f(x) = -\dfrac13 x + 1, &&
			g(x) = \dfrac32 x - 2, &&
			h(x) = -\dfrac16 x + 5.
	\end{align*}
}

\end{document}