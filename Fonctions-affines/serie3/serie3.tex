\documentclass[12pt]{paper}
\usepackage[french]{babel}
\usepackage[
a4paper,
margin=2cm,
nomarginpar,% We don't want any margin paragraphs
]{geometry}
\usepackage{fancyhdr}
\usepackage{array}
\usepackage{amsmath,amsfonts,amsthm,amssymb,mathtools,}
\newcolumntype{P}[1]{>{\centering\arraybackslash}p{#1}}


\usepackage{stackengine}
\newcommand\xrowht[2][0]{\addstackgap[.5\dimexpr#2\relax]{\vphantom{#1}}}

% theorems

\theoremstyle{plain}
\newtheorem{theorem}{Th\'eor\`eme}
\newtheorem*{sol}{Solution}
\theoremstyle{definition}
\newtheorem{ex}{Exercice}


% corps
\newcommand{\C}{\mathcal{C}}
\newcommand{\R}{\mathbb{R}}
\newcommand{\Rnn}{\mathbb{R}^{2n}}
\newcommand{\Z}{\mathbb{Z}}
\newcommand{\N}{\mathbb{N}}
\newcommand{\Q}{\mathbb{Q}}

% domain
\newcommand{\D}{\mathbb{D}}


% date
\usepackage{advdate}
\AdvanceDate[0]

% plots
\usepackage{tikz}
\usepackage{multicol}
\usepackage{pgfplots}

% for calligraphic C
\usepackage{calrsfs}

% euro
\usepackage{lmodern,textcomp}


% SOLUTION SWITCH
\newif\ifsolutions
				\solutionstrue
				\solutionsfalse

\ifsolutions
	\newcommand{\exe}[2]{
		\begin{ex} #1  \end{ex}
		\begin{sol} #2 \end{sol}
	}
\else
	\newcommand{\exe}[2]{
		\begin{ex} #1  \end{ex}
	}
	
\fi

\begin{document}
\pagestyle{fancy}
\fancyhead[L]{Première}
\fancyhead[C]{\textbf{Fonctions affines 3 $\star$ \ifsolutions -- Solutions  \fi}}
\fancyhead[R]{\today}


\exe{[$\star$]
    Soit $f(x) = ax + b (x \in \R)$ une fonction affine de paramètres $a,b\in\R$ et $P(x_P;y_P) \in \R^2$ un point du plan.

    \begin{enumerate}
        \item 
        Montrer que la fonction affine $g$ donnée par
            \[ g(x) = a(x-x_P) + y_P, \]
        pour tout $x\in\R$ vérifie que $\C_f$ est parallèle à $\C_g$ et que $P \in \C_g$.
        \item
        Montrer que si $P\in\C_f$, alors $f=g$.
    \end{enumerate}

}{}

\exe{[$\star$]
    Soient $f(x) = 3x^2 + 17x - 11$ et $g(x) = 2x^2 + 17x - 10$ pour tout $x \in \R$ deux fonctions quadratiques.

    Déterminer entièrement $\C_f \cap \C_g$.
}{}

\exe{[$\star$]
    Soient $f(x) = x$ et $g(x) = x^3 - 3x^2 + 4x - 1$ pour tout $x \in \R$ deux fonctions polynomiales.

    \begin{enumerate}
        \item 
        Montrer que $(x-1)^3 = x^3 - 3x^2 + 3x - 1$ pour tout $x\in\R$.
        \item
        Déterminer entièrement $\C_f \cap \C_g$.
        \item
        Créer une fonction polynomiale $h$ de degré $5$ telle que $\C_f \cap \C_h = \{ (1;1) \}.$
    \end{enumerate}
}{}

\exe{[$\star$]
    Soient $f(x) = -2x^2 + 7x + 2$ et $g(x) = -3x^2 + 2x +16$ pour tout $x \in \R$ deux fonctions quadratiques.
    
    \begin{enumerate}
        \item 
        Déterminer l'autre solution de $f(x)-g(x)=0$, sachant que $x=2$ en est une et donc que $f(x)-g(x) = (x-2)h(x)$, où $h$ est affine.
        \item
        Déterminer entièrement $\C_f \cap \C_g$.
        \item
        Créer une fonction polynomiale $F$ de degré $2$ telle que $\C_f \cap \C_F = \{ (2;0), (-1; -7) \}.$
    \end{enumerate}
}{}

\exe{[$\star$]
    Soit $f(x) = ax+b$ et $g(x) = a'x + b'$ pour tout $x \in \R$ deux fonctions affines de paramètres $a,a',b,b' \in\R$.

    Montrer que si $a\neq a'$, alors
        \[ \C_f \cap \C_g = \left\{ \left(\ \dfrac{b-b'}{a-a'} ; \dfrac{ab' - ba'}{a-a'} \right) \right\}. \]
    Comparer avec les réultats obtenus dans la série d'exercices précédente.
}{}



\end{document}
