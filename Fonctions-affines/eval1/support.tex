\documentclass[14pt]{beamer}
\usepackage[french]{babel}


\usetheme{CambridgeUS}
\usecolortheme{rose}
\beamertemplatenavigationsymbolsempty

\usepackage{amsmath,amsfonts,calrsfs}

\theoremstyle{plain}
\newtheorem*{dfn}{Définition}
\newtheorem*{thm}{Théorème}


\newcommand{\R}{\mathbb{R}}
\newcommand{\C}{\mathcal{C}}

\begin{document}

\begin{frame}

\begin{dfn}[Fonction affine]
	Un fonction $f$ est dite \emph{affine} s'il existe deux paramètres $a,b\in\R$ tels que
		\[ f(x) = ax + b \qquad \text{ pour tout } x\in\R. \]
	$a$ est le \emph{coefficient directeur} et $b$ l'\emph{ordonnée à l'origine}.
\end{dfn}

\begin{thm}[Propriété fondamentale]
	Soit $f$ une fonction, $\C_f$ sa courbe représentative, et $(x;y)\in\R^2$ un point du plan.
		\begin{align*}
			(x;y) \in \C_f && \iff && y = f(x).
		\end{align*}
\end{thm}

\end{frame}

\end{document}