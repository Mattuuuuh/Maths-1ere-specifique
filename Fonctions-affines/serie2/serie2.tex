\documentclass[12pt]{paper}
\usepackage[french]{babel}
\usepackage[
a4paper,
margin=2cm,
nomarginpar,% We don't want any margin paragraphs
]{geometry}
\usepackage{fancyhdr}
\usepackage{array}
\usepackage{amsmath,amsfonts,amsthm,amssymb,mathtools,}
\newcolumntype{P}[1]{>{\centering\arraybackslash}p{#1}}


\usepackage{stackengine}
\newcommand\xrowht[2][0]{\addstackgap[.5\dimexpr#2\relax]{\vphantom{#1}}}

% theorems

\theoremstyle{plain}
\newtheorem{theorem}{Th\'eor\`eme}
\newtheorem*{sol}{Solution}
\theoremstyle{definition}
\newtheorem{ex}{Exercice}


% corps
\newcommand{\C}{\mathcal{C}}
\newcommand{\R}{\mathbb{R}}
\newcommand{\Rnn}{\mathbb{R}^{2n}}
\newcommand{\Z}{\mathbb{Z}}
\newcommand{\N}{\mathbb{N}}
\newcommand{\Q}{\mathbb{Q}}

% domain
\newcommand{\D}{\mathbb{D}}


% date
\usepackage{advdate}
\AdvanceDate[0]

% plots
\usepackage{tikz}
\usepackage{multicol}
\usepackage{pgfplots}

% for calligraphic C
\usepackage{calrsfs}

% euro
\usepackage{lmodern,textcomp}


% SOLUTION SWITCH
\newif\ifsolutions
				\solutionstrue
				%\solutionsfalse

\ifsolutions
	\newcommand{\exe}[2]{
		\begin{ex} #1  \end{ex}
		\begin{sol} #2 \end{sol}
	}
\else
	\newcommand{\exe}[2]{
		\begin{ex} #1  \end{ex}
	}
	
\fi

\begin{document}
\pagestyle{fancy}
\fancyhead[L]{Première}
\fancyhead[C]{\textbf{Fonctions affines 2 \ifsolutions -- Solutions \fi}}
\fancyhead[R]{\today}

\exe{[Interpolation]
	Pour chacune des paires de points $A,B \in \R^2$ suivantes, calculer les paramètres de la fonction affine $f$ dont la courbe passe par ces deux points.
	
	\begin{multicols}{2}
	\begin{enumerate}
		\item $A(x_A;y_A), B(x_B; y_B)$.
		\item $A(2;8), B(4,7)$.
		\item $A(4;7), B(2;8)$.
		\item $A(-3; -3), B(-2; -1)$.
		\item $A(2;5), B(-10; 5)$.
		\item $A(-3;4), B(12;-11)$.
	\end{enumerate}
	\end{multicols}
}
{
	On utilise la convention $a, b \in \R$ pour désigner respectivement le coefficient directeur et l'ordonnée à l'origine de la fonction affine $f$.
	$f(x) = ax+b$ pour tout $x\in\R$.

	\begin{enumerate}
		\item 
			D'après le cours,
			\begin{align*}
				a = \dfrac{y_A - y_B}{x_A - x_B}, && b = \dfrac{x_A y_B - x_B y_A}{x_A - x_B}.
			\end{align*}
		\item 
			\begin{align*}
				a = \dfrac{8-7}{2-4} = -\dfrac12, && b = \dfrac{2\cdot7 - 4\cdot8}{2-4} = 9.
			\end{align*}
			
			Visualisation : le point $A$ est à gauche du point $B$ car son abscisse est plus petite, mais est plus haut que le point $B$, car son ordonnée est plus grande. 
			La droite $(AB)$ doit donc être décroissante, ce qu'on a bien trouvé car $a<0$.
		\item
			\begin{align*}
				a = \dfrac{7-8}{4-2} = -\dfrac12, && b = \dfrac{4\cdot8 - 2\cdot7}{4-2} = 9.
			\end{align*}
			
		Les formules ne dépendent bien sûr pas de l'ordre de $A$ et $B$ car la droite $(AB)$ est égale à la droite $(BA)$.
					
		\item
			\begin{align*}
				a = \dfrac{-3-(-1)}{-3-(-2)} = 2, && b = \dfrac{(-3)\cdot(-1) - (-3)\cdot(-2)}{-3 - (-2)} = 3.
			\end{align*}
			
		\item
			\begin{align*}
				a = \dfrac{5-5}{2-(-10)} = 0, && b = \dfrac{2\cdot5 - 5 \cdot(-10)}{2-(-10)} = 5.
			\end{align*}
			
			Visualisation : les deux points sont à la même hauteur car leurs abscisses sont égales.
			La droite $(AB)$ est donc constante (ou horizontale), d'où $a=0$, et sa hauteur est celle des points, d'où $b=5$.
			
		\item $A(-3;4), B(12;-11)$.
			\begin{align*}
				a = \dfrac{4-(-11)}{-3-12} = -1, && b = \dfrac{(-3)\cdot(-11) - 4\cdot12}{-3-12} = 1.
			\end{align*}
			
			Vérification : on vérifie que $A$ et $B$ vérifient bien l'équation $y=f(x) = -x +1$ en remplaçant les valeurs par les coordonnées.
			Ainsi pour $A$ on a bien $y=4$ à gauche, et $f(-3) = -(-3) + 1 = 4$ à droite.
			Pour $B$, on a $y=-11$ à gauche, et $f(12) = -12 + 1 = -11$ à droite.
	\end{enumerate}


}

\exe{[Intersection]
	Pour chacune des paires de fonctions affines $f,g$, calculer l'intersection des droites $\C_f \cap \C_g$.
	
	\begin{multicols}{2}
	\begin{enumerate}
		\item $f(x) = 2x + 1, g(x) = -x+1$ $(x\in\R)$.
		\item $f(x) = -x + 1, g(x) = 2x + 1$ $(x\in\R)$.
		\item $f(x) = 3+7x, g(x) = 2$ $(x\in\R)$.
		\item $f(x) = 9-2x, g(x) = 17-x$ $(x\in\R)$.
		\item $f(x) = 2x+1, g(x) = 2x+1$ $(x\in\R)$.
		\item $f(x) = 2x+1, g(x) = 2x+2$ $(x\in\R)$.
	\end{enumerate}
	\end{multicols}
}{

	\begin{enumerate}
		\item $f(x) = 2x + 1, g(x) = -x+1$ $(x\in\R)$.
			On cherche un point $P(x_P, y_P) \in \R^2$ vérifiant les deux équations
				\begin{align*}
					y_P = f(x_P) && \text{ et } && y_P = g(x_P)
				\end{align*}
			Comme bien sûr $y_P = y_P$, on peut résoudre l'équation
				\[ f(x_P) = g(x_P) \]
			pour trouver $x_P$.
			Ici, on pose calmement
				\begin{align*}
					f(x_P) &= g(x_P) \\
					2x_P + 1 &= -x_P + 1 \\
					3x_P &= 0 \\
					x_P &= 0,
				\end{align*}
			ce qui nous fournit $x_P = 0$.
			Reste plus qu'à utiliser que $y_P = f(x_P) = g(x_P)$ pour calculer $y_P$.
			
			D'une part, 
				\[ f(x_P) = f(0) = 2\cdot0 + 1 = 1. \]
			D'autre part, on aurait pû calculer
				\[ g(x_P) = g(0) = -0 + 1 = 1. \]
			Rien de surprenant à ce qu'on trouve la même valeur $y_P = 1$, car $x_P$ a été trouvé vérifiant $f(x_P) = g(x_P)$.
			
			En conclusion, $\C_f \cap \C_g = \{ (0;1) \}$.
			
		\item 
			Similairement,
			\begin{align*}
				f(x_P) &= g(x_P) \\
				-x_P + 1 & = 2x_P + 1 \\
				x_P &= 0,
			\end{align*}
			et $y_P = f(0) = 1$.
			Sans surprise, $(0;1)$ est à nouveau le point d'intersection de $\C_f$ et de $\C_g$ car se sont les mêmes fonctions que ci-dessus mais échangées.
		
		\item 
			Similairement,
			\begin{align*}
				f(x_P) &= g(x_P) \\
				3 + 7x_P & = 2 \\
				7x_P &= -1 \\
				x_P &= \dfrac{-1}7,
			\end{align*}
			\[
				y_P = f(x_P) = 3+7\cdot \dfrac{-1}7  = 2.
			\]
			Il aurait été encore plus facile de calculer $g(x_P) = 2$ à la place !
			
			En conclusion, $\C_f \cap \C_g = \{ \left(-\dfrac72 ; 2\right) \}$.
		
		\item
			Similairement,
			\begin{align*}
				f(x_P) &= g(x_P) \\
				9-2x_P & = 17-x_P \\
				-8 &= x_P,
			\end{align*}
			\[
				y_P = f(x_P) = 9-2 \cdot(-8) = 25.
			\]
		
			En conclusion, $\C_f \cap \C_g = \{ (-8;25) \}$.
		
		\item 			
		Similairement,
			\begin{align*}
				f(x_P) &= g(x_P) \\
				2x_P + 1 & = 2x_P + 1 \\
				0 &= 0,
			\end{align*}
			cette équation étant évidemment vraie, tous les $x_P \in \R$ vérifient l'équation.
			\[
				y_P = f(x_P) = 2x_P + 1.
			\]
			N'importe quel $(x_P ; 2x_P+1) \in \R$ appartient donc à l'intersection des droites.
			
			En conclusion, $\C_f \cap \C_g = \{ (x_P ; 2x_P+1) \text{ tq. } x_P \in \R \} = \{ (x_P ; y_P) \in \R^2 \text{ tq. } y_P = 2x_P + 1 \} = \C_f = \C_g$.
			Il n'est pas surprenant qu'on trouve que toute la droite appartienne à l'intersection, car $f$ et $g$ sont identiques : tous les points de la droite $\C_f$ sont des points d'intersection.
		
		
		\item 
			Similairement,
			\begin{align*}
				f(x_P) &= g(x_P) \\
				2x_P + 1 & = 2x_P + 2 \\
				1 &= 2,
			\end{align*}
			cette équation étant évidemment fausse, aucun $x_P \in \R$ ne vérifie l'équation.
			
			En conclusion, $\C_f \cap \C_g = \emptyset$, l'ensemble vide.
			En fait, les droites $\C_f$ et $\C_g$ sont parallèles car elles ont le même coefficient directeur $2$.
			Or les droites ne sont pas égales car leurs ordonnées à l'origine sont différentes, donc il n'y a aucun point d'intersection.
	\end{enumerate}

}

\exe{[Parallélisme]
	Pour chacun des couples de point $P$ et fonction affine $f$, trouver la fonction affine $g$ telle que $\C_f$ et $\C_g$ soient parallèles et que $P \in \C_g$.
	
	
	\begin{multicols}{2}
	\begin{enumerate}
		\item $P=(1;2)$ et $f(x) = 2x+1$ $(x\in\R)$.
		\item $P=(1;2)$ et $f(x) = 7-x$ $(x\in\R)$.
		\item $P=(1;2)$ et $f(x) = 1600$ $(x\in\R)$.
		\item $P=(1;2)$ et $f(x) = -3x + 5$ $(x\in\R)$.
	\end{enumerate}
	\end{multicols}
}{
	On pose $a,b\in\R$ les paramètres de la fonction affine $g(x)=ax+b$, $x\in\R$.
	\begin{enumerate}
		\item 
			L'information de parallélisme donne immédiatement $a=2$.
			D'où $g(x) = 2x+b$ pour tout $x\in\R$.
			
			Reste à trouver $b$ avec l'information d'appartenance.
			Celle-ci est équivalente à la contrainte
				\begin{align*}
					2 &= g(1) \\
					2 &= 2\cdot1 + b \\
					b &= 0
				\end{align*}
			D'où $g(x) = 2x$ ($x\in\R$) est la fonction affine recherchée.
		
		
		\item 
			L'information de parallélisme donne immédiatement $a=-1$.
			D'où $g(x) = -x+b$ pour tout $x\in\R$.
			
			Reste à trouver $b$ avec l'information d'appartenance.
			Celle-ci est équivalente à la contrainte
				\begin{align*}
					2 &= g(1) \\
					2 &= -1 + b \\
					b &= 3
				\end{align*}
			D'où $g(x) = -x+3$ ($x\in\R$) est la fonction affine recherchée.
			
		\item 
			L'information de parallélisme donne immédiatement $a=0$.
			D'où $g(x) =b$ pour tout $x\in\R$.
			
			Reste à trouver $b$ avec l'information d'appartenance.
			Celle-ci est équivalente à la contrainte
				\begin{align*}
					2 &= g(1) \\
					2 &=  b
				\end{align*}
			D'où $g(x) = 2$ ($x\in\R$) est la fonction affine recherchée.
			
		\item 
			L'information de parallélisme donne immédiatement $a=-3$.
			D'où $g(x) = -3x+b$ pour tout $x\in\R$.
			
			Reste à trouver $b$ avec l'information d'appartenance.
			Celle-ci est équivalente à la contrainte
				\begin{align*}
					2 &= g(1) \\
					2 &= -3\cdot1 + b \\
					b &= 5
				\end{align*}
			D'où $g(x) = -3x+5$ ($x\in\R$) est la fonction affine recherchée.
	\end{enumerate}

}


\exe{[Lecture graphique]

	Déterminer les paramètres des fonction affines $f,g,h$ dont les courbes sont représentées ci-dessous.

	\begin{center}
		\begin{tikzpicture}[>=stealth, scale=1.5]
		\begin{axis}[xmin = -10, xmax=10, ymin=-10, ymax=10, axis x line=middle, axis y line=middle, axis line style=<->, xlabel={}, ylabel={}, xtick = {-10, -8, ..., 8, 10}, ytick = {-10, -8, ..., 8, 10}, grid=both]
		
			\addplot[red, thick, domain =-9:9, samples=2] {-x}  node[above=6pt] {$(\mathcal{C}_f)$};
			\addplot[red, thick, dotted, domain =-10:-9, samples=2] {-x} ;
			\addplot[red, thick, dotted, domain =9:10, samples=2] {-x};
		
		
			\addplot[green, thick, domain =-9:9, samples=2] {x/2+1}  node[below=6pt] {$(\mathcal{C}_g)$};
			\addplot[green, thick, dotted, domain =-10:-9, samples=2] {x/2+1} ;
			\addplot[green, thick, dotted, domain =9:10, samples=2] {x/2+1};
		
		
			\addplot[black, thick, domain =-9:9, samples=2] {7}  node[above=10pt, left] {$(\mathcal{C}_h)$};
			\addplot[black, thick, dotted, domain =-10:-9, samples=2] {7} ;
			\addplot[black, thick, dotted, domain =9:10, samples=2] {7};
		
			
		\end{axis}
	\end{tikzpicture}
	\end{center}
}{
	Pour la fonction $f(x) = ax+b (x\in\R)$, on peut par exemple choisir deux points lui appartenant et interpoler linéairement à partir de ces deux points.
	Par unicité de la droite passant par deux points, l'interpolation trouvée doit nécessairement être $f$.
	Prenons $A(-4;4)$ et $B(2;-2)$ et calculons les paramètres de la droite $(AB) = \C_f$.
	
	\begin{align*}
		a = \dfrac{4 - (-2)}{-4 - 2} = -1, && b = \dfrac{(-4)\cdot(-2) - 4\cdot2}{-4-2} = 0.
	\end{align*}
	
	\hrule\vspace{10pt}
	
	Pour la fonction $g(x) = ax+b (x\in\R)$, on peut faire idem en choisissant de bons points pour faciliter les calculs.
	Par exemple, $A(0;1)$ et $B(-2;0)$ sont pratiques car il y a beaucoup de zéros qui annulent les expressions.
	En fait, le point $A$ donne directement l'ordonnée à l'origine $b=1$, mais on peut calculer à l'aide du théorème du cours nonobstant.
	\begin{align*}
		a = \dfrac{1-0}{0-(-2)} = \dfrac12, && b = \dfrac{0 \cdot0 - 1 \cdot (-2)}{0-(-2)} = 1.
	\end{align*}
	
	\hrule\vspace{10pt}
	
	Pour la fonction $h(x) = ax+b (x\in\R)$, on procède par parallélisme et appartenance.
	La fonction est constante, donc $a=0$.
	N'importe quel point lui appartenant a pour ordonnée $b=7$, ce qui conclut.

}

\end{document}