\documentclass[12pt]{paper}
\usepackage[french]{babel}
\usepackage[
a4paper,
margin=2cm,
nomarginpar,% We don't want any margin paragraphs
]{geometry}
\usepackage{fancyhdr}
\usepackage{array}
\usepackage{amsmath,amsfonts,amsthm,amssymb,mathtools,}
\newcolumntype{P}[1]{>{\centering\arraybackslash}p{#1}}


\usepackage{stackengine}
\newcommand\xrowht[2][0]{\addstackgap[.5\dimexpr#2\relax]{\vphantom{#1}}}

% theorems

\theoremstyle{plain}
\newtheorem{theorem}{Th\'eor\`eme}
\newtheorem*{sol}{Solution}
\theoremstyle{definition}
\newtheorem{ex}{Exercice}


% corps
\newcommand{\C}{\mathbb{C}}
\newcommand{\R}{\mathbb{R}}
\newcommand{\Rnn}{\mathbb{R}^{2n}}
\newcommand{\Z}{\mathbb{Z}}
\newcommand{\N}{\mathbb{N}}
\newcommand{\Q}{\mathbb{Q}}

% domain
\newcommand{\D}{\mathbb{D}}


% date
\usepackage{advdate}
\AdvanceDate[0]

% plots
\usepackage{tikz}
\usepackage{multicol}

% for calligraphic C
\usepackage{calrsfs}

% euro
\usepackage{lmodern,textcomp}


% SOLUTION SWITCH
\newif\ifsolutions
				\solutionstrue
				\solutionsfalse

\ifsolutions
	\newcommand{\exe}[2]{
		\begin{ex} #1  \end{ex}
		\begin{sol} #2 \end{sol}
	}
\else
	\newcommand{\exe}[2]{
		\begin{ex} #1  \end{ex}
	}
	
\fi

\begin{document}
\pagestyle{fancy}
\fancyhead[L]{Première G5}
\fancyhead[C]{\textbf{Fonctions affines \ifsolutions -- Solutions \fi}}
\fancyhead[R]{\today}

\exe{
	Un vase droit, de base carrée de $5$cm de côté, a une hauteur de $20$cm.
	On y dépose une couche initial de sable de $2$cm
	
	\begin{enumerate}
		\item Quel est le volume du vase ?
		\item Quel est le volume de la couche initiale de sable ?
	\end{enumerate}
	
	On note $x$ la hauteur supplémentaire de sable que l'on rajoute.
	\begin{enumerate}
		\item À quel intervalle appartient $x$ ?
		\item Exprimer, en fonction de $x$, le volume total $V(x)$ de sable dans le vase.
		\item Calculer $V(0), V(3),$ et $V(18)$.
		\item Quelle hauteur de sable a été rajoutée si le volume total est de $335 \text{cm}^3$ ?
		\item Quelle hauteur minimale de sable a été rajoutée si le volume total dépasse $440 \text{cm}^3$ ?
	\end{enumerate}
}{}

\exe{ \hspace{1cm} \\
	\begin{multicols}{2}
		\begin{center}
		\begin{tikzpicture}[scale=0.8]
		% real line
		\draw[black, thick] (0,0) -- (7,0);
		\draw[black,thick] (7,0) -- (7,7);
		\draw[black,thick] (7,7) -- (2,7);
		\draw[black,thick] (2,7)--(2,2);
		\draw[black, thick](2,2)--(0,2);
		\draw[black,thick](0,2)--(0,0);
		
		\draw[black,thick, dotted](2,0)--(2,2);
		
		\draw[<->, thick] (2,-.2) -- (7,-.2) node [midway, below] {$5$} ;
		\draw[<->, thick] (0,-.2) -- (2,-.2) node [midway, below] {$x$} ;
		\draw[<->, thick] (-.2,0) -- (-.2,2) node [midway, left] {$x$} ;
		\draw[<->, thick] (7.2,0) -- (7.2,7) node [midway, right] {$12$} ;
	\end{tikzpicture}
	\end{center}
	
	Considérons la figure ci-contre. La longueur du côté du carré de gauche doit rester inférieure à la longueur du rectangle de droite.
	\begin{enumerate}
		\item À quel intervalle appartient $x$ ?
		\item Exprimer le périmètre de la figure en fonction de $x$. Est-ce une fonction affine ? Si oui, donner son coefficient directeur de son ordonnée à l'origine.
		\item Donner l'ensemble des valeurs de $x$ pour lesquelles le périmètre est supérieur ou égal à $50$.
	\end{enumerate}
	
	\end{multicols}
}
{}

\exe{
	Soit $(d)$ un droite, courbe de la fonction
		\[ f(x) = ax + b, \]
	et soient $A(x_A,y_A)$ et $B(x_B,y_B)$ deux points distincts appartenant à la droite.
	
	Montrer que 
		\[ a  = \dfrac{y_B - y_A}{x_B-x_A}. \]
}{}

\exe{
	Soit $(d)$ une droite du plan donnée par
		\[ (d) = \{ (x,y) \text{ tq. } x,y\in\R, \text{et }  y=3x+4 \} \] 
	et $(d')$ une droite parallèle à $(d)$ et passant par le point $(1;-4)$.
	Décrire $(d')$ : quelle équation tout point $(x,y)$ vérifie-t-il si et seulement s'il appartient à $(d')$ ?
}{}

\exe{
	Soient 
		\begin{align*}
		(d) &= \{ (x,y) \text{ tq. } x,y\in\R, \text{et }  y=3x+4 \} \\
		(d') &= \{ (x,y) \text{ tq. } x,y\in\R, \text{et }  y=-x-7 \}
		\end{align*}
	\begin{enumerate}
		\item Donner le point d'intersection des deux droite : l'élément de l'ensemble $(d) \cap (d')$.
		\item Donner les variations de $(d)$ et de $(d')$.
		\item Pour quelles valeurs de $x\in\R$ la droite $(d)$ est-elle au-dessus de $(d')$ ?	
	\end{enumerate}
}{}


\end{document}